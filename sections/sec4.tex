\chap SPORY

\rule{59}
\secc Spory

\begitems \style N
\bold{Všeobecně}
* Pokud není uvedeno jinak v P 60.2 nebo jiném pravidle či předpise, všechny spory v souvislosti s těmito pravidly musí být řešeny v~souladu s dále uvedenými ustanoveními.
* Ustanovení P 60 se netýkají následujících záležitostí:
  \begitems \style a
  * všech sporů vyplývajících z rozhodnutí učiněných podle Antidopingových pravidel uvedených v Oddíle 4 bez omezení, sporů vyplývajících z porušení pravidel o dopingu. Tyto spory musí být řešeny v souladu s ustanoveními P 42;
  * porušení pravidel zákazu sázení a dalších protikorupčních zásad uvedených v Oddíle 1. Tato provinění budou řešena Etickou komisí IAAF v souladu s P 9 a Etickým kodexem;
  * všech protestů vznesených před nějakou soutěží v souvislosti se statutem atleta pro účast v této soutěži. V souladu s P 146.1 je možné proti rozhodnutí technického delegáta v těchto případech podat odvolání k Jury. Rozhodnutí Jury (nebo technického delegáta v případě, že jury nebyla ustanovena nebo protest nebyl podán) je konečné a není proti němu žádné odvolání, ani k CAS, možné. Pokud záležitost není možno uspokojivě vyřešit před soutěží, a atletu bylo povoleno startovat \uv{pod protestem}, musí být tato záležitost předložena Radě IAAF a její rozhodnutí bude konečné a není proti němu žádné odvolání, ani k CAS, možné;
  * všechny protesty nebo jiné spory vyplývající z činnosti na sportovním poli, mj. vč. protestů týkajících se výsledků nebo průběhu disciplíny. Podle P 146.3 lze proti rozhodnutí vrchního rozhodčího podat protest k Jury. Rozhodnutí Jury (nebo vrchního rozhodčího v případě, že jury nebyla ustanovena nebo protest nebyl podán) je konečné a není proti němu žádné odvolání, ani k CAS, možné.
  * sporů vzniklých na základě rozhodnutí Etické komise podle Etického kodexu. Takové spory musí být řešeny podle ustanovení Procesních pravidel Etické komise.
  \enditems

\bold{Spory týkající se atletů, členů zabezpečovacího týmu a ostatních osob}
* Každá členská federace musí mít ve svém statutu ustanovení, že pokud není uvedeno jinak ve speciálním pravidle nebo předpise, veškeré spory, týkající se atletů, jejich doprovodu nebo jiných osob pod pravomocí tohoto člena, jakkoliv vzniklé, ať se týkají záležitostí spojených s dopingem nebo nikoliv, budou postoupeny slyšení před příslušným orgánem zřízeným nebo jinak schváleným touto federací. Takové slyšení musí vyhovovat následujícím zásadám: včasné jednání před spravedlivým a nezaujatým orgánem, právo jednotlivce na informaci o obvinění proti němu vznesenému, právo na předložení důkazu, vč. práva povolat a vyslýchat svědky, právo na zastoupení právníkem a překladatelem (placeným tímto jednotlivcem), včasné a odůvodněné písemné rozhodnutí.
* Kterýkoliv atlet, člen zabezpečovacího týmu nebo jiná osoba, kdo se
  \begitems \style a
  * zúčastnil atletické soutěže nebo startoval v disciplíně, kdy buď některý ze spoluúčastníků, podle jeho znalosti, měl zastavenou činnost nebo neměl oprávnění k soutěžení podle těchto pravidel, nebo se soutěže konaly v zemi nebo oblasti člena s~pozastavenou činností. To se netýká atletických soutěží vyhrazených veteránským věkovým kategoriím (ve smyslu P 141);
  * zúčastnil atletické soutěže, která nebyla schválená podle P 2;
  * provinil proti ustanovením P 4 (účast v mezinárodních soutěžích) či některému, tam uvedenému nařízení;
  * provinil proti ustanovením P 5 (občanství a jeho změna) či některému tam uvedenému nařízení;
  * provinil proti ustanovením P 6 (peněžní úhrady) či některému, tam uvedenému nařízení;
  * provinil proti ustanovením P 7 (zástupci atleta) nebo některému tam uvedenému nařízení;
  * provinil proti ustanovením P 8 (reklamy a propagace během mezinárodních soutěží) nebo některému tam uvedenému nařízení;
  * poruší některé jiné pravidlo (mimo pravidel uvedených v P 60.2),
  \enditems
může být prohlášen za osobu nezpůsobilou k účasti podle tohoto P 60.
* V případě údajného porušení P 60.4, musí být dodržen následující postup:
  \begitems \style a
  * Obvinění musí být v písemné formě předloženo členské zemi, kam atlet, člen podpůrného týmu nebo jiná osoba přísluší (nebo jehož pravidly se cítí vázána) a ta včas zahájí vyšetřování faktů týkajících se případu.
  * Pokud na základě vyšetřování dojde členská federace k přesvědčení, že obvinění je oprávněné, musí tento člen, dříve než v~daném případě rozhodně, obviněného neprodleně informovat o vzneseném obvinění a o jeho právu na slyšení před tím, než je o provinění rozhodnuto. Pokud po provedeném šetření dojde členská federace k názoru, že k potrestání obviněné osoby není dostatek důkazů, musí členská federace o této skutečnosti ihned uvědomit IAAF a písemně odůvodnit své rozhodnutí a~zastavení případu.
  * Je-li potvrzeno, že došlo k porušení P 60.4, atlet, člen podpůrného týmu nebo jiná dotčená osoba bude vyzvána, aby své údajné jednání písemně vysvětlila a to do 7 dnů od vyrozumění. Pokud do této doby není učiněno žádné vysvětlení nebo vysvětlení není dostatečné, příslušná členská federace může atletu, členu podpůrného týmu nebo jiné dotčené osoba prozatímně zastavit činnost a o svém rozhodnutí musí okamžitě informovat IAAF. Pokud členská federace o takovém zastavení činnosti nerozhodne, může tak učinit IAAF. Proti takovému dočasnému zastavení činnosti se nelze odvolat, ale dotčená osoba má právo na řádné slyšení před příslušným orgánem členské federace ve smyslu P 60.5.e).
  * Pokud poté, co atlet, člen podpůrného týmu nebo jiná osoba byla o údajném provinění informována, do 14 dnů od doručení obvinění, písemně nepotvrdí, že si slyšení přeje, má se zato, že se svého práva na slyšení vzdala a přijímá obvinění o porušení příslušného ustanovení P 60.4.
  * Pokud atlet, člen podpůrného týmu nebo jiná osoba potvrdí, že si slyšení přeje, musí ji být poskytnuty veškeré příslušné důkazy a nejdéle do dvou měsíců po sdělení obvinění se musí uskutečnit slyšení v souladu se zásadami uvedenými v P 60.3. Jakmile je datum slyšení stanoveno, členská federace musí o~něm, uvědomit IAAF, která má právo zúčastnit se jako pozorovatel. Bez ohledu na případnou účast při slyšení, IAAF má právo podat proti rozhodnutí odvolání k CAS v souladu s P 60.14 a P 60.16.17.
  * Pokud po provedeném slyšení příslušný orgán členské federace rozhodne, že obviněná osoba porušila příslušné pravidlo nebo předpis, musí tuto osobu zbavit způsobilosti k účasti v mezinárodních a domácích soutěžích na dobu uvedenou ve směrnicích Rady IAAF, nebo na ni uvalit jinou sankci ve smyslu příslušných zásad schválených Radou. Totéž musí členská federace učinit v případě, že se obviněný vzdal práva na slyšení. Pokud uvedená směrnice neexistuje, orgán, který slyšení vedl, sám rozhodne o délce nezpůsobilosti dané osoby nebo o jiné sankci.
  * Členská federace musí do pěti dnů od rozhodnutí o něm písemně informovat IAAF a písemně sdělit i důvody, které k~rozhodnutí vedly.
  \enditems

\bold{Spory mezi členskou federací a IAAF}
* Pokud členská federace pověří provedením slyšení jakýkoliv orgán, výbor nebo tribunál, působící v rámci této federace nebo mimo ni, případně pokud je z nějakého důvodu za poskytnutí slyšení obviněné osobě odpovědný jakýkoliv orgán, výbor nebo tribunál působící mimo rámec členské federace, rozhodnutí tohoto orgánu, výboru nebo tribunálu je pro účely dále uvedeného ustanovení P 60 považováno za rozhodnutí členské federace a tak je třeba pojem \uv{Člen (členská federace)} v tomto pravidle vykládat a chápat.
* Každá členská federace musí do svého statutu, pravidel nebo předpisů zahrnout ustanovení, podle kterého všechny spory vzniklé mezi touto členskou federací a IAAF musí být předloženy Radě, která musí v závislosti na okolnostech daného případu stanovit postup rozhodnutí o tomto sporu.
* V případě, že IAAF hodlá členskou federaci suspendovat za porušení pravidel, musí být této federaci zasláno předběžné písemné sdělení důvodů suspendování a musí dostat odpovídající příležitost, aby mohla být vyslechnuta v souladu s postupem uvedeným v článku 14.10 statutu IAAF.

\bold{Spory mezi členskými federacemi}
* Každá členská federace musí do svého statutu zahrnout ustanovení, podle kterého všechny spory vzniklé mezi ní a jinou členskou federací musí být předloženy Radě, která musí v závislosti na okolnostech daného případu stanovit postup rozhodnutí o tomto sporu.

\bold{Odvolání proti rozhodnutím podle P 60.4}
* Proti všem rozhodnutím podle P 60.4  je možné v souladu s dále uvedenými ustanoveními podat odvolání k CAS. Všechna taková rozhodnutí zůstávají do rozhodnutí o odvolání v platnosti, pokud není stanoveno jinak, (viz P 60.22).
* V následujících příkladech jsou uvedena rozhodnutí, která mohou být podle těchto pravidel  podmíněna odvoláním podle P 60.4:
  \begitems \style a
  * Členská federace rozhodla, že atlet, člen zabezpečovacího týmu nebo jiná osoba porušila P 60.4;
  * Členská federace rozhodla, že atlet, člen zabezpečovacího týmu nebo jiná osoba neporušila P 60.4;
  * Členská federace rozhodla, že atlet, člen zabezpečovacího týmu nebo jiná osoba porušila P 60.4, ale neuvalila na viníka sankce podle směrnic schválených Radou;
  * Členská federace rozhodla, že pro obvinění podle 60.4 není dostatek důkazů (viz P 60.5.b);
  * Členská federace uskutečnila slyšení podle P 60.5 a atlet, člen podpůrného týmu nebo jiná dotčená osoba se domnívají, že členská federace vedením slyšení došla k mylnému rozhodnutí;
  * Členská federace uskutečnila slyšení podle P 60.5 a IAAF se domnívá, že členská federace vedením slyšení došla k mylnému rozhodnutí.
  \enditems
* V případech týkajících se atletů mezinárodní úrovně nebo členů jejich podpůrného týmu, lze proti rozhodnutí příslušného orgánu členské federace podat odvolání výhradně u CAS v souladu s dále uvedenými ustanoveními P 60.23 až 60.28.
* V případech netýkajících se atletů mezinárodní úrovně nebo členů jejich podpůrného týmu, pokud  v P 60.17 není uvedeno jinak, lze proti rozhodnutí příslušného orgánu členské federace podat odvolání k národnímu odvolacímu orgánu členské federace v souladu s jejími pravidly. Každá členská federace musí mít zavedený odvolací proces na národní úrovni, který respektuje následující zásady:
  \begitems \style -
  * bez zbytečných odkladů provedené slyšení před řádným nestranným, nezaujatým orgánem,
  * právo na zastoupení právním zástupcem (na náklady osoby podávající odvolání),
  * právo na překladatele (na náklady osoby podávající odvolání,
  * bez zbytečných odkladů učiněné písemné rozhodnutí.
  \enditems
Proti rozhodnutí národního odvolacího orgánu lze v souladu s~ustanovením P 60.16 podat odvolání k CAS.

\bold{Strany oprávněné podat odvolání}
* V jakémkoliv případě týkajícím se atletů mezinárodní úrovně či člena jejich podpůrného týmu, mají právo podat odvolání k CAS následující osoby:
  \begitems \style a
  * atlet nebo jiná osoba, která je dotčená rozhodnutím, proti němuž má být odvolání vedeno,
  * další strana případu, v němž bylo rozhodnutí učiněno,
  * IAAF,
  * MOV, pokud rozhodnutí může ovlivnit způsobilost k účasti na OH.
  \enditems
* V kterémkoliv případě, který se netýká atletů mezinárodní úrovně nebo člena jejich podpůrného týmu, mají právo podat odvolání k příslušnému národnímu orgánu osoby uvedené v pravidlech daného člena, ale alespoň následující osoby:
  \begitems \style a
  * atlet nebo jiná osoba, která je dotčená rozhodnutím, proti němuž má být odvolání vedeno,
  * další strana případu, v němž bylo rozhodnutí učiněno,
  * členská federace,
  \enditems
IAAF nemá právo podat odvolání proti rozhodnutí národního odvolacího orgánu členské federace, ale má právo zúčastnit se slyšení před příslušným orgánem na národní úrovni jako pozorovatel. Účast IAAF na takovém slyšení nemá vliv na její právo na odvolání k~CAS proti rozhodnutí národního orgánu na národní úrovni podle P 60.16.
* V kterémkoliv případě, který se netýká atleta mezinárodní úrovně nebo člena jeho podpůrného týmu, mají právo podat odvolání k~CAS proti rozhodnutí národního odvolacího orgánu
  \begitems \style a
  * IAAF a
  * MOV, pokud rozhodnutí může ovlivnit způsobilost k účasti na OH.
  \enditems
* V kterémkoliv případě, který se netýká atleta mezinárodní úrovně nebo jeho doprovodu, IAAF nebo WADA (pouze v případech týkajících se dopingu) mají právo podat, v souladu s P 60, odvolání proti rozhodnutí příslušného orgánu členské federace přímo k~CAS za následujících okolností:
  \begitems \style a
  * členská federace nemá odvolací orgán na národní úrovni, nebo
  * žádná ze stran uvedených v P 60.15 nepodala odvolání k národnímu odvolacímu orgánu členské federace, nebo
  * pravidla členské federace to umožňují.
  \enditems
* Každá strana, která podá odvolání podle ustanovení těchto pravidel má právo na asistenci CAS při získávání relevantních informací od orgánu, proti jehož rozhodnutí má být odvolání zaměřeno a~taková informace musí být poskytnuta, pokud to CAS nařídí.

\bold{Protistrany odvolacího řízení před CAS}
* Pokud není dále uvedeno jinak, je podle těchto pravidel protistranou (odpůrcem) u odvolacího řízení před CAS orgán, který učinil rozhodnutí, které je předmětem odvolacího řízení. Pokud člen pověřil vedením slyšení podle těchto pravidel jiný orgán, komisi nebo tribunál ve smyslu P 60.6,  je protistranou odvolání proti takovému rozhodnutí podaného k CAS člen.
* Pokud odvolání k CAS podá IAAF, je IAAF oprávněna připojit se dle svého uvážení k odvolání jako další protistrana k ostatním protistranám, vč. atleta, člena podpůrného týmu atleta nebo jiné osoby či organizace, která může být rozhodnutím dotčena.
* V kterémkoliv případě, kdy IAAF není jednou ze stran odvolání podanému k CAS, může se IAAF přesto rozhodnout, že se odvolání zúčastní jako strana, pokud to bude považovat za vhodné. Pokud se IAAF rozhodne k účasti, má plná práva odpůrce a spolu s další stranou může jmenovat arbitra. Pokud mezi odpůrci nedojde k dohodě o tom, kdo má být tímto arbitrem, rozhoduje volba IAAF.

\bold{Odvolání IAAF k CAS}
* O případném odvolání IAAF k CAS (nebo zda se IAAF má zúčastnit odvolání jako strana ve smyslu P 60.21)  rozhoduje Rada nebo orgán jí zvolený. Rada (nebo orgán jí zvolený) současně rozhodne, zda do rozhodnutí CAS bude atletu, kterého se daná záležitost týká, zastavena činnost.

\bold{Odvolací řízení před CAS}
* Pokud Rada nerozhodne jinak, od data, kdy obdržel písemné sdělení o důvodech pro rozhodnutí,  proti němuž se lze odvolat (v~angličtině nebo francouzštině, je-li stranou, která by mohla podat odvolání, IAAF), má odvolatel 60 dnů na to aby k CAS podal  formální odvolání podle P 60.15. Pokud odvolání nepodává IAAF, musí odvolávající se strana současně se sdělením CAS podat kopii odvolání IAAF. Během 15 dnů od termínu pro podání odvolání, musí odvolatel své podání k CAS věcně doplnit. Odpůrce musí své vyjádření podat CAS do třiceti dní poté, co obdržel doplněné podání odvolatele.
* Všechna odvolání k CAS musí mít formu nového slyšení s uvedením všeho, co se případu týká aby komise CAS mohla svým rozhodnutím nahradit rozhodnutí příslušného orgánu členské federace, pokud dojde k závěru, že rozhodnutí takového orgánu nebo IAAF bylo chybné nebo procedurálně nesprávné. Komise CAS může v každém případě zvýšit nebo snížit sankce uložené napadeným rozhodnutím.
* Při všech odvoláních k CAS týkajících se IAAF, jsou CAS a komise CAS vázány Statutem, Pravidly a Předpisy IAAF (včetně procedurálních směrnic). V případě jakékoliv kolize mezi platnými pravidly CAS a Statutem, Pravidly a Předpisy IAAF, mají materiály IAAF přednost.
* Při všech odvoláních k CAS týkajících se IAAF a řídícími zákony budou zákony Monackého knížectví a jednací řečí angličtina, pokud se strany nedohodnou jinak.
* Komise CAS může ve vhodných případech přiznat straně náhradu výdajů nebo příspěvek na výdaje spojené s odvolacím řízením před CAS.
* Rozhodnutí CAS je konečné a závazné pro všechny strany, všechny členské federace a není proti němu právně přípustné žádné odvolání. Rozhodnutí CAS má okamžitý účinek a všechny členské federace musí učinit veškeré kroky pro zajištění jeho účinnosti. O rozhodnutí CAS bude generální sekretář IAAF v následující zprávě informovat všechny členské federace.
\enditems

\endinput