\sec ČINOVNÍCI

\rule{109}
\secc Mezinárodní činovníci

Pro soutěže uvedené v~P1.1.a), b), c) a f) mají být jmenováni následující mezinárodní činovníci:
\begitems \style a
* organizační delegát(i),
* technický delegát(i),
* lékařský delegát,
* delegát anti-dopingové kontroly,
* mezinárodní techničtí činovníci (ITO),
* mezinárodní rozhodčí chůze,
* mezinárodní měřič tratí pro soutěže na silnici,
* mezinárodní startér,
* mezinárodní rozhodčí cílové kamery,
* odvolací komise -- jury.
\enditems

Počet činovníků pro každou funkci, jak, kdy a kým budou jmenováni, musí být uvedeno v~příslušných technických předpisech IAAF (předpisech příslušné oblastní asociace).

Pro soutěže uvedené v~P1.1.a) a P 1.1.e) může IAAF ustanovit komisaře pro reklamu.
Pro soutěže uvedené v~P 1.1.c),f),j) jakákoliv taková jmenování provede příslušná oblastní asociace, a pro soutěže vyjmenované v~P 1.1.b) příslušná pořádající organizace a pro soutěže podle P1.1.d),h) a i) příslušná členská organizace IAAF.

POZN. 1: Mezinárodní činovníci mají mít odlišné oblečení nebo označení.

POZN. 2: Mezinárodní činovníci uvedení v~e) až i) mají být pokud možno zařazeni(delegováni) z~úrovně IAAF nebo oblasti, v~souladu s~příslušnou politikou IAAF.

Výdaje na cestu a ubytování každého činovníka jmenovaného IAAF nebo oblastní asociací podle tohoto pravidla nebo P3.2, musí být tomuto činovníku uhrazeno podle příslušných předpisů.

Pozn. 1: Pro soutěže v~rámci ČR je počet činovníků dán soutěžním řádem.

Pozn. 2: Pro soutěže řízené ČAS jsou delegováni pouze vedoucí činovníci a to složkou, na jejíž úrovni je soutěž pořádána.
Ostatní činovníky deleguje pořadatel.

\secc Organizační delegáti

Organizační delegáti udržují úzké spojení s~pořadatelem soutěží a~podávají pravidelně zprávy IAAF (nebo oblastní asociaci, či jinému příslušnému řídícímu orgánu) a zabývají se, je-li to nutné, otázkami týkajícími se povinností a finančních závazků pořádající členské země a pořadatele dané soutěže.
Spolupracují s~technickými delegáty.

\secc Techničtí delegáti

Techničtí delegáti spolu s~pořadatelem, který jim musí poskytnout veškerou pomoc, jsou odpovědní za to, že veškerá technická opatření plně odpovídají technickým pravidlům IAAF a manuálu IAAF Atletická zařízení (\uv{IAAF Track and Field Facilities Manual}).

Techničtí delegáti určení pro soutěže trvající déle než jeden den:
\begitems \style a
* zajistí, že příslušnému orgánu jsou předloženy návrhy časového pořadu soutěží a účastnických výkonnostních limitů;
* schválí seznam náčiní, které bude použito, a schválí zda atleti budou moci použít vlastní náčiní nebo náčiní poskytnuté pořadatelem;
* zajistí, že příslušné technické předpisy jsou doručeny všem účastnícím se členům v~dostatečném předstihu před soutěžemi;
* zodpovídají za všechny technické přípravy nutné pro uskutečnění atletických disciplín;
* kontrolují přihlášky a mají právo je odmítnout pro technické důvody nebo v~souladu s~P146.1 (Odmítnutí pro jiné, než technické důvody, musí být založeno na ustanoveních IAAF nebo Oblastní Asociace, či jiného příslušného řídícího orgánu);
* určí kvalifikační limity pro soutěže v~poli a jak se bude postupovat z~kvalifikačních kol v~atletických soutěžích;
* v~souladu pravidly a příslušnými technickými předpisy určí nasazování a pořadí atletů a schválí startovní listiny;
* na požádání předsedají Technické poradě a informují Technické činovníky (ITO).
* Před soutěží zajistí předložení písemných zpráv o~jejích přípravách a po ukončení zajistí jejich zahrnutí do výsledné zprávy spolu s~doporučeními pro budoucí soutěže.
\enditems

Techničtí delegáti jmenovaní pro jednodenní soutěže poskytnou pořadateli veškerou pomoc a rady a zajistí přeložení zprávy o~uskutečněné soutěži.

Příslušné informace jsou uvedené ve směrnici IAAF pro technické delegáty (IAAF Technical Delegates Guidelines) a je možno je stáhnout z~webových stránek IAAF.

\secc Lékařský delegát

Lékařský delegát musí:
\begitems \style a
* rozhodovat s~konečnou platností o~všech lékařských záležitostech;
* zajistit, aby v~místě tréninků, rozcvičování a konání soutěží byly k~dispozici odpovídající prostředky pro lékařská vyšetření, ošetření a pohotovostní péči a lékařská péče byla poskytována též v~místě ubytování atletů;
* provádět vyšetření a vystavovat lékařská osvědčení v~souladu s~P142.4;
* mít oprávnění nařídit atletu nenastoupit k~soutěži nebo okamžitě odstoupit ze soutěže.
\enditems

POZN. 1: Oprávnění podle bodů c) a d) výše mohou být Lékařským delegátem (nebo tam, kde delegát není nominován nebo není k~dispozici) přenesena na lékaře nominovaného pořadatelem. Tento lékař má být označen páskou na ruce, vestou nebo podobným výrazným oblečením.

POZN. 2: Každý atlet, který nenastoupil nebo odstoupil ze soutěže v~běhu nebo chůzi ve smyslu P113.d), musí být ve výsledkové listině označen jako DNS, resp. DNF. Kterýkoliv atlet, který příkazu lékaře neuposlechne, bude vyloučen z~další účasti v~dané soutěži.

POZN. 3: Každý atlet, který nenastoupil nebo odstoupil ze soutěže v~poli ve smyslu P113.d), musí být ve výsledkové listině označen jako DNS, pokud neabsolvoval žádný pokus. Pokud nějaký pokus absolvoval, dosažený výkon zůstává v~platnosti a atlet bude podle něj uveden ve výsledcích. Kterýkoliv atlet, který příkazu lékaře neuposlechne, bude vyloučen z~další účasti v~dané soutěži.

POZN. 4: Každý atlet, který nenastoupil nebo odstoupil ze soutěže ve víceboji ve smyslu P113.d), musí být ve výsledkové listině označen jako DNS, pokud se nepokusil o~start v~první disciplíně. Pokud se o~start v~první disciplíně pokusil, bude uplatněno ustanovení P200.10. Kterýkoliv atlet, který příkazu lékaře neuposlechne, bude vyloučen z~další účasti v~soutěži.


\secc Anti-dopingový Delegát

Anti-dopingový delegát musí ve spolupráci s~pořadatelem zajistit, že pro dopingovou kontrolu jsou připraveny vhodné prostředky.
Musí být odpovědný za všechny záležitosti týkající se dopingové kontroly.

\secc Mezinárodní techničtí činovníci -- ITO

\begitems \style N
* Jsou-li delegováni ITO, Techničtí delegáti určí jednoho z~delegovaných ITO jako vedoucího ITO, pokud tento již nebyl určen dříve příslušným orgánem. Pokud je to možné, vedoucí ITO, ve spolupráci s~Technickými delegáty, určí pro každou disciplínu programu soutěží alespoň jednoho ITO. Každý ITO se stává vrchním rozhodčím (referee) disciplíny, která je mu přidělena.
* Při soutěžích v~přespolních bězích, bězích na silnici, bězích do vrchu a bězích krajinou musí ITO, pokud je jmenovaný, poskytnout veškerou nezbytnou podporu pořadateli soutěže. Musí být přítomni po celou dobu konání příslušné soutěže. Ručí za to, že soutěže proběhnou v~souladu s~technickými pravidly a technickými předpisy IAAF a rozhodnutími technických delegátů. ITO musí působit jako vrchní rozhodčí každé disciplíny, která je mu přidělená.
\enditems

Příslušné informace jsou uvedené ve směrnici IAAF pro ITO (IAAF ITO Guidelines) a je možno je stáhnout z~webových stránek IAAF.

\secc Mezinárodní rozhodčí chůze (IRWJ -- MRCH)

Rozhodčí chůze jmenovaní pro mezinárodní soutěže uvedené v~P1.1.a) musí být mezinárodními rozhodčími chůze IAAF.

POZN.: Rozhodčí chůze delegovaní pro soutěže uvedené v~P1.1.b), c), e), f), g) a j) musí být mezinárodními rozhodčími chůze IAAF nebo mezinárodními rozhodčími chůze oblasti.

\secc Mezinárodní měřič tratí pro soutěže na silnici

Pro soutěže uvedené v~P1.1 musí být určen mezinárodní měřič tratí pro soutěže na silnici, který ověří tratě pro soutěže konané zcela nebo částečně mimo stadion.

Měřič musí být uveden na Panelu mezinárodních měřičů tratí vedeném IAAF/AIMS (kvalifikační stupeň A~či B).

Tratě musí být přeměřeny v~dostatečném předstihu před datem konání soutěží.

Měřič ověří a potvrdí, že trať odpovídá pravidlům IAAF pro soutěže na silnici (viz P240.2, P240.3 a P230.11 a příslušné Poznámky).
Musí rovněž zajistit dodržení pravidel P260.20 a P260.21 v~případě dosažení světového rekordu.

Měřič spolupracuje s~pořadatelem při přípravě trati a sleduje průběh soutěže, aby bylo zaručeno, že trať absolvovaná atlety je shodná s~tratí změřenou a odsouhlasenou.
Příslušné osvědčení (certifikát) předá Technickému delegátu.

Pozn.: Pro soutěže ČAS musí být trať vyměřena nebo ověřena měřičem tratí ČAS.

\secc Mezinárodní startér a mezinárodní rozhodčí cílové kamery

Pro všechny soutěže uvedené v~P1.1.a), b), c) a f) konané na stadionu musí IAAF, příslušná oblastní asociace či příslušný řídící orgán jmenovat mezinárodního startéra a mezinárodního rozhodčího cílové kamery.

Mezinárodní startér startuje závody a vykonává další úkoly, přidělené technickým delegátem a dohlíží na kontrolu a provoz startovního informačního systému.
Mezinárodní rozhodčí cílové kamery rovněž dohlíží na veškerou činnost cílové kamery a stává se vedoucím rozhodčím cílové kamery.

Příslušné informace jsou uvedené ve směrnici IAAF pro startování (IAAF Starting Guidelines) a směrnici IAAF pro cílovou kameru (IAAF Photo Finish Guidelines) a je možno je stáhnout z~webových stránek IAAF.

\secc Odvolací komise -- jury

Při všech soutěžích uvedených v~P1.1.a), b), c) a f) musí být ustanovena jury, která obvykle sestává ze tří, pěti nebo sedmi osob.
Jeden ze členů musí být předsedou a další sekretářem.
Pokud je to účelné, sekretářem může být osoba, která není členem jury.

V~případech, kdy se jedná o~protest vztahující se k~P230, alespoň jedním členem jury musí být mezinárodním rozhodčím chůze IAAF (nebo oblasti).

Členové jury se nesmí zúčastnit žádného jednání jury týkajícího se protestu spojeného přímo nebo nepřímo s~atletem jejich vlastní národní federace.
Předseda jury musí vyzvat každého člena, jehož se toto ustanovení týká, aby v~daném případě odstoupil, pokud to příslušný člen jury již neučinil sám.
IAAF nebo příslušný řídící orgán musí určit jednoho nebo více náhradníků, kteří zastoupí kteréhokoliv člena jury, který se nemůže zúčastnit projednávání daného případu.

Obdobně bude jury ustanovena tam, kde to pořadatel považuje za žádoucí nebo nutné pro řádný průběh soutěží.

Hlavním úkolem jury je jednat o~protestech podaných v~souladu s~P146 a o~sporných otázkách vzniklých v~průběhu soutěží a jí předložených k~řešení.

Pozn. 1: Rozhodnutí jury je konečné, jury však může sama své rozhodnutí změnit, budou-li předloženy přesvědčivé důkazy.

Pozn. 2 : Při soutěžích v~rámci ČAS musí být hlavní rozhodčí členem jury.

Pozn. 3 : Pokud v~soutěžích ČAS nebyla jury ustanovena, řeší veškerá odvolání k~protestům s~konečnou platností hlavní rozhodčí.

\secc Činovníci závodů

Pořadatel soutěží ustanoví všechny činovníky podle pravidel té členské země, v~níž se závody konají, v~případě soutěží uvedených v~P1.1.a), b), c) a f) podle pravidel a způsobu řízení příslušného řídícího orgánu.

Následující seznam zahrnuje činovníky nezbytné pro hlavní mezinárodní soutěže.
Pořadatel jej může pozměnit podle místních okolností.

\bold{SPRÁVNÍ ČINOVNÍCI}
\begitems \style -
* ředitel soutěží (P121)
* hlavní rozhodčí a přiměřený počet asistentů (zástupců) (P122)
* technický ředitel a přiměřený počet asistentů (zástupců) (P123)
* manažer prezentace soutěží (P124)
\enditems

\bold{SOUTĚŽNÍ ČINOVNÍCI -- sbor rozhodčích}
\begitems \style -
* vrchní rozhodčí svolavatelny
* vrchní rozhodčí atletických soutěží
* vrchní rozhodčí soutěží v~poli
* vrchní rozhodčí soutěží ve vícebojích
* vrchní rozhodčí soutěží mimo stadion
* vrchní rozhodčí videa
* vrchník a přiměřený počet rozhodčích soutěží na dráze (P126)
* vrchník a přiměřený počet rozhodčích pro každou disciplínu soutěží v~poli (P126)
* vrchník a přiměřený počet asistentů a pět rozhodčích pro každou soutěž v~chůzi na dráze (P230)
* vrchník a přiměřený počet asistentů a osm rozhodčích pro každou soutěž v~chůzi mimo stadion (P230)
* další rozhodčí soutěží v~chůzi, podle potřeby, vč. zapisovatelů, rozhodčích u~tabule varování, atd. (P230)
* vedoucí úsekový rozhodčí a přiměřený počet úsekových rozhodčích (P127)
* vedoucí časoměřič a přiměřený počet časoměřičů (P128)
* vrchník cílové kamery a přiměřený počet asistentů (P128 a P165)
* vrchník čipového časoměrného systému a přiměřený počet asistentů (P128 a P165)
* vrchník startů a dostatečný počet startérů a zástupců startéra (P126)
* asistent startéra (P130)
* vedoucí počítač kol a přiměřený počet počítačů kol (P131)
* sekretář soutěží a přiměřený počet asistentů (P132)
* manažer technického informačního střediska (závodní kanceláře) a přiměřený počet asistentů (P132.5)
* hlavní pořadatel a dostatečný počet pořadatelů na hřišti (P133)
* měřiči rychlosti větru (P134)
* vrchník měření (exaktního) a přiměřený počet asistentů (P135)
* vrchník a dostatečný počet rozhodčích svolavatelny (P126)
* komisař pro reklamu (P137)
\enditems

\bold{DALŠÍ ČINOVNÍCI}
\begitems \style -
* hlasatelé
* statistici
* lékaři
* pomocníci pro atlety, činovníky a media
\enditems

Pozn.: Při soutěžích ČAS jsou sekretář soutěže a hlavní pořadatel považováni za správní činovníky.

Vrchní rozhodčí a vrchníci mají být zřetelně označeni oblečením, páskou na rukávu nebo odznakem.

Pozn.: Obdobně musí být označeni hlavní rozhodčí a jeho zástupci.

Pokud je to považováno za vhodné, je možno jmenovat i další pomocníky. Je však nutné dbát, aby na sportovišti bylo co nejméně činovníků a dalších osob.

{\bf Další SOUTĚŽNÍ ČINOVNÍCI -- sbor rozhodčích v~rámci ČAS}
\begitems \style -
* rozhodčí kontroly náčiní
* úřední měřič dráhy
\enditems

\secc Ředitel soutěží (Competition Director)

Ředitel soutěží připravuje technickou organizaci soutěží ve spolupráci s~technickými delegáty, pokud jsou pro soutěže jmenováni, ručí za její provedení a společně s~technickými delegáty řeší všechny technické problémy.

Řídí vzájemnou součinnost účastníků soutěže a pomocí komunikačního systému je ve styku se všemi klíčovými činovníky.

\secc Hlavní rozhodčí (Meeting Manager), zástupce HR

Hlavní rozhodčí je zodpovědný za správný průběh soutěží.
Kontroluje, zda se dostavili všichni činovníci, podle potřeby ustanovuje náhradníky a~je oprávněn odvolat kteréhokoliv činovníka, který by se neřídil pravidly.
Ve spolupráci s~hlavním pořadatelem dbá, aby se uvnitř soutěžního prostoru zdržovaly jen oprávněné osoby.

POZN.: Pro soutěže delší než čtyři hodiny nebo konané po více dní se doporučuje, aby hlavní rozhodčí měl jednoho nebo více zástupců.

\begitems \style N \itemnum=30
* Hlavní rozhodčí je delegován řídícím orgánem, který závody povolil a~je nadřízený všem soutěžním činovníkům. Nedostaví-li se, přebírá jeho funkci některý z~přítomných členů komise rozhodčích nominovaných řídícím orgánem nebo pořadatelem. Nedá-li se tímto způsobem hlavní rozhodčí určit, zvolí jej přítomní rozhodčí ze svého středu hlasováním.
* Nevykonává sám žádnou jinou funkci rozhodčího.
* Spolu s~ředitelem závodů provede kontrolu vybavení závodiště a~rozhodne, ve kterých soutěžních sektorech se uskuteční jednotlivé disciplíny soutěží v~poli.
* Podle časového programu a rozmístění jednotlivých soutěžních sektorů na závodišti rozhodne, kolik komisí činovníků bude zapotřebí a které disciplíny jednotlivým komisím přidělí a určí vrchníky jednotlivých komisí, pokud tak již neučinil pořadatel nebo řídící orgán.
* Musí být členem jury. Pokud nebyla ustanovena, s~konečnou platností sám rozhoduje ve všech sporných otázkách, týkajících se výkladu pravidel, které se během soutěží vyskytnou a řeší veškerá odvolání k~protestům.
* Rozhoduje o~případném přerušení soutěže, stane-li se závodiště nebo některý soutěžní sektor pro další soutěž nezpůsobilým, např. povětrnostními vlivy.
* Je oprávněn napomenout nebo vyloučit kteréhokoliv atleta pro nevhodné chování či jednání nebo pro přestupek proti pravidlům, pokud tak neučinil příslušný vrchní rozhodčí nebo vrchník.
* Při docílení výkonu lepšího nebo rovného platnému rekordu ČR zabezpečí vyplnění rekordního protokolu a přílohy.

\bold{Zástupce hlavního rozhodčího}

* Pro soutěže, kde nejsou jmenování vrchní rozhodčí, je jmenován alespoň jeden zástupce hlavního rozhodčího, který vykonává konkrétní úkoly podle pokynů hlavního rozhodčího, vč. pravomoci podle odstavce 37.
\enditems

\secc Technický ředitel (Technical Manager)

Technický ředitel je zodpovědný za to, že:
\begitems \style a
* atletické dráhy, rozběžiště, vrhačské kruhy a výseče, doskočiště, všechny sektory pro soutěže v~poli i veškeré vybavení a náčiní odpovídají těmto pravidlům;
* vybavení a náčiní budou rozmístěna a odstraněna podle technického organizačního plánu schváleného technickými delegáty;
* soutěžní sektory budou po technické stránce připraveny podle tohoto plánu;
* bude provedena kontrola a označení veškerého náčiní povoleného pro soutěže podle P187.2;
* před soutěžemi obdržel nebo byl seznámen s~veškerými certifikáty podle P148.1.
\enditems

Pozn.1: Zajištěním úkolu dle bodu d) je při soutěžích ČAS pověřen rozhodčí kontroly náčiní (viz P138.31).

Pozn. 2: Technický ředitel je zodpovědný za to, že jsou připraveny všechny zápisy.

\secc Manažer prezentace soutěží (Event Presentation Manager)

Manažer prezentace soutěží, spolu s~ředitelem závodů a ve spolupráci s~Organizačním delegátem a technickým delegátem, plánuje provedení prezentace atletů v~jednotlivých disciplínách.
Stará se, aby tento plán byl dodržen, ve spolupráci s~ředitelem závodů a příslušným delegátem řeší vzniklé problémy.
Řídí součinnost mezi členy týmu prezentace soutěží a pomocí komunikačního systému je v~kontaktu s~každým z~nich.

Pomocí hlášení a technických prostředků, které jsou k~dispozici, musí zajistit, že veřejnost je informovaná o~atletech účastnících se soutěží, vč. startovních listin a průběžných a konečných výsledků.
Oficiální výsledky (umístění, časy, výšky, vzdálenosti a body) každé disciplíny musí být po jejich obdržení předávány dále tak rychle, jak je to jen možné.

Při soutěžích konaných podle P1.1.a) určí hlasatele v~anglickém a~francouzském jazyce IAAF.

\secc Vrchní rozhodčí (Referee)

\begitems \style N
* Dle potřeby je alespoň jeden vrchní rozhodčí určen pro svolavatelnu, soutěže (atletické a chodecké) na dráze, pro soutěže v~poli, pro víceboje a pro soutěže (atletické a chodecké) mimo stadion. Pokud je třeba, musí být určen jeden (nebo více) vrchních rozhodčích videa. Vrchník běhů na dráze, který je určený pro dohled nad starty, je označován jako vrchník startů.

Vrchní rozhodčí videa musí pracovat v~řídící místnosti videa. Radí se s~ostatními vrchními rozhodčími a je s~nimi ve spojení.

* Vrchní rozhodčí dbá na dodržování pravidel a předpisů (a ostatních řádů příslušné soutěže). Rozhoduje o~jakýchkoliv protestech a námitkách týkajících se průběhu soutěže a o~všech situacích, které vznikají v~průběhu soutěží (včetně prostoru pro rozcvičení, svolavatelny a po ukončení soutěže až vč. vyhlášení vítězů) a které nejsou přímo řešeny pravidly (nebo příslušnými předpisy).

Vrchní rozhodčí nesmí působit jako rozhodčí nebo úsekový rozhodčí, ale může jednat nebo rozhodovat podle pravidel na základě vlastního pozorování a může změnit (zrušit)  rozhodnutí rozhodčího.

POZN.: Pro účely tohoto pravidla a použitelných předpisů, vč. předpisů týkajících se reklamy, se vyhlašování vítězů koná až tehdy, když skončily veškeré související činnosti (vč. fotografování, vítězného kola, kontaktů s~diváky atd.)

* Vrchní rozhodčí soutěží na dráze a vrchní rozhodčí soutěží mimo stadion mohou rozhodnout o~pořadí v~závodě pouze tehdy, když příslušní rozhodčí v~cíli nejsou schopni rozhodnout sami. Nemají oprávnění zasahovat ani do záležitostí spadajících do pravomoci vrchníka chodeckých soutěží.

Příslušný vrchní rozhodčí soutěží na dráze má pravomoc rozhodnout kteroukoliv záležitost týkající se startů, pokud nesouhlasí s~rozhodnutím startovní skupiny, vyjma případů, kdy se jedná o~zřejmý nezdařený start označený Startovním informačním systémem certifikovaným IAAF, pokud z~jakéhokoliv důvodu vrchní rozhodčí nerozhodne, že údaje tohoto Systému jsou zřejmě nepřesné.

Vrchní rozhodčí vícebojů má pravomoc rozhodovat o~průběhu soutěží ve vícebojích. Má též pravomoc zasahovat do průběhu jednotlivých disciplín víceboje.

* Příslušný vrchní rozhodčí musí zkontrolovat všechny konečné výsledky a řešit případné sporné otázky. Ve spolupráci s~vedoucím rozhodčím pro elektronické měření délek dohlíží na měření rekordních výkonů. Po ukončení každé disciplíny musí být zápis okamžitě zcela dokončen, podepsán příslušným vrchním rozhodčím (nebo jinak schválen) a předán sekretáři soutěží.
* Příslušný vrchní rozhodčí má právo napomenout nebo vyloučit ze soutěže kteréhokoliv atleta, který se proviní nesportovním nebo nevhodným jednáním nebo jednáním ve smyslu P144, P162.5, P163.14, P163.15.c), P180.5, P180.19, P230.7.d), P230.10.h) nebo P240.8.h). Napomenutí může být atletovi sděleno ukázáním žluté karty, vyloučení ukázáním červené karty. Napomenutí a vyloučení musí být uvedena v~soutěžním zápise a musí být sdělena závodní kanceláři a ostatním vrchním rozhodčím.

Pravomoc vrchního rozhodčího svolavatelny v~disciplinárních záležitostech sahá od prostoru pro rozcvičení po soutěžní prostor. Všechny ostatní záležitosti spadají pod pravomoc vrchního rozhodčího zodpovědného za disciplínu, v~níž atlet soutěží nebo soutěžil.

Příslušný vrchní rozhodčí (pokud je to možné po konzultaci s~ředitelem soutěží) může varovat nebo vykázat ze soutěžního prostoru (nebo ostatních navazujících prostorů, vč. svolavatelny, rozcvičovacího prostoru a míst vyhrazených trenérům) kohokoliv, kdo jedná nesportovním či nevhodným způsobem nebo poskytuje atletům pravidly nedovolenou pomoc.

POZN. 1: Pokud to okolnosti vyžadují, vrchní rozhodčí může vyloučit atleta bez předchozího varování (viz též poznámku k~P144.2).

POZN. 2: Vrchní rozhodčí pro soutěže mimo stadion musí, pokud je to proveditelné (např. podle P144, P230.10, nebo P240.8), před diskvalifikací vydat varování. Pokud je proti rozhodnutí vrchního rozhodčího podán protest, vše další se řídí P146.

POZN. 3 : Pokud si je vrchní rozhodčí vědom, že atlet již dostal žlutou kartu, pak při vyloučení podle tohoto pravidla nejprve ukáže žlutou kartu a vzápětí nato červenou kartu.

POZN. 4: Pokud vrchní rozhodčí ukáže žlutou kartu a neví o~již dříve udělené žluté kartě, má tato druhá žlutá karta stejný důsledek jako červená karta. Příslušný vrchní rozhodčí informuje atleta nebo jeho družstvo o~vyloučení.

* Vrchní rozhodčí může znovu své rozhodnutí přehodnotit (ať v~první instanci nebo při rozhodování o~protestu) na základě jakéhokoliv důkazu, pokud jeho nové rozhodnutí bude ještě možné uplatnit. Takové nové rozhodnutí může být uplatněno pouze před vyhlášením vítězů příslušné disciplíny nebo před jakýmkoliv rozhodnutím jury.
* Nastanou-li podle mínění vrchního rozhodčího v~průběhu soutěže takové okolnosti, že je v~zájmu spravedlnosti, aby byla soutěž nebo kterákoliv část soutěže v~některé disciplíně opakována, má právo prohlásit takovou soutěž nebo kteroukoliv její část za neplatnou a~nařídit její opakování buď tentýž den či později při jiné příležitosti, podle svého uvážení (viz též P146.4 a P163.2).
* Pokud v~soutěži konané podle těchto pravidel soutěží atlet s~tělesným postižením, příslušný vrchní rozhodčí může tomuto atletu umožnit nebo povolit odchylku od kteréhokoliv příslušného pravidla (vyjma P144.3), pokud takováto odchylka neposkytne atletu jakoukoliv výhodu oproti kterémukoliv atletu ve stejné disciplíně. V~případě pochybností nebo pokud bude takové rozhodnutí předmětem sporu, musí být případ předán jury.

POZN.: Toto pravidlo nepovoluje účast vodičům pro běžce se zrakovým postižením, pokud tak není uvedeno v~předpisech příslušné soutěže.

\itemnum=30
* Pokud vrchní rozhodčí nebyli jmenováni nebo nejsou přítomni v~soutěžním sektoru, přecházejí jejich pravomoci podle tohoto pravidla na zástupce hlavního rozhodčího, případně na příslušné vrchníky.
\enditems

\secc Rozhodčí, vrchníci (Judges, Chief Judges)

\begitems \style N
\bold{Obecná ustanovení}
* Vrchník soutěží na dráze a vrchníci jednotlivých disciplín v~poli řídí činnosti rozhodčích příslušné disciplíny. Pokud tak nebylo provedeno dříve příslušným orgánem, přidělí jednotlivým rozhodčím jejich úkoly.
* Rozhodčí mohou kdykoliv přehodnotit kterékoliv své původní rozhodnutí, pokud se zmýlili a pokud lze nové rozhodnutí uplatnit. Alternativně, nebo pokud nové rozhodnutí učinil následně vrchní rozhodčí nebo jury, musí vrchnímu rozhodčímu nebo jury podat všechny dostupné informace.

\bold{Soutěže na dráze a soutěže na silnici končící na dráze}
* Rozhodčí rozhodují o~pořadí, v~němž závodníci dosáhli cíle. Všichni rozhodčí musí sledovat závod ze stejné strany atletického oválu. V~případě, že nemohou dospět k~rozhodnutí, předají případ vrchnímu rozhodčímu, který rozhodne.

POZN:  Rozhodčí mají být v~rovině cíle ve vzdálenosti alespoň 5~m od okraje atletické tratě. Mají mít vyvýšené stanoviště.

\bold{Soutěže v~poli}
* Rozhodčí posoudí a zaznamenají každý pokus a změří každý zdařený pokus atletů ve všech disciplínách soutěží v~poli. Při soutěži ve skoku do výšky a skoku o~tyči přesně změří výšku laťky při každém zvýšení laťky, zejména při pokusu o~rekord. Zápis o~všech pokusech vedou alespoň dva rozhodčí, kteří své záznamy porovnávají na konci každého soutěžního kola. Příslušný rozhodčí signalizuje zdařený pokus zvednutím bílého praporku a nezdařený pokus červeným praporkem.

Vrchníci jednotlivých disciplín vykonávají v~rozsahu své působnosti veškeré pravomoci vrchního rozhodčího v~případě, že pro danou soutěž nebyl ani vrchní rozhodčí ani zástupce hlavního rozhodčího jmenován.
\enditems

\secc Úsekoví rozhodčí pro atletické a chodecké soutěže (Umpires Running and Race Walking Events)

\begitems \style N
* Úsekoví rozhodčí jsou asistenty vrchního rozhodčího bez práva konečného rozhodnutí.
* Jsou rozmístěni vrchním rozhodčím tak, aby mohli zblízka sledovat soutěž a v~případě porušení nebo nedodržení pravidel (kromě pravidla P 230.2) atletem či jinou osobou, podají vrchnímu rozhodčímu neprodleně písemnou zprávu o~události.
* Jakékoliv porušení pravidel signalizuje úsekový rozhodčí příslušnému vrchnímu rozhodčímu zvednutím žlutého praporku nebo jakýmkoliv spolehlivým prostředkem schváleným Technickým delegátem.
* Dostatečný počet úsekových rozhodčích musí být též pověřen sledováním předávacích území při závodech štafet.
\enditems

POZN. 1: Zpozoruje-li úsekový rozhodčí, že atlet běžel v~jiné dráze než ve své vlastní či předávka proběhla mimo předávací území, ihned vhodným materiálem vyznačí na dráze místo, kde k~porušení pravidel došlo nebo si udělá poznámku na papír či do elektronického prostředku.

POZN. 2 : Úsekový rozhodčí musí vrchnímu rozhodčímu oznámit jakékoliv porušení pravidel, i když atlet (nebo družstvo v~bězích štafet) závod nedokončí.

\secc Časoměřiči, rozhodčí cílové kamery a rozhodčí čipové časomíry (Timekeepers, Photo Finish Judges and Transponder Timing Judges)

\begitems \style N
* V~případě ručního měření časů se podle počtu startujících musí určit dostatečný počet časoměřičů. Jeden z~nich musí být určen vedoucím časoměřičem a ten přidělí úkoly jednotlivým časoměřičům. V~případě použití plně automatické cílové kamery nebo čipové časomíry časoměřiči zajišťují záložní měření časů.
* Časoměřiči, rozhodčí cílové kamery a rozhodčí čipové časomíry musí jednat v~souladu s~ustanoveními P165.
* V~případě automatického měření časů musí být určen vrchník cílové kamery a potřebný počet asistentů.

Pozn.: Vedoucí rozhodčí cílové kamery zodpovídá za funkci cílové kamery. Ve spolupráci s~technickou obsluhou se před začátkem soutěží seznámí s~činností zařízení. Dohlíží na umístění a~odzkoušení tohoto zařízení a kontrolu nulového časového údaje.

* V~případě použití čipové časomíry musí být určen vrchník této časomíry a potřebný počet asistentů.

Pozn.: Vrchník rozhodčích čipové časomíry zodpovídá za funkci systému čipové časomíry. Ve spolupráci s~technickou obsluhou se před začátkem soutěží seznámí s~činností zařízení a odpovídá za provedení kontroly správné funkce systému (Směrnice pro ovládání čipového systému jsou dostupné na webových stránkách IAAF)
\enditems

\secc Vrchník startů, startér a zástupci startéra (Start Coordinator, Starter, Recaller)

\begitems \style N
* Vrchník startů (dle IAAF koordinátor startů) má tyto úkoly:
  \begitems \style a
  * Přiděluje úkoly jednotlivým členům startovní skupiny. Nicméně, v~případě soutěží uvedených v~P 1.1.a) a oblastních mistrovství nebo her, technický delegát rozhoduje o~tom, které závody budou startovat mezinárodní startéři.
  * Sleduje, jak členové startovního týmu plní své povinnosti.
  * Informuje startéra poté, co dostal příslušný pokyn od ředitele závodů, že je vše připraveno k~zahájení startovacího procesu (tj. že časoměřiči, rozhodčí a, pokud byli jmenováni, vedoucí rozhodčí cílové kamery, vedoucí rozhodčí pro čipový časoměrný systém a měřič rychlosti větru jsou připraveni).

  Pozn.: Při soutěžích ČAS dává pokyn, že vše je připraveno k~zahájení startovacího procesu vrchník běhů nebo vrchní rozhodčí pro atletické soutěže, pokud byl jmenován.

  * Působí jako styčný pracovník mezi technickými pracovníky společnosti spravující časomíru a rozhodčími.
  * Uchovává všechny záznamy týkající se provedených startů, vč. všech záznamů ukazujících reakční časy nebo křivky nezdařených startů, pokud jsou k~dispozici.
  * Dbá na to, že na základě P162.8 a P200.8.c) je dodržováno ustanovení P162.9.
  \enditems
* Startér musí mít dokonalý přehled o~všech závodnících na startu. Pokud je pro nízký start použit Startovní informační systém, platí ustanovení P162.6.
* Startér musí zaujmout takové místo, aby během startovního procesu měl zcela pod dohledem všechny atlety. Doporučuje se, zejména při odstupňovaných startovních čarách, použít pro současný přenos startovních povelů a startovního signálu, vč. vrácení startů, všem běžcům reproduktory umístěné v~jednotlivých drahách.

POZN.: Startér musí zaujmout takové místo, že má celé startovní pole v~úzkém zorném poli. Při závodech z~nízkého startu musí být v~postavení, kdy si může být jist, že závodníci před výstřelem pistole nebo aktivace startovního zařízení, jsou zcela v~klidu. (Všechna taková startovací zařízení se pro účely pravidel nazývají \uv{pistole}). Při závodech z~odstupňovaného startu, kde nejsou použity reproduktory, zaujme startér takové místo, v~němž je jeho vzdálenost od každého atleta přibližně stejná. Tam, kde to není možné, se takto umístí pistole nebo schválené startovací zařízení a odpálí se elektricky.

* Musí být jmenován jeden či více zástupců startéra, kteří startérovi pomáhají.

POZN.: Pro závody v~bězích na 200 m, 400 m, 400 m překážek, 4x100 m, 4x200 m, 1-2-3-400 m a 4x400 m musí být alespoň dva zástupci startéra.

* Každý zástupce startéra se musí postavit tak, aby viděl každého atleta jemu přiděleného.
* Startér nebo každý zástupce startéra musí vrátit nebo zrušit start, pokud zjistí jakékoliv porušení pravidel. Po vrácení nebo zrušení startu musí své zjištění sdělit startérovi, který rozhodne, zda a~který atlet bude varován nebo vyloučen (viz též P162.7 a P162.10).
* O~napomenutí a vyloučení atleta podle P162.7, P162.8 a P200.8.c) rozhoduje pouze startér (viz též P125.3).
\enditems

\secc Asistenti startéra (Starter’s Assistants)

\begitems \style N
* Asistenti startéra musí kontrolovat, zda závodníci startují ve správném rozběhu či běhu a mají správně umístěna startovní označení.
* Musí umístit atlety do správné dráhy nebo ve správném pořadí na startovní čáře. Atlety přitom seřadí na shromažďovací čáře ve vzdálenosti 3~m za startovní čárou. (V~případě, že startovní čáry nejsou v~přímce, pak atlety umístí v~obdobné vzdálenosti před jednotlivými startovními čárami.) Poté uvědomí startéra, že je vše připraveno. Pokud je nařízen opakovaný start, asistenti startéra běžce opět seřadí.
* Asistenti startéra zodpovídají za přípravu štafetových kolíků pro první běžce jednotlivých družstev ve štafetových bězích.
* Po povelu startéra atletům \uv{Připravte se}, musí asistenti startéra dohlížet na dodržování P 162.3 a P162.4 (zaujmutí správného postavení na startu).
* V~případě nezdařeného startu asistenti startéra postupují dle P 162.9.
\enditems

\secc Počítači kol (Lap Scorers)

\begitems \style N
* Počítači kol vedou záznamy o~počtu uběhnutých kol všemi atlety při bězích delších než 1500 m. Pro závody delší než 5000 m a pro chodecké soutěže je pod vedením vrchního rozhodčího určen větší počet počítačů kol, kteří na záznamové karty zapisují v~každém kole mezičasy, sdělované úředními časoměřiči, těch atletů, za něž odpovídají. V~takovém případě sleduje každý počítač kol nejvíce čtyři atlety (šest atletů při chodeckých soutěžích).

Místo ručního záznamu počtu absolvovaných kol může být použit počítačem řízený systém, který může zahrnovat čipy nesené každým z~atletů.

* Jeden počítač kol na cílové čáře je odpovědný za tabuli oznamující počet kol, která mají závodníci ještě uběhnout. Tabule musí být měněna každé kolo, jakmile vedoucí atlet vběhne na začátek cílové rovinky. Samostatná signalizace počtu kol se provádí pro atlety, kteří byli nebo budou předběhnuti o~jedno nebo více kol.

Poslední kolo musí být signalizováno každému atletovi, obvykle zvoněním.
\enditems

\secc Sekretář soutěží (Competition Secretary), Technické informační centrum (TIC)

\begitems \style N
* Sekretář soutěží shromažďuje úplné výsledky všech disciplín, které mu předávají vrchní rozhodčí, vedoucí časoměřič, vrchník cílové kamery nebo vrchník čipového časoměrného systému a rozhodčí pro měření síly větru. Kompletní údaje předá ihned hlasateli k~vyhlášení, zaznamená výsledky a zápis (výsledkovou listinu) pak předá řediteli závodu.

Při použití počítačového výsledkového systému musí obsluha každého počítače u~soutěže v~poli zajistit, aby úplné výsledky soutěže byly zaneseny do počítačového systému. Výsledky atletických soutěží musí být do systému zaneseny pod dohledem vrchníka cílové kamery. Hlasatel a ředitel závodů musí mít přes počítač přístup k~výsledkům.

* V~disciplínách, kdy atleti soutěží za rozdílných podmínek (jako je použitá hmotnost náčiní nebo rozdílná výška překážek), musí být takové rozdíly zřetelně uvedené ve výsledcích nebo každá kategorie musí mít samostatné výsledky.
* Tam, kde soutěžní řád soutěže nespadající pod P1.1.a) dovoluje účast atletů
  \begitems \style a
  * s~asistencí jiné osoby, např. vodícího běžce,
  * používajících mechanické pomůcky, které nejsou schválené podle P144.3.d),
  \enditems
musí být jejich výsledky uvedené samostatně a pokud možno musí být uvedena též klasifikace jejich postižení.
* Ve startovní listině a výsledcích je třeba používat podle potřeby tyto standardní zkratky:
  \begitems
  * Nestartoval -- DNS
  * Nedokončil -- DNF (u~soutěží v~běhu a chůzi)
  * Bez zdařeného pokusu -- NM
  * Diskvalifikován -- DQ (následuje příslušné pravidlo)
  * Zdařený pokus ve skoku do výšky a skoku o~tyči -- O

  Pozn.: Platí i ve vrhu a hodech, kdy se neměří každý pokus.

  * Nezdařený pokus -- X
  * Vynechaný pokus -- \uv{--}
  * Odstoupil ze soutěže (soutěže v~poli nebo ve víceboji) -- r
  * Postoupil do dalšího kola na základě umístění nebo splnění kvalifikačního limitu -- Q
  * Postoupil do dalšího kola na základě dosaženého výkonu -- q
  * Kvalifikoval se  splněním limitu v~soutěžích v~poli -- Q
  * Kvalifikoval se bez splnění limitu v~soutěžích v~poli -- q
  * Postoupil do dalšího kola rozhodnutím vrchního rozhodčího -- qR
  * Postoupil do dalšího kola rozhodnutím Jury -- qJ
  * Ohnuté koleno v~chodeckém závodě -- \uv{$>$}
  * Ztráta kontaktu v~chodeckém závodě -- \uv{$\sim$}
  * Žlutá karta -- YC
  * Druhá žlutá karta -- YRC
  * Červená karta -- RC
  \enditems

Pokud je atlet v~nějaké disciplíně diskvalifikován pro porušení kteréhokoliv pravidla, musí být v~oficiálních výsledcích uvedený odkaz na pravidlo, podle nějž byl diskvalifikován.

Pokud je atlet v~nějaké disciplíně diskvalifikován pro nesportovní nebo nevhodné chování, musí být v~oficiálních výsledcích uvedeny důvody pro tuto diskvalifikaci.

* Pro soutěže uvedené v~P1.1 a), b), c), f) a g) a pro další soutěže konané déle než jeden den se doporučuje zřídit technické informační centrum (TIC). Hlavním úkolem TIC je zajistit hladkou komunikaci mezi jednotlivými delegacemi družstev, pořadateli, technickým delegátem a vedením soutěže ve všech technických a~dalších záležitostech týkajících se soutěže.

Pozn.: Při soutěžích ČAS uvedené úkoly TIC  plní závodní kancelář.
\enditems

\secc Pořadatel (Marshal)

Pořadatel má dozor nad závodištěm a nesmí dovolit vstup na závodiště a pobyt v~prostoru závodiště nikomu, kromě činovníků a atletů připravených k~soutěži nebo jiných akreditovaných osob.

\secc Měřič rychlosti větru (Wind Gauge Operator)

Měřič rychlosti větru měří rychlosti větru ve směru běhu v~disciplínách, ke kterým byl přidělený.
Zapíše a~podepíše zjištěné údaje a~předá je sekretáři soutěží.

Pozn.: Při soutěžích ČAS se naměřené údaje předávají podle dispozic vrchního rozhodčího, většinou vrchníkovi příslušné disciplíny nebo sekretáři soutěží.

\secc Rozhodčí měření (exaktního) (Measurement Judge (Scientific))

Při použití elektronického nebo video či jiného exaktního zařízení pro měření délek je určen jeden vrchník měření a jeden nebo více asistentů pro dozor nad tímto zařízením.

Před zahájením soutěží se sejde s~technickou obsluhou, seznámí se s~daným zařízením.

Před započetím soutěží dohlíží na umístění měřících přístrojů, respektující technické požadavky výrobce a laboratoře provádějící kalibraci přístroje.

Pro zajištění správné činnosti zařízení, spolu s~rozhodčími a vrchním rozhodčím pro soutěže v~poli, dohlíží na srovnávací měření s~výsledky získanými měřením pomocí kalibrovaného a ověřeného ocelového pásma, které musí být provedeno jak před, tak i po soutěži.
Zpráva o~správnosti měření, podepsaná všemi účastníky testu musí být přiložena k~výsledkům.

V~průběhu soutěží má celkový dohled nad měřeními.
Ohlásí vrchnímu rozhodčímu, že zařízení pracuje správně a přesně.

Pozn.: Tuto funkci může vykonávat přímo vrchní rozhodčí pro soutěže v~poli.

\secc Rozhodčí svolavatelny (Call Room Judges)

Vrchník svolavatelny:
\begitems \style a
* ve spolupráci s~ředitelem soutěží připraví a zveřejní časový program ve svolavatelně pro každou disciplínu, alespoň začátek a konec příchodu a čas, kdy atleti odejdou ze svolavatelny (poslední svolavatelny) do soutěžního prostoru;
* dohlíží na přesun atletů z~prostoru pro rozcvičení do soutěžního prostoru a zajišťuje, že po kontrole ve svolavatelně se atleti dostaví do soutěžního prostoru a ve stanovenou dobu jsou připraveni ke své soutěži.
\enditems

Rozhodčí svolavatelny kontrolují, zda závodníci mají národní nebo klubový dres, schválený jejich národní federací, neutrální atleti mají oblečení schválené IAAF, že startovní označení jsou na oblečení správně upevněna a odpovídají startovní listině a zda obuv, počet a rozměry hřebů na tretrách, reklamy na oblečení a taškách atletů odpovídají pravidlům a platným předpisům a neschválený materiál není vnesen do sportoviště.

Jakýkoliv nevyřešený problém nebo záležitost hlásí rozhodčí vrchnímu rozhodčímu pro svolavatelnu.

\secc Komisař pro reklamu (Advertising Commissioner)

Komisař reklamy (je-li nominován) dohlíží na dodržování platných pravidel a předpisů, týkajících se reklamy a spolu s~vrchním rozhodčím svolavatelny rozhoduje o~všech nevyřešených problémech a~případech, které vznikly ve svolavatelně.


\bold{Pravidla 138 a 139 platí pouze pro soutěže ČAS:}

\secc Rozhodčí kontroly náčiní

Rozhodčí kontroly náčiní ověřuje, zda náčiní připravené k~soutěži, ať dodané pořadatelem či vlastní náčiní atletů, odpovídá podmínkám pro připuštění k~soutěži.
O~provedené kontrole učiní záznam do protokolu o~kontrole.
Náčiní, které odpovídá parametrům pro připuštění k~soutěži, označí značkou prokazující jeho regulérnost.
Náčiní, které nevyhoví, musí vyřadit a nepřipustit k~soutěži.
O~takovém případě informuje vrchního rozhodčího pro soutěže v~poli či vrchníka příslušné disciplíny.

Podléhá vrchnímu rozhodčímu soutěží v~poli, pokud není jmenován, hlavnímu rozhodčímu.

\secc Hlasatel

Hlasatel oznamuje obecenstvu jména a dle případu i čísla atletů startujících v~jednotlivých disciplínách a všechny důležité informace, jako je složení rozběhů, vylosování drah nebo pořadí na startu a mezičasy.
Výsledky (pořadí, výkony) všech atletů ohlásí co nejdříve po obdržení příslušných údajů.

Hlasatel musí zaznamenat čas vyhlášení výsledků.

\endinput