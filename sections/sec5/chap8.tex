\sec ATLETICKÉ SOUTĚŽE MIMO DRÁHU

\secc Běhy na silnici

\begitems \style N
Vzdálenosti
* Standardními vzdálenostmi pro závody mužů i žen pro soutěže na silnici jsou 5 km, 10 km, 15 km, 20 km, půlmaratón, 25 km, 30 km, maratón (42,195 km), 100 km a silniční běhy štafetové.

POZN.: Doporučuje se, aby se běh rozestavný konal na trati o délce maratónu, nejlépe na okruhu o délce 5 km, s jednotlivými úseky o délce 5km, 10km, 5km, 10km, 5km a 7,195 km. Pro juniorské silniční běhy rozestavné se doporučuje trať půlmaratónu s úseky 5km, 5km, 5km a 6,098 km.

Trať
* Závod se musí běžet na upravených cestách. Pokud to však silniční provoz nebo jiné okolnosti nedovolují, je možno řádně vyznačenou trať vést po stezkách pro cyklisty nebo chodnících pro pěší podél silnice, nikoliv však po měkké půdě jako jsou travnaté okraje cest apod. Start a cíl mohou být na atletickém sportovišti.

POZN. 1: Doporučuje se, aby u silničních tratí standardní délky přímková vzdálenost mezi startem a cílem nepřesahovala 50 % její délky závodu. Podmínky pro uznání rekordu jsou uvedeny v P 260.21.b).

POZN. 2: je přijatelné umístit start, cíl a některé další úseky tratě závodu na travnatý, či jiný nedlážděný podklad. Takových úseků by však mělo být minimum.

* Délka trati musí být měřena po nejkratší možné dráze, kterou by atlet po cestě vymezené pro závod mohl proběhnout.

Při všech soutěžích uvedených v P 1.a) a podle možností i v P 1.b), c), je po celé délce tratě výraznou barvou vyznačena čára měření, kterou nelze zaměnit s jiným značením.

Délka trati nesmí v žádném případě být kratší, než je úředně udávaná délka závodu. Při soutěžích, jež jsou uvedeny v P 1.a), b), c) a v závodech přímo řízených IAAF nesmí odchylka měření překročit 0,1 % délky trati (tj. 42 m pro maratón) a délka trati má být předem ověřena měřičem tratí s oprávněním IAAF.

POZN. 1: Pro měření délky tratí se doporučuje metoda "kalibrovaného jízdního kola".

POZN. 2: Při vyměřování trati se doporučuje použít "koeficient zamezení zkrácení trati" (tzv. SCP-faktor), aby byla vyloučena možnost, že při pozdějším přeměřování trati bude zjištěno, že trať je kratší. Pro měření metodou "kalibrovaného jízdního kola" má tento koeficient hodnotu 0,1 %, tzn., že každý kilometr trati bude mít "měřenou délku" 1001 metr.

POZN. 3: Počítá-li se s tím, že část závodu bude vymezena pomocí dočasně umístěných pomůcek, jako jsou kužele, plůtky apod., musí být jejich poloha stanovena před započetím měření a dokumentace o jejich umístění musí být součástí každé zprávy o měření tratě.

POZN. 4: Doporučuje se, aby u silničních tratí standardní délky výškové převýšení cíle oproti místu startu nebylo větší než 1:1000, tj. 1 m na 1 km. Podmínky pro uznání rekordu jsou uvedeny v P 260.28.c).

POZN. 5: Protokol o měření tratě platí po dobu 5 let a pak je třeba trať přeměřit, i když nebyla změněna.

* Vzdálenosti v kilometrech musí být na trati viditelně vyznačené pro všechny atlety.
* Při silničních bězích rozestavných musí být délka každého úseku a startovní čára vyznačeny čarami o šířce 50 mm vedenými napříč tratě. Obdobně musí být provedeny čáry 10 m před a 10 m za koncovou čarou každého úseku vyznačující předávací území. Každá předávka, pokud není pořadatelem určeno jinak, musí zahrnovat fyzický kontakt mezi předávajícím a přebírajícím atletem a musí proběhnout zcela uvnitř tohoto území.

Start
* Závod musí být odstartován výstřelem z pistole, děla, signálem sirény nebo podobného zařízení. Použijí se povely pro běhy delší než 400 m (viz P162.3). V závodech, kterých se účastní velký počet atletů, musí být dána výstražná znamení 5 minut, 3 minuty a 1 minutu před startem.  Po povelu "Připravte se!" se závodníci shromáždí na startovní čáře způsobem učeným pořadatelem, se startér přesvědčí, že žádný atlet se nohou (nebo kteroukoliv částí těla) nedotýká startovní čáry nebo země za ní (míněno ve směru běhu) a pak závod odstartuje.

Pozn.: V soutěži družstev je startovní prostor rozdělen do sektorů. Členové každého družstva se ve svém sektoru seřadí do zástupu.

Bezpečnost závodu
* Pořadatelé závodu v běhu na silnici musí bezpodmínečně zajistit bezpečnost atletů. Při soutěžích uvedených v pravidle 1.a), b), c) musí pořadatelé zajistit uzavření silnice, po níž závod probíhá, pro motorizovanou dopravu v obou směrech.

Osvěžovací a občerstvovací stanice
* \begitems \style a
  * Voda a další vhodné občerstvení musí být k dispozici na startu a v cíli všech závodů.
  * Při všech závodech do 10 km včetně, musí být ve vhodných intervalech, přibližně po 2 až 3 kilometrech, zřízeny osvěžovací stanice (pouze voda na pití a osvěžení těla), pokud k tomu povětrnostní podmínky opravňují. Při všech závodech musí být k dispozici voda na pití a osvěžení ve vhodných intervalech, přibližně po 5 km. Při závodech delších než 10 km může být k dispozici v těchto bodech i jiné občerstvení než voda.

  POZN. 1: Tam, kde to povětrnostní podmínky vyžadují, s ohledem na povahu závod, může být voda nebo občerstvení rozmístěno po trati v kratších intervalech.

  POZN. 2: Pokud to bude považováno za vhodné s ohledem na určité klimatické podmínky, mohou být rovněž zřízeny stanice, kde bude vytvářena vodní mlha na trati.

  * Občerstvení může zahrnovat nápoje, energetické doplňky, potraviny nebo jiné položky než vodu. Organizační výbor, podle povětrnostních podmínek, určí, jaké občerstvení poskytne.
  * Občerstvení obvykle poskytuje pořadatel, ale může dovolit, aby si jej připravili závodníci sami. V takovém případě si musí také rozhodnout, kde jim má být podáno. Občerstvení, které si připraví atlet, musí být od okamžiku, kdy je předáno atletem či jeho zástupcem, pod dozorem činovníků určených pořadatelem. Tito činovníci musí zajistit, že občerstvení nebude jakýmkoliv způsobem změněno nebo modifikováno.
  * Pořadatel vymezí bariérami, stoly nebo značkami na zemi území, v němž občerstvení podáváno nebo přebíráno. Musí být přímo na vyměřené trati. Občerstvení musí být umístěno tak, aby bylo atletům snadno přístupné nebo jim může být podáváno přímo do rukou pověřenými osobami. Tyto osoby musí zůstávat ve vymezeném území a nesmí vstoupit na trať nebo překážet atletům. Žádný činovník nebo pověřená osoba se za žádných okolností nesmí pohybovat vedle atleta, když přebírá občerstvení nebo vodu.
  * Při soutěžích uvedených v P 1.1.a), b), c) a f) mohou v kterékoliv době být v území vymezeném dané členské federaci současně pouze dva její činovníci. Za žádných okolností nesmí kterýkoliv činovník běžet vedle atleta, když přebírá občerstvení.

  POZN.: Při soutěžích, kde je členská federace reprezentována více než třemi atlety, mohou technické předpisy dovolit účast dalších činovníků u stolů občerstvovací stanice.

  * Atlet může kdykoliv nést vodu nebo občerstvení v rukou nebo je mít uchycené na těle, pokud je nese od startu nebo je převzal či obdržel na oficiální stanici.
  * Atlet, který obdrží nebo přijme občerstvení nebo vodu mimo určenou stanici pro občerstvení, pokud se nejedná o lékařské důvody a není přijato od nebo pod dozorem rozhodčích závodu, nebo přijme občerstvení od jiného atleta, bude při prvním provinění varován vrchním rozhodčím ukázáním žluté karty. Při druhém takovém provinění vrchní rozhodčí atleta diskvalifikuje ukázáním červené karty. Atlet pak musí trať ihned opustit.

  POZN.: Atlet může občerstvení obdržet od jiného atleta nebo mu předat občerstvení, vodu nebo houby, pokud je nese od startu nebo je vzal či obdržel na oficiální stanici. Nicméně trvalá pomoc mezi dvěma nebo více atlety takovým způsobem může být považovaná za nedovolenou dopomoc a mohou být proto uplatněna varování nebo diskvalifikace, jak je uvedeno výše.
  \enditems

Průběh soutěží
* Při závodech na silnici smí atlet opustit vyznačenou trať se svolením rozhodčího a za jeho dozoru, pokud si odchodem z trati nezkrátí předepsanou vzdálenost.
* Pokud se příslušný vrchní rozhodčí dozví od rozhodčího, úsekového rozhodčího nebo jinak, že atlet opustil vyznačenou trať a zkrátil si tak předepsanou vzdálenost, musí takového běžce diskvalifikovat.
* Na všech klíčových místech musí být v pravidelných intervalech rozmístěni úsekoví rozhodčí. Další úsekoví rozhodčí by se měli pohybovat podél trati.
\enditems

\endinput