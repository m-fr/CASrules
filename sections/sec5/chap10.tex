\sec SVĚTOVÉ REKORDY

\secc Světové rekordy

\begitems \style N
Předkládání a schvalování výkonů
* Rekordního výkonu musí být dosaženo v bona fide soutěži, řádně vypsané, oznámené a před dnem konání schválené členskou federací země nebo teritoria, kde se konala a proběhla podle a v souladu s těmito pravidly. V soutěži jednotlivců musí startovat a s bona fide úsilím soutěžit alespoň tři atlet, v rozestavném běhu alespoň dvě družstva. Vyjma soutěží v poli konaných v souladu s P 147 a soutěží konaných mimo stadion podle P 230 a P 240 nesmí být uznán jako rekord výkon dosažený ve smíšené soutěži žen a mužů.

POZN.: Čistě ženské rekordy v soutěžích na silnici se řídí podmínkami stanovenými v P 261.

* IAAF uznává následující kategorie světových rekordů:
  \begitems \style a
  * světové rekordy,
  * světové halové rekordy,
  * světové juniorské (U20) rekordy,
  * světové juniorské (U20) halové rekordy.
  \enditems

POZN. 1: Pro účely tohoto pravidla, pokud z textu nevyplývá jinak, se pojem „světový rekord“ vztahuje na všechny kategorie rekordů podle tohoto pravidla.

POZN. 2: Jako světové rekordy podle P 260.2.a), 260.2.b) budou uznány výkony dosažené na jakýchkoliv sportovištích, které odpovídají P 260.12 nebo P 260.13.

* Atlet (nebo atleti, v případě běhu rozestavného), který dosáhne světového rekordu, musí:
  \begitems \style a
  * být oprávněn k účasti v soutěži v souladu s těmito pravidly;
  * podléhat pravomoci členské federace
  * v případě výkonu předloženého v souladu s P 260.2.b), nebo 260.2.d), pokud údaj nebyl již předtím potvrzen IAAF, svoje datum narození doložit pasem, rodným listem nebo podobným úředním dokumentem, který, pokud není doložen k žádosti, musí být IAAF bez průtahů doložen atletem nebo národní federací atleta;
  * v případě běhu rozestavného, všichni být oprávněni reprezentovat členskou federací v souladu s P5.1.
  * po ukončení disciplíny bezpodmínečně podrobit se dopingové kontrole provedené v souladu s právě platnými pravidly IAAF a Antidopingovými směrnicemi IAAF. Odebraný vzorem musí být zaslán do laboratoře akreditované u WADA a výsledek zaslán spolu s ostatními podklady požadovanými IAAF zaslán ke schválení rekordu. Pokud je výsledek takové kontroly pozitivní nebo test nebyl proveden, IAAF v žádném případě takový výkon jako rekord neschválí.
  \enditems

POZN. 1: V případě běhu rozestavného se testu musí podrobit všichni členové družstva.

POZN. 2: Pokud atlet přizná, že nějaký čas před dosažením výkonu na úrovni světového rekordu užil v té době zakázanou látku nebo použil nedovoleného postupu, pak, dle stanoviska Útvaru Integrity Atletiky (Athletics Integrity Unit), IAAF nebude takový výkon nadále pokládán za rekord.

* Dosáhne-li atlet nebo družstvo výkonu, který je roven, nebo lepší než stávající světový rekord, členská federace IAAF země, kde bylo rekordního výkonu dosaženo, musí bez prodlení shromáždit veškeré podklady požadované IAAF pro schválení rekordu. Žádný výkon nesmí být považován za světový rekord, pokud nebyl schválen IAAF. Členská federace oznámí ihned IAAF, že hodlá předložit ke schválení rekordní výkon.
* Předložený výkon musí být lepší nebo roven výkonu na úrovni stávajícího rekordu v disciplíně, kde IAAF vede světové rekordy. Výkon rovný hodnotě stávajícího rekordu, má stejné postavení jako původní rekordní výkon.
* Úřední formulář IAAF pro přihlášení rekordu musí být po vyplnění zaslán leteckou poštou kanceláři IAAF do 30 dní. Týká-li se přihláška cizího atleta nebo cizího družstva, je třeba zaslat kopii přihlášky ve stejné lhůtě členské federaci atleta nebo družstva.

POZN.: Potřebné formuláře jsou členským zemím zašle kanceláří IAAF na požádání nebo jsou k dispozici na webových stránkách IAAF.

* Členská federace v zemi, kde bylo dosaženo rekordu, zašle spolu s úřední přihláškou:
  \begitems \style a
  * tištěný program závodů (nebo totéž v elektronické podobě);
  * kompletní výsledky dané disciplíny, vč. všech informací požadovaných podle tohoto pravidla;
  * v případě výkonu dosaženého na dráze cílovou fotografii a nulový test v případě běžeckého závodu;
  * veškeré informace požadované k předložení v souladu s tímto pravidlem, pokud takovou informaci má nebo by měla mít.
  \enditems
* Výkony dosažené v kvalifikačních soutěžích, při rozhodování rovnosti výkonů ve skoku vysokém nebo skoku o tyči, či dosažené v kterékoliv disciplíně nebo její části, která je následně prohlášena za neplatnou v souladu s P 125.7 nebo P 146.4.b), nebo dosažené v jednotlivých disciplínách víceboje, bez ohledu na to, zda atlet dokončil celý víceboj, mohou být předloženy k ratifikaci jako světový rekord.
* President a první jednatel IAAF (Chief Executive Officer IAAF) jsou pověřeni, aby společně rozhodli o schválení světových rekordů. Jsou-li na pochybách, zda výkon má či nemá být schválen jako rekord, předloží věc k rozhodnutí Radě IAAF.
* Pokud byl výkon schválen jako světový rekord, IAAF:
  \begitems \style a
  * informuje členskou zemi, která žádost o uznání světového rekordu podala, členskou zemi atleta a příslušnou oblastní asociaci;
  * předá oficiální plaketu IAAF o uznání světového rekordu jeho držiteli;
  * doplní listinu světových rekordů po každém schválení nového rekordu. Listina musí obsahovat výkony uznávané IAAF od počátku své existence, nejlepší výkony kdy dosažené jednotlivcem nebo družstvy v každé disciplíně uvedené v P 261, P 262, P 263 a P 264;
  * formálně zveřejní (oběžníkem zaslaným členským federacím) stav této listiny vždy k 1.1. každého roku.
  \enditems
* Pokud výkon nebyl uznán jako světový rekord, IAAF sdělí důvody, které k rozhodnutí vedly.

Specifické podmínky
* Vyjma soutěží konaných na silnici, pro schválení výkonu jako rekord:
  \begitems \style a
  * musí být rekordního výkonu dosaženo na atletickém závodišti, které odpovídá P 140, případně P 149.2.
  * v bězích od 200 m výše musí být výkonu bezpodmínečně dosaženo na dráze, která není delší než 402,336 m (440 yardů) a závod musí být odstartován z některého místa na jejím obvodu. Toto omezení neplatí pro závody v běhu překážkovém, kde je vodní příkop umístěn vně 400 m běžeckého oválu;
  * musí být rekordního výkonu v běhu dosaženo na oválné dráze, na níž poloměr příslušné dráhy nepřesahuje 50 m, vyjma případů, kdy je zatáčka tvořena dvěma různými poloměry, z nichž delšímu nepřísluší více než 60° ze 180° zatáčky;
  * v bězích na otevřeném stadionu musí být výkonu dosaženo pouze na oválných drahách, které odpovídají P160.
  \enditems
* Pro schválení světového rekordu v hale, výkonu:
  \begitems \style a
  * musí být dosaženo na IAAF certifikovaném sportovišti nebo sportovišti, které odpovídá ustanovením P211 resp. P213;
  * musí být v bězích na tratích 200 m a delších dosaženo na drahách, jejichž jmenovitá délka není větší než 201,2 m (220 yardů);
  * může být dosaženo na oválném okruhu, jehož jmenovitá délka je menší než 201,2 m (220 yardů), pokud absolvovaná délka běhu splňuje povolenou toleranci pro danou trať;
  * na oválném okruhu musí být dosaženo v dráze, kde poloměr půdorysného průmětu dráhy běhu v klopené zatáčce s plynulým zakřiveným nepřesáhne 27 m a pro běhy na více kol musí být každý z obou přímých úseků dlouhý alespoň 30 m;
  * každá přímá trať musí vyhovovat ustanovením P212.
  \enditems
* Pro schválení světového rekordu v běžeckém a chodeckém závodu:
  \begitems \style a
  * musí být výkony změřeny úředními časoměřiči, plně automatickou časomírou s obrazovým záznamem (u níž byl proveden nulový test v souladu s P165.19), nebo pomocí čipového systému (viz P165) odpovídajícího pravidlům IAAF;
  * při závodech do a vč. 800 m (vč. 4x200 m a 4x400 m) se jako rekordy uznávají pouze výkony změřené plně automatickou časomírou, odpovídající pravidlům IAAF;
  * pro všechny rekordy na otevřeném závodišti do 200 m vč. je třeba předložit údaje o síle větru měřené v souladu s P163.8) až P163.13). Přesahuje-li rychlost větru naměřená ve směru běhu v průměru hodnotu + 2,0m/s, nebude výkon uznán jako rekord;
  * v závodě běženém v drahách nebude výkon uznaný jako rekord, pokud atlet porušil P163.3, ani v případě individuálního závodu v rámci víceboje, kdy byl atletovi zaznamenán nezdařený start povolený v souladu s P200.8.c).
  * U všech rekordů v bězích do 400 m vč., uvedených v P261 a P263, musí být použity startovní bloky připojené ke startovnímu informačnímu systému, které byl schválený IAAF. Toto zařízení musí pracovat správně a zjistit reakční časy startujících, které musí být uvedené ve výsledcích závodů.
  \enditems
* Pro světové rekordy dosažené na několik vzdáleností dosažených ve stejném závodě:
  \begitems \style a
  * musí být závod vypsán pouze na jedinou vzdálenost;
  * závod vypsaný na dosažení vzdálenosti v určitém čase však může být spojen se závodem na určitou vzdálenost (např. závod na 1 hodinu se závodem na 20 km - viz P164.4);
  * atlet může předložit ke schválení více výkonů dosažených v jednom závodě;
  * je rovněž možné, aby několik atletů předložilo ke schválení výkony dosažené ve stejném závodě.
  * Nelze uznat atletovi rekord na kratší vzdálenost, pokud nedokončil závod na celou vypsanou vzdálenost.
  \enditems
* Při světovém rekordu ve štafetových bězích:

Čas dosažený prvním atletem družstva nemůže být předložen ke schválení jako rekord.

* Pro schválení světového rekordu v soutěžích v poli:
  \begitems \style a
  * Výkony musí být změřeny třemi rozhodčími v poli buď pomocí ověřeného pásma nebo tyče z oceli či pomocí vědeckého měřícího přístroje, jehož přesnost byla ověřena v souladu s P148.
  * Pro soutěž ve skoku do dálky a trojskoku na otevřeném závodišti musí být bezpodmínečně předložen údaj o rychlosti větru měřené v souladu s ustanovením P184.10) až 12). Pokud rychlost větru naměřená ve směru skoku dosáhla více než + 2,0 m/s, nebude výkon schválen jako rekord.
  * V soutěžích v poli může být jako světový rekord uznáno více výkonů dosažených v dané soutěži za předpokladu, že každý z těchto výkonů bude roven či lepší než předcházející nejlepší výkon v daném okamžiku.
  * Ve vrhačských soutěžích musí být použité náčiní ověřené před zahájením soutěže v souladu s P123. Pokud vrchní rozhodčí během soutěže zjistí nebo se domnívá, že byl překonaný nebo vyrovnaný rekord, ihned označí náčiní, jímž bylo výkonu dosaženo, a nechá ověřit, zda stále odpovídá pravidlům nebo se jeho charakteristiky změnily. Obvykle je náčiní zkontrolováno opět po ukončení soutěže v souladu s P123.
  \enditems
* Pro dosažení světového rekordu ve vícebojích musí v jednotlivých disciplínách, kde je měřena rychlost větru, být splněny podmínky předepsané P 200.8. Navíc v soutěžích, při nichž je požadováno měření rychlost větru, průměrná rychlost větru (daná podílem součtu hodnot naměřených v jednotlivých disciplínách a počtu těchto disciplín) nesmí překročit hodnotu + 2,0 m/s.
* Pro schválení světového rekordu v chodeckém závodě musí při závodě působit alespoň tři rozhodčí, kteří jsou buď mezinárodními rozhodčími chůze IAAF, nebo mezinárodními rozhodčími chůze příslušné oblasti a všichni podepsali formulář žádosti o uznání rekordu.
* Pro schválení světového rekordu v chodeckém závodě na silnici:
  \begitems \style a
  * trať musí být změřena oprávněným měřičem IAAF/AIMS tř. A nebo B, který zajistí, že příslušná zpráva o měření a další informace vyžadované tímto pravidlem IAAF dostane na vyžádání k dispozici;
  * okruh nesmí být kratší než 1,0 km a delší než 2,5 km, start a cíl na stadionu je možný,
  * kterýkoliv z měřičů, který trať původně vyměřil nebo jiný vhodně kvalifikovaný činovník určený měřičem (po konzultaci s příslušným orgánem), musí podle kompletní dokumentace o vyměření vč. map předem ověřit, že trať pro závod je vyznačená v souladu s tratí vyměřenou a zdokumentovanou úředním měřičem. Musí jet ve vedoucím voze během závodu, nebo jinak ověřit, že závodníci absolvují stejnou trať.
  * trať musí být ověřena (tj. přeměřena) co nejpozději před závodem, v den závodu nebo co nejdříve po závodě. Je vhodné, aby přeměření provedl jiný měřič třídy A než ti, kteří provedli původní měření.

  POZN.: Pokud byla trať původně změřena alespoň dvěma měřiči třídy tř. A nebo B, nebo jedním „A“ a jedním „B“, ověření (přeměření) podle P 260.20.d) není požadováno.

  * u výkonu dosaženého na dílčí vzdálenosti daného závodu musí být splněny podmínky dané ustanoveními P260. Délka dílčího úseku musí být vyměřena a následně vyznačena jako součást vyměřování celé tratě a musí být ověřena v souladu s ustanoveními P260.20.d).
  \enditems
* Pro dosažení světového rekordu v běhu na silnici:
  \begitems \style a
  * trať musí být změřena oprávněným měřičem IAAF/AIMS tř. A nebo B, který zajistí, že příslušná zpráva o měření a další informace vyžadované tímto pravidlem IAAF dostane na vyžádání k dispozici;
  * místo startu a místo cíle, měřeno po teoretické přímé lince mezi nimi, nesmí být vzájemně více vzdáleny, než je 50 % celkové délky závodu.
  * Celkový výškový pokles mezi startem a cílem nesmí překročit hodnotu 1:1000, tj. 1 m na 1 km (0,1 %).
  * kterýkoliv z měřičů, který trať vyměřil nebo jiný vhodně kvalifikovaný činovník určený měřičem, (po konzultaci s příslušným orgánem) musí podle kompletní dokumentace o vyměření vč. map předem ověřit, že trať pro závod je vyznačená v souladu s tratí vyměřenou a zdokumentovanou úředním měřičem. Musí jet ve vedoucím voze během závodu, nebo jinak ověřit, že závodníci absolvují stejnou trať.
  * trať musí být ověřena (tj. přeměřena) co nejpozději před závodem, v den závodu nebo co nejdříve po závodě. Je vhodné, aby přeměření provedl jiným měřiče třídy A než ti, kteří provedli původní měření.

  POZN.: Pokud byla trať původně změřena alespoň dvěma měřiči třídy tř. A nebo B, nebo jedním „A“ a jedním „B“, ověření (přeměření) podle tohoto P260.21.e) není požadováno.

  * U výkonu dosaženého na dílčí vzdálenosti daného závodu, musí být splněny podmínky dané ustanoveními P260. Délka dílčího úseku musí být vyměřena a následně vyznačena jako součást vyměřování celé tratě a musí být ověřena v souladu s ustanoveními P260.21.e).
  * při běhu rozestavném musí soutěž proběhnout po úsecích 5 km, 10 km, 5 km, 10 km, 5 km, 7,195 km. Jednotlivé úseky musí být vyměřeny a vyznačeny jako součást měření celé tratě s tolerancí ± 1 % délky úseku a musí být ověřena v souladu s ustanoveními P260.21.e).
  \enditems

Pozn.: Doporučuje se, aby národní federace uplatňovaly obdobná pravidla pro uznávání vlastních rekordů.

\itemnum=30
* Rekordy České republiky jsou vedeny samostatně v kategorii dospělých, juniorů, dorostu a žactva ve všech disciplínách uvedených v P261, P262 a P263 a v dalších disciplínách podle rozsahu závodění v jednotlivých věkových kategoriích.

Rekord České republiky může vytvořit pouze atlet, který má státní občanství ČR.

* Pro schválení rekordů dospělých platí podmínky uvedené v odstavcích 1, 5, 7, 8, 12 až 21 tohoto pravidla.

Pozn.: Pro uznání rekordu dosaženého na silnici platí, že vyměření tratě musí provést alespoň jeden měřič, který je držitelem certifikátu ČASu.

Postup při schvalování rekordů ČR je uveden ve Směrnici ČAS č. 3/2004, část IV, článek 13 a 14.

Pozn.: Za řádné závody ve smyslu tohoto pravidla se považují, jak soutěže řízené ČAS, tak soutěže řízené Krajskými atletickými svazy (KAS).

* V bězích rozestavných se vedou rekordy jak oddílových/klubových družstev, tak reprezentačních výběrů. Pro vytvoření oddílového či klubového rekordu musí být všichni členové družstva, které rekordního výkonu dosáhlo, příslušníky téhož atletického oddílu/klubu.
* V běžeckých soutěžích žactva a dorostu nemůže být jako rekord příslušné věkové kategorie uznán výkon dosažený v závodě, v němž startoval alespoň jeden atlet vyšší věkové kategorie. Nicméně výkon dosažený za takových podmínek je platným výkonem a může být veden pouze jako nejlepší výkon dané kategorie.
\enditems

\secc Disciplíny, v nichž se vedou světové rekordy

\table{ll}{
Plně automatické měření výkonů  & PAM\cr
Ručně měřené výkony             & RM\cr
Čipovou časomírou měřené výkony & TM\cr
}

MUŽI

běžecké disciplíny, víceboje a soutěže v chůzi:

\table{ll}{
pouze PAM   & 100 m,  200 m,  400 m, 800 m,\cr
            & 110 m přek., 400 m přek.,\cr
            & 4 x 100 m, 4 x 200 m, 4 x 400 m, 100-200-300-400 m\cr
            & desetiboj.\cr
PAM i RM    & 1000 m, 1500 m, 1 míle, 2000 m, 3000 m,\cr
            & 5000 m, 10 000 m, 20 000 m, 1 hodina,\cr
            & 25 000 m, 30 000 m, 3000 m přek.,\cr
            & 4 x 800 m, 1200-400-800-1600 m, 4 x 1500 m.\cr
            & chůze na dráze: 20 000 m, 30 000 m, 50 000 m.\cr
PAM, RM, TM & závody na silnici: 5 km, 10 km, půlmaratón,\cr
            & maratón, 100 km, Maratonský běh rozestavný\cr
            & Chůze na silnici: 20 km, 50 km\cr
Skoky       & skok do výšky, skok o tyči, skok do dálky, trojskok.\cr
Vrh a hody  & Vrh koulí, hod diskem, hod kladivem, hod oštěpem.\cr
Víceboje    & Desetiboj.\cr
}

ŽENY

běžecké disciplíny, víceboje a soutěže v chůzi

\table{ll}{
pouze PAM    & 100 m, 200 m, 400 m, 800 m,\cr
             & 100 m přek., 400 m přek.,\cr
             & 4 x 100 m, 4 x 200 m, 4 x 400 m, 100-200-300-400 m\cr
             & sedmiboj, desetiboj.\cr
PAM i RM     & 1000 m, 1500 m, 1 míle, 2000 m, 3000 m,\cr
             & 5000 m, 10 000 m, 20 000 m, 1 hodina,\cr
             & 25 000 m, 30 000 m, 3000 m přek.,\cr
             & 4 x 800 m, 1200-400-800-1600 m, 4x1500 m,\cr
             & chůze na dráze: 10 000 m, 20 000 m, 50 000 m\cr
PAM, RM, TM  & závody na silnici:5 km, 10 km, 15 km, 20 km, půlmaratón,\cr
             & maratón, 100 km, Maratonský běh rozestavný\cr
             & Chůze na silnici: 20 km, 50 km\cr
Skoky        & skok do výšky, skok o tyči, skok do dálky, trojskok.\cr
Vrh a hody   & Vrh koulí, hod diskem, hod kladivem, hod oštěpem.\cr
}

POZN. 1: Vyjma chodeckých soutěží na dráze, IAAF vede dva světové rekordy žen dosažené na silnici: jednak rekordy dosažené v ryze ženské soutěži, jednak rekordy dosažené ve smíšené soutěži.

POZN. 2: Jako ryze ženské mohou být pořádány soutěže s odlišnou dobou startu mužské a ženské kategorie. Časový rozdíl musí být zvolen tak, aby byla vyloučena možnost dopomoci, udávání tempa nebo kolize, zejména tam, kde závod probíhá vícekolově na jednom okruhu.

Pozn.: Běh rozestavný 1-2-3-400 m je v anglickém originále označený jako „Medley Relay“, česky „smíšená štafeta“, běh rozestavný 12-4-8-1600 m jako „Distance Medley Relay“, česky „dlouhá smíšená štafeta“. S ohledem na české zvyklosti jsou v textu tyto běhy rozestavné (běžně označované jako štafety) uvedeny ve zkráceném číselném tvaru 1-2-3-400 m, resp. 12 4 8 1600 m.

Rekordy v silničních bězích na 5 km budou ustanoveny po 1.1.2018. Výkon v soutěži mužů musí být lepší než 13:10 a v soutěži žen lepší než 14:45. Pokud takové výkony nebudou v roce 2018 dosaženy, budou nejlepší výkony roku 2018 uznány jako rekordy k 1.1.2019.

\secc Světové juniorské rekordy

\table{ll}{
Plně automatické měření výkonů  & PAM\cr
Ručně měřené výkony             & RM\cr
Čipovou časomírou měřené výkony & TM\cr
}

JUNIOŘI (U20)

běžecké disciplíny, víceboje a soutěže v chůzi

\table{ll}{
pouze PAM   & 100 m, 200 m, 400 m, 800 m,\cr
            & 110 m přek., 400 m přek.,\cr
            & 4 x 100 m, 4 x 200 m, 4 x 400 m,\cr
            & desetiboj.\cr
PAM i RM    & 1000 m, 1500 m, 1 míle, 3000 m,\cr
            & 5000 m, 10 000 m, 3000 m přek.,\cr
            & chůze na dráze: 20 000 m, 30 000 m, 50 000 m.\cr
PAM, RM, TM & chůze na silnici: 10 km\cr
Skoky       & skok do výšky, skok o tyči, skok do dálky, trojskok.\cr
Vrh a hody  & Vrh koulí, hod diskem, hod kladivem, hod oštěpem.\cr
}

JUNIORKY (U20)

běžecké disciplíny, víceboje a soutěže v chůzi

\table{ll}{
pouze PAM   & 100 m, 200 m, 400 m, 800 m,\cr
            & 100 m přek., 400 m přek.,\cr
            & 4 x 100 m, 4 x 400 m,\cr
            & sedmiboj, desetiboj.\cr
PAM i RM    & 1000 m, 1500 m, 1 míle, 3000 m,\cr
            & 5000 m, 10 000 m, 3000 m přek.\cr
            & chůze na dráze: 10 000 m,\cr
PAM, RM, TM & Chůze na silnici: 20 km\cr
Skoky       & skok do výšky, skok o tyči, skok do dálky, trojskok.\cr
Vrh a hody  & Vrh koulí, hod diskem, hod kladivem, hod oštěpem.\cr
}

\secc Disciplíny světových halových rekordů

\table{ll}{
Plně automatické měření výkonů  & PAM\cr
Ručně měřené výkony             & RM\cr
}

MUŽI

běžecké disciplíny, víceboje a soutěže v chůzi

\table{ll}{
pouze PAM & 50 m, 60 m, 200 m, 400 m, 800 m,\cr
          & 50 m přek., 60 m přek.,\cr
          & 4x200 m, 4x400 m,\cr
          & sedmiboj.\cr
PAM či RM & 1000 m, 1500 m, 1 míle, 3000 m, 5000 m,\cr
          & 4x800 m.\cr
          & chůze: 5000 m\cr
Skoky     & skok do výšky, skok o tyči, skok do dálky, trojskok.\cr
Vrhy      & vrh koulí.\cr
}

ŽENY

běžecké disciplíny, víceboje a soutěže v chůzi

\table{ll}{
Pouze PAM & 50 m, 60 m, 200 m, 400 m, 800 m,\cr
          & 50 m přek., 60 m přek.,\cr
          & 4x200 m, 4x400 m\cr
          & pětiboj.\cr
PAM či RM & 1000 m, 1500 m, 1 míle, 3000 m, 5000 m,\cr
          & 4x800 m.\cr
          & chůze: 3000 m,\cr
Skoky     & skok do výšky, skok o tyči, skok do dálky, trojskok.\cr
Vrhy      & vrh koulí.\cr
}

\secc Disciplíny světových juniorských halových rekordů

\table{ll}{
Plně automatické měření výkonů  & PAM\cr
Ručně měřené výkony             & RM\cr
}

Junioři U20

běžecké disciplíny, víceboje a soutěže v chůzi

\table{ll}{
pouze PAM & 60 m, 200 m, 400 m, 800 m,\cr
          & 60 m přek.,\cr
          & sedmiboj.\cr
PAM či RM & 1000 m, 1500 m, 1 míle, 3000 m, 5000 m,\cr
Skoky     & skok do výšky, skok o tyči, skok do dálky, trojskok.\cr
Vrhy      & vrh koulí.\cr
}

Juniorky U20

běžecké disciplíny, víceboje a soutěže v chůzi

\table{ll}{
Pouze PAM & 60 m, 200 m, 400 m, 800 m,\cr
          & 60 m přek.\cr
          & pětiboj,\cr
PAM či RM & 1000 m, 1500 m, 1 míle, 3000 m, 5000 m,\cr
Skoky     & skok do výšky, skok o tyči, skok do dálky, trojskok.\cr
Vrhy      & vrh koulí.\cr
}

\secc Jiné rekordy

\begitems \style N
* Rekordy her, mistrovství mítinků a podobné rekordy mohou být zavedeny příslušným orgánem, který danou soutěž řídí, nebo pořadatelem.
* Jako rekord má být veden nejlepší výkon dosažený při kterémkoliv konání příslušné soutěže v souladu s těmito pravidly, přičemž může být zanedbána naměřená rychlost větru, pokud v příslušných řádech dané soutěže není ustanoveno jinak.
\enditems

\endinput