\chap MEZINÁRODNÍ SOUTĚŽE

\secc Mezinárodní soutěže

\begitems \style N
* Mezinárodními soutěžemi se rozumí:
  \begitems \style a
  *
    \begitems \style i
    * Soutěže zahrnuté do Světového atletického seriálu,
    * programy atletických soutěží Olympijských her.
    \enditems
  * Program atletických soutěží oblastních, regionálních nebo skupinových her, které nejsou pod výhradním dozorem IAAF.
  * Regionální nebo skupinová atletická mistrovství, která nejsou určena výhradně pro účastníky z~jediné oblasti.
  * Utkání mezi družstvy z~různých oblastí, složená z~reprezentantů členských federací nebo oblasti nebo kombinovaná družstva.
  * Mezinárodní zvací mítinky a~soutěže uznávané IAAF a~patřící do jednotlivých kategorií světové struktury soutěží IAAF.
  * Oblastní mistrovství nebo jiné soutěže sportovců různých oblastí organizované oblastní asociací.
  * Atletická součást oblastních, regionálních nebo skupinových her a~regionální nebo skupinová atletická mistrovství určená výhradně pro účastníky z~jediné oblasti.
  * Utkání mezi družstvy reprezentujícími dvě nebo více členských federací nebo kombinovaná družstva členských federací z~jediné oblasti, vyjma soutěží kategorie dorostu a~juniorů.
  * Mezinárodní zvací mítinky a~soutěže, jiné než jak je uvedeno v~P1.1.e), kde startovné, peněžité ceny nebo hodnota nepeněžitých cen přesahuje částku 50~000~USD celkově nebo částku 8~000~USD pro kteroukoliv disciplínu.
  * Oblastní programy obdobné těm, které jsou uvedené v~P1.1.e).
  \enditems
* Pravidla musí být uplatňována následujícím způsobem:
  \begitems \style a
  * Pravidla týkající se způsobilosti atletů (Oddíl 2), sporů (Oddíl 4) a~technická pravidla (Oddíl 5) musí být uplatňována při všech mezinárodních soutěžích.
  * Jiné mezinárodní organizace uznávané IAAF mohou mít a~při jimi pořádaných soutěžích uplatňovat pravidla ještě více omezující způsobilost atletů při jimi pořádaných soutěžích.
  * Pravidla týkající se dopingu (Oddíl 3) musí být uplatňována při všech mezinárodních soutěžích (pokud není v~Oddíle 3 uvedeno jinak), s~výjimkou těch, kde MOV nebo jiná mezinárodní organizace uznávaná v~této oblasti IAAF, provádí antidopingovou kontrolu podle svých pravidel, např. při OH, přičemž tato pravidla jsou uplatňována přiměřeně.
  * Pravidlo týkající se reklamy (Pravidlo 8) musí být uplatňováno při všech mezinárodních soutěžích uvedených v~P1.1.a), b), c), d) a~e). Oblastní asociace mohou vyhlásit svá vlastní pravidla pro reklamu při soutěžích uvedených v~P1.1.f), g), h), i) a~j). Pokud taková pravidla neexistují, platí příslušné pravidlo IAAF.
  * Pravidla 2 až 7 a~9 musí být uplatňována při všech mezinárodních soutěžích, pokud v~kterémkoliv pravidle není rozsah jeho platnosti uveden jinak.
  \enditems
\enditems

\secc Oprávnění pořádat mezinárodní soutěže

\begitems \style N
* IAAF, ve spolupráci s~oblastními asociacemi, je zodpovědná za dozor nad světovým systémem soutěží. Aby se vyloučily nebo minimalizovaly vzájemné konflikty, musí IAAF koordinovat svůj kalendář soutěží s~programy oblastních asociací. Všechny mezinárodní soutěže musí být schváleny IAAF nebo oblastní asociací v~souladu s~tímto pravidlem. Pro jakoukoliv kombinaci nebo zahrnutí mezinárodních mítinků do soustavy nebo ligy soutěží je nutný souhlas IAAF nebo příslušné oblastní asociace včetně nezbytných předpisů nebo smluvních podmínek pro takovou činnost. Organizace může být svěřena třetí straně. V~případě, že oblastní asociace nedokáže ve smyslu těchto pravidel řádně vést a~organizovat mezinárodní soutěže, je IAAF oprávněna zasáhnout a~podniknout kroky, které bude považovat za nezbytné.
* Pouze IAAF má právo organizovat atletické soutěže OH a~soutěže, které jsou součástí Světového atletického seriálu.
* IAAF organizuje Světová mistrovství v~každém lichém roce.
* Oblastní asociace mají právo pořádat oblastní mistrovství a~mohou pořádat další soutěže mezi oblastmi podle svého uvážení.

\bold{Soutěže vyžadující povolení IAAF}

* \begitems \style a
  * Pro pořádání všech mezinárodních soutěží uvedených v~P1.1.b), c), d) a~e) je třeba souhlasu IAAF.
  * Žádost o~pořadatelství musí členská federace, v~jejíž zemi nebo na jejímž území se má mezinárodní soutěž konat, předložit IAAF nejméně 12~měsíců před konáním soutěže, případně v~jiném termínu stanoveném IAAF.
  \enditems

\bold{Soutěže vyžadující povolení oblastní asociace}

* \begitems \style a
  * Pro pořádání všech mezinárodních soutěží uvedených v~P1.1.g), h), i) a~j) je třeba souhlasu IAAF. Souhlas s~mezinárodním zvacím mítinkem, kde startovné, peněžité ceny nebo hodnota nepeněžitých cen celkově přesahuje částku 250~000~USD nebo částku 25~000~USD pro kteroukoliv jednu disciplínu, nelze poskytnout dříve, než oblastní asociace tuto záležitost předem projedná s~IAAF.
  * Žádost o~pořadatelství musí členská federace, v~jejíž zemi nebo na jejímž území se má mezinárodní soutěž konat, předložit příslušné oblastní asociaci nejméně 12~měsíců před konáním soutěže, případně v~jiném termínu stanoveném touto oblastní asociací.
  \enditems

\bold{Soutěže schvalované členskou federací}

* Členské federace mohou schvalovat národní soutěže a~v~souladu s~ustanoveními P 4.2 a~P 4.3 na nich mohou startovat cizí atleti. Při účasti cizích atletů, startovné ceny za umístění nebo nepeněžní odměny všem atletům na těchto národních soutěžích nesmí celkově překročit částku 50~000~USD nebo 8~000~USD. Na kterékoliv takové soutěži nesmí startovat atlet, který ztratil oprávnění k~atletickým soutěžím podle pravidel IAAF, dané členské federace nebo národní federace, jejímž je členem.
\enditems

\secc Předpisy týkající se řízení mezinárodních soutěží

\begitems \style N
* Rada může připravit předpisy pro řízení mezinárodních soutěží konaných podle pravidel a~vztahy mezi atlety, zástupci atletů, pořadateli mítinků a~členskými federacemi. Tato pravidla mohou být Radou podle potřeby upravována nebo měněna.
* IAAF a~oblastní asociace mohou jmenovat jednoho nebo více delegátů, kteří se zúčastní každé mezinárodní soutěže vyžadující souhlas IAAF, resp. oblastní asociace a~budou dbát na dodržování příslušných pravidel a~předpisů. Na žádost IAAF, resp. oblastní asociace, tito představitelé do 30~dní od ukončení daného mezinárodního mítinku podají zprávu o~jeho průběhu.
\enditems

\secc Požadavky na účast v~mezinárodních soutěžích

\begitems \style N
* Každý atlet se smí zúčastnit mezinárodních atletických soutěží pokud:
  \begitems \style a
  * je členem klubu, který spadá pod pravomoc člena IAAF;
  * spadá přímo pod pravomoc člena IAAF;
  * souhlasí s~dodržováním pravidel člena;
  * mu byla udělena zvláštní způsobilost Radou IAAF, účastnit  se mezinárodních soutěží jako neutrální sportovec a~atlet splňuje podmínky této způsobilosti stanovené Radou IAAF;
  * pro soutěže, na nichž IAAF odpovídá za dopingovou kontrolu, svým podpisem formuláře IAAF stvrdil, že souhlasí s~dodržováním pravidel, předpisů a~směrnic pro mezinárodní soutěže IAAF (měněných čas od času), a~že veškeré spory, které by mohl mít s~IAAF nebo členskou federací, v~souladu s~pravidly IAAF předloží rozhodčí komisi a~souhlasí s~tím, že takový spor nebude řešit u~soudu nebo orgánu, které nejsou uvedeny v~pravidlech IAAF.
  \enditems
* Členské federace mohou požadovat, aby se žádný atlet nebo klub, patřící pod pravomoc některé členské federace, nesměl zúčastnit atletických soutěží v~cizí zemi nebo oblasti bez písemného souhlasu této členské federace. V~takovém případě žádná členská federace pořádající soutěž nesmí dovolit žádnému cizímu atletu nebo klubu jiné členské federace účast v~soutěži bez důkazu, že tento atlet nebo klub je způsobilý a~oprávněn k~účasti v~soutěži v~dané zemi nebo teritoriu. Členské federace uvědomí IAAF o~všech požadavcích na oprávnění. Pro usnadnění dodržování tohoto pravidla, bude IAAF na svých internetových stránkách uchovávat seznam členů s~těmito povinnostmi. Toto pravidlo se nevztahuje na neutrální atlety.
* Žádný atlet nesmí být organizován v~zahraničí bez předchozího souhlasu své původní federace, pokud pravidla této federace udělení souhlasu vyžadují. Ani pak federace země, v~níž atlet sídlí, nemůže přihlásit tohoto atleta do soutěže v~jiné zemi, aniž by měla předchozí souhlas původní federace. Ve všech případech, jichž se týkají tato pravidla, národní federace země nebo teritoria, kde má atlet své sídlo, musí zaslat původní národní federaci písemnou žádost a~ta musí do 30~dní odeslat písemnou odpověď na takovou žádost. V~obou případech musí být zaslání provedeno způsobem zahrnujícím potvrzení příjmu. Přijatelným způsobem je e-mailová zpráva s~automatickou odezvou. Pokud do 30 dní nedojde odpověď původní národní federace, má se za to, že souhlasí. V~případě negativní odpovědi na žádost o~souhlas, což musí být odůvodněno, může atlet nebo národní federace země nebo teritoria, kde má atlet své sídlo, podat odvolání k~IAAF. IAAF vydá směrnice pro podávání odvolání podle tohoto pravidla a~tyto směrnice budou k~dispozici na webových stránkách IAAF.
\enditems

POZN.: Ustanovení P4.3. se týkají atletů, kteří dosáhnou věku alespoň 18 let do 31. prosince příslušného roku.
Netýká se atletů, kteří nejsou občany nějaké země nebo teritoria nebo jsou politickými uprchlíky nebo neutrálními atlety.

\secc Občanství a~změna občanství

\begitems \style N
* Při mezinárodních soutěžích uvedených v~P1.1a), b), c), f) nebo g), musí být členská federace reprezentována pouze atlety, kteří jsou občany země nebo teritoria, kterou tato federace zastupuje, a~kteří vyhovují ustanovením tohoto pravidla.
* Atlet, který nikdy nesoutěžil v~mezinárodní soutěži uvedené v~P1.1.a), b), c), f) nebo g), je způsobilý reprezentovat členskou federaci IAAF v~mezinárodních soutěžích podle pravidla P1.1..a),b)c)f) nebo g), pokud je občanem země nebo teritoria z~důvodu narození nebo má rodiče či prarodiče, kteří se v~zemi nebo se v~teritoriu narodili.
* S~ohledem na znění P 5.4, atlet, který reprezentoval členskou federaci v~mezinárodní soutěži uvedené v~P 1.a),b),c),f) nebo g), není oprávněn reprezentovat jinou členskou federaci v~některé ze soutěží podle pravidla P1.1.a),b),c),f) nebo g).
* Atlet, který reprezentoval členskou federaci v~mezinárodní soutěži uvedené v~P 1.a),b),c),f) nebo g), je oprávněn reprezentovat jinou členskou federaci v~některé ze soutěží uvedených v~tomto odstavci pouze za následujících okolností:
  \begitems \style a
  * země nebo teritorium členské federace se následně stane součástí jiné země, které je nebo se sama následně stane členskou federací;
  * země nebo teritorium členské federace přestane existovat a~atlet se právně stane občanem nově vytvořené, smluvně nebo jinak na mezinárodní úrovni uznané země, která se následně stane členskou federací;
  * teritorium členské federace nemá národní olympijský výbor a~atlet se kvalifikuje pro účast na Olympijských hrách za (mateřskou) zemi pod níž teritorium spadá. V~tomto případě reprezentace mateřské země nijak neovlivní oprávnění atleta reprezentovat teritorium v~jiných mezinárodních soutěžích uvedených v~P1.1.a), b) c), f) nebo g).
  \enditems
* Ve smyslu P 21.2, za oprávnění atleta soutěžit podle těchto pravidel vždy ručí členská federace, pod níž atlet spadá. Tíha důkazů, zda je atlet oprávněn ve smyslu ustanovení P5 leží na atletovi a~jeho členské federaci. Členská federace musí IAAF poskytnout dokumentaci prokazující oprávnění atleta a~v~případě nutnosti jakékoliv další dokumenty jednoznačně prokazující oprávnění atleta. Pokud to bude požadováno, členská federace musí poskytnout ověření kopie všech dokumentů, jež považuje za rozhodující pro prokázání oprávnění atleta ve smyslu tohoto pravidla.
* Toto pravidlo se nevztahuje na neutrální atlety.
\enditems

\secc Platby poskytované atletům

Atletika je otevřeným sportem a~v~souladu s~Pravidly a~předpisy atleti smějí být placení v~hotovosti nebo jiným vhodným způsobem za přítomnost, účast a~výkony v~atletických soutěžích nebo v~jakékoli jiné komerční záležitosti vztahující se k~jejich činnosti v~atletice.

\secc Zástupci atletů

\begitems \style N
* Členská federace může povolit, aby atlet, ve spolupráci s~členskou federací, využíval služeb oprávněných zástupců při plánování, organizování a~sjednávání svého atletického programu a~sponzorských smluv. Alternativně mohou atleti zastupovat sami sebe nebo pověřit tímto úkolem příbuzného na nesmluvním základě.
* Atleti, kteří jsou zařazení mezi 30 nejlepších ve standardních disciplínách na konci kalendářního roku, nesmějí během následujícího roku uzavřít nebo prodloužit smlouvu se zástupcem, který nemá oprávnění k~této činnosti.
* Členská federace je zodpovědná za oprávnění udělená zástupům atletů. Každý člen má pravomoc nad zástupci jednajícími jménem jeho atletů, zástupci, kteří mají kancelář ve stejné zemi (nebo teritoriu) nebo zástupci, kteří jsou státními příslušníky téže země.
* Pro usnadnění působnosti členských federací v~této oblasti, Rada IAAF může vydat Pokyny pro řízení činnosti oprávněných zástupců atletů. Tyto Pokyny mohou obsahovat povinná opatření, která budou převzata do systémů pro řízení činnosti oprávněných zástupců v~rámci každé členské federace.
* Je podmínkou členství v~IAAF, aby stanovy členské federace obsahovaly ustanovení, že veškeré dohody mezi atlety a~jejich zástupci musí být v~souladu s~Pravidly a~předpisy IAAF týkajícími se zástupců atletů.
* Zástupce atleta musí být bezúhonný a~dobré pověsti. Musí mít dostatečné vzdělání a~znalosti pro svoji činnost zástupce v~souladu s~předpisy IAAF.
* Každá členská federace musí předat IAAF roční seznamy všech jí oprávněných zástupců atletů. IAAF každoročně vydá oficiální seznam oprávněných zástupců atletů.
* Kterýkoliv atlet nebo zástupce atleta, který se nebude řídit pravidly a~předpisy IAAF, může být postižen sankcemi ve smyslu pravidel a~předpisů IAAF.
\enditems

\secc Reklama a~propagace během soutěží

\begitems \style N
* Reklama a~výstavy zboží musí být povoleny během všech mezinárodních soutěží uvedených v~P 1.2.c), pokud tato reklama a~nabídka bude splňovat ustanovení tohoto pravidla a~kteréhokoliv předpisu vydaného v~této souvislosti.
* Rada může kdykoliv vydat předpisy s~podrobnou směrnicí pro formu reklamy a~způsob jakým propagační a~jiné materiály mohou být nabízeny během mezinárodních soutěží konaných podle těchto pravidel. Tyto předpisy se musí držet alespoň následujících zásad:
  \begitems \style a
  * Během mezinárodních soutěží konaných podle těchto pravidel je povolena pouze reklama komerční nebo charitativní povahy. Nesmí být dovolena žádná reklama, jejímž předmětem je podpora politických případů nebo zájem jakékoliv nátlakové skupiny, ať místní nebo zahraniční.
  * Nesmí se objevit žádná reklama, která je podle názoru IAAF nevkusná, rozptylující, útočná, hanlivá nebo nevhodná. Nesmí se objevit žádná reklama, která byť jen částečně, stíní pohled televizních kamer. Všechny reklamy musí vyhovovat bezpečnostním předpisům.
  * Reklama tabákových výrobků je zakázána. Pokud není výslovně povoleno Radou, je zakázána reklama na alkoholické výrobky.
  \enditems
* Rada může předpisy podle tohoto pravidla kdykoliv měnit.
\enditems

\endinput