\sec VÍCEBOJE

\secc Soutěže ve vícebojích

\begitems \style N
Soutěže mužů, U20 juniorů a U18 dorostenců, -- pětiboj a desetiboj
* Pětiboj se skládá z pěti disciplín, které se musí uskutečnit v jediném dnu v následujícím pořadí: skok do dálky, hod oštěpem, 200 m, hod diskem, 1500 m.
* Desetiboj se skládá z dále uvedených disciplín, které atlet musí absolvovat v určeném pořadí ve dvou po sobě jdoucích dnech:

1. den -- 100 m, skok do dálky, vrh koulí, skok do výšky, 400 m.

2. den -- 110 m přek., hod diskem, skok o tyči, hod oštěpem, 1500 m.

POZN: Dva po sobě jdoucí dny znamenají dvě po sobě jdoucí 24 hodinové období.

Soutěž žen a U20 juniorek – sedmiboj a desetiboj
* Sedmiboj se skládá z dále uvedených disciplín, které atlet musí absolvovat v určeném pořadí ve dvou po sobě jdoucích dnech:

1. den -- 100 m přek., skok do výšky, vrh koulí, 200 m.

2. den -- skok do dálky, hod oštěpem, 800 m.

* Desetiboj se skládá z dále uvedených disciplín, které atlet musí absolvovat ve dvou, po sobě jdoucích dnech v pořadí podle P200.2, nebo v následně uvedeném pořadí:

1. den -- 100 m, hod diskem, skok o tyči, hod oštěpem, 400 m.

2. den -- 100 m přek., skok do dálky, vrh koulí, skok do výšky, 1500 m

POZN: Dva po sobě jdoucí dny znamenají dvě po sobě jdoucí 24 hodinové období.

U18 Dorostenky (sedmiboj)
* Sedmiboj sestává z dále uvedených disciplín, které atlet musí absolvovat v určeném pořadí ve dvou po sobě jdoucích dnech:

1. den -- 100 m přek., skok do výšky, vrh koulí, 200 m.

2. den -- skok do dálky, hod oštěpem, 800 m.

POZN: Dva po sobě jdoucí dny znamenají dvě po sobě jdoucí 24 hodinové období.

Všeobecná ustanovení
* Podle uvážení vrchního rozhodčího musí mít každý atlet, kdykoliv je to možné, mezi koncem jedné a začátkem další disciplíny přestávku alespoň 30 minut. Je-li to možné, pak začátek první disciplíny druhého dne má být alespoň 10 hodin po ukončení poslední disciplíny prvního dne.
* V jednotlivých disciplínách víceboje, vyjma poslední, určuje složení jednotlivých běhů a skupin podle okolností technický delegát nebo vrchní rozhodčí vícebojů tak, že atlet s podobnou výkonností v dané disciplíně, dosaženou v předem stanoveném údobí, jsou umístěni do stejného běhu nebo skupiny. V každé skupině mají být alespoň 3 závodníci, nejlépe však 5 a více atletů.

Pokud tak nelze učinit s ohledem na časový pořad, je obsazení běhů nebo skupin další disciplíny stanoveno až a když jsou atlet k dispozici po absolvování předcházející disciplíny.

V poslední disciplíně víceboje budou jednotlivé běhy sestaveny tak, že poslední běh tvoří vedoucí závodníci podle pořadí po předposlední disciplíně.

Technický delegát nebo vrchní rozhodčí vícebojů, podle okolností, má právo změnit složení kterékoliv skupiny, je-li to podle jeho mínění žádoucí.

* Pro všechny disciplíny, z nichž se každá soutěž skládá, platí příslušná pravidla IAAF s následujícími výjimkami:
  \begitems \style a
  * ve skoku do dálky a při vrhu koulí a všech hodech má každý atlet pouze tři pokusy.
  * pokud není k dispozici automatická časomíra, musí být čas každého atleta měřen nezávisle třemi časoměřiči.
  * v běžeckých soutěžích je v každém běhu možný pouze jediný nezdařený start bez diskvalifikace atleta (atletů), který (kteří) jej způsobil (i). Kterýkoliv atlet, který způsobil další nezdařený start v témže běhu, musí být startérem ze závodu vyloučen. (Viz též P162 8).
  * Ve vertikálních skocích musí být zvyšování laťky jednotné po celou dobu soutěže, a to po 3 cm ve skoku vysokém a po 10 cm ve skoku o tyči.
  \enditems
* V každé běžecké disciplíně může být použit pouze jediný systém měření časů. Nicméně v případě rekordu musí být použity údaje plně automatického časoměrného systému, bez ohledu na skutečnost, zda údaje tohoto systému byly použity pro bodování ostatních startujících v dané disciplíně.
* Kterémukoliv atletovi, který se nepokusí o start nebo neprovede pokus v jedné z disciplín víceboje, nesmí být povolena účast v následujících disciplínách, ale nadále musí být považován za atleta, který soutěž vzdal. Nesmí být proto uveden v konečném pořadí.

Kterýkoliv atlet, který se rozhodl, že v soutěži ve víceboji nebude dále pokračovat, musí o svém rozhodnutí okamžitě uvědomit vrchního rozhodčího víceboje.

Pozn.: Není tedy možné, aby v některé disciplíně víceboje startoval atlet, který ve víceboji nesoutěží.

* Po ukončení každé disciplíny musí být všem atletům oznámeny získané body podle platných bodovacích tabulek IAAF, jak v právě ukončeném závodě, tak i celkové součty po všech ukončených disciplínách.

Pořadí atletů v soutěži je dáno součtem získaných bodů.

* Dosáhnou-li dva atleti nebo více atletů stejného bodového součtu na kterémkoliv místě konečného pořadí soutěže, rovnost umístění zůstává a těmto atletům bude přiznáno stejné umístění.
\itemnum=30
* V kategoriích mládeže se víceboje konají též v níže uvedeném rozsahu:

\table{|3{l|}}{\crl
Kategorie         & den & Disciplíny\crl
žáci starší 9-boj & 1.  & 100 m př., disk, tyč, oštěp\crlp{2-3}
                  & 2.  & 60 m, dálka, koule, výška, 1000 m\crl
žáci mladší 5-boj & 1.  & 60 m př., míček, 60 m, dálka, 800 m\crl
dorostenky 10-boj & 1.  & 100 m, disk, tyč, oštěp, 400 m\crlp{2-3}
                  & 2.  & 100 m př., dálka, koule, výška, 1500 m\crl
žákyně st. 7-boj  & 1.  & 100 m př., výška, koule, 150 m\crlp{2-3}
                  & 2.  & dálka, oštěp, 800 m\crl
žákyně ml. 5-boj  & 1.  & 60 m př., míček, 60 m, dálka, 800 m\crl
}

Vyjma pětibojů mladšího žactva jsou všechny ostatní víceboje dvoudenní. Pořadí jednotlivých disciplín musí být zachováno tak, jak je výše uvedeno.

Váha a rozměry nářadí a náčiní je dána údaji platnými pro samostatné závody v jednotlivých věkových kategoriích.

* Kromě rozsahů uvedených v bodech 1, 2, 3 a 31 je možno pořádat víceboje v běžeckých nebo vrhačských disciplínách (trojboj, resp. čtyřboj) a sprinterský trojboj (60 m, 100 m, 200 m). Pro bodové hodnocení se použijí běžné bodovací tabulky.
\enditems

\endinput