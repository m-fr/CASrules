\nonum\sec Definice pojmů

\dt Antidopingová organizace (Anti-doping Organization)
\dd Signatář odpovědný za přijetí pravidel pro zavedení, uplatňování a prosazování antidopingových kontrol. K~nim patří např. Mezinárodní olympijský výbor, další organizátoři velkých sportovních soutěží, které provádějí testrování při svých soutěžích, WADA a~národní antidopingové organizace.
\dend
\dt Antidopingová pravidla (Anti-doping Rules)
\dd Antidopingová pravidla IAAF schválená čas od času Kongresem IAAF nebo Radou IAAF.
\dend
\dt Antidopingové předpisy (Anti-doping Regulations)
\dd Antidopingové předpisy schválené čas od času Radou IAAF.
\dend
\dt Atlet
\dd Kterákoliv osoba, která je členem IAAF, její členské federace a oblastní asociace na základě dohody, členství, přidružení, autorizace, akreditace nebo účastí na jejích aktivitách nebo soutěžích nebo kterýkoliv jiný účastník atletických soutěží, který podléhá pravomoci kteréhokoliv signatáře nebo jiné sportovní organizace uznávající Kodex.
\dend
\dt Atypický nález (Atypical Finding)
\dd Zpráva laboratoře nebo jiného schváleného pracoviště, která vyžaduje další zkoumání podle mezinárodního standardu laboratoří nebo příslušných technických dokladů před vydáním rozhodnutí o~nepříznivém nálezu.
\dend
\dt Během soutěže (In-Competition)
\dd Pojmem \uv{během soutěže} se rozumí časový úsek začínající dvanáct (12) hodin před začátkem disciplíny, jíž se má atlet zúčastnit, do konce této disciplíny a proces odběru vzorku po této disciplíně.
\dend
\dt Bez předchozího oznámení (No Advance Notice)
\dd Dopingová kontrola bez předchozího varování atleta, kdy je atlet trvale pod dohledem od oznámení kontroly do poskytnutí vzorku.
\dend
\dt Cílené testování (Target Testing)
\dd Výběr pro testy, kdy je v~určité době zvolena určitá skupina atletů nebo určití atleti na základě nikoliv nahodilého výběru.
\dend
\dt Disciplína (Event)
\dd Jeden závod nebo soutěž (např. běh na 100 m nebo hod oštěpem).
\dend
\dt Diskvalifikace
\dd Viz bod Důsledky porušení antidopingových pravidel.
\dend
\dt Dopingová kontrola (Doping Control)
\dd Všechny kroky a procesy od stanovení systému kontrol po konečné rozhodnutí o~případných odvoláních včetně všech kroků a procesů mezitím, jako jsou informace o~pobytu, odběr a zacházení se vzorky, laboratorní analýza, výjimky pro terapeutické použití zakázaných látek, správa výsledků a slyšení.
\dend
\dt Držení (Possession)
\dd Skutečné, fyzické držení nebo nepřímé držení zakázané látky nebo zakázaného postupu (které lze zjistit pouze tehdy, pokud daná osoba měla výhradní vládu nad zakázanou látkou/postupem nebo areálem, v~němž se zakázaná látka nacházela nebo kde docházelo k~zakázanému postupu);  nicméně, pokud daná osoba neměla výhradní vládu nad zakázanou látkou/postupem nebo areálem, v~němž se zakázaná látka nacházela nebo kde docházelo k~zakázanému postupu, o~nepřímé držení se jedná pouze v~případě, že daná osoba o~zakázané látce/zakázaném postupu věděla a měla v~úmyslu mít ji pod kontrolou. Naproti tomu se nejedná o~porušení pravidel o~dopingu pouhým držením zakázaných prostředků, pokud daná osoba před jakýmkoliv oznámením o~porušení pravidel, podnikla kroky, prokazující, že nikdy neměla v~úmyslu mít takové prostředky v~držení a výslovně se jich zřekla prohlášením poskytnutým IAAF, členské federaci nebo anti-dopingové organizaci. Bez ohledu na cokoliv, co je v~rozporu s~touto definicí, nákup (včetně elektronickými nebo jinými prostředky) zakázané látky nebo zakázaného postupu znamená, spadá pod definici „držení“.
\dend
\dt Důsledky porušení antidopingových pravidel (Consequences of Anti-Doping Rule Violations)
\dd Porušení antidopingových pravidel atletem nebo jinou osobou může vyústit buď v~diskvalifikaci, což znamená zrušení výsledků dosažených atletem v~příslušné disciplíně nebo soutěži a následné odebrání jakéhokoliv titulu, odměny, medaile, bodů, ceny a startovného a/nebo ve ztrátu způsobilosti, což znamená, že atlet nebo jiná osoba má po určitou dobu zákaz účasti v~jakékoliv soutěži nebo činnosti nebo podpory ve smyslu ustanovení P 40.
\dend
\dt Informace o~pobytu (Whereabouts Filing)
\dd Písemná informace o~pobytu během následujícího čtvrtletí poskytnutá atletem, který je na seznamu atletů podléhajících testování, nebo poskytnutá jeho jménem.
\dend
\dt Kodex (Code)
\dd Světový antidopingový kodex.
\dend
\dt Metabolit (Metabolite)
\dd Jakákoliv látka vytvořená bio-transformačním procesem.
\dend
\dt Mezinárodní soutěž (International Competition)
\dd Pro účely těchto pravidel o~dopingu se tím rozumí mezinárodní soutěže podle P 35.7, jak jsou každoročně uvedené na webových stránkách IAAF.
\dend
\dt Mezinárodní standard  (International standard)
\dd Standard přijatý WADA na podporu Kodexu.
\dend
\dt Mimosoutěžní (Out-of-Competition)
\dd Libovolný časový úsek, který nespadá pod pojem \uv{během soutěže}.
\dend
\dt Národní antidopingová organizace (National Anti-Doping Organization)
\dd Orgán ustanovený každou zemí nebo teritoriem, který má prvotní pravomoc a odpovědnost za přijímání a uplatňování antidopingových pravidel, řízení odběru kontrolních vzorků, správu výsledků testů a vedení slyšení, vše na národní úrovni.
\dend
\dt Nelegální šíření (Trafficking)
\dd Prodej, přeprava, zasílání, dodávání nebo distribuce zakázaných látek nebo zakázaných postupů sportovcům, jejich doprovodu nebo jiným osobám, buď přímo nebo prostřednictvím jedné nebo více třetích stran, ale s~výjimkou prodeje nebo distribuce (lékařskými nebo jinými osobami) zakázaných látek nebo zakázaných postupů ke skutečným a legálním terapeutickým účelům.
\dend
\dt Neplnoletost (Minor)
\dd Fyzická osoba, která podle platných zákonů země svého sídla nedosáhla věku plnoletosti.
\dend
\dt Nepříznivý analytický nález (Adverse Analytical Finding)
\dd Zpráva laboratoře nebo jiného oprávněného zkušebního ústavu, podle níž byla ve vzorku zjištěna přítomnost zakázané látky nebo jejího metabolitu nebo příznak užití takové látky nebo byl nalezen důkaz o~užití zakázaného postupu.
\dend
\dt Nezpůsobilost (Ineligibility)
\dd Viz bod \uv{Důsledky porušení antidopingových pravidel}.
\dend
\dt Opomenutí podat informace o~pobytu (Filing Failure)
\dd Nezaslání písemné informace o~pobytu nebo zmeškaný test.
\dend
\dt Osoba (Person)
\dd Kterákoliv fyzická osoba (včetně atleta nebo člena podpůrného týmu atleta) nebo organizace nebo jiná entita.
\dend
\dt Ovlivňování (Tampering)
\dd Pozměňování věci k~nežádoucímu účelu nebo nežádoucím způsobem ve vztahu k~dopingové kontrole, nežádoucí ovlivňování průběhu dopingové kontroly nebo disciplinárního řízení, zásahy směřující ke změně výsledku kontroly nebo bránění řádnému průběhu celého procesu.
\dend
\dt Panel expertů BPS (ABP Expert Panel)
\dd Odborná skupina o~třech členech vybraných IAAF, kteří odpovídají za vyhodnocování Biologických pasů sportovce (Athlete Biological Passport) podle anti-dopingových předpisů. Budou vybrány osoby se znalostmi v~oblasti klinické hematologie, laboratorní praxe v~oblasti lékařství a hematologie a sportovního lékařství nebo fyziologie zaměřené na hematologii.
\dend
\dt Příznak (Marker)
\dd Sloučenina, skupina sloučenin nebo biologických parametrů, které nepřímo ukazují na použití zakázané látky nebo zakázaného postupu.
\dend
\dt Zabezpečovací tým atleta (Athlete Support Personnel)
\dd Kouč, trenér, vedoucí, oprávněný zástupce atleta, agent, činovník družstva, lékař nebo zdravotník, nebo jiná osoba spolupracující, ošetřující nebo napomáhající atletu, který se účastní nebo připravuje na atletické soutěže.
\dend
\dt Pokus (Attempt)
\dd Úmyslné jednání, které je podstatným krokem směrem k~porušení antidopingových pravidel. Pokud  ale osoba, která se pokusila tato pravidla porušit, od svého pokusu upustí dříve, než to zjistila nezúčastněná osoba, není takové jednání považováno za pokus o~porušení antidopingových pravidel.
\dend
\dt Pořadatel velkých sportovních soutěží (Major Event Organization)
\dd Kontinentální asociace Národních olympijských výborů a dalších multi-sportovních organizací, které řídí jakékoliv kontinentální, regionální nebo jiné mezinárodní soutěže.
\dend
\dt Prozatímní zastavení činnosti (Provisional Suspension)
\dd Atlet nebo jiná osoba má dočasně zastavenou činnost a nesmí se zúčastnit žádné soutěže do konečného rozhodnutí během slyšení uskutečněného v~souladu s~těmito pravidly.
\dend
\dt Seznam atletů podléhajících testování (Registered Testing Pool)
\dd Seznam sestavený IAAF, uvádějící přední atlety, podléhající soutěžním a mimosoutěžním kontrolám v~rámci testovacího programu IAAF
\dend
\dt Seznam zakázaných prostředků (Prohibited List)
\dd Seznam zakázaných látek a zakázaných postupů vydaný WADA .
\dend
\dt Signatáři (Signatories)
\dd Subjekty, které podepsaly Kodex a souhlasí s~jeho uplatňováním, včetně mezinárodního olympijského výboru, mezinárodních federací, národních olympijských výborů, pořadatelů velkých sportovních soutěží, národních protidopingových organizací a WADA.
\dend
\dt Soutěž (Competition)
\dd Disciplína nebo sled disciplín probíhajících jeden nebo více dní.
\dend
\dt Systém správy a řízení antidopingových záležitostí (ADAMS -- Anti-Doping Administration and Management System)
\dd Systém správy a řízení antidopingových záležitostí je internetová databáze pro ukládání, uchovávání sdílení a sdělování údajů napomáhající účastníkům systému a WADA v~antidopingových činnostech v~návaznosti na legislativu ochrany dat.
\dend
\dt Testování (Testing)
\dd Části dopingové kontroly zahrnující plánování rozsahu testů, odběr vzorků, nakládání se vzory a přepravu vzorků do laboratoře.
\dend
\dt TUE (Therapeutic Use Exemption)
\dd Výjimka pro terapeutické použití zakázané látky nebo postupu.
\dend
\dt Účastník (Participant)
\dd Kterýkoliv atlet nebo člen podpůrného týmu atleta.
\dend
\dt Užití (Use)
\dd Aplikace, přijmutí, vstříknutí nebo konzumace jakýmikoliv prostředky jakékoliv zakázané látky nebo zakázaného postupu.
\dend
\dt Podstatná spolupráce (Substantial Assistance)
\dd Pro účely ustanovení P 40.5.c), osoba poskytující významnou spolupráci musí jednak v~písemném prohlášení sdělit všechny jemu známé informace o~poručení dopingových pravidel, jednak plně spolupracovat při vyšetřování a rozhodování v~jakémkoliv případě týkajícím se takové informace, vč. např. svědectví při slyšení, pokud to bude vyžádáno členy komise provádějící slyšení. Poskytnutá informace musí být důvěryhodná a musí obsahovat důležitou část projednávaného případu nebo přinést dostatek informací na jejich základě je možné případ otevřít.
\dend
\dt Vzorek (Sample/Specimen)
\dd Jakýkoliv biologický materiál odebraný pro účely testu.
\dend
\dt WADA (World Anti-Doping Agency)
\dd Světová antidopingová agentura.
\dend
\dt Zakázaná látka (Prohibited Substance)
\dd Kterákoliv látka popsaná v~seznamu zakázaných prostředků.
\dend
\dt Zakázaný postup (Prohibited Method)
\dd Kterýkoliv postup popsaný v~seznamu zakázaných prostředků.
\dend
\dt Zanedbané oznámení (Filing Failure)
\dd Zanedbání povinnosti atleta přesně a úplně oznamovat svůj pobyt buď ve smyslu Antidopingových předpisů či pravidel nebo předpisů členské federace či antidopingové organizace, pod jejíž pravomoc atlet spadá a~které odpovídají mezinárodnímu standardu pro testování.
\dend
\dt Zanedbání (Whereabouts Failure)
\dd Nepodané oznámení o~pobytu nebo neúčast na testování.
\dend
\dt Zmeškaný test (Missed Test)
\dd Chyba atleta, kdy ve lhůtě 60 minut není k~dispozici pro provedení testu na místě a v~době, kde a~kdy se má podle jím podaného oznámení nacházet, jak je uvedeno v~Antidopingových předpisech nebo pravidlech či předpisech členské federace nebo antidopingové organizace, pod jejíž pravomoc atlet spadá a které odpovídají mezinárodnímu standardu pro testování.
\dend
\dt Ztráta způsobilosti (Ineligibility)
\dd Viz bod Důsledky porušení antidopingových pravidel.
\dend
\dt Žádná chyba nebo nedbalost (No Fault or No Negligence)
\dd V~případě projednávaném podle P 38 bylo zjištěno, že sportovec i~při vynaložení maximálního úsilí nemohl vědět nebo tušit, že použil nebo mu byly podány zakázané látky nebo se podrobil zakázanému postupu.
\dend
\dt Žádná závažná chyba nebo závažná nedbalost (No Significant Fault or No Significant Negligence)
\dd V~případě projednávaném podle P 38 bylo zjištěno, že s~ohledem na celkové okolnosti a vzhledem ke kritériím pro přiznání statutu \uv{Žádná chyba nebo nedbalost}, chyba nebo zanedbání ze strany atleta nebylo podstatné ve vztahu k~porušení anti-dopingových pravidel.
\dend
\dt Žádné oznámení předem (no Advance Notice)
\dd Dopingová kontrola, která se koná bez předchozího oznámení sportovci, kdy sportovec je trvale pod dozorem od okamžiku uvědomění o~kontrole do odběru vzorku.
\dend

\endinput