\sec HALOVÉ SOUTĚŽE

\secc Platnost pravidel soutěží na otevřeném hřišti

S výjimkami uvedenými v ustanoveních tohoto oddílu a požadavku měření rychlosti větru podle P 163 pro soutěže v hale platí pravidla pro soutěže na otevřeném závodišti, uvedená v kapitolách I až V.

\secc Kryté sportoviště

\begitems \style N
* Celé sportoviště musí být zcela uzavřené a pod střechou. Musí být vybaveno osvětlením, topením a ventilací, zajišťujícími nutné podmínky pro soutěže.
* Halové sportoviště má mít oválnou atletkou dráhu a rovinku pro běhy na krátké vzdálenosti na hladkých a překážkových tratích, rozběhové dráhy a doskočiště pro skok do výšky, skok do dálky, trojskok a pro skok o tyči. Dále má mít vrhačský kruh a sektor pro dopad náčiní pro vrh koulí, umístěný trvale nebo podle potřeby. Všechna zařízení mají odpovídat specifikacím uvedeným v manuálu IAAF Atletická zařízení ("IAAF Track and Field Facilities Manual").
* Všechny běžecké dráhy, rozběhové dráhy a odrazové plochy musí být pokryty umělým materiálem, jehož vrstva má být dostatečná pro použití obuvi s hřeby o délce 6 mm.

Pokud je tato vrstva odlišné tloušťky, uvědomí správa haly atlety o přípustné délce hřebů (viz P 143.4).

Halové atletické soutěže uvedené v P 1.a), b), c) a f) mají být konány pouze na závodištích, která mají platný certifikát IAAF.

Doporučuje se, aby se na takových závodištích konaly i soutěže uvedené v P 1.d), e), f), g) h) i) a j).

* Podklad, na němž je položen umělý povrch drah, rozběžišť a odrazových ploch, musí být buď tuhý, tj. z betonu nebo, jedná-li se o montovanou konstrukci (např. dřevěné či překližkové desky upevněné na trámových nosnících), musí mít, jak je to jen technicky možné, po celé ploše jednotnou pružnost. Na odrazových plochách pro skokanské soutěže musí být tato podmínka prověřena před každou soutěží.

Podkladová plocha, na níž je povrch odraziště položen, musí být bezpodmínečně pevná, nebo bez jakýchkoliv pružících dílů.

POZN. 1: „Pružným dílem“ se rozumí takový díl, který je záměrně navržen tak, že atletům poskytuje výhodu.

POZN. 2: Manuál IAAF pro Atletická zařízení ("IAAF Track and Field Facilities Manual"), kterou je možno obdržet v kanceláři IAAF nebo stáhnout z webových stránek IAAF, obsahuje podrobnější přesné specifikace pro návrh a stavbu atletických sportovišť, včetně náčrtků pro měření a značení dráhy.

POZN. 3: Současné standardní formuláře, které jsou požadované při podání certifikační žádosti a pro zprávu o měření, stejně jako Postupy pro certifikační systém jsou dostupné v kanceláři IAAF, nebo mohou být staženy z webových stránek IAAF.
\enditems

\secc Rovinka pro běhy na krátké vzdálenosti

* Příčný sklon rozběhové dráhy nemá překročit hodnotu 1:100 (1 \%), pokud zvláštní okolnosti neospravedlňují výjimku udělenou IAAF a sklon ve směru běhu nesmí v kterémkoliv místě mít hodnotu větší než 1:250 (0,4 %) a celkově hodnotu větší než 1:1000 (0,1 %).

Dráhy
* Rovinka má mít nejméně 6 a nejvýše 8 drah vzájemně vymezených a z obou stran ohraničených bílými čarami širokými 50 mm. Jednotlivé dráhy musí mít stejnou šířku 1,22 m ± 0,01 m, včetně čáry po pravé straně.

POZN.: Dráhy okruhů postavených před 1. 1. 2004 mohou mít jednotlivé dráhy maximální šířku 1,25 m.  Nicméně, při opravě povrchu nebo celkové změně, musí šířka drah odpovídat tomuto pravidlu.

Start a cíl
* Před startovní čárou musí být volný prostor alespoň 3 m a za cílovou čarou musí být zcela volný prostor bez jakýchkoliv překážek o délce alespoň 10 m, upravený tak, aby dobíhající závodníci mohli zastavit bez nebezpečí zranění.

POZN.: Doporučuje se, aby za cílovou čarou byla volná vzdálenost alespoň 15 metrů.
\enditems

\secc Běžecký ovál a dráhy

\begitems \style N
* Doporučuje se jmenovitá délka 200 m. Běžecký ovál musí sestávat ze dvou paralelně vedených, rovných úseků a dvou zatáček stejného poloměru, které mohou být klopené. Vnitřní okraje trati musí být ohraničené buď obrubníkem z vhodného materiálu asi 50 mm vysokým i širokým nebo bílou čarou širokou 50 mm. Vnitřní hrana obrubníku nebo čáry musí být po celém obvodu dráhy vodorovná, s nejvyšším sklonem 1:1000 (0,1 %). Obrubník na obou rovinkách může být vynechán a nahrazen bílou čárou o šířce 50 mm. POZN.: Všechna měření musí být provedena v souladu s P160.2.

Dráhy
* Běžecká trať má mít nejméně 4 a nejvýše 6 drah. Jmenovitá šířka drah musí být v rozmezí 0,90 m až 1,10 m, vč. čáry po pravé straně. Všechny dráhy musí být stejně široké s tolerancí ± 0,01 m vzhledem ke zvolené šířce. Jednotlivé dráhy musí být vzájemně vymezeny bílými čarami širokými 50 mm.

Klopení zatáček
* Úhel klopení dráhy musí být stejné v kterémkoliv radiálním řezu všemi drahami v zatáčkách a v kterémkoliv kolmo na dráhy vedeném řezu všemi drahami na rovinkách. Rovinky mohou být ploché nebo s příčným sklonem nejvýše 1:100 (1 %) směrem k vnitřní dráze. Pro usnadnění přechodu z rovinek do klopené zatáčky, může přechodový úsek plynule ve vertikálním i horizontálním směru zasahovat do rovného úseku trati.

Vyznačení vnitřního okraje
* Je-li vnitřní okraj dráhy vymezen pouze bílou čarou, musí být ještě vyznačen kužel, nebo praporky. Kužele musí být alespoň 200 mm vysoké. Praporky musí mít rozměr 0,25 m x 0,20 m a výšku nad plochou dráhy alespoň 0,45 m a musí být skloněny pod úhlem 120° k povrchu běžeckého oválu. Kužele nebo praporky musí být umístěny tak, že jejich ven směřující plocha splývá s vnitřním okrajem bílé hraniční čáry. Vzájemné vzdálenosti kuželů nebo praporků nesmí být větší než 1,5 m v zatáčkách a 10 m na rovinkách.

POZN.: Pro všechny halové soutěže řízené přímo IAAF se doporučuje vnitřní vymezení běžeckého oválu obrubníkem.
\enditems

\secc Start a cíl na běžeckém oválu

\begitems \style N
* Technické informace o stavbě a značení běžeckého oválu o délce 200 m a opatřeného klopenými zatáčkami jsou uvedeny v příslušné příručce IAAF Atletická zařízení ("IAAF Track and Field Facilities Manual"). Základní principy, které by měly být dodrženy, jsou uvedeny níže.

Základní požadavky
* Start a cíl musí být vyznačeny bílými čarami širokými 50 mm, vedenými kolmo na podélné dělící čáry drah na rovince a ve směru normály v daném místě zakřivení zatáčky.
* Cílová čára, pokud je to možné, má být pouze jedna pro všechny běhy a musí být umístěna na rovném úseku trati tak, aby co největší část tohoto úseku byla před cílem.
* Základním požadavkem pro všechny startovní čáry, přímé, odstupňované či zakřivené je, aby nejkratší možná vzdálenost od startu do cíle byla pro všechny atlety stejná.
* Pokud je to možné, nemají být startovní čáry (a středové čáry předávacích území) v nejstrmější části klopení zatáček.

Průběh závodů
* \begitems \style a
  * Běhy do 300 m včetně musí být běženy celé v drahách.
  * Běhy delší než 300 m, avšak kratší než 800 m, musí být běženy v drahách od startu až po čáru seběhnutí vyznačenou na konci druhé zatáčky.
  * V bězích na 800 m může mít každý atlet přidělenu samostatnou dráhu nebo jedna dráha může být přidělena až dvěma atletům nebo se může se běžet i ve skupinách podle P 163.5.b), nejlépe vybíhajících z drah 1 a 3. V těchto případech závodníci mohou opustit své dráhy nebo ti, kteří běží ve vnější skupině, se mohou připojit k vnitřní skupině za čárou seběhnutí, vyznačenou na konci první zatáčky, nebo pokud je závod běžen se dvěma zatáčkami v drahách, na konci druhé zatáčky. Může být též použita jediná zakřivená čára.
  * Běhy nad 800 m musí být běženy ze zakřivené startovní čáry bez ohledu na vyznačené dráhy. Pokud je použit skupinový start, je čára seběhnutí na konci první nebo druhé zatáčky.
  \enditems

Pokud atlet toto pravidlo nedodrží, bude diskvalifikován.

Čára seběhnutí (k obrubníku) je oblouková čára vyznačená na výběhu z každé zatáčky, vyznačená v šíři 50 mm napříč všemi drahami vyjma první. Pro usnadnění identifikace této obloukové čáry musí být na čarách, bezprostředně před průsečíky obloukové čáry s čarami vymezujícími jednotlivé dráhy, umístěny malé kužely či hranoly (50 x 50 mm) nebo jiné značky, jejichž výška nepřesáhne 0,15 m, a jejichž barva je s výhodou odlišná od barvy obloukové čáry.

POZN. 1: Při soutěžích neuvedených v P 1.1.a), b), c) a f) se účastníci mohou dohodnout, že běh na 800 m se nepoběží v drahách.

POZN. 2: Při méně než šesti drahách lze použít skupinového startu, který umožňuje soutěžit šesti atletům současně.

Startovní a cílová čára pro dráhu o jmenovité délce 200 m
* Startovní čára první dráhy je na hlavní rovince. Její poloha musí být stanovena tak, aby startovní čára v krajní vnější dráze (běhy na 400 m) byla v místě, kde úhel klopení nepřesahuje 12°.

Cílovou čarou pro všechny běhy na běžeckém oválu je prodloužená startovní čára první dráhy, vedená napříč jednotlivými drahami, kolmo na jejich hraniční čáry.
\enditems

\secc Losování drah v soutěžích na dráze

\begitems \style N
* Pro všechny běhy, které se běží zcela či částečně v drahách v zatáčce, a kde jsou kvalifikační kola, se o dráhy losuje takto:
  \begitems \style a
  * o vnější dvě dráhy losují dva závodníci nebo družstva nejvýše postavená,
  * o další dvě dráhy závodníci nebo družstva na třetím a čtvrtém místě,
  * o zbývající dráhy losují zbývající účastníci běhu.

  Postavení jednotlivých účastníků běhu se stanoví takto:
  * pro běhy prvního kola podle tabulky dosažených výkonů v předem stanoveném údobí,
  * pro další kola v souladu s postupem uvedeným v P 166.3.b), i), v případě běhu na 800 m v souladu s P166.3.b.ii)
  \enditems
* Pro všechny ostatní běhy se pořadí drah stanoví v souladu s P 166.4 a P 166.5.
\enditems

\secc Oblečení, obuv a startovní čísla

Část hřebů vystupujících z podrážky nebo paty obuvi nesmí přesahovat délku 6 mm (nebo hodnotu udanou pořadatelem, která nesmí překročit maximum dané P 143.4.

\secc Běhy překážkové

\begitems \style N
* Překážkové běhy v hale se pořádají na vzdálenosti 50 m nebo 60 m, na rovné trati.
* Rozmístění překážek pro závody:

\table{|6{l|}}{\crl
                             & U18        & U20     & muži    & U18        & U20 \cr
                             & dorostenci & junioři &         & dorostenky & Juniorky \cr
                             &            &         &         &            & Ženy \crl
výška překážky               & 0,914 m    & 0,991 m & 1,067 m & 0,762 m    & 0,838 m \crl
délka tratě                  & \mspan5[c|]{50 m / 60 m} \crl
Počet překážek               & \mspan5[c|]{4 / 5} \crl
od startu k 1. překážce      & \mspan3[c|]{13,72 m}               & \mspan2[c|]{13,00 m} \crl
mezi překážkami              & \mspan3[c|]{9,14 m}                & \mspan2[c|]{8,50 m} \crl
od poslední překážky do cíle & \mspan3[c|]{8,86 m / 9,72 m}       & \mspan2[c|]{11,50 m / 13,00 m} \crl
}

\itemnum=30
* Pro žákovské soutěže platí:

\table{|5{l|}}{\crl
                             & Žáci st.    & Žáci ml.    & žákyně st.  & žákyně ml. \crl
výška překážky               & 0.840 m     & 0.762 m     & 0.762 m     & 0.762 m \crl
délka tratě                  & \mspan4[c|]{50 m / 60 m} \crl
Počet překážek               & 4 / 5       & 5 / 6       & 4 / 5       & 5 / 6 \crl
od startu k 1. překážce      & 13,00 m     & 11,70 m     & 13,00 m     & 11,70 m \crl
mezi překážkami              & 8,50 m      & 7,70 m      & 8,20 m      & 7,70 m \crl
od poslední překážky do cíle & 11,5 / 13 m & 7,5 / 9,8 m & 12,4 / 14 m & 7,5 / 9,8 m \crl
}
\enditems

\secc Běhy štafetové

\begitems \style N
* Při běhu na 4x200 m musí být celý první úsek a první zatáčka druhého úseku, až po nejbližší hranu čáry seběhnutí, popsanou v P214.9, běženy v drahách. Všechna předávací území musí být dlouhá 20 m a druhý, třetí a čtvrtý atlet musí zahájit běh uvnitř předávacího území.
* Běh na 4x400 m musí být běžen v souladu s P 214.6.b).
* Běh na 4x800 m musí být běžen v souladu s P 214.6.c).
* Běžci na třetím a čtvrtém úseku závodu na 4x200 m a na druhém, třetím a čtvrtém úseku závodu na 4x400 a na všech úsecích závodu na 4x800 m se za řízení určeným rozhodčím řadí do vyčkávací pozice na počátku předávacího území ve stejném pořadí (od vnitřní dráhy po vnější), v jakém přibíhající členové jednotlivých družstev vběhli do poslední zatáčky. Své postavení na počátku předávacího území musí očekávající běžci zachovat a nesmí je již měnit, i když se pořadí přibíhajících běžců mezitím změní. Pokud některý atlet toto ustanovení nedodrží, jeho družstvo bude diskvalifikováno.

POZN.: Vzhledem k úzkým drahám může při bězích štafetových v halách dojít k neúmyslným kolizím snadněji než při závodech na otevřeném závodišti. Proto se doporučuje, pokud je to možné, ponechat mezi jednotlivými družstvy volnou dráhu. Použity by tedy byly dráhy 1, 3 a 5 a dráhy 2,4 a 6 by zůstaly volné.
\enditems

\secc Skok do výšky

\begitems \style N
Rozběhová plocha a odrazová plocha
* Při použití přenosných rohoží platí ustanovení o odrazové ploše pro horní povrch těchto rohoží.
* Atlet může zahájit svůj rozběh z klopené části běžeckého oválu, avšak posledních 15 m musí být na úseku vyhovujícím ustanovení P182.3, 182.4 a 182.5.
\enditems

\secc Skok o tyči

\begitems \style N
Rozběhová dráha
* Atlet se může začít rozbíhat z klopené části běžeckého oválu, ale posledních 40 m rozběhu m je na rozběžišti odpovídajícím ustanovení P183.6 a P183.7.
\enditems

\secc Horizontální skoky

\begitems \style N
Rozběhová dráha
* Atlet se může začít rozbíhat z klopené části běžeckého oválu, pokud posledních 40 m rozběhu je na rozběžišti odpovídajícím ustanovení P184.1 a P184.2.
\enditems

\secc Vrh koulí

\begitems \style N
Výseč pro dopad náčiní
* Povrch výseče pro dopad náčiní musí být tvořen takovým materiálem, že na něm koule po dopadu zanechá otisk, a který současně utlumí odskoky náčiní.
* Pokud je to nutné pro bezpečnost diváků, atletů a činovníků, výseč pro dopad náčiní musí být na vzdáleném konci a po obou bocích, co nejblíže ke kruhu, ohraničena bariérou nebo ochrannou sítí vysokou nejméně 4,0 m, a dostatečně pevnou, aby zadržela kouli při přímém dopadu nebo po odskoku od povrchu výseče.
* S ohledem na omezený prostor krytého závodiště, plocha ohraničená bariérou (viz bod 3), nemusí být dostatečně velká pro celou výseč 34,92°. Zmenšená plocha výseče musí vyhovovat následujícím podmínkám.
  \begitems \style a
  * Bariéra na konci plochy výseče musí být alespoň o 0,5 m dále, než je platný světový rekord soutěže mužů nebo žen při mezinárodních soutěžích, jinak podle úrovně soutěže.
  * Postranní čáry vymezující výseč musí být vyznačeny symetricky vůči ose kruhu.
  * Postranní čáry výseče, která není standardní výsečí s úhlem 34,92°, mohou vybíhat paprskovitě ze středu kruhu (obdobně jako je tomu u standardní výseče) nebo mohou být rovnoběžné navzájem a s osou kruhu. Vzdálenost obou vzájemně rovnoběžných postranních čar musí být nejméně 9,0 m.
  \enditems

Konstrukce náčiní
* Podle typu povrchu výseče pro dopad náčiní (viz P222.1) musí být koule zhotovena z plného kovu či s kovovým pláštěm nebo alternativně s pláštěm z měkké plastické hmoty či gumy, naplněným vhodným materiálem. V jedné soutěži není možno použít oba typy náčiní.

Koule z plného kovo nebo s kovovým pláštěm
* Koule z plného kovu nebo s kovovým pláštěm musí vyhovovat ustanovením P 188.4 a P 188.5 pro soutěže na otevřeném závodišti.

Koule s plastovým nebo gumovým pláštěm
* Koule s pláštěm z měkkého plastu nebo z gumy musí být naplněna vhodným materiálem a pevnost pláště musí být taková, aby bez poškození vydržel dopad náčiní na podlahu běžné sportovní haly. Musí být kulového tvaru s hladkým povrchem.

Informace pro výrobce: Hladkostí se rozumí, že jeho nerovnosti jsou menší než 1,6 μm, tj. hrubost povrchu označená číslem N7 a menší.

* Náčiní musí vyhovovat následujícím parametrům:

\table{|l|l|r|r|}{\crl
            & Minimální hmotnost & \mspan2[r|]{ \  } \cr
            & pro připuštění     & \mspan2[r|]{průměr (mm)} \crlp{3-4}
Kategorie   & k soutěži (kg)     & min. & max. \crl
Muži        & 7,260              & 110  & 145 \crl
Junioři     & 6,000              & 105  & 140 \crl
Dorostenci  & 5,000              & 100  & 135 \crl
Žáci starší & 4,000              &  95  & 130 \crl
Žáci mladší & 3,000              &  85  & 120 \crl
Ženy        & 4,000              &  95  & 130 \crl
Juniorky    & 4,000              &  95  & 130 \crl
Dorostenky  & 3,000              &  85  & 120 \crl
Žákyně st.  & 3,000              &  85  & 120 \crl
Žákyně ml.  & 2,000              &  75  & 115 \crl
}

\itemnum=30
* Pro ostatní kategorie platí stejné parametry jako pro náčiní určené pro soutěže na otevřeném závodišti.
\enditems

\secc Víceboje

\begitems \style N
U18 Dorostenci, U20 junioři, muži (pětiboj)
* Pětiboj sestává z pěti disciplín, které musí proběhnout během jediného dne v následujícím pořadí:

60 m přek., skok do dálky, vrh koulí, skok do výšky, 1000 m.

U18 Dorostenci, U20 junioři, muži (sedmiboj)
* Sedmiboj sestává ze sedmi disciplín, které musí proběhnout během dvou po sobě následujících dnů v pořadí:

první den: 60 m, skok do dálky, vrh koulí, skok do výšky,

druhý den: 60 m přek., skok o tyči, 1000 m.

U18 Dorostenky, U20 juniorky, ženy (pětiboj)
* Pětiboj sestává z pěti disciplín, které musí proběhnout během jediného dne v následujícím pořadí:

60 m přek., skok do výšky, vrh koulí, skok do dálky, 800 m.

Rozběhy a rozdělení do skupin
* V každé skupině mají být alespoň 3, nejlépe však 4 a více atletů, dle potřeby rozdělených do rozběhů, nebo skupin.
\enditems

\endinput