\sec BĚHY PŘESPOLNÍ, BĚHY DO VRCHU A TRAILOVÉ BĚHY

\secc Běhy přespolní

\begitems \style N
Vzdálenosti
* Běhy při Mistrovství světa v přespolním běhu se mají konat na přibližně následujících vzdálenostech:

\table{lr}{
Muži     & 10 km \cr
Muži U20 & 8 km \cr
Ženy     & 10 km \cr
Ženy U20 & 6 km \cr
}

Pro soutěže chlapců a dívek (dorostu) jsou doporučené vzdálenosti

\table{lr}{
Chlapci U18 & 6 km \cr
dívky U18   & 4 km \cr
}

Doporučuje se použít uvedené vzdálenosti i pro další mezinárodní a národní soutěže.

Trať
* \begitems \style a
  * Trať musí být vytýčena v otevřené nebo lesnaté krajině, v terénu pokrytém, pokud možno, trávou, s přírodními překážkami, které může stavitel trati využít pro ztížení či zpestření tratě.
  * Terén musí být natolik prostorný, aby v něm bylo nejen možno vytýčit vlastní závodní trať, ale též zřídit veškerá potřebná zařízení.
  \enditems
* Pro mistrovské a mezinárodní soutěže a, pokud možno, pro všechny soutěže:
  \begitems \style a
  * Trať závodu musí probíhat po okruhu o délce mezi 1500 a 2000 m. Je-li to nutné, je možno vytvořit menší okruh pro dodržení vypsané délky v různých bězích, přičemž tento menší okruh musí závodníci absolvovat v počátečních fázích závodu. Doporučuje, aby každý velký okruh měl celkové převýšení alespoň deset metrů.
  * Pokud je to možné, je třeba využít existujících přírodních překážek. Nicméně je třeba se vyvarovat příliš vysokým překážkám, stejně tak i hlubokým příkopům, nebezpečným stoupáním či klesáním, hustému křoví a obecně všem překážkám, které by ztěžovaly závod více, než je účelné. Doporučuje se nestavět umělé překážky, jsou-li však nezbytné pro dosažení potřebné obtížnosti závodu, musí se podobat přírodním překážkám, s nimiž se lze v otevřené krajině potkat. V závodě s velkým počtem atletů nesmějí být na prvních 1500 m trati úzká místa nebo překážky, které by atlety brzdily v běhu.
  * Křižování cest nebo úseky s makadamovým povrchem je třeba omezit na minimum. Není-li možno se tomu vyhnout, je třeba, aby taková místa byla pokryta travou, zeminou nebo rohožemi.
  * Vyjma prostorů startu a cíle trať nesmí mít žádné jiné dlouhé rovné úseky. Nejvhodnější je ”přírodní”, zvlněná trať s plynulými zatáčkami a krátkými rovnými úseky.
  \enditems
* \begitems \style a
  * Trať musí být jasně vyznačena páskou po obou stranách. Doporučuje se, aby po jedné straně byl koridor široký 1,0 m, bezpečně ohraničený na vnější straně, který by mohli využívat činovníci závodu a pracovníci médií (toto opatření je povinné pro mistrovské soutěže). Významné úseky musí být bezpečně ohraničené, jedná se zejména o prostor startu, vč. oblasti pro rozcvičení a svolavatelny) a prostoru cíle (vč. mix-zóny). To těchto prostor mohou mít přístup pouze oprávněné osoby.
  * Veřejnost smí mít povoleno křižovat trať pouze v počátečním stadiu závodu na vymezených místech, střežených pořadateli.
  * Doporučuje se, aby kromě prostorů startu a cíle byla trať široká 5,0 m, vč. míst s překážkami.
  * Pokud se příslušný vrchní rozhodčí dozví od rozhodčího, úsekového rozhodčího nebo jinak, že atlet opustil vyznačenou trať a zkrátil si tak předepsanou vzdálenost, musí takového běžce diskvalifikovat.
  \enditems
* Pro běhy rozestavné při přespolních bězích musí být předávací území vyznačena čarami 50 mm širokými, vedenými napříč tratí. Všechny úkony při předávkách, které, pokud není pořadateli určeno jinak, musí zahrnovat fyzický kontakt mezi předávajícím a přebírajícím atletem, musí proběhnout uvnitř předávacího území.

Start
* Závod musí být odstartován výstřelem. Použijí se povely pro běhy delší než 400 m (viz P 162.2.b).

V závodech, kde startuje velký počet atletů, mají být pět, tři a jednu minutu před vlastním startem dána výstražná znamení.

Pro soutěže družstev mohou být připraveny startovní boxy a členové družstva se ve svém boxu řadí do zástupu. V ostatních soutěžích musí být závodníci řazeni způsobem určeným pořadatelem. Po povelu "Připravte se!" se startér přesvědčí, že žádný atlet se nohou (nebo kteroukoliv částí těla) nedotýká startovní čáry nebo země za ní (míněno ve směru běhu) a pak závod odstartuje.

Bezpečnost závodu
* Pořadatelé závodu v přespolním běhu musí bezpodmínečně zajistit bezpečnost atletů.

Osvěžovací a občerstvovací stanice
* Voda a jiné vhodné občerstvení musí být k dispozici na startu a v cíli všech závodů. Při všech závodech musí být v každém kole k dispozici osvěžovací stanice, pokud povětrnostní podmínky takové opatření opodstatňují.

POZN.: Pokud to podmínky opodstatňují, na základě charakteru závodu, povětrnostní podmínky a stav a fyzická připravenost většiny atletů, může být voda a houby umístěny podél trati ve více pravidelných intervalech.

Průběh soutěží
* Pokud se příslušný vrchní rozhodčí dozví od rozhodčího, úsekového rozhodčího nebo jinak, že atlet opustil vyznačenou trať a zkrátil si tak předepsanou vzdálenost, musí takového běžce diskvalifikovat

Družstva a náhradníci
\itemnum=30
* Ustanovení týkající se počtu členů družstva, náhradníků a počtu atletů, jejichž výsledek se počítá pro konečné umístění družstva, se může u jednotlivých soutěží lišit. Jako základní počty lze použít následující počty:

V závodě mužů může být do družstva přihlášeno nejvýše 12 atletů, z nichž nejméně 6, ale nejvýše 9 smí v závodě startovat, a z nich pak 6 bude bodovat.

Pro závody žen a juniorské soutěže mohou být družstva nejméně o 4 a nejvýše o 8 členech, z nichž nejvýše 6 může být připuštěno na start a z nich pak 4 bodují.

* Je možno připustit i start jednotlivců. Rovněž tak členům neúplných družstev, která v den závodů nemají dostatek atletů pro bodování, by mělo být povoleno startovat jako jednotlivci.

Bodování
* \begitems \style a
  * Po ukončení závodu zjistí rozhodčí umístění bodujících členů každého družstva, sečtou jejich umístění a družstvo s nejnižším součtem prohlásí vítězem.
  * Při zjišťování pořadí musí být vyřazena konečná pořadí atletů startujících pouze jako jednotlivci a podle toho upravena pořadí členů soutěžících družstev.
  * V případě shodného součtu se rozhodne ve prospěch družstva, jehož poslední bodující člen skončil na místě bližším vítězi závodu.
  \enditems
\enditems

\secc Běhy do vrchu

\begitems \style N
Typy běhů do vrchu
* \begitems \style a
  * Většina běhů do vrchu se koná s hromadným startem, kdy všichni závodníci odstartují společně nebo podle pohlaví či věkových kategorií.
  * Při bězích štafetových v rámci běhů do vrchu, se skladba, délky úseků a typ tratě mohou široce měnit v závislosti na přírodních podmínkách a plánech pořadatele.
  * Běhy do vrchu s intervalovým startem jsou považovány za závody na čas. Výsledky jsou stanoveny podle dosažených časů.
  \enditems

Závodní trať
* \begitems \style a
  * Běhy do vrchu se konají v terénu vedeném převážně mimo cesty. V případě podstatných převýšení na trati je přijatelný zpevněný povrch.
  * Každý běh do kopce je specifický vzhledem ke svým přírodním podmínkám určujícím charakter běhu. Přednostně je třeba používat existující cesty a stezky. Pořadatel odpovídá za péči o životní prostředí.
  * Délka běhu může být v rozmezí od 1 km po maraton, respektující požadované technické detaily.
  * Trať může být vedena do kopce, nahoru a dolů nebo smíšeně.
  * Průměrný sklon má být alespoň 5 \% (čili 50 m na kilometr) a nemá překročit 20 \% (čili 200 m na kilometr). Nejvhodnější průměrné převýšení je 100 m na 1 km, přičemž trať musí dovolovat běh po celé délce.
  * Celá trať musí být zřetelně vyznačena. Závodníci musí mít k dispozici podrobnou mapu tratě, vč. jejího profilu.
  \enditems

Start
* Použijí se povely pro běhy delší než 400 m (viz P 162.2.b). V závodech, kde startuje velký počet atletů, mají být pět, tři a jednu minutu před vlastním startem dána výstražná znamení.

Bezpečnost závodu a lékařské zabezpečení
* Pořadatelé závodu v běhu do vrchu musí bezpodmínečně zajistit bezpečnost atletů. Je třeba respektovat specifické faktory, jako je nadmořská výška ve vztahu k povětrnostním podmínkám a dostupná infrastruktura.

Osvěžovací a občerstvovací stanice
* Voda a jiné vhodné občerstvení musí být k dispozici na startu a v cíli všech závodů. Další osvěžovací a občerstvovací stanice mají být zřízeny na vhodných místech podél tratě.

Průběh soutěží
* Pokud se příslušný vrchní rozhodčí dozví od rozhodčího, úsekového rozhodčího nebo jinak, že atlet opustil vyznačenou trať a zkrátil si tak předepsanou vzdálenost, musí takového běžce diskvalifikovat.
\enditems

\secc Trailové běhy (Trail races) – běhy rozličným terénem

\begitems \style N
Trať
* \begitems \style a
  * Běhy krajinou se konají v členitém terénu (vč. prašných cest, lesních stezek a jednotlivě vedených pěšin) ve volné přírodě (jako jsou hory, pouště, lesy nebo planiny), převážně mimo silnice.
  * Zpevněné povrchy (makadam) nebo pevné povrchy (asfalt, beton) jsou přijatelné, ale nesmí pokrývat víc než 20 % celkové délky závodu.  Délka závodu ani jeho převýšení, vč. klesání, nejsou omezeny.
  * Pořadatel musí před závodem oznámit změřenou délku a celkové stoupání i klesání závodu a musí atletům poskytnout mapu a detailní profil tratě spolu s popisem obtížnosti překážek, které závodníci budou muset překonat.
  * Trať musí být vyznačena tak, že závodníci budou mít dostatek informací pro absolvování závodu bez odchylek.
  \enditems

Vybavení
* \begitems \style a
  * Běhy krajinou nemají určeno, že pro jejich absolvování je zapotřebí určitá technika nebo speciální vybavení.
  * Nicméně pořadatel může určit nebo doporučit povinné bezpečnostní vybavení odpovídající podmínkám, které je jsou nebo které je možno očekávat na trati, aby se závodnici vyhnuli stresovým situacím nebo v případě nehody mohli dát zprávu a v bezpečí vyčkat příchodu pomoci.
  * Pláštěnka, píšťalka, zásoba vody a jídla jsou minimum, co by každý atlet měl mít u sebe.
  * Pokud je pořadatelem povoleno, závodníci mohou používat hole, jako jsou hole pro „nordic walking“.
  \enditems

Start
* Závod musí být odstartován výstřelem. Použijí se povely pro běhy delší než 400 m (viz P 162.2.b). V závodech, kde startuje velký počet atletů, mají být pět, tři a jednu minutu před vlastním startem dána výstražná znamení.
Bezpečnost
* Pořadatelé závodu musí zajistit bezpečnost atletů a činovníků a musí mít připravený plán vč. technických prostředků pro zajištění zdraví, bezpečnosti a případné záchrany atletů a ostatních účastníků závodu.

Stanice pomoci
* Jelikož běhy krajinou jsou založeny na soběstačnosti účastníků, v úsecích mezi stanicemi pomoci musí každý atlet být nezávislý na jiných osobách z hlediska oblečení, komunikace, stravy a jídla. S ohledem na tuto skutečnost musí stanice pomoci být pořadatelem naplánovány a rozmístěny tak, aby respektovaly soběstačnost atletů, ale současně odpovídaly požadavkům na zdraví a bezpečnost atletů.

Průběh závodů
* Pokud se příslušný vrchní rozhodčí dozví od rozhodčího, úsekového rozhodčího nebo jinak, že atlet opustil vyznačenou trať a zkrátil si tak předepsanou vzdálenost, musí takového běžce diskvalifikovat.
* Pomoc může být poskytnuta pouze na stanicích pomoci.
* Pořadatelé každého běhu krajinou musí vydat soutěžní řád určující situace, které mohou vést k penalizaci nebo diskvalifikaci běžců.
\enditems

\endinput