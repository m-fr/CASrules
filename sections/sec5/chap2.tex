\sec VŠEOBECNÁ SOUTĚŽNÍ PRAVIDLA

\secc Atletická zařízení

Atletické soutěže je možno konat na jakémkoliv závodišti s~jednotným povrchem, který odpovídá specifikacím Manuálu IAAF pro Atletická zařízení (\uv{IAAF Track and Field Facilities Manual}).

Soutěže pod otevřeným nebem uvedené v~P1.1.a) mohou být pořádány pouze na zařízeních, která mají Certifikát IAAF třídy 1.
Doporučuje se, aby se na těchto zařízeních konaly i soutěže pod otevřeným nebem uvedené v~pravidle P 1.1.b) až j), pokud jsou k~dispozici.

V~každém případě musí být pro zařízení, na nichž se mají konat soutěže pod otevřeným nebem uvedené v~P 1.1.b) až j), požadován Certifikát IAAF třídy 2.

POZN. 1: IAAF Manuál atletických zařízení (\uv{The IAAF Track and Field Facilities}), který je možno obdržet v~kanceláři IAAF nebo stáhnout z~webových stránek IAAF, obsahuje podrobnější, přesné specifikace pro návrh a stavbu atletických sportovišť, včetně náčrtků pro měření a značení dráhy.

POZN. 2: Aktuální standardní formuláře žádosti o~certifikaci atletických zařízení a protokol o~jejich měření jakož i postupy systému certifikace je možno získat přímo v~kanceláři IAAF nebo na webových stránkách IAAF.

POZN. 3: Pro chodecké soutěže na silnici a atletické soutěže konané na silnici nebo ve volné krajině (cross-country), běhy po stezkách a~horské tratě, viz P230.11, P240.2, P240.3, P250.1-3, P251.1 a P252.1.

POZN. 4: Pro halové soutěže viz P211.

Pozn.: Výklad pojmů:
\begitems
* Atletická zařízení -- tímto pojmem se ve smyslu těchto pravidel rozumí jak atletické dráhy, tak sektory jednotlivých soutěží v~poli, tj. sektory pro jednotlivé atletické a vrhačské disciplíny a to jak na stabilních sportovištích (stadiony) tak i budovaná dočasně mimo stabilní sportoviště.
* Soutěžní prostor -- zahrnuje jak příslušnou část závodiště pro konání určité soutěže, tj. atletký ovál, rozběžiště s~doskočištěm nebo vrhačský kruh s~výsečí pro dopad náčiní, tak i přilehlé prostory, v~nichž se závodníci v~průběhu soutěže mohou pohybovat. Vymezení soutěžního sektoru je v~pravomoci příslušného vrchního rozhodčího nebo vrchníka (viz též P144.1. POZN.).
* Sportoviště -- tímto pojmem se ve smyslu těchto pravidel rozumí nejen veškerá atletická zařízení, ale též prostory pro diváky, které je obklopují (např. stadion s~tribunami).
\enditems

\secc Věkové skupiny, kategorie podle pohlaví

\begitems \style N
\bold{Kategorie podle věku}
* Soutěže podle těchto pravidel mohou být rozděleny do následujících věkových skupin:
  \begitems
  * Pod 18 (U18) chlapci a dívky, kteří do 31.12. v~roce konání soutěže dovršili 16 nebo 17 let.

  Pozn.: pro tuto věkovou kategorii se v~ČR používá pojem dorostenci/dorostenky,

  * Pod 20 (U20) závodníci a závodnice, kteří do 31.12. v~roce konání soutěže dovršili 18 nebo 19 let.


  Pozn.: pro tuto věkovou kategorii se v~ČR používá pojem junioři a juniorky,

  * Veteráni a veteránky (Master Men and Women): Všichni atleti, kteří dosáhli věku 35 let.

  Pozn.: rozhoduje datum narození.

  POZN. 1: Všechny záležitosti týkající se veteránských soutěží upravuje příručka IAAF/WMA, schválená Radou IAAF a Radou WMA.

  POZN. 2: Přípustnost k~soutěži IAAF, vč. minimálního věku účastníků soutěží se řídí příslušnými propozicemi a předpisy.
  \enditems
* Atlet je oprávněn zúčastnit se soutěží ve věkové skupině podle těchto pravidel, pokud věkem spadá do rozsahu příslušné věkové skupiny. Atlet musí být schopen prokázat svůj věk předložením platného pasu nebo jiného průkazu povoleného předpisy soutěže. Atlet, který tuto podmínku nesplní nebo odmítne splnit, není oprávněn se soutěží zúčastnit.

POZN.: Sankce za nedodržení tohoto pravidla jsou uvedeny v~P22.2.

\bold{Kategorie podle pohlaví atletů}

* Soutěže podle těchto pravidel se dělí na mužské, ženské a univerzální soutěže. Pokud je pořádaná smíšená soutěž mimo stadion nebo ve výjimečných případech uvedených v~P147, musí být vyhlášeny nebo vyznačeny samostatné klasifikace pro mužské a ženské účastníky. V~případě univerzální soutěže je vyhlášena pouze jednotná výsledková klasifikace.
* Atlet je oprávněn startovat v~mužské (nebo univerzální) soutěži, pokud je zákonně uznáván jako osoba mužského pohlaví a je oprávněn k~účasti v~soutěžích podle těchto pravidel a předpisů.
* Atletka je oprávněna startovat v~ženské (nebo univerzální) soutěži, pokud je zákonně uznávána jako osoba ženského pohlaví a je oprávněna k~účasti v~soutěžích podle těchto pravidel a předpisů.
* Rada IAAF schvaluje předpisy určující oprávněnost k~účasti v~ženských soutěžích pro:
  \begitems \style a
  * osoby ženského pohlaví, které se podrobily změně z~mužského na ženské pohlaví;
  * osoby ženského pohlaví s~hyperandrogenismem.
  \enditems

  Závodníci, odmítající se podrobit příslušným předpisům, nejsou oprávněni k~účasti v~soutěžích IAAF.
  
  POZN.: Postihy za nedodržení P141 viz P22.2 .

\itemnum=30
* Kategorie soutěží řízených ČAS zahrnují kromě kategorií muži, ženy, junioři, juniorky, dorostenci, dorostenky, rovněž kategorie žáci a žákyně starší, žáci a žákyně mladší.

Věkové hranice v~těchto kategoriích žactva upravuje soutěžní řád.

\secc Přihlášky

\begitems \style N
* Soutěže podle pravidel IAAF jsou vyhrazeny sportovcům způsobilým k~účasti.
* Způsobilost atleta k~účasti na soutěžích mimo jeho vlastní zemi je dána ustanovením P 4.2. Způsobilost je uznána pokud Technickému delegátu nejsou předloženy námitky (viz též P 146.1).

\bold{Účast v~současně probíhajících disciplínách}
* Je-li atlet přihlášen k~soutěži na dráze a soutěži v~poli nebo k~několika soutěžím v~poli, které probíhají současně, může příslušný vrchní rozhodčí povolit tomuto atletovi v~právě probíhajícím soutěžním kole nebo pro každý pokus ve skoku do výšky a skoku o~tyči, provést tento pokus mimo pořadí stanovené před započetím soutěže. Pokud se atlet následně k~příslušnému pokusu nedostaví, bude to považováno za vzdání se pokusu, jakmile uplyne doba vymezená pro provedení tohoto pokusu.

POZN.: Při soutěži v~poli vrchní rozhodčí nedovolí atletovi vykonat pokus mimo dané pořadí v~posledním kole, ale může to povolit během dřívějších kol pokusů. Ve vícebojích je možné povolit změnu pořadí ve kterémkoliv kole pokusů.

\bold{Nedostavení se ke startu}
* Při všech soutěžích uvedených v~P 1.1.a), b), c) a f) musí být atlet vyloučen ze všech dalších startů (včetně všech soutěží, kde startuje současně) v~soutěži, včetně štafet, v~případě že:
  \begitems \style a
  * bylo dáno konečné potvrzení startu v~soutěži, ale atlet se soutěže nezúčastnil,

  POZN.: Závazný časový údaj pro konečné potvrzení startu musí být předem oznámen.

  * atlet postoupil z~kvalifikační soutěže či rozběhu do dalšího kola, ale pak dále nesoutěžil;
  * atlet nesplnil podmínku soutěžit čestně, s~\uv{bona fide} úsilím, tj. v~dobrém úmyslu, s~vynaložením dostatečného úsilí. O~splnění této podmínky rozhoduje příslušný vrchní rozhodčí a odpovídající záznam musí být uveden v~úředních výsledcích.

  POZN.: Podmínka soutěžení s~dostatečným úsilím se netýká jednotlivých disciplín soutěží ve víceboji.

  \enditems
Lékařské potvrzení vystavené a založené na vyšetření atleta lékařským delegátem jmenovaným podle P113 nebo, pokud lékařský delegát nebyl jmenován, lékařem jmenovaným pořadatelem, může být přijato jako dostatečný důvod neúčasti atleta v~soutěži, do níž byl řádně přihlášen nebo v~níž startoval v~předcházejícím kole. V~tomto případě mu lze povolit start v~dalších disciplínách (vyjma v~dalších disciplínách vícebojů) v~následujících dnech soutěží. Jiné ospravedlnitelné důvody, tj. skutečnosti nezávislé na vlastní činnosti atleta, jako problémy s~dopravou, mohou být, po ověření, technickým delegátem rovněž přijaty.

Pozn.: Ustanovení a), b), c) platí v~soutěžích ČAS, pokud v~příslušném soutěžním řádu není uvedeno jinak.

\bold{Nedostavení se do svolavatelny}
* S~přihlédnutím k~dodatečným sankcím podle P142.4 a vyjma jak uvedeno níže, atlet musí být vyloučen z~účasti ve kterékoliv disciplíně, ke které se neprezentoval ve svolavatelně v~určeném čase dle časového rozpisu (viz P136). Ve výsledkové listině bude uvedeno DNS.

O~vyloučení rozhoduje příslušný vrchní rozhodčí (vč. možnosti startu atleta pod protestem, nelze-li rozhodnout okamžitě) a ve výsledkové listině musí být uveden odpovídající odkaz.

Oprávněné důvody (např. faktory působící nezávisle na startujícím, jako je problém s~oficiální dopravou, chyba v~oznámeném čase příchodu do svolavatelny) mohou být po ověření vrchním rozhodčím přijaty a atletům účast povolena.

\bold{Souběh soutěží různých věkových kategorií}
\itemnum=30
* Každá věková kategorie musí mít pro každou disciplínu vlastní zápis.
* Soutěží-li v~dané disciplíně současně atleti různých věkových kategorií, hodnotí se výsledek soutěže jednotně v~rámci nejvyšší kategorie. Při soutěžích ve vrhu a hodech musí všichni atleti startující v~takovéto soutěži používat náčiní platné pro nejvyšší věkovou kategorii v~soutěži zúčastněnou.
\enditems

\secc Oblečení, obuv a startovní označení

\begitems \style N
\bold{Oblečení}
* Při všech soutěžích musí být oděv atletů čistý, upravený a~musí být nošen tak, aby nevzbuzoval pohoršení. Oděv musí být zhotoven z~látky neprůhledné i za mokra. Závodníci nesmí nosit oděv bránící rozhodčím v~pozorování. Tílko má mít stejnou barvu vpředu i vzadu. Při všech soutěžích uvedených v~P1.1.a), b), c), f) a g) a při reprezentaci své národní federace v~soutěžích podle P1.1.d), h) musí atlet startovat v~jednotném oblečení, úředně schváleném jejich národní federací a při všech soutěžích uvedených v~P1.1.a), b), c), f) a g) musí neutrální závodníci startovat v~jednotném oblečení schváleném IAAF. Vyhlašování vítězů a jakékoliv čestné kolo se ve smyslu tohoto pravidla považuje za součást soutěže.

POZN.: Příslušný řídící orgán může v~předpisech soutěže stanovit, že tílka atletů musí mít povinně stejnou barvu vpředu i~vzadu.

\bold{Obuv}
* Závodníci mohou soutěžit bosí nebo s~obuví na jedné či obou nohách. Účelem atletické obuvi je ochrana a zpevnění nohou a~pevný záběr. Obuv však nesmí být vyrobena tak, aby atletům poskytovala jakoukoliv nečestnou pomoc nebo výhodu. Jakýkoliv typ použité obuvi musí být k~dispozici všem v~duchu univerzality atletiky.

POZN. 1: Uzpůsobení obuvi tak, aby vyhovovala charakteristikám chodidla atleta, je povoleno, pokud úprava je provedena v~souladu s~principy těchto pravidel.

POZN. 2: Pokud IAAF obdrží důkaz, že obuv používaná při soutěžích neodpovídá těmto pravidlům či duchu těchto pravidel, může nechat takovou obuv přezkoumat, a pokud nevyhoví, může zakázat používání takové obuvi při soutěžích.

\bold{Počet hřebů}
* Podrážka a podpatek obuvi musí být upraveny tak, že je možno použít až 11 hřebů. Může být používán libovolný počet hřebů, ale nejvýše 11.

\bold{Rozměry hřebů}
* Část hřebu, vyčnívající z~podrážky nebo z~podpatku, nesmí být delší než 9 mm, vyjma skoku do výšky a hodu oštěpem, kde nesmí být delší než 12 mm. Hřeb musí být proveden tak, že alespoň polovinou své délky bližší hrotu projde měrkou se čtvercovým otvorem o~straně 4 mm. Pokud výrobce povrchu nebo správa stadionu požaduje menší rozměr, je třeba tomuto požadavku vyhovět.

POZN.: Povrch musí být vhodný pro použití dle tohoto pravidla.

Pozn. : Pro povrchy drah z~ostatních materiálů (škvára, antuka či jiné povrchy na nesyntetické bázi), může mít hřeb délku nejvýše 25 mm.

\bold{Podrážka a podpatek}
* Podrážka i podpatek mohou mít drážky, žebrování, vroubkování nebo výstupky, pokud jsou všechny zhotoveny ze stejného nebo podobného materiálu jako základní podrážka.

Pro skok do výšky a skok do dálky musí být tloušťka podrážky max. 13 mm a tloušťka podpatku pro skok do výšky max. 19~mm. Pro všechny ostatní soutěže může být podrážka i podpatek libovolné tloušťky.

POZN.: Tloušťka podrážky a podpatku musí být měřena mezi vnitřním horním povrchem a vnějším spodním povrchem, včetně výše uvedených prvků a včetně jakéhokoliv typu vnitřní vložky boty.

\bold{Vložky nebo doplňky obuvi}
* Závodníci nesmějí použít žádné prvky, ať vně nebo uvnitř, jehož důsledkem by bylo zvýšení tloušťky obuvi nad povolené maximum nebo by poskytlo jeho uživateli jakoukoliv výhodu oproti obuvi popsané v~předchozích odstavcích.

\bold{Startovní označení (bib)}
* Každý atlet musí mít, během soutěže, dvě startovní označení, která musí být nošena viditelně, jedno na prsou a jedno na zádech, vyjma atletických soutěží, kde stačí jedno, buď na zádech, nebo na prsou. Je dovoleno, aby na jednom nebo všech startovních označeních bylo místo čísla uvedeno jméno atleta nebo jiné vhodné označení. Pokud jsou použita čísla, musí souhlasit s~číslem atleta uvedeným v~programu nebo ve startovní listině. Pokud atlet během soutěže obléká  teplákovou (sportovní) soupravu nebo v~ní soutěží, musí mít na ni startovní označení připevněna obdobným způsobem.
* Žádný atlet nemá dovoleno absolvovat jakoukoliv části soutěže bez vhodného startovního označení (dvou označení) a/nebo identifikace.
* Startovní označení musí být nošena tak, jak byla vydána a~nesmí být sestřižena, složena nebo jinak pozměněna. Při soutěžích na dlouhých tratích startovní označení smějí být perforována pro lepší proudění vzduchu, ale perforace nesmějí být v~místech jakéhokoliv značení.
* Je-li použito cílové kamery, může pořadatel vyžadovat, aby závodníci měli na bocích trenýrek nebo spodní části těla další číselná označení přilnavého typu.
* Pokud některý atlet jakýmkoliv způsobem toto pravidlo nedodrží a:
  \begitems \style a
  * odmítne jakkoliv se řídit příkazem vrchního rozhodčího,
  * účastní se soutěže,
  \enditems
bude diskvalifikován.

\itemnum=30
* Pokud rozhodčí nebo startér připustí účast atleta v~soutěži bez požadovaného počtu startovních označení či v~jiném než oddílovém dresu, není to důvod k~diskvalifikaci a výsledek atletem dosažený je platný. Nicméně tato skutečnost, pokud je zjištěna následně může být důvodem pro udělení žluté karty.
* Oblečení, obuv a startovní označení při soutěžích řízených ČAS upravuje soutěžní řád.
\enditems

\secc Napomáhání atletům

\begitems \style N
\bold{Lékařské ošetření a asistence}
* Lékařské vyšetření či ošetření nebo fyzioterapie mohou být poskytnuty přímo v~soutěžním prostoru členy oficiálního lékařského týmu jmenovaného pořadatelem, jasně označenými páskou na rukávu, vestou nebo podobným zřetelným označením. V~mimosoutěžním prostoru může být taková péče poskytnuta v~místech určených k~tomuto účelu též lékařským personálem akreditovaných družstev, který byl pro tento účel schválený lékařským nebo technickým delegátem. V~žádném případě nesmí lékařský zásah způsobit zpoždění soutěže nebo změnu určeného pořadí atletů. Ošetření či asistence poskytnuté jakoukoliv jinou osobou bezprostředně před soutěží, poté co atlet opustil svolavatelnu, nebo během soutěže, je napomáháním.

POZN.: Soutěžní prostor, který je obvykle ohrazený, je pro tento účel definovaný jako prostor, v~němž se soutěž koná a~kam je přístup povolen pouze atletům a osobám oprávněným v~souladu s~příslušnými pravidly a předpisy.

* Kterýkoliv atlet, který poskytuje nebo přijímá dopomoc ze soutěžního prostoru v~průběhu disciplíny (vč. dle P163.14, P163.15, 230.10 a P240.8), musí být vrchním rozhodčím varován a poučen, že při opakování bude z~další účasti v~disciplíně vyloučen.

POZN.: V~případech spadajících pod P144.3.a) může dojít k~diskvalifikaci bez předchozího napomenutí.

\bold{Nedovolené napomáhání}
* Pro účely tohoto pravidla jsou následující příklady konání atletů považovány za napomáhání, a proto nejsou dovolené:
  \begitems \style a
  * Udávání tempa osobami, které nejsou účastníky stejného běhu, nebo běžci či chodci, kteří jsou či budou předběhnutí o~kolo nebo jakýmikoliv technickými prostředky (mimo prostředků uvedených v~P144.4.d).
  * Držení či používání video přístrojů, rádií, CD, vysílaček, mobilních telefonů či podobných přístrojů v~soutěžním prostoru.
  * Vyjma obuvi vyhovující ustanovení P143, použití jakékoliv technologie či technického prostředku, poskytujícího uživateli výhodu, kterou by neměl použitím prostředků specifikovaných nebo dovolených pravidly.
  * Použití mechanické pomůcky, pokud atlet neprokáže, že získání výhody oproti těm, kteří takovou pomůcku nemají, je nepravděpodobné.
  * Pomoc, rada nebo jiná podpora od kteréhokoliv činovníka soutěže, která nesouvisí nebo není vyžadovaná vzhledem k~úloze tohoto činovníka v~probíhající soutěži (např. trenérská rada, označení místa odrazu ve atletické soutěži, vyjma sdělení o~nezdařeném pokusu v~horizontálních skocích, času nebo ztráty na soupeře v~závodě atd.).
  * Fyzická pomoc jiného atleta (kromě pomoci postavit se) umožňující další dopředný pohyb v~závodě.
  \enditems

\bold{Dovolené napomáhání}
* Pro účely tohoto pravidla se nepovažuje za napomáhání, a~proto je dovoleno:
  \begitems \style a
  * Komunikace mezi atlety a jejich trenéry, kteří se nenacházejí v~soutěžní prostor.

  Pro usnadnění této komunikace a nenarušování průběhu soutěže má být trenérům vyhrazeno místo na tribuně v~blízkosti soutěžního prostoru.

  POZN.: Trenéři a ostatní osoby splňující ustanovení P230.10 a P240.8 mohou se svými atletmi komunikovat.

  * Lékařské vyšetření či ošetření nebo fyzioterapie poskytnutá v~soutěžním prostoru a umožňující atletu zúčastnit se, nebo dokončit soutěž podle P144.1.
  * Jakýkoliv druh osobního zabezpečení (např. bandáž, náplast, pás, výztuž, chladič zápěstí, dýchací pomůcka, apod.) pro lékařské účely. Vrchní rozhodčí ve spolupráci s~Lékařským delegátem má právo každý případ ověřit, pokud to považuje za žádoucí (viz též P 187.4 a P187.5).
  * Monitor tepové frekvence či rychlosti nebo krokoměr nebo podobné přístroje, které atlet nese nebo má u~sebe během soutěže, pokud takové zařízení nelze využít ke komunikaci s~jinou osobou.
  * Pozorování záznamů předchozího pokusu (pokusů) pořízených osobami nacházejícími se mimo soutěžní prostor (viz P144.1. POZN.) atlety účastnícími se soutěží v~poli. Tyto záznamy nesmějí být vneseny do daného soutěžního prostoru.
  * Pokrývky hlavy, rukavice, obuv, součásti oblečení poskytnuté atletům na oficiálních stáncích nebo jinak schválené příslušným vrchním rozhodčím.
  \enditems
\enditems

\secc Účinky vyloučení (diskvalifikace)

\begitems \style N
\bold{Diskvalifikace za porušení technického pravidla‚ jiného než P125.5 a P162.5}
* Je-li atlet vyloučen z~další účasti v~dané disciplíně pro porušení technických pravidel IAAF v~určitém kole soutěže (vyjma vyloučení podle P 125.5 nebo P 162.5), jakýkoliv do té doby dosažený výkon v~daném kole soutěže nesmí být platným. Nicméně výkony dosažené v~předcházejícím kole disciplíny nebo v~předcházejících disciplínách víceboje zůstávají v~platnosti. Takováto diskvalifikace v~disciplíně nebrání, aby atlet nemohl soutěžit ve všech dalších disciplínách daných soutěží.

\bold{Diskvalifikace za porušení P125.5, vč. P162.5}
* Pokud je atlet vyloučen ze soutěže pro porušení P 125.5, bude v~dané disciplíně diskvalifikován. Pokud k~druhému napomenutí dojde v~jiné disciplíně, bude atlet diskvalifikován pouze v~té, kde byl napomenut podruhé. Jakýkoliv do té doby dosažený výkon v~daném kole nesmí být platným. Avšak výkony dosažené v~předcházejícím kvalifikačním kole této disciplíny, jiných disciplínách či v~předcházejících disciplínách víceboje zůstávají v~platnosti. Taková diskvalifikace nedovoluje atletu startovat ve všech dalších disciplínách (včetně dalších individuálních disciplín a disciplín víceboje, dalších disciplínách, kterých se účastní souběžně a ve štafetách) této soutěže.
* Pokud je přestupek vážný, ředitel soutěží jej ohlásí příslušnému řídícímu orgánu pro zvážení dalšího disciplinárního jednání.

Pozn.: Při soutěžích řízených ČAS je třeba považovat neuposlechnutí příkazu rozhodčího za nesportovní chování a atlet se vystavuje nebezpečí okamžitého vyloučení z~další účasti v~dané disciplíně a všech disciplín konaných téhož dne.
\enditems

\secc Protesty a odvolání

\begitems \style N
* Protesty týkající se práva atleta na účast v~soutěži musí být podány před zahájením soutěží a to technickému delegátu(tům). Proti jeho rozhodnutí je možno se odvolat k~jury. Nelze-li protest uspokojivě vyřešit před zahájením soutěží, je atletu povoleno \uv{startovat pod protestem} a záležitost musí být předána příslušnému řídícímu orgánu.

Pozn.: Pokud nebyla jury ustanovena v~soutěžích ČAS, podávají se protesty hlavnímu rozhodčímu.

* Protesty týkající se výsledků nebo průběhu disciplíny musí být podány do 30 minut po úředním vyhlášení výsledků této disciplíny.

Pořadatel soutěže zodpovídá za to, že čas vyhlášení výsledků každé disciplíny bude zaznamenán.

* Každý protest musí být podán ústně vrchnímu rozhodčímu buď samotným atletem nebo někým jednajícím jeho jménem nebo oficiálním představitelem družstva. Taková osoba nebo družstvo může podat protest pouze tehdy, pokud soutěží ve stejném kole soutěže, k~níž se protest (nebo následné odvolání) vztahuje (nebo se účastní soutěže, kde rozhodují body získané členy družstva). Aby rozhodnutí bylo spravedlivé, vrchní rozhodčí musí zvážit všechny dostupné důkazy, která považuje za nezbytné, včetně filmových záběrů či obrazu úředního nebo jiného videozáznamu, který je k~dispozici. Vrchní rozhodčí může rozhodnout o~protestu sám nebo jej předat jury. Proti rozhodnutí vrchního rozhodčího je možno podat odvolání k~jury. Pokud vrchní rozhodčí není k~dosažení nebo není přítomen, je možné mu předat protest prostřednictvím Technického informačního střediska (TIC).
* Soutěže na dráze:
  \begitems \style a
  * Pokud atlet vznese okamžitý protest proti tomu, že jeho start byl označen jako nezdařený, vrchní rozhodčí atletkých soutěží, pokud má jakékoliv pochybnosti, může povolit atletovi pokračování v~soutěži "pod protestem", aby byla zachována práva všech dotčených. Takové pokračování v~soutěži pod protestem však nemůže být přiznáno v~případě, že start byl označen jako nezdařený příslušným Startovním informačním systémem, schváleným IAAF, pokud vrchní rozhodčí z~jakéhokoliv důvodu nerozhodne, že informace tohoto Systému je zřejmě nepřesná.
  * Protest může být podán proti tomu, že startér nezdařený start nevrátil, nebo dle P162.5 start nepřerušil. Takový protest může být podán pouze ve prospěch atleta, který běh dokončil. Je-li protest uznán, kterýkoliv atlet, který nezdařený start způsobil nebo jehož chování by mohlo vést k~přerušení startu a na nějž se vztahuje varování nebo diskvalifikace podle P162.5, P162.7, P162.8 nebo P200.8c), musí být varován, nebo diskvalifikován. Bez ohledu na to, zda dojde, či nedojde k~varování, nebo diskvalifikaci, vrchní rozhodčí má právo prohlásit soutěž nebo část soutěže za neplatnou a nařídit opakování soutěže nebo její části, je-li to dle jeho názoru v~zájmu spravedlnosti.

  POZN.: Atlet má právo na protest a odvolání podle P146.4 b) bez ohledu na skutečnost, zda byl či nebyl použit Startovní informační systém.

  * Pokud je předmětem protestu neoprávněné vyloučení atleta ze soutěže pro nezdařený start a protest je uznán po ukončení závodu, atletu má být umožněn samostatný běh, aby dosáhl v~soutěži výkonu a mohl případně postoupit do následujícího kola. Žádný atlet nesmí postoupit do následujícího kola, aniž by soutěžil ve všech kolech, pokud vrchní rozhodčí nebo jury rozhodnou za daných okolností případu jinak, např. pro krátkou dobu před následujícím kolem nebo délku závodu.

  POZN.: Toto ustanovení může vrchní rozhodčí nebo jury uplatnit za jiných okolností, kdy to považuje za oprávněné (viz P163.2).
  \enditems
* Pokud při soutěži v~poli atlet vznese okamžitý ústní protest proti prohlášení jeho pokusu za nezdařený, vrchní rozhodčí soutěže může, pokud má jakékoliv pochybnosti, nařídit, aby byl pokus změřen a zaznamenán, aby tak byla zachována práva zúčastněných.

Pokud k~pokusu, který je předmětem protestu došlo:
  \begitems \style a
  * v~průběhu prvních tří kol některé horizontální soutěže v~poli, v~níž soutěží více než 8 atletů a protestující atlet by postoupil do kteréhokoliv následujícího kola soutěže, jen pokud by protest či následné odvolání byly uznány; nebo
  * v~průběhu vertikální soutěže v~poli, kdy protestující atlet by postoupil na vyšší výšku pouze v~případě, že jeho protest nebo následné odvolání bude uznáno, vrchní rozhodčí, pokud má jakékoliv pochybnosti, může atletovi povolit pokračování v~soutěži \uv{pod protestem}, aby byla zachována práva všech dotčených.
  \enditems
* Výkon dosažený za okolností, proti nimž byl podán protest nebo jakýkoliv jeho další výkon během pokračování v~soutěži pod protestem se však stane platným pouze v~případě, kdy protestu vyhoví vrchní rozhodčí nebo jury rozhodne ve prospěch protestujícího atleta.
* Odvolání k~jury musí být podáno do 30 minut od:
  \begitems \style a
  * oficiálního oznámení výsledku soutěže vycházejícího rozhodnutím vrchního rozhodčího, nebo
  * oznámení těm, kdo podali protest, že výsledek nebyl změněn.
  \enditems

Odvolání musí být písemné a podepsané atletem, osobou jednající jeho jménem nebo oficiálním zástupcem družstva a~musí být doloženo vkladem 100 USD nebo jejich ekvivalentem. Vklad propadne, bude-li protest zamítnut. Atlet nebo družstvo mohou podat protest pouze tehdy, pokud soutěží ve stejném kole soutěže, k~níž se protest (nebo následné odvolání) vztahuje (nebo se účastní soutěže, kde rozhodují body získané členy družstva).

POZN.: Po oznámení svého rozhodnutí musí vrchní rozhodčí okamžitě informovat TIC o~čase tohoto oznámení. Pokud vrchní rozhodčí nemůže své rozhodnutí sdělit příslušnému družstvu/atletovi ústně, je oficiálním časem okamžik, kdy pozměněný výsledek nebo  toto rozhodnutí bylo vyvěšeno v~TIC (závodní kanceláři).

Pozn.: Pro soutěže konané na území České republiky je výše vkladu dána soutěžním řádem nebo rozpisem příslušné soutěže, nejvýše však ekvivalentem výše uvedené částky.

* Jury musí vyslechnout všechny zúčastněné osoby, vč. Vrchního rozhodčího (vyjma případu, kdy jeho rohodnutí chce jury potvrdit). Pokud je jury na pochybách, mohou být vzaty v~úvahu další dostupné důkazy. Pokud tyto důkazy, včetně jakéhokoliv dostupného video záznamu, nejsou přesvědčivé, zůstává v~platnosti rozhodnutí vrchního rozhodčího nebo vrchníka chůze.
* Jury může přehodnotit své rozhodnutí, jsou-li předložené nové přesvědčivé důkazy, pokud takové nové rozhodnutí může být ještě uplatněno. Nové rozhodnutí může být běžně uplatněno pouze před vyhlášením vítězů příslušné disciplíny, pokud příslušný řídící orgán nerozhodne, že okolnosti změnu ospravedlňují.
* Rozhodnutí, které se týká záležitosti, která není v~těchto pravidlech řešena, musí předseda jury ohlásit prvnímu jednateli IAAF (dříve generální sekretář).
* Rozhodnutí jury (pokud jury nebyla ustanovena, nebo nerozhodovala, pak rozhodnutí vrchního rozhodčího) je konečné. Proti konečnému rozhodnutí neexistuje odvolání, včetně odvolání k~CAS (Arbitrážnímu soudu pro sport.

Pozn.: V~soutěžích ČAS, pokud není jury, rozhoduje s~konečnou platností hlavní rozhodčí.

Pozn::Pokud při soutěži nebyla ustanovena funkce vrchního rozhodčího, podávají se protesty stejným způsobem vrchníkovi příslušné disciplíny.
\enditems

\secc Smíšené soutěže

\begitems \style N
* Univerzální soutěže jako jsou štafety a soutěže družstev, v~kterých muži a ženy soutěží společně, nebo kdy muži a ženy soutěží v~jedné společné klasifikaci, jsou dovolené v~souladu s~příslušnými předpisy odpovědného orgánu.
* V~soutěžích jiných, než uvedeno v~P147.1, konaných zcela na stadiónech, nejsou smíšené soutěže mezi mužskými a ženskými účastníky běžně přípustné.

Nicméně smíšené soutěže v~bězích na 5000 m a delších lze pořádat, vyjma soutěží uvedených v~P 1.1.a) až h). V~soutěžích uvedených v~P 1.1.i) a j) lze smíšené soutěže pořádat pouze tehdy, pokud je to schválené příslušnou oblastní asociací.

POZN. 1: Pokud se smíšené soutěže konají v~soutěžích v~poli, musí být vedeny samostatné zápisy a vyhlášeny výsledky samostatně pro jednotlivá pohlaví. Pro soutěže v~bězích musí být ve výsledcích u~každého atleta uvedeno pohlaví.

POZN. 2: V~bězích jsou podle tohoto pravidla povoleny smíšené soutěže pouze tehdy, pokud není dostatek účastníků v~mužské nebo ženské soutěži.

POZN. 3: V~žádném případě však nelze ve smíšených atletkých soutěžích připustit, aby běžci jednoho pohlaví působili jako vodiči nebo pomáhali atletům druhého pohlaví.

Pozn.: POZN. 1 až 3 platí obdobně i pro soutěže řízené ČAS.
\enditems

\secc Dozor a Měření

\begitems \style N
* Přesnost značení a instalace atletických zařízení podle P140 a 149.2 musí být zkontrolovaná příslušně kvalifikovaným dozorem, který příslušnému orgánu nebo majiteli zařízení či jeho provozovateli vystaví příslušné certifikáty s~detaily kontrolních měření. Musí mít přístup ke všem plánům a výkresům stadionu a protokolu o~posledních měřeních pro jejich ověření.
* Při atletických  soutěžích uvedených v~P 1.1.a) ,b), c) a f) musí být veškeré vzdálenosti měřeny kalibrovaným ocelovým pásmem nebo tyčí nebo vědeckým měřícím přístrojem. Ocelové pásmo, tyč nebo vědecký měřicí přístroj musí být vyrobený a kalibrovaný podle mezinárodních norem. Přesnost měřícího zařízení použitého v~soutěži musí být ověřena příslušnou organizací akreditovanou u~národního metrologického úřadu.

Při jiných soutěžích než uvedených v~P1.1.a), b), c) a f) může být použito i pásmo ze skelných vláken.

POZN.: Pokud se týká schválení rekordních výkonů, viz P 260.17.a).
\enditems

\secc Platnost výkonu

\begitems \style N
* Výkon je platný pouze tehdy, pokud byl atletem nebo atletkou dosažen v~oficiální soutěži, konané v~souladu s~pravidly IAAF.
* Výkony se obvykle dosahují na stadionech, výkony dosažené mimo tradiční atletická zařízení (jako jsou soutěže konané na náměstích měst, na jiných sportovních zařízení, na pláži apod.) nebo na dočasných zařízeních vybudovaných na stadionu, budou platné a uznány pro jakékoliv účely pouze tehdy, pokud budou dosaženy za těchto podmínek:
  \begitems \style a
  * příslušný řídící orgán (viz P1 až P3) soutěž schválil;
  * soutěž řídili jmenovaní rozhodčí uvedení na seznamu národních rozhodčích;
  * při soutěži bylo použito vybavení odpovídající těmto pravidlům;
  * disciplína se konala v~soutěžním prostoru nebo na zařízení, které odpovídá těmto pravidlům a bylo ověřeno v~souladu s~P148 v~den konání soutěže.

  POZN.: Standardní formulář požadovaný pro ohlášení, že soutěžní prostor nebo zařízení odpovídá předpisům, lze získat z~kanceláře IAAF nebo jej stáhnout z~webových stránek IAAF.

  Pozn.: Standardní formulář požadovaný pro ohlášení, že místo konání soutěže nebo zařízení odpovídá předpisům ČAS, lze získat v~kanceláři ČAS.
  \enditems

* Výkony dosažené v~souladu s~těmito pravidly v~kvalifikačních kolech, při rozhodování rovnosti umístění ve skoku vysokém a skoku o~tyči nebo v~kterékoliv disciplíně, která byla podle ustanovení P125.7, P146.4.b), 163.2, nebo P180.20 zrušena, nebo v~jedné z~disciplín víceboje, bez ohledu na to, zda atlet dokončil celý víceboj, nebo v~chodeckých soutěžích, kde byl atlet potrestán setrváním v~trestné zóně budou prohlášeny za platné pro účely statistické, rekordních zápisů, tabulek a~účastnických limitů.
\enditems

\secc Videozáznamy

Při soutěžích uvedených v~P 1.1.a), b) a c) a tam, kde je to možné i ve všech dalších soutěžích, musí být pořizován, podle příkazů technického delegáta, úřední video záznam všech soutěží. Tento záznam, pokud je pořízen, slouží jako pomůcka pro vrchního rozhodčího videa, pokud je jmenován, a musí dostatečně přesně zaznamenat výkony i jakékoliv porušení pravidel.

Podrobnější informace je uvedena ve směrnici IAAF pro video záznamy a vrchního rozhodčího videa (IAAF Video Recording and Video Referee Guidelines), který lze stáhnout z~webových stránek IAAF.

\secc Bodování

V~utkáních, kde je výsledek určen počtem bodů, se na způsobu bodování všechny zúčastněné země nebo družstva dohodnou před započetím utkání, pokud není uvedeno jinak v~příslušném předpisu nebo soutěžním řádu.

Pozn.: Pro soutěže ČAS je způsob bodování dán Soutěžním řádem.

\endinput