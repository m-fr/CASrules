\sec LÉKAŘSKÉ ZÁLEŽITOSTI

\secc IAAF a lékařské záležitosti

\begitems \style N
* IAAF vykonává lékařský dohled podle těchto pravidel prostřednictvím následujících osob nebo orgánů:
  \begitems \style a
  * Lékařská a antidopingová komise,
  * Manažer pro lékařské záležitosti.
  \enditems

\noindent{\bf Lékařská a antidopingová komise}
* Lékařská a antidopingová komise je ustanovena jako komise Rady podle Článku 6.11.j) Statutu IAAF, aby poskytovala IAAF všeobecná doporučení ve všech  lékařských záležitostech.
* Lékařská a antidopingová komise se schází alespoň jednou ročně, obvykle začátkem roku, aby zkontrolovala aktivity IAAF v~lékařské oblasti za předcházejících 12 měsíců a stanovila svůj program na následující rok. Lékařská a antidopingová komise poskytuje konzultace k~problémům lékařského charakteru vzniklým během roku podle okamžité potřeby.
* Lékařská a antidopingová komise odpovídá na následující specifické úkoly, které ji podle těchto pravidel příslušejí:
  \begitems \style a
  * stanovení zásad a vydávání stanovisek k~problémům lékařského charakteru v~atletice;
  * zveřejňování všeobecných informací z~oblasti sportovního lékařství se zaměřením na atletiku pro praktické lékaře;
  * konzultace Radě, podle potřeby, k~předpisům vztahujícím se k~lékařským problémům vzniklým v~atletice;
  * organizace nebo účast na seminářích zaměřených na aktuální problémy sportovního lékařství v~atletice;
  * vydávání doporučení a směrnic k~organizaci lékařské služby během mezinárodních soutěží;
  * zveřejňování studijních materiálů týkajících se lékařské péče v~atletice s~ohledem na rostoucí všeobecné povědomí atletů a~členů atletických podpůrných týmů o~lékařské medicíně;
  * zaujímání stanoviska k~jednotlivým problémům lékařské povahy, které mohou v~atletice vzniknout a vydávat svá doporučení k~těmto problémům; a
  * spolupracovat dle potřeby s~MOV a dalšími organizacemi zabývajícími se sportovním lékařstvím.
  \enditems
* Předseda může dle svého uvážení pověřit řešením jednotlivých úkolů pracovní skupiny. Podle potřeby může pro získání speciálních odborných znalostí povolat externí odborníky.

\noindent{\bf Manažer pro lékařské záležitosti}
* Manažer pro lékařské záležitosti je lékařsky kvalifikovaný člen Lékařské a antidopingové komise, který zodpovídá za:
  \begitems \style a
  * koordinaci různých úkolů svěřených Lékařské a antidopingové komisi (nebo pracovním skupinám) ve smyslu těchto lékařských pravidel;
  * monitorování a zavedení jakýchkoliv zásad, vyhlášek, doporučení nebo směrnic, které byly vydány Lékařskou a antidopingovou komisí;
  * řízení administrace udělených TUE podle antidopingových předpisů;
  * rozhodování o~nezpůsobilosti atletů, pokud je to podle předpisů požadováno;
  * skutečnost, že problémy lékařské povahy, které se objeví během aktivit IAAF, budou řešeny.
  \enditems
* Manažer pro lékařské záležitosti může kdykoliv během své práce požádat o~radu předsedu Lékařské a antidopingové komise nebo jinou vhodnou osobu. Alespoň jednou ročně, případně častěji na vyžádání, podává zprávu Lékařské a antidopingové komisi.
* Lékařské informace projednávané Lékařskou a antidopingovou komisí podle těchto pravidel musí zůstat přísně důvěrnými materiály ve smyslu příslušných zákonů na ochranu osobnosti.
\enditems

\secc Atleti

\begitems \style N
* Atleti odpovídají za své zdraví a za absolvování lékařských prohlídek.
* Vstupem na pole mezinárodních soutěžích berou na sebe zodpovědnost za jakoukoliv ztrátu, zranění, nebo škodu, které mohou utrpět v~důsledku své účasti na těchto soutěžích a nemohou tuto zodpovědnost přenést na IAAF (resp. její členské federace, ředitele, činovníky, zaměstnavatele, dobrovolné spolupracovníky nebo agenty).
\enditems

\secc Členské federace

\begitems \style N
* Bez ohledu na ustanovení P 49, členské federace musí zajistit, aby atleti spadající pod jejich pravomoc byli při účasti mezinárodních soutěží fyzicky zdraví a jejich výkonnost odpovídala úrovni vrcholové atletiky.
* Každá členská federace musí zajistit příslušné a trvalé lékařské sledování svých atletů vlastními nebo externími orgány. Dále se doporučuje, aby každá členská federace zajistila předsezónní lékařské vyšetření každého atleta startujícího v~mezinárodních soutěžích uvedených v~P 1.1.a) a f). Toto vyšetření by mělo být provedené formou uvedenou v~materiálu IAAF Medical Guidelines (metodika lékařské péče o~atlety).
* Každá členská federace musí jmenovat alespoň jednoho lékaře národního týmu, který by se staral o~lékařskou péči před a pokud je to možné, během mezinárodních soutěží uvedených v~P 1.1.a) a P 1.1.f).
\enditems

\secc Lékařská/bezpečnostní služba při mezinárodních soutěžích

\begitems \style N
* Organizační výbory jsou zodpovědné za zajištění a provedení odpovídající lékařské služby a bezpečnostních opatření během mezinárodních soutěží. Rozsah lékařské služby a bezpečnostních opatření se může měnit v~závislosti na okolnostech, zejména rozsahu a~povaze soutěže, kategorii a počtu zúčastněných atletů a jejich doprovodu, počtu diváků, úrovni zdravotnictví země, kde se soutěže konají a prostředí, v~němž soutěže probíhají, např. klimatických podmínkách, nadmořské výšce apod.
* Lékařská a antidopingová komise musí vydávat a obnovovat praktické směrnice napomáhající organizačnímu výboru poskytnout odpovídající lékařskou službu a zajistit nezbytná bezpečnostní opatření během mezinárodní soutěže.
* Pro určité kategorie nebo disciplíny (např. silniční závody nebo chodecké soutěže) mohou být podle těchto pravidel vzneseny speciální požadavky na lékařskou službu nebo bezpečnostní opatření.
* Lékařská služba a bezpečnostní opatření během mezinárodních soutěží musí minimálně zahrnovat:
  \begitems \style a
  * všeobecnou zdravotní péči o~atlety a akreditované osoby v~místě konání hlavních soutěží a v~místě ubytování atletů;
  * první pomoc a záchrannou službu pro atlety, činovníky soutěží, dobrovolné pracovníky, media a diváky v~místě konání hlavních soutěží;
  * bezpečnostní dozor;
  * koordinaci záchranných a evakuačních plánů; a
  * koordinaci jakýchkoliv speciálních lékařských služeb dle potřeby.
  \enditems
* Organizační výbor každé mezinárodní soutěže uvedené v~P 1.1.a), musí jmenovat lékařského ředitele pro koordinaci lékařské služby a požadavků na bezpečnost během soutěží. Lékařský ředitel je prostředníkem mezi IAAF a organizačním výborem při zajišťování lékařské služby a bezpečnosti.
* Pro mezinárodní soutěže uvedené v~P 1.1.a), IAAF jmenuje Lékařského delegáta, který podle P 113 musí zajistit, že v~místě konání soutěží jsou k~dispozici dostatečná zařízení pro lékařská vyšetření, ošetření a záchrannou službu a v~místě ubytování atletů je zajištěna lékařská služba.
\enditems

\endinput