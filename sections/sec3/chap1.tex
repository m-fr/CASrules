\sec PRAVIDLA O DOPINGU

\rule{29}
\secc Rozsah Antidopingových pravidel

\begitems \style N
* Tato Antidopingová pravidla platí pro IAAF, její členy, oblastní asociace a atlety, jejich doprovod a ostatní osoby, které jsou ve spojení s IAAF, jejími členy nebo oblastními asociacemi prostřednictvím smluv, členství, přidružení, pověření, akreditace nebo účastí na jejich aktivitách či soutěžích.
* Všichni členové a oblastní asociace musí dodržovat tato Antidopingová pravidla a Procedurální směrnice. Antidopingová pravidla a Procedurální směrnice musí být zahrnuty přímo nebo musí být na ně odkaz v pravidlech každého člena a oblastní asociace a každý člen a oblastní asociace musí do svých pravidel zahrnout procesní předpisy nutné pro účinné začlenění Antidopingových pravidel a Procedurálních směrnic (a jejich případných změn). V pravidlech každého člena a oblastní asociace musí být výslovně uvedeno, že všichni atleti, členi jejich zabezpečovacího týmu a ostatní osoby, které jsou pod jejich pravomocí, musí být vázáni těmito Antidopingovými pravidly a Procedurálními směrnicemi.
* Pro získání způsobilosti soutěžit, účastnit se nebo být jakkoliv akreditován na mezinárodní soutěži, musí atlet, (a pokud je to použitelné) členi jeho zabezpečovacího týmu a ostatní osoby podepsat písemné prohlášení, že uznávají a souhlasí s těmito Antidopingovými pravidly a Procedurálními směrnicemi. Formu tohoto prohlášení určí Rada IAAF. Členem vystavená záruka, že jeho atlet je oprávněn startovat v mezinárodních soutěžích (viz P 21.2. výše), musí obsahovat sdělení, že atlet takové prohlášení podepsal v požadované formě a kopie tohoto prohlášení byla zaslána kanceláři IAAF.
* Tato Antidopingová pravidla a Procedurální směrnice musí být dodrženy při všech dopingových kontrolách prováděných pod pravomocí IAAF, jejích členů a oblastních asociací.
* Každý člen odpovídá za to, že veškeré testy na národní úrovni a správa výsledků těchto testů proběhnou v souladu s těmito Antidopingovými pravidly a Procedurálními směrnicemi. Je na členu, zda provede testy a bude jejich výsledky spravovat sám nebo svoji odpovědnost, zcela nebo částečně, převede na národní antidopingovou organizaci nebo třetí stranu, a to buď z vlastního rozhodnutí nebo na základě příslušného národního zákonodárství nebo předpisů. Ve vztahu k takovým zemím se v těchto Antidopingových pravidlech a Procedurálních směrnicích uvedené odkazy na člena nebo národní federaci, tam kde je to vhodné, týkají národní antidopingové organizace nebo třetí strany (nebo jejich činovníků).
* IAAF sleduje činnost svých členských federací při uplatňování anti-dopingových pravidel, vč. kontrol během soutěží a v mimosoutěžním období prováděných na národní úrovni přímo členskou organizací nebo příslušnou národní anti-dopingovou organizací nebo třetí stranou, v souladu s P30.5. Pokud IAAF zjistí, že uvedené činnosti členské federace jsou nedostatečné, nebo nepřiměřené, jak ve vztahu k úspěchům atletů členské federace v mezinárodních soutěžích nebo z jiného důvodu, může Rada IAAF požadovat, aby členská federace podnikla dle Rady nezbytné kroky k zajištění dostatečné úrovně anti-dopingových opatření na příslušném území nebo teritoriu. Pokud členská federace nesplní požadavky Rady, mohou na ni být uvaleny sankce podle P 44.
* Oznámení podle těchto Antidopingových pravidel určené atletu, členu jeho zabezpečovacího týmu nebo jiné osobě spadající pod pravomoc člena, může být učiněno doručením takového oznámení příslušnému členu. Tento člen zodpovídá za okamžité navázání kontaktu s osobou, které je oznámení určeno.
\enditems

\secc Organizace antidopingových orgánů v rámci IAAF

\begitems \style N
* Činnosti v rámci těchto Antidopingových pravidel zajišťují v rámci IAAF následující osoby nebo orgány
  \begitems \style a
  * Rada;
  * Lékařská a antidopingová komise;
  * Dopingová revizní komise;
  * Správce IAAF pro záležitosti dopingu.
  \enditems

Rada
* Je povinností Rady v zastoupení Kongresu kontrolovat a dohlížet na činnosti IAAF v souladu s jejími cíli (viz článek 6.12.a) Statutu IAAF). Jedním z těchto cílů je podpora fair play ve sportu, zejména snaha o vedoucí roli v boji proti dopingu, jak v atletice, tak ve sportu v širším měřítku a vytvářet a udržovat programy zjišťování, odstrašování a výchovy, zaměřené na vymýcení dopingu ve sportu (viz článek 3.8 Statutu IAAF).
* Podle Statutu IAAF má Rada následující pravomoci pro kontrolu a dohled nad činností IAAF:
  \begitems \style a
  * ustanovovat komise či subkomise, podle okamžité potřeby nebo s dlouhodobou účinností, které považuje za nutné pro správnou činnost IAAF (viz článek 6.11.j) Statutu IAAF).
  * vydávat jakékoliv prozatímní změny Pravidel, které považuje za nezbytné v údobí mezi Kongresy a určovat data, od nichž takové změny nabudou účinnosti. Prozatímní změny musí být uvedeny na pořad jednání nejbližšímu Kongresu, který rozhodne, zda  tyto změny budou platit i nadále (viz článek 6.11.c) Statutu IAAF).
  * schvalovat, odmítat nebo měnit Procedurální směrnice (viz článek 6.11.i) Statutu IAAF) a
  * pozastavit členství nebo uvalit jiné sankce na člena pro porušení Pravidel v souladu s ustanovením článku 14.7 (viz článek 6.11.b) Statutu IAAF).
  \enditems

Lékařská a antidopingová komise
* Lékařská a antidopingová komise je ustanovena jako komise Rady dle Článku 6.11.j) Statutu IAAF. Jejím účelem je poskytovat IAAF všeobecné rady ve všech záležitostech týkajících se dopingu a obdobných záležitostí, včetně problémů ve vztahu k těmto Antidopingovým pravidlům a  Procedurálním směrnicím.
* Lékařská a antidopingová komise sestává až z 15 členů, kteří se musí sejít alespoň jednou v roce, běžně na konci každého kalendářního roku s cílem revize antidopingových aktivit IAAF během předcházejících 12 měsíců a stanovit antidopingový program IAAF na následující rok. Tento program podléhá schválení Rady. Lékařská a antidopingová komise také podle potřeby v průběhu roku působí jako poradní orgán.
* V rámci Antidopingových pravidel Lékařská a antidopingová komise odpovídá za plnění následujících úkolů :
  \begitems \style a
  * vydávání Procedurálních směrnic a jejich změn tak často, jak je třeba. Procedurální směrnice obsahují, přímo či v odkazech, následující dokumenty vydané WADA :
    \begitems \style i
    * Seznam zakázaných látek a postupů;
    * Mezinárodní standard testování;
    * Mezinárodní standard pro laboratoře;
    * Mezinárodní standard výjimek pro terapeutické použití a
    * Mezinárodní standard pro ochranu osobnosti a osobních údajů
    spolu s jakýmikoliv dodatky nebo úpravami  těchto dokumentů nebo další postupy a směrnice, které mohou být považovány za nezbytné k doplnění Antidopingových pravidel či k uskutečňování antidopingového programu IAAF
    Procedurální směrnice a jejich jakékoliv změny nebo úpravy, pokud v Antidopingových pravidlech není jinak stanoveno, musí být schváleny Radou. S jejich schválením Rada musí určit datum, od kterého Procedurální směrnice nebo navrhované změny, vstoupí v platnost. Kancelář IAAF musí o tomto datu uvědomit všechny členy a Procedurální směrnice a k nim navrhované změny zveřejnit na webových stránkách IAAF.
    \enditems
  * poskytování odborné pomoci Radě ohledně změn Antidopingových pravidel podle okamžité potřeby. Jakékoliv nezbytné změny Antidopingových pravidel v údobí mezi Kongresy musí být schválené Radou a v souladu s článkem 6.11.c) Statutu IAAF sděleny členům.
  * plánování, zavádění a sledování antidopingových informací a antidopingových výchovných programů. Tyto programy mají poskytovat nejnovější a přesné informace týkající se :
    \begitems \style i
    * seznamu zakázaných látek a zakázaných postupů;
    * zdravotních důsledků dopingu;
    * postupů při dopingových kontrolách;
    * právech a odpovědnosti atleta.
    \enditems
  * rozhodování o udělení výjimky pro terapeutické použití podle P 34.5 níže.
  * stanovování všeobecných směrnic pro zařazování do seznamu atletů IAAF podléhajících kontrolám.
  \enditems

Při vykonávání svých výše uvedených povinností Lékařská a antidopingová komise může dle potřeby povolat další odborníky pro získání lékařských nebo vědeckých poznatků.
* O svých aktivitách musí Lékařská a antidopingová komise podávat zprávu Radě před každým jejím jednáním. Ve všech případech týkajících se dopingu a podobných záležitostí komunikuje s kanceláří IAAF prostřednictvím jejího Lékařského a antidopingového oddělení.

Dopingová revizní komise
* Dopingová revizní komise, ustanovená podle článku 6.11.j) Statutu IAAF, je subkomisí Rady a má následující úkoly :
  \begitems \style a
  * rozhoduje, zda podle P 38.9 mají být předány CAS k arbitráži případy, kdy příslušný člen opominul ve stanovené lhůtě 3 měsíců provést slyšení atleta či jiné osoby;
  * v zastoupení Rady zjišťuje, zda v daném případě nastaly zvláštní/mimořádné okolnosti ve smyslu P 40.4 a P 40.5) v případech, které jsou ji předány ve smyslu P 38.16.
  * rozhoduje, zda daný případ má být předložen a podroben arbitráži CAS podle P 42.15 a zda v takovém případě má být, do doby než CAS rozhodne, atletu znovu zastavena činnost;
  * rozhoduje, zda se IAAF má zúčastnit případů řešených CAS, kde není původním účastníkem podle P 42.19 a zda v těchto případech má být, do doby než CAS rozhodne, atletu znovu zastavena činnost;
  * rozhoduje ve všech případech o prodloužení doby pro podání odvolání IAAF k CAS ve smyslu P 42.13 a
  * rozhoduje v případech, které jsou jí předloženy ve smyslu P 45.4, zda výsledky kontrol provedených jinou sportovní organizací než IAAF, podle pravidel a postupů, které jsou odlišné od příslušných řádů IAAF, mají být IAAF uznány.
  \enditems

Při výkonu svých povinností může Dopingová revizní komise požádat Lékařskou a antidopingovou komisi nebo Radu o názor nebo pomoc při řešení daného případu nebo na jakýkoliv problém všeobecného charakteru, který může vzniknout.
* Dopingová revizní komise sestává ze tří členů, z nichž jeden musí mít právní kvalifikaci. Prezident může podle okamžité potřeby jmenovat dalšího člena nebo členy této komise.
* Osvých aktivitách musí Dopingová revizní komise podávat zprávu Radě před každým jejím jednáním.

Správce IAAF pro záležitosti dopingu
* Správce IAAF pro záležitosti dopingu je vedoucím Lékařského a antidopingového oddělení IAAF. Je zodpovědný za realizaci antidopingového programu vypracovaného Lékařskou a antidopingovou komisí podle P 31.5. Jednou ročně podává v tomto smyslu zprávu Lékařské a antidopingové komisi pro její výroční schůzi nebo častěji, je-li k tomu vyzván.
* Správce IAAF pro záležitosti dopingu odpovídá za stálý dohled nad dopingovými případy vzniklými v souvislosti s těmito antidopingovými pravidly. Je zejména odpovědný za správu výsledků kontrol v souladu s P 37 a za rozhodnutí o prozatímním zastavení činnosti podle P 38 a vedení záznamů o  oznámeních zanedbaných nebo testech zmeškaných atlety mezinárodní úrovně v souladu s postupy stanovenými Anti-dopingovými předpisy.
* Správce IAAF pro záležitosti dopingu si pro svoji činnost může kdykoliv vyžádat stanovisko předsedy Lékařské a antidopingové komise, Dopingové revizní komise nebo dalších osob podle svého uvážení.
\enditems

\secc Porušení Antidopingových pravidel

\begitems \style N
* Dopingem je jedno nebo vícenásobné porušení pravidel o dopingu ve smyslu ustanovení P 32.2 těchto antidopingových pravidel.
* Atleti a ostatní osoby jsou zodpovědné za znalosti toho, co představuje porušení pravidel o dopingu a znalost látek a metod, které jsou na seznamu zakázaných látek a postupů. Za porušení pravidel o dopingu se považuje :
  \begitems \style a
  * Přítomnost zakázané látky či jejích metabolitů nebo příznaků ve vzorku odebraného atletu.
    \begitems \style i
    * Je osobní povinností každého atleta dbát, aby se do jeho tělesných tkání a tekutin nedostaly zakázané látky. Atleti odpovídají za jakoukoliv zakázanou látku, jejíž přítomnost byla v jejich tělech zjištěna. K průkazu porušení Antidopingového pravidla dle P 32.2.a) není nutné dokázat úmysl, chybu, nedbalost nebo vědomé užití ze strany atleta.
    * Dostatečným důkazem porušení pravidel o dopingu podle P 32.2 je zjištění alespoň jedné z následujících skutečností: přítomnost zakázané látky či jejích metabolitů nebo příznaků ve vzorku A, pokud atlet odmítne analýzu vzorku B; nebo pokud byl vzorek B analyzován, přítomnost zakázané látky či jejích metabolitů nebo příznaků ve vzorku B.
    * Vyjma těch zakázaných látek, u nichž je horní prahové množství (jehož překročení je porušením pravidel) výslovně uvedeno v Seznamu zakázaných prostředků, je zjištěná přítomnost jakéhokoliv množství zakázané látky ve vzorku atleta porušením Antidopingového pravidla.
    * Seznam zakázaných prostředků může, jako výjimku ze všeobecných ustanovení podle P 32.2.a), obsahovat zvláštní kritéria pro hodnocení zakázaných látek, které mohou být produkované vnitřními pochody v lidském těle.
    \enditems
  * Užití nebo pokus o užití zakázané látky nebo zakázaného postupu;
    \begitems \style i
    * úspěch nebo neúspěch při užití zakázané látky nebo zakázaného postupu je nepodstatný. Zcela dostačuje použití zakázaného prostředku nebo pokus o jeho použití, aby došlo k porušení Antidopingových pravidel;
    * doznání se k použití či pokusu o použití zakázaného prostředku může být učiněno ústně, ověřitelným způsobem, nebo písemně. Nicméně prohlášení není přípustné, pokud je učiněno po osmi letech od události, k níž se vztahuje.
    \enditems
  * Odmítnutí nebo nedostavení se, bez závažného důvodu, k dopingové kontrole na vyzvání odpovědného činovníka nebo jakákoliv snaha vyhnout se dopingové kontrole.
  * Porušení příslušných požadavků, které se týkají podrobení se testu v mimosoutěžním období, vč. neposkytnutí údajů o pobytu a zmeškání testu ve smyslu znění pravidel odpovídajících mezinárodnímu standartu pro testování. Jakákoliv kombinace tří zmeškaných testů nebo neposkytnutí údajů o pobytu v průběhu osmnácti měsíců určených IAAF nebo jinou anti-dopingovou organizací, pod jejíž pravomoc atlet spadá, musí být považováno za poručení pravidel o dopingu.
  * Ovlivnění nebo pokus o ovlivnění kterékoliv části procesu dopingové kontroly
  * Vlastnictví zakázané látky nebo zakázaného postupu;
    \begitems \style i
    * vlastnictvím atleta se rozumí vlastnictví mimosoutěžně zakázané látky nebo postupu v kterékoliv době nebo místě, pokud atlet neprokáže, že toto vlastnictví je na základě výjimky udělené pro terapeutické použití ve smyslu P 34.5 nebo se jinak přijatelně neospravedlní;
    * vlastnictvím členy zabezpečovacího týmu atleta se rozumí vlastnictví mimosoutěžně zakázané látky nebo postupu ve spojitosti s atletem, soutěží či tréninkem, pokud člen zabezpečovacího týmu atleta neprokáže, že toto vlastnictví je na základě výjimky udělené pro terapeutické použití ve smyslu P 34.9, nebo se jinak přijatelně neospravedlní.
    \enditems
  * Nelegální rozšiřování jakékoliv zakázané látky či zakázaného postupu.
  * Podání či pokus o podání zakázané látky nebo zakázaného postupu atletu během soutěže, podání či pokus o podání zakázané látky nebo zakázaného postupu atletu v mimosoutěžním údobí, či asistence, povzbuzování, napomáhání, navádění, zakrývání nebo účast na jakékoliv spoluúčasti při porušení Antidopingových pravidel nebo pokusu o jejich porušení.
  \enditems
\enditems

\secc Měřítka posuzování důkazů o dopingu

\begitems \style N
Tíha důkazů a jejich měřítko.
* IAAF, členská federace nebo jiný prošetřující orgán nesou tíhu důkazů, že došlo k porušení Antidopingových pravidel ve smyslu těchto pravidel.

Měřítkem důkazů o porušení Antidopingových pravidel, které jsou předloženy IAAF, členskou federací či jiným prošetřujícím orgánem, musí být jejich přijatelnost pro projednávající orgán s přihlédnutím k závažnosti učiněného obvinění. Měřítkem důkazu musí být více než pouhá pravděpodobnost skutku, ale méně než důkaz nad rámec oprávněných pochybností.

* Pokud podle těchto Antidopingových pravidel leží na straně atleta, atletického doprovodu nebo jiných osob obviněných z porušení Antidopingových pravidel tíha důkazů pro vyvrácení tohoto obvinění, nebo je na nich, aby doložili určitá fakta nebo skutečnosti, je měřítkem důkazů míra pravděpodobnosti, vyjma ustanovení P 40.4 (specifikovaná látka) a P 40.6 (přitěžující okolnosti¨, kdy atlet musí podat více než uspokojující důkazy.

Způsoby doložení skutečností a domněnek
* Fakta spojená s porušením Antidopingových pravidel mohou být doložena jakýmikoliv spolehlivými prostředky, mj. doznání, svědectví třetích osob, prohlášení svědka, zpráva experta, průkazné dokumenty, závěry učiněné na základě dlouhodobých záznamů jako je biologický pas a jiné analytické informace
V dopingových případech lze pro důkazy uplatnit následující měřítka :
  \begitems \style a
  * Laboratoře akreditované u WADA jsou považovány za pracoviště, kde analýzy vzorků a manipulace s nimi probíhají v souladu s Mezinárodními standardy pro laboratoře. Atlet nebo jiná osoba může vznesená obvinění vyvrátit, pokud prokáže, že došlo k odchylkám od těchto standardů. Atlet nebo jiná osoba může obvinění odmítnout poukazem na skutečnost, že došlo k odchylce od těchto standardů, která odůvodněně mohla způsobit nepříznivý analytický nález.
    Pokud atlet nebo jiná osoba odmítne obvinění poukazem na vzniklou odchylku od těchto standardů, která odůvodněně mohla způsobit nepříznivý analytický nález, IAAF, členská federace nebo jiný prošetřující orgán nese tíhu důkazu, že taková odchylka nezpůsobila nepříznivý analytický nález.
  * Odchylky od jakéhokoliv mezinárodního standardu nebo jiného anti-dopingového pravidla nebo zásad, které nezpůsobily nepříznivý analytický nález nebo jiné porušení anti-dopingových pravidel, nemohou výsledky zneplatnit. Pokud atlet nebo jiná osoba prokáží, že odchylka od jiného mezinárodního standardu nebo jiného anti-dopingového pravidla nebo zásad ke které došlo, mohla odůvodněně způsobi nepříznivý analytický nález nebi jiné porušení anti-dopingových pravidel, pak IAAF, členská federace nebo jiný prošetřující orgán nese tíhu důkazu, že taková odchylka nezpůsobila nepříznivý analytický nález, nebo poručení anti-dopingového pravidla
  * Skutečnosti vyplývající z rozhodnutí soudu nebo profesionálního disciplinárního tribunálu s příslušnou pravomoci, které není předmětem odvolání, jsou nezvratným důkazem proti atletu nebo jiné osobě, jichž se týká, pokud tento atlet nebo jiná osoba neprokáže, že takové rozhodnutí porušuje zásady přirozené spravedlnosti.
  * Komise řídící slyšení, jehož předmětem je porušení pravidel o dopingu, může dojít k závěru který je pro atleta nebo jinou osobu obviněnou z porušení pravidel o dopingu nepříznivý, na základě skutečnosti, že tento atlet nebo jiná osoba, poté, co byla v přiměřeném předstihu pozvání na slyšení, se odmítla slyšení zúčastnit (buď osobně telefonicky, podle rozhodnutí komise) a zodpovědět otázky komise IAAF, členské federace nebo jiného orgánu vyšetřujícího porušování pravidel o dopingu.
  \enditems
\enditems

\secc Seznam zakázaných prostředků

\begitems \style N
* Součástí těchto Antidopingových pravidel je Seznam zakázaných prostředků, vydávaný a revidovaný WADA.

Zveřejnění a revize seznamu zakázaných prostředků
* IAAF zpřístupní Seznam zakázaných prostředků a zveřejní jej na svých webových stránkách. Každý člen musí platný Seznam zakázaných prostředků zpřístupnit (na svých webových stránkách nebo jinak) všem atletům a dalším osobám, které spadají pod jeho pravomoc.
* Pokud není v Seznamu zakázaných prostředků a/nebo v jakékoliv jeho revidující listině stanoveno jinak, podle těchto Antidopingových pravidel vstupuje Seznam zakázaných prostředků a jeho revize v platnost tři (3) měsíce poté, co WADA takový seznam zveřejní, aniž by bylo třeba dalšího kroku ze strany IAAF.

Seznam zakázaných látek a zakázaných postupů
* Zakázané látky a zakázané postupy: Seznam zakázaných látek a zakázaných postupů obsahuje ty látky a metody, které jsou zakázané, které jsou zakázané vždy (jak v soutěžním, tak v mimosoutěžním období), neboť jsou schopné zvyšovat výkonnost v budoucích soutěžích nebo zakrýt užití dopingu, a dále ty látky a metody, které jsou zakázané pouze během soutěží. Zakázané látky a zakázané postupy mohou být na seznamu uvedené pouze jako obecná kategorie (např. anabolika) nebo s odkazem na určitou látku nebo metodu.
* Látky specifikované: Pro účely uplatnění P 40 (sankce proti jednotlivým osobám) všechny zakázané látky jsou uvedené jako látky blíže specifikované, vyjma látek třídy anabolik a hormonů a modulátorů takto na seznamu uvedených. Zakázané metody nejsou specifikovanými látkami.
* Nové třídy zakázaných látek: V případě, že WADA rozšíří seznam zakázaných látek o novou třídu takových látek, výkonný výbor WADA stanoví, zda některé nebo všechny látky nové třídy spadají pod označení specifikované látky ve smyslu P 34.5.
* Rozhodnutí WADA o položkách zařazených na seznam zakázaných látek a zakázaných postupů a jejich klasifikace do tříd tohoto seznamu je konečné a nemůže být předmětem námitek atleta nebo jiných osob založených na tvrzení, že látka nebo metoda nebyla maskovacím činidlem nebo neměla schopnost zvýšit výkonnost, nepředstavovala zdravotní riziko nebo porušení ducha sportu.

Terapeutické užití
* WADA přijala mezinárodní standard pro uplatňování výjimek pro terapeutické použití zakázané látky nebo postupu („TUE“).
* Atleti s lékařsky stanovenou diagnózou, vyžadující podávání zakázané látky nebo použití zakázaného postupu, musí nejdříve požádat o vydání TUE. TUE však bude vystaveno pouze v případech jasné a nezbytné klinické potřeby, kdy atlet nemůže získat žádnou výhodu při soutěži.
  \begitems \style a
  * Atleti mezinárodní výkonnosti musí získat TUE od IAAF ještě před účastí na mezinárodní soutěži, a to bez ohledu na skutečnost, že již obdrželi TUE na národní úrovni. IAAF zveřejní seznam mezinárodních soutěží, pro něž je požadovaná TUE udělená IAAF. Atleti mezinárodní úrovně musí o TUE požádat Lékařskou a antidopingovou komisi. Podrobnosti takového kroku jsou popsány v Procedurální směrnici. O udělení TUE v rámci ustanovení tohoto pravidla bude IAAF informovat  národní federaci atleta a WADA (přes ADAMS nebo jinak).
  * Ostatní atleti musí TUE získat od své národní federace nebo od tohoto orgánu, který byl udělováním TUE pověřen touto federací nebo, který je jinak oprávněn udělovat TUE v dané zemi nebo v oblasti působnosti této národní federace. Národní federace je povinna každé udělení TUE okamžitě oznámit IAAF a WADA. (přes ADAMS nebo jinak).
  * WADA může, o své iniciativě, kdykoliv přezkoumat TUE udělená atletům mezinárodní úrovně i ostatním atletům, kteří jsou na listině národních federací. Dále, na žádost kteréhokoliv takového atleta, jemuž byla TUE odepřena, WADA může toto odmítnutí přezkoumat. Pokud WADA zjistí, že udělení nebo odmítnutí TUE odporuje mezinárodnímu standardu pro TUE, může takové rozhodnutí změnit.
  * Přítomnost zakázané látky či jejích metabolitů nebo jejích příznaků (P32.2.a), užití nebo pokus o užití zakázané látky nebo zakázaného postupu (P 32.2.b), vlastnictví zakázané látky nebo zakázaného postupu (P 32.2.f) nebo podání zakázané látky nebo zakázaného postupu (P 32.2.h), k nimž došlo v souladu s mezinárodním standardem pro TUE není považováno za porušení pravidel o dopingu.
  \enditems
\enditems

\secc Dopingové kontroly

\begitems \style N
* Podle těchto Antidopingových pravidel může být každý atlet podroben kontrole během soutěže, které se účastní nebo kdykoliv a kdekoliv mimosoutěžnímu testování. Atleti se musí podrobit dopingové kontrole kdykoliv jsou k ní vyzváni odpovědným činovníkem.
* Povinností každého člena IAAF (a jednotlivých oblastních asociací) je zařadit do vlastního statutu následující ustanovení:
  \begitems \style a
  * oprávnění provádět soutěžní a mimosoutěžní dopingové kontroly. Zprávu o kontrolách musí každý člen jednou ročně předkládat IAAF (viz P 43.5);
  * oprávnění IAAF provádět dopingové kontroly na národním mistrovství člena (nebo příslušné oblastní asociace);
  * oprávnění IAAF provádět neohlášené mimosoutěžní dopingové kontroly atletů členské federace;
  * podmínku členství nebo přidružení k národní federaci a podmínku účasti v soutěžích konaných či organizovaných v rámci národní federace, spočívající v souhlasu atleta podrobit se jakékoliv soutěžní nebo mimosoutěžní kontrole prováděné členem IAAF nebo kterýmkoliv orgánem, oprávněným ke kontrolám podle těchto Antidopingových pravidel.
  \enditems
* IAAF a její členové mohou delegovat dopingové kontroly podle tohoto pravidla na kteréhokoliv člena, WADA, vládní agenturu, národní antidopingovou organizaci nebo jinou třetí stranu, kterou považují za dostatečně kvalifikovanou k tomuto účelu.
* Kromě kontrol prováděných IAAF a jejími členy (a orgány, na něž IAAF nebo její členové mohou přenést svoji odpovědnost za testování ve smyslu ustanovení P 35.3), mohou být atleti podrobeni kontrolám
  \begitems \style a
  * během soutěží, kdy testy provádí organizace nebo orgán oprávněný provádět kontroly během dané soutěže, které se účastní;
  * mimosoutěžním, prováděným buď (i) WADA nebo (ii) národní antidopingovou organizací země nebo teritoria, kde se nacházejí nebo (iii) MOV či jeho jménem, během Olympijských her.
  \enditems
Nicméně, vždy jen jedna organizace může být odpovědná za iniciování a řízení dopingových kontrol během soutěže. Při mezinárodních soutěžích je odběr vzorků zahájen a řízen IAAF (viz P 35.7) nebo jiným řídícím orgánem mezinárodní sportovní organizace v případě mezinárodních soutěží, kdy IAAF není výhradním řídícím orgánem (např. MOV při Olympijských hrách nebo Federace her britského společenství národů při Hrách britského společenství). Rozhodne-li se IAAF nebo jiný řídící orgán mezinárodní sportovní organizace neprovádět dopingové kontroly při určité mezinárodní soutěži, národní antidopingová organizace země nebo teritoria, kde se tato mezinárodní soutěž koná, může se souhlasem IAAF a WADA zahájit a řídit takové dopingové kontroly.
* IAAF a její členové musí okamžitě hlásit všechny dopingové kontroly provedené během soutěží útvaru pro sbírání, třídění a distribuci informací při WADA (WADA clearing house), přičemž členská federace musí kopii své zprávy zaslat IAAF, aby se zamezilo nežádoucímu dvojímu testování.
* Dopingové kontroly prováděné IAAF a jejími členy ve smyslu tohoto pravidla musí v podstatě odpovídat Mezinárodnímu standardu pro testování (a dalším aplikovatelným ustanovením Procedurálních směrnic), které platí v době prováděných kontrol.

Kontroly během soutěže
* IAAF odpovídá za zahájení a řízení dopingových kontrol během těchto mezinárodních soutěží:
  \begitems \style a
  * Mistrovství světa;
  * Soutěže světové atletické série;
  * Mezinárodní zvací mítinky podle P 1.1;
  * mítinky IAAF;
  * Silniční závody IAAF (vč. maratonů IAAF);
  * V dalších mezinárodních soutěžích dle rozhodnutí Rady IAAF na doporučení Lékařské a anti-dopingové komise. Úplný seznam těchto mítinků bude každoročně uveden na webových stránkách IAAF.
  \enditems
* Pro výše uvedené soutěže určí Rada na doporučení Lékařské a antidopingové komise vhodný počet atletů, kteří budou vyzvání k dopingové kontrole. Výběr atletů pro dopingovou kontrolu musí být prováděn následovně:
  \begitems \style a
  * na základě finálového umístění a/nebo nahodile;
  * z rozhodnutí IAAF (prostřednictvím odpovídajícího orgánu nebo činovníka), jakýmkoliv způsobem, který si určí, vč. cíleného výběru;
  * kdokoliv, kdo překoná či vyrovná světový a/nebo oblastní rekord (viz P 260.6 a P 260.8)
  \enditems
* Pokud IAAF provádění dopingových kontrol delegovala na jiný orgán ve smyslu P 35.3, může na příslušnou mezinárodní soutěž jmenovat svého představitele, který zajistí řádné uplatňování těchto Antidopingových pravidel a Procedurálních směrnic.
* Po konzultaci s příslušnou členskou organizací (resp. s příslušnou oblastní asociací) IAAF může provádět nebo asistovat při dopingových kontrolách při národním mistrovství tohoto člena nebo mistrovství oblastní asociace.
* Ve všech ostatních případech (vyjma případu, kdy jsou dopingové kontroly prováděny podle předpisů řídícího orgánu jiné sportovní organizace, např. MOV při OH) je za zahájení a řízení dopingových kontrol zodpovědná členská federace, která kontroly provádí nebo na jejímž území se soutěž koná. Pokud členská federace podle P 35.3 delegovala zodpovědnost za dopingové kontroly, pak tato federace zodpovídá, že tyto dopingové kontroly na daném území odpovídají těmto Antidopingovým pravidlům a směrnicím.

Mimosoutěžní kontroly
* IAAF zaměří své mimosoutěžní kontroly na atlety mezinárodní výkonnosti. Nicméně může, podle svého rozhodnutí, kdykoliv provádět mimosoutěžní kontroly kteréhokoliv atleta. Vyjma výjimečných okolností, bude kontrola v mimosoutěžním období provedena bez oznámení atletovi nebo jeho doprovodu nebo národní federaci. Atleti uvedení na Seznam atletů podléhajících testování musí ve smyslu P 35.17 podávat zprávy o svém pobytu.
* Je povinností každé členské federace, činovníka členské federace a ostatních osob pod pravomocí členské federace napomáhat IAAF, případně jiné členské federaci, WADA a dalším orgánům oprávněným provádět dopingové kontroly, v provádění mimosoutěžních kontrol podle tohoto pravidla. Každá členská federace, činovník členské federace nebo jiná osoba pod pravomocí členské federace, která by bránila v provádění, znemožňovala, kladla jakékoliv překážky nebo ovlivňovala kontroly, vystavuje se nebezpečí postihu podle těchto Antidopingových pravidel.
* Mimosoutěžní kontroly podle těchto Antidopingových pravidel musí být prováděny na látky a postupy uvedené na Seznamu zakázaných látek a prostředků jako látky a postupy zakázané jak během soutěží tak v mimosoutěžním období nebo za účelem získání dat pro biologický pas atleta nebo z obou důvodů současně.
* Přehled mimosoutěžních kontrol provedených u jednotlivých atletů a jednotlivých federací bude zveřejňován jednou ročně.

Informace o pobytu
* IAAF vytvoří Seznam atletů podléhajících testování. Tito atleti jsou povinni řídit se předpisy pro sdělování svého pobytu uvedenými v těchto pravidlech a v antidopingových předpisech. IAAF tento Seznam zveřejní na svých webových stránkách a bude jej čas od času inovovat.
* Atleti, uvedení na Seznamu osob podléhajících kontrolám, povinni předkládat písemné informace o místě svého pobytu ve smyslu antidopingových předpisů. Odpovědnost za tyto informace je plně na každém jednotlivém atletu. Každá národní federace je povinna na vyžádání IAAF nebo příslušného zkušebního subjektu, poskytnout IAAF součinnost při získávání informací o pobytu atleta a musí mít pro tyto účely příslušná ustanovení ve svých předpisech. Informace poskytnuté atletem, ve smyslu tohoto pravidla budou tam, kde je to vhodné, poskytnuty WADA a kterémukoliv dalšímu orgánu oprávněnému provádět kontroly atletů, ovšem za jednoznačné podmínky, že takové informace budou použity výhradně pro účely dopingových kontrol.
* Pokud atlet, který je na seznamu atletů podléhajících testování, nedodrží povinnost sdělit IAAF místa svého pobytu, bude to považováno za zanedbané oznámení ve smyslu P 32.2.d), pokud jinak byly dodrženy příslušné podmínky Anti-dopingových pravidel. Pokud atlet, který je na Seznam atletů podléhajících testování není k dispozici k provedení kontroly v místě, kde se podle svého oznámení má zdržovat bude to považováno za zmeškaný test ve smyslu P 32.2.d), pokud jinak byly dodrženy příslušné podmínky Anti-dopingových pravidel. Bude považováno za porušení pravidel o dopingu ve smyslu ustanovení 32.2.d), pokud se atlet třikrát dopustí zanedbaného podání (což může být jakákoliv kombinace třikrát zanedbaného podání a/nebo zmeškaného testu) v průběhu 18 měsíců. Pro účely ustanovení P 32.2.d), IAAF může použít sdělení o zanedbaném oznámení a/nebo zmeškaného testu poskytnuté jinou Antidopingovou organizací, pod jejíž pravomoc atlet spadá, pokud byly jinak dodržena pravidla vyhovující mezinárodnímu standardu pro testování.
* Pokud atlet, který je na seznam atletů podléhajících testování nebo člen jeho zabezpečovacího týmu vědomě poskytne nepřesné nebo zavádějící informace o svém pobytu, bude to považováno za obejití povinnosti poskytnout vzorek a porušení P 32.2.c) nebo ovlivňování nebo pokus o ovlivňování průběhu dopingové kontroly a porušení P 32.2.e). Pokud byl člen požádán o spolupráci s IAAF při získávání informací o pobytu atleta ve smyslu P 35.17 nebo jinak souhlasil s předložením informací o pobytu svého atleta a neověřil si, zda předané informace jsou platné a přesné, bude to porušením ustanovení P 44.2.e)

Návrat k soutěžím po zanechání činnosti nebo jiném období nečinnosti
* Pokud atlet, který je povinen poskytovat údaje o svém pobytu, si již nepřeje podrobovat se mimosoutěžním kontrolám proto, že zanechal závodní činnosti nebo se rozhodl dále nesoutěžit z jiných důvodů, je povinen o tom informovat IAAF předepsanou formou. Tento atlet se může vrátit k závodní činnosti pouze tehdy, pokud o svém úmyslu předepsanou formou uvědomí IAAF v předstihu 12 měsíců a dá se v tomto údobí k dispozici pro mimosoutěžní kontroly zasláním zprávy o svém pobytu v souladu s P 35.17. Atlet, který odmítne se dostavit nebo se nedostaví k dopingové kontrole s odůvodněním, že zanechal závodní činnosti nebo že nadále nezávodí z jiného důvodu, ale neuvědomil o tom IAAF podle tohoto pravidla, poruší tím pravidla o dopingu ve smyslu P 32.2.c).
\enditems

\secc Analýza vzorků

\begitems \style N
* Všechny vzorky odebrané podle těchto Antidopingových pravidel musí být analyzovány v souladu s těmito všeobecnými principy:
  \begitems \style a
  Využití akreditované laboratoře
  * Pro účely P 32.2.a) (Přítomnost zakázané látky nebo použití zakázaného postupu) musí být vzorky zaslány k analýze pouze laboratoři s WADA akreditací nebo jinak schválené WADA. Pokud se jedná o kontroly prováděné IAAF podle P 35.7, mohou být vzorky zaslané pouze laboratořím s WADA akreditací (nebo pokud je to vhodné, hematologickým laboratořím nebo mobilním testovacím jednotkám), které jsou schválené IAAF.

  Zaměření analýzy
  * Vzorky musí být analyzovány na zakázané látky a zakázané postupy uvedené na Seznamu zakázaných prostředků a na další látky, které určí WADA na základě svého monitorovacího programu. a/nebo ke stanovení profilu příslušných parametrů v moči krvi nebo tkáni atleta, včetně DNA nebo genomického profilu, pro antidopingové účely. Takto získané informace mohou být použity pro cílený test nebo jako podpora tvrzení o porušení pravidel o dopingu podle P 32.2 nebo obou.

  Rozbor vzorků
  * Bez písemného souhlasu atleta nesmí být žádný vzorek použit k jinému účelu, než jak je popsáno v P 36.1.b). Vzorky použité se souhlasem atleta jiným účelům, než uvedeným v P 36.1.b), musí být zbaveny jakýchkoliv identifikačních prvků, aby nebylo později možné zpětně zjistit, komu byly odebrány.

  Standardy pro analýzu vzorků a zprávu.
  * Laboratoře musí vzorky analyzovat a podat zprávu o výsledcích v souladu s mezinárodním standardem pro laboratoře. Shoda s mezinárodním standardem pro laboratoře (na rozdíl od jiných alternativních standardů nebo postupů) stačí, aby provedený postup byl považován za správný. Mezinárodní standard pro laboratoře musí obsahovat jakékoliv technické dokumenty, které byly vydány na základě tohoto standardu
  \enditems
* Všechny vzorky odebrané atletům při dopingových kontrolách, za něž odpovídá IAAF, se okamžitě stávají majetkem IAAF.
* Pokud v kterémkoliv stadiu kontroly vznikne jakýkoliv problém týkající se rozboru nebo interpretace výsledků, osoba zodpovědná za analýzu v dané laboratoři (nebo hematologické laboratoři nebo mobilní testovací jednotce) si může vyžádat konzultaci Správce IAAF pro záležitosti dopingu.
* Pokud v kterémkoliv stadiu kontroly vznikne jakýkoliv problém týkající se vzorku, laboratoř (nebo mobilní testovací jednotka) může provést jakýkoliv další test nebo testy pro vyjasnění vzniklého problému a podle těchto testů může IAAF rozhodnout, zda nepříznivý výsledek rozboru je zapříčiněn vzorkem.
* Vzorek odebraný podle P 36.2 může být z rozhodnutí IAAF nebo WADA (se souhlasem IAAF) kdykoliv znovu analyzován pro účely P 36.1.b). Všechny ostatní vzorky odebrané v atletice mohou být znovu analyzovány výhradně na příkaz zkušebního orgánu nebo IAAF (se souhlasem zkušebního orgánu) nebo WADA. Okolnosti a podmínky pro opakovaný test vzorků musí odpovídat požadavkům mezinárodního standardu pro laboratoře.
* Pokud rozbor odhalí přítomnost zakázané látky či použití zakázané látky nebo zakázaného postupu, akreditovaná laboratoř musí nepříznivý nález okamžitě písemně potvrdit, buď IAAF, pokud se jedná o test IAAF nebo příslušné členské federaci, v případě národního testu (s kopií pro IAAF). V případě národního testu musí členská federace informovat IAAF o nepříznivém výsledku testu, vč. jména atleta, bezpodmínečně do dvou týdnů po obdržení takové zprávy z laboratoře akreditované u WADA.
\enditems

\secc Správa výsledků

\begitems \style N
* Po obdržení oznámení o positivním nálezu dopingové kontroly či jiném porušení pravidel o dopingu ve smyslu těchto Antidopingových pravidel, bude případ řešen podle následujících procesních ustanovení.
* Pokud se jedná o atleta mezinárodní úrovně, bude správa výsledků řízena Správcem IAAF pro záležitosti dopingu, ve všech ostatních případech národního testu, pověřenou osobou nebo orgánem národní federace, pod jejíž pravomoc atlet nebo osoba spadá. Pověřená osoba nebo orgán dané národná federace musí Správce IAAF pro záležitosti dopingu průběžně informovat o průběhu případu. Při vyřizování každého případu je možno kdykoliv požádat Správce IAAF pro záležitosti dopingu o pomoc nebo informaci.

Pro účely tohoto pravidla a následného P 38 se každá zmínka o Správci IAAF pro záležitosti dopingu vztahuje, pokud je to namístě, též na pověřenou osobu nebo orgán členské federace (nebo orgán, na nějž členská federace delegovala odpovědnost za správu výsledků) a odkazy na atleta se týkají rovněž atletického doprovodu nebo jiných osob.

* Po obdržení oznámení o positivním výsledku dopingové kontroly Správce IAAF pro záležitosti dopingu zahájí šetření zda
  \begitems \style a
  * atletu bylo uděleno TUE vztahující se na zakázanou látku;
  * existuje zřejmá odchylka od Mezinárodního standardu pro testování nebo Mezinárodního standardu pro laboratoře, která by mohla ovlivnit platnost nálezu.
  \enditems
* Pokud je při průzkumu podle P 37.3 zjištěno, že nebylo uděleno TUE a neexistuje zřejmá odchylka od Mezinárodního standardu pro testování (či jiného použitelného ustanovení Procedurálních směrnic) nebo Mezinárodního standardu pro laboratoře, která by mohla ovlivnit platnost nálezu, Správce IAAF pro záležitosti dopingu musí atleta ihned informovat o
  \begitems \style a
  * pozitivním analytickém nálezu;
  * ustanovení antidopingových pravidel, které bylo porušeno;
  * časovém limitu, v němž atlet musí přímo nebo prostřednictvím své národní federace podat vysvětlení pozitivního nálezu;
  * jeho právu vyžádat si výsledek analýzy „B“ vzorku, případně se analýzy „B“ vzorku vzdát. Současně musí atletu sdělit, že v případě vyžádané analýzy „B“ vzorku nese veškeré náklady atlet. Pokud se však prokáže rozdíl mezi výsledky analýz „A“ a „B“ vzorků, nese náklady organizace odpovědná za zadání testu;
  * datu, kdy byl znám výsledek testu „B“ vzorku, pokud byl takový test atletem vyžádán. Toto datum má být sděleno do dvou týdnů poté, co byl atletu oznámen pozitivní výsledek první analýzy. Pokud příslušná laboratoř není schopna ve stanoveném termínu provést test „B“ vzorku, musí laboratoř provést tento test v nejbližším termínu pro ni možném. Pro takovouto změnu termínu testu „B“ vzorku není přípustný žádný jiný důvod;
  * možnosti zúčastnit se osobně nebo prostřednictvím zástupce, otevření „B“ vzorku a jeho analýzy, v případě, že si tuto analýzu vyžádá; a
  * právu atleta vyžádat si kopie celého rozboru obou vzorků včetně informací požadovaných Mezinárodním standardem pro laboratoře.
  \enditems

Správce IAAF pro záležitosti dopingu zašle příslušnému členu a WADA kopii výše zmíněné zprávy předané atletovi. Pokud se Správce IAAF pro záležitosti dopingu rozhodne, že nepříznivý analytický nález nepředstavuje porušení pravidel o dopingu, uvědomí o tom atleta, člena a WADA.

* Podle mezinárodních standardů, za určitých okolností, se požaduje, aby laboratoře oznámily přítomnost zakázaných látek, které mohou být produkovány endogenně, jako atypický nález k dalšímu zkoumání. Na základě atypického nálezu zjištěného u vzorku A, Správce IAAF pro záležitosti dopingu provede prvotní šetření zda (a) atypický nález odpovídá TUE, která byla udělena podle mezinárodního standardu pro TUE nebo (b) existuje zřejmá odchylka od antidopingových předpisů nebo Mezinárodního standardu pro laboratoře, která způsobila atypický nález. Pokud počáteční šetření neodhalí soulad s TUE nebo odchylku od antidopingových předpisů nebo Mezinárodního standardu pro laboratoře, která atypický nález způsobila, Správce IAAF pro záležitosti dopingu musí provést šetření požadované Mezinárodními standardy. Po ukončení šetření, je informována WADA, zda atypický nález je nebo není považován za nepříznivý analytický nález. Pokud je atypický nález považován za nepříznivý analytický nález, je o tom, v souladu s P 37.4, informován atlet. Správce IAAF pro záležitosti dopingu nevydá oznámení o atypickém nálezu, dokud nedokončí své šetření a nerozhodne o tom, zda IAAF zahájí další řízení o atypickém nálezu, pokud existuje jedna z následujících okolností:
  \begitems \style a
  * pokud Správce IAAF pro záležitosti dopingu určí, že vzorek B má být analyzován před uzavřením jeho šetření podle P 37.5, IAAF může provést analýzu vzorku B poté, co to oznámila atletovi, přičemž toto oznámení zahrnuje popis atypického nálezu a pokud to lze, informaci podle P 37.4. b) až g) výše.
  * pokud Správce IAAF pro záležitosti dopingu obdrží žádost, buď od organizátora velké soutěže krátce před jeho mezinárodními soutěžemi, nebo žádost od sportovní organizace, které má ve velmi krátkém termínu vybrat účastníky mezinárodní soutěže, zda u některého z atletů uvedených na seznamu účastníků velké soutěže nebo týmu sportovní organizace byl zjištěn a je v řízení atypický nález, Správce IAAF pro záležitosti dopingu jméno takového atleta sdělí žadateli poté, co o takovém nálezu informoval dotčeného atleta.
  \enditems
* Atlet může přijmout výsledek analýzy „A“ vzorku vzdáním se svého práva na analýzu „B“ vzorku. Nicméně si IAAF může kdykoliv vyžádat analýzu „B“ vzorku, pokud věří, že tato analýza bude pro posouzení daného případu závažná.
* Atlet nebo jeho zástupce musí mít možnost zúčastnit se celého procesu analýzy „B“ vzorku. Přítomen analýze může být rovněž zástupce národní federace atleta, stejně tak jako zástupce IAAF. Přesto, že si vyžádal analýzu „B“ vzorku, dočasné zastavení závodní činnosti atleta (viz P 38.2) zůstává v platnosti.
* Po ukončení analýzy „B“ vzorku musí být kompletní laboratorní zpráva s kopií příslušných údajů požadovaných Mezinárodním standardem pro laboratoře, zaslána Správci IAAF pro záležitosti dopingu. Pokud si to vyžádá, musí atlet dostat kopii této zprávy, vč. všech příslušných údajů.
* Po obdržení laboratorní zprávy o výsledku analýzy „B“ vzorku, Správce IAAF pro záležitosti dopingu musí provést jakékoliv dodatečné vyšetřování požadované podle Seznamu zakázaných prostředků. Po ukončení tohoto následného vyšetřování, musí Správce IAAF pro záležitosti dopingu okamžitě vyrozumět atleta o výsledku tohoto dodatečného vyšetřování a zda IAAF na tvrzení o porušení pravidel o dopingu trvá či netrvá.
* V případě jakéhokoliv porušení pravidel o dopingu kde není nepříznivý nález nebo atypický nález, Správce IAAF pro záležitosti dopingu musí provést jakákoliv šetření, která mohou být požadována příslušnými antidopingovými zásadami a pravidly přijatými na základě Kodexu nebo která jinak považuje za nutná a po ukončení takového šetření musí okamžitě uvědomit dotčeného atleta, zda je to považováno za porušení pravidel o dopingu. Pokud je tomu tak, atlet musí dostat příležitost buď přímo, nebo prostřednictvím své národní federace, aby během lhůty stanovené Správcem IAAF pro záležitosti dopingu, podal vyjádření k tomuto obvinění z porušení pravidel o dopingu.
* Osoby, které se účastní dopingové kontroly, musí učinit vše, aby do ukončení analýzy „B“ vzorku (nebo ukončení jakéhokoliv dodatečného vyšetřování k „B“ vzorku, které může být požadované dle Seznamu zakázaných prostředků podle P 37.9), zůstaly veškeré informace v tajnosti. Identita atletů, jejichž vzorky byly testovány s pozitivním výsledkem nebo kteří údajně porušili pravidla o dopingu, může být zveřejněna pouze poté, co atlet nebo jiná osoba obdrželi sdělení ve smyslu P 37.4 nebo P 37.10 a za normálních okolností ne dříve, než je na ně uvalen prozatímní zákaz činnosti podle ustanovení P 38.2 a P 38.3.
* Správce IAAF pro záležitosti dopingu kdykoliv vyzvat člena aby prošetřil možné porušení těchto antidopingových pravidel jedním či více atlety nebo jinou osobou spadající pod pravomoc člena (pokud možno ve spolupráci s národní antidopingovou organizací v zemi nebo teritoriu dotčeného člena a/nebo jiného příslušného národního orgánu nebo subjektu. Nevyhovění nebo odmítnutí výzvy člena provést takové šetření na žádost IAAF nebo podat zprávu a takovém šetření v průběhu přiměřené lhůty určené Správcem IAAF pro záležitosti dopingu může vést k uvalení sankcí na člena v souladu s P 44.
* Případy týkající se zřejmě zmeškaných testů nebo opomenutí podat informace o pobytu atletem, který je na seznamu testovaných atletů spravuje IAAF v souladu s postupy uvedenými v antidopingových předpisech. Případy týkající se zřejmě zmeškaných testů nebo opomenutí podat informace o pobytu atletem, který je na národním seznamu testovaných atletů, kde atlet měl být testován IAAF či jejím jménem, spravuje IAAF v souladu s postupy uvedenými v antidopingových předpisech. Případy týkající se zřejmě zmeškaných testů nebo opomenutí podat informace o pobytu atletem, který je na národním seznamu testovaných atletů, kde atlet měl být testován jinou antidopingovou organizací nebo jejím jménem, vede tato jiná antidopingová organizace v souladu s postupy Mezinárodního standardu pro testování.
* Správu výsledků v rámci programu biologického pasů atletů má na starosti IAAF v souladu s postupy danými těmito Antidopingovými pravidly.

Pokud podle těchto anti-dopingových předpisů IAAF řeší případ jako porušení anti-dopingových přepisů, správce IAAF pro záležitosti dopingu může atletu současně zastavit činnost až do rozhodnutí národní federace. Alternativně se atlet může dobrovolně rozhodnout svoji závodní činnost přerušit a své rozhodnutí písemně sdělit IAAF. Proti rozhodnutí o dočasném zastavení činnosti atleta není odvolání přípustné. Atlet, kterému byla dočasně zastavena činnost, nebo který se tak sám rozhodnul, má přesto plné právo na slyšení u své členské federace ve smyslu P 38.9.

* Správu výsledků testů provedených MOV nebo jinou sportovní organizací, která testuje sportovce během mezinárodních soutěží, které IAAF přímo neřídí (např. Hry Britského společenství národů nebo Panamerické hry) až do uvalení sankcí, kromě diskvalifikace atleta v dané soutěži, provádí IAAF v souladu s těmito anti-dopingovými pravidly.
\enditems

\secc Disciplinární řízení

\begitems \style N
* V případech, kdy bylo zjištěno porušení antidopingového pravidla podle těchto Antidopingových pravidel, dojde k disciplinárnímu řízení, které probíhá ve třech krocích:
  \begitems \style a
  * prozatímní zastavení činnosti,
  * slyšení,
  * sankce nebo zproštění obvinění.
  \enditems

Prozatímní zastavení činnosti
* Pokud ve lhůtě stanovené Správcem IAAF pro záležitosti dopingu podle P 37.4.c), atlet nebo jeho národní federace nepodá žádné, nebo dostatečné vysvětlení zjištěného porušení Antidopingového pravidla, musí být, na rozdíl od nepříznivého analytického nálezu specifikované látky, tomuto atletovi zastavena činnost, a to prozatímně, rozhodnutím jeho národní federace. V případě atleta mezinárodní úrovně, je atletu zastavena činnost Správcem IAAF pro záležitosti dopingu. Ve všech ostatních případech musí zákaz činnosti písemně oznámit atletu jeho národní federace. Případně se atlet může sám rozhodnout, že prozatímně zanechá závodní činnosti a své rozhodnutí sdělí písemně své národní federaci. V případě nepříznivého analytického nálezu specifikované látky nebo v případě kteréhokoliv porušení pravidel o dopingu jiného než nepříznivý analytický nález, Správce IAAF pro záležitosti dopingu může atletovi prozatímně zastavit činnost do vyřešení jeho případu národní federací. Prozatímní zastavení činnosti vstupuje v platnost dnem oznámení atletu v souladu s těmito antidopingovými pravidly.
* V kterémkoliv případě, kdy člen atletu prozatímně zastaví činnost nebo se atlet dobrovolně rozhodne svoji činnost prozatímně pozastavit, musí tuto skutečnost člen okamžitě oznámit IAAF a atlet je pak podroben dále uvedenému disciplinárnímu řízení. Dobrovolné prozatímní vzdání se činnosti nabývá účinnosti dnem, kdy IAAF obdrží od atleta písemné potvrzení této skutečnosti. Pokud v rozporu s výše uvedeným ustanovením, podle názoru Správce IAAF pro záležitosti dopingu, členská federace prozatímní zákaz činnosti řádně nenařídí, musí tak učinit Správce IAAF pro záležitosti dopingu sám a o této skutečnosti člena písemně informovat. Členská federace pak musí zahájit dále uvedené disciplinární řízení.
* Proti rozhodnutí o prozatímním zákazu činnosti se nelze odvolat. Atlet, jemuž byla prozatímně zastavena činnost nebo se tak dobrovolně rozhodl, má právo na urychlené slyšení u členské federace podle P 38.9.
* Pokud je nařízeno prozatímní zastavení činnosti (nebo dobrovolně přijato) na základě nepříznivého analytického nálezu u vzorku A následná analýza vzorku B (ať již na vyžádání IAAF nebo atleta) nepotvrdí výsledek analýzy A vzorku, pak atlet nesmí být postižen žádným dalším zastavením činnosti na základě porušení P 32.2.a) (přítomnost zakázané látky nebo jejích metabolitů nebo příznaků). Pokud byl atlet (nebo jeho tým) vyloučen ze soutěže pro porušení P 32.2.a) a následný vzorek B nepotvrdil nález u vzorku A, je možné, pokud by to jinak neovlivnilo soutěž, povolit návrat atleta nebo jeho týmu do soutěže a atlet nebo jeho tým mohou v soutěži pokračovat.
* Pokud atlet nebo jiná osoba ukončí činnost v době, kdy jsou zpracovávány výsledky jeho kontroly, organizace zodpovědná za tyto výsledky podle těchto antidopingových pravidel má stále pravomoc dokončit celý proces. Pokud atlet nebo jiná osoba ukončí činnost v době před zahájením zpracovávání výsledku jeho kontroly, organizace, která by podle těchto antidopingových pravidel měla pravomoc tyto výsledky zpracovávat v době, kdy atlet nebo jiná osoba pošila pravidla o dopingu, má pravomoc celé šetření provést.

Slyšení
* Každý atlet musí mít právo požádat o slyšení před příslušným tribunálem své národní federace dříve, než bude rozhodnuto o jakýchkoliv sankcích v souladu s těmito Antidopingovými pravidly. Pokud atlet obdržel souhlas s působením v zahraničí podle P 4.3, má právo vyžádat si slyšení před příslušným tribunálem své národní federace nebo před příslušným tribunálem federace, kde působí. V průběhu slyšení musí být dodrženy následující zásady: zahájení řízení v přiměřeném termínu, spravedlivé a nestranné obsazení komise řídící slyšení, právo na zastoupení poradcem na náklady atleta nebo jiné osoby, právo na včasnou a správnou informaci o vzneseném obvinění z porušení pravidel o dopingu, právo na vyjádření ke vznesenému obvinění z porušení pravidel o dopingu a z toho vyplývajících důsledků, právo každé strany na předložení důkazů, včetně práva na povolání a výpověď svědků (záleží na rozhodnutí komise, zda připustí svědeckou výpověď po telefonu nebo písemně), právo atleta nebo jiné osoby na tlumočníka při slyšení, přičemž je na komisi určit osobu a náklady na tlumočení, včasné písemné odůvodněné rozhodnutí s vysvětlením a důvody pro stanovený zákaz činnosti.
* Je-li atletu sděleno, že jeho vysvětlení nebylo přijato a podle P 38.2 je mu prozatímně zastavena činnost, musí mu být také sděleno jeho právo požádat o slyšení. Pokud atlet do 14 dní po obdržení takového sdělení písemně nepotvrdí své národní federaci nebo jinému příslušnému orgánu, že si přeje slyšení, má se za to, že se svého práva na slyšení vzdal a uznává, že se dopustil daného porušení pravidel o dopingu. Tuto skutečnost musí členská federace do 5 pracovních dní písemně potvrdit IAAF.
* Pokud je slyšení vyžádáno atletem, musí být svoláno bez odkladu a toto slyšení se musí uskutečnit do 3 měsíců od data, kdy členská federace obdrží žádost atleta. Člen musí IAAF průběžně informovat o všech případech čekajících na slyšení a všech datech, na něž byla slyšení nařízena. IAAF má právo zúčastnit se všech slyšení jako pozorovatel. Nicméně, účast IAAF na slyšení, nebo jakákoliv účast IAAF na případu, nemá vliv na právo IAAF podat proti rozhodnutí člena odvolání k CAS podle P 42. Pokud není slyšení svoláno ve lhůtě tří měsíců, IAAF může, v případě, kdy se jedná o atleta mezinárodní úrovně, rozhodnout o předání případu rozhodčímu-samosoudci určenému CAS. Případ bude projednán v souladu s pravidly CAS (pravidla týkající se odvolání, vyjma časových lhůt pro podání odvolání). Za průběh slyšení a jeho náklady odpovídá členská federace a proti rozhodnutí rozhodčího-samosoudce je možné se odvolat k CAS podle P 42. Opomenutí člena svolat slyšení ve lhůtě 3 měsíců podle tohoto pravidla může vyústit v uvalení sankcí podle P 44.
* Atlet si může zvolit, že se místo slyšení písemně přizná porušení pravidel o dopingu a přijme následná opatření navržená členem v souladu s ustanovením P 40. Pokud atlet přijme navržená opatřené své členské federace a slyšení se neuskuteční, musí člen svá opatření předložit IAAF a odůvodnit je. Proti opatřením člena následující poté, co atlet přijal důsledky vyplývající z porušení těchto pravidel o dopingu, je možné se odvolat v souladu s P 42.
* Slyšení se musí konat před orgánem ustanoveným nebo jinak odsouhlaseným členskou federací. Pokud člen pověří provedením slyšení nějaký orgán, výbor nebo tribunál (ať již jsou součástí jeho organizace nebo stojí mimo ni) nebo z jakéhokoliv důvodu je za slyšení atleta podle těchto pravidel odpovědný jakýkoliv národní orgán, výbor nebo tribunál nespadající pod pravomoc člena, pak rozhodnutí tohoto orgánu, výboru nebo tribunálu musí být pro účely P 42 považováno za rozhodnutí člena a pojem „člen“ musí být v tomto pravidle takto vykládán.
* Při slyšení, orgán projednávající případ musí nejdříve rozhodnout, zda ano nebo nikoliv, došlo k porušení pravidel o dopingu. Člen či vyšetřující orgán nese tíhu důkazů o porušení Antidopingového pravidla k dostatečné spokojenosti tribunálu (viz P 33.1).
* Pokud příslušný tribunál uváží, že k porušení pravidel o dopingu nedošlo, musí být o tomto rozhodnutí písemně uvědomen Správce IAAF pro záležitosti dopingu a to do 5 pracovních dní od takového rozhodnutí, vč. kopie zdůvodnění tohoto rozhodnutí. Případ musí být pak přezkoumán Dopingovou revizní komisí (Doping Review Board), která rozhodne, zda případ má nebo nemá být předán CAS k arbitrážnímu jednání podle P 42.15. Pokud se komise rozhodne tak učinit, může současně, pokud je to přiměřené, obnovit až do rozhodnutí odvolacího řízení před CAS prozatímní zastavení činnosti atleta.
* Pokud příslušný tribunál uváží, že k porušení pravidel o dopingu došlo, musí mít atlet, před rozhodnutím o zastavení činnosti na jakoukoliv dobu, příležitost doložit, že v jeho případě existují výjimečné okolnosti ospravedlňující snížení sankcí, které jinak podle P 40 platí pro dané provinění.

Výjimečné / zvláštní okolnosti
* Všechna rozhodnutí týkající se výjimečných okolností, přijatá podle těchto Antidopingových pravidel, musí být harmonizována tak, aby pro všechny atlety byly zaručeny stejné právní podmínky, bez ohledu na jejich národnost, sídlo, úroveň nebo zkušenosti. Proto v otázkách výjimečných okolností musí platit následující zásady :
  \begitems \style a
  * je osobní povinností každého atleta postarat se o to, aby se do tkání nebo tekutin jeho těla nedostaly žádné zakázané látky. Atleti se upozorňují na to, že budou bráni k zodpovědnosti za jakékoliv zakázané látky, zjištěné v jejich tělech ( viz P 32.2.a.i)
  * výjimečné okolnosti budou připuštěny pouze v případech, kde okolnosti budou skutečně mimořádné a nebudou se vyskytovat v převážné většině případů.
  * vzhledem k osobním povinnostem atleta podle ustanovení P 38.15.a), nebudou za skutečně výjimečné okolnosti považována obvinění, že zakázaná látka nebo zakázaný postup byly atletu poskytnuty jinou osobou bez jeho vědomí, obvinění, že zakázaná látka byla užita omylem, obvinění, že zakázaná látka byla obsažena v doplňku stravy nebo obvinění, že atletu byl jeho doprovodem podán lék bez ohledu na skutečnost, že obsahuje zakázanou látku.
  * výjimečné okolnosti mohou nicméně nastat v případě, kdy atlet poskytne dostatečný důkaz nebo pomoc IAAF, své národní federaci, orgánu činnému v trestním řízení nebo profesnímu disciplinárnímu orgánu, vedoucí k tomu, že IAAF, národní federace atleta nebo orgán činný v trestním řízení nebo profesní disciplinární orgán, zjistí porušení pravidel o dopingu jinou osobou nebo se tato osoba dopustila trestné činnosti nebo porušila profesní pravidla.
  * Zvláštní okolnosti mohou nastat v případě nepříznivého analytického nálezu specifikované látky, kdy atlet může prokázat, jak se tato látka dostala do jeho těla nebo držení a že tato látka nebyla určena ke zvýšení jeho sportovní výkonnosti nebo k zamaskování užití látky zvyšující výkonnost.
  \enditems
* O tom, zda se jedná o výjimečných/zvláštních okolnostech v případech týkajících se atletů mezinárodní výkonnostní úrovně musí rozhodnout Dopingová revizní komise (viz P 38.20).
* Pokud atlet usiluje o přiznání výjimečných/ zvláštních okolností svému případu, příslušný tribunál musí uvážit, s přihlédnutím k předloženým důkazům a za přísného dodržení zásad uvedených v P 38.15, zda podle jeho názoru mohou být okolnosti daného případu výjimečné/zvláštní. V případě spadajícím pod P 32.2.a), musí být daný atlet v každém případě schopný doložit, jak se zakázaná látka dostala do jeho těla, aby mu byla doba zákazu činnosti zkrácena.
* Pokud po prozkoumání předložených svědectví příslušný tribunál rozhodne, že žádné výjimečné či zvláštní okolnosti v daném případě nejsou, musí na atleta uvalit sankce předepsané v P 40. Členská federace musí do 5 pracovních dní o rozhodnutí tribunálu písemně uvědomit atleta a IAAF.
* Pokud po prozkoumání předložených svědectví příslušný tribunál rozhodne, že existují okolnosti daného případu, které mohou být výjimečné, potom, pokud se jedná o atlety mezinárodní výkonnostní úrovně, musí
  \begitems \style a
  * ohlásit případ, prostřednictvím Generálního sekretáře, Dopingové revizní komisi a předat ji veškeré materiály a/nebo důkazy, které podle jeho názoru dokládají výjimečnou povahu okolností; a
  * vyzvat atleta a/nebo jeho národní federaci, aby podpořili doporučení tribunálu nebo učinili nezávislé podání na jeho podporu;
  * odložit slyšení atleta do rozhodnutí Dopingové revizní komise o výjimečnosti okolností případu.
  \enditems

Do rozhodnutí Dopingové revizní komise o výjimečnosti/zvláštnosti okolností případu, zůstává prozatímní zastavení činnosti atleta v platnosti.

* Po obdržení doporučení příslušného tribunálu, Dopingová revizní komise musí otázku mimořádnosti/zvláštnosti okolností daného případu přezkoumat pouze na základě písemných materiálů, které jí byly předloženy. Dopingová revizní komise má pravomoc:
  \begitems \style a
  * vyměňovat si názory k případu prostřednictvím e-mailu, telefonu, faxu nebo osobním jednáním;
  * vyžádat si další důkazy nebo dokumenty;
  * vyžádat si od atleta další vysvětlení;
  * je-li potřeba, vyžádat si přítomnost atleta na svém jednání.
  \enditems

Po přezkoumání obdržených písemných materiálů, dalších důkazů nebo dokumentů, další vysvětlení atleta a za přísného dodržení zásad uvedených v P 38.15, Dopingová revizní komise musí rozhodnout, zda v daném případě existují výjimečné okolnosti a pokud ano, do které kategorie spadají. Zda se nejedná o žádnou chybu nebo žádnou nedbalost ze strany atleta (viz P 40.5.a) nebo žádnou podstatnou chybu nebo žádnou podstatnou nedbalost ze strany atleta (viz P 40.5.b) nebo významné svědectví nebo pomoc ze strany atleta, jejímž výsledkem je odhalení porušení pravidel o dopingu jinou osobou (viz P 40.5.c). Toto rozhodnutí musí Generální sekretář předat členské federaci.

* Pokud Dopingová revizní komise rozhodne, že se v daném případě nejedná o výjimečné okolnosti, musí být takové rozhodnutí závazné pro příslušný tribunál, který musí na atleta uvalit sankce předepsané v P 40. Členská federace musí o rozhodnutí příslušného tribunálu, včetně rozhodnutí Posudkové komise pro záležitosti dopingu, do 5 pracovních dní písemně uvědomit atleta a IAAF.
* Pokud Dopingová revizní komise rozhodne, že se v daném případě jedná o výjimečné/zvláštní okolnosti, příslušný tribunál rozhodne o sankcích proti atletovi podle s P 40.4, nebo P 40.5, podle toho, do které kategorie dle P 38.20 Dopingová revizní komise okolnosti jeho případu zařadila. Členská federace musí o příslušném rozhodnutí tribunálu, vč. plného odůvodnění, do 5 pracovních dní písemně uvědomit atleta a IAAF.
* Atlet má právo dožadovat se u CAS přezkoumání rozhodnutí Dopingové revizní komise o výjimečných/zvláštních okolnostech. Ve všech případech musí přezkoumání rozhodnutí Dopingové revizní komise o výjimečných okolnostech probíhat tak, jak je uvedeno v P 42.21.
* V případech, kdy se nejedná o atleta mezinárodní výkonnostní úrovně, musí příslušný tribunál zvážit, při přísném dodržování zásad stanovených v P 38.15, zda v daném případě existují výjimečné/zvláštní okolnosti a podle toho rozhodnout o sankcích proti atletovi. Členská federace musí o rozhodnutí tribunálu do 5 pracovních dní písemně uvědomit atleta a IAAF. Pokud příslušný tribunál dojde k závěru, že v případě atleta existují výjimečné/zvláštní okolnosti, musí ve svém písemném rozhodnutí uvést všechny faktické údaje, na jejichž základě tak rozhodl a uvalil sankce.
\enditems

\secc Zrušení výsledků

Porušení pravidel o dopingu ve spojitosti s kontrolou v průběhu soutěže automaticky znamená vyloučení atleta z dané disciplíny se všemi následnými důsledky pro tohoto atleta, včetně odebrání všech titulů, odměn, medailí, bodů a cen a startovného (appearance money).

\secc Sankce proti jednotlivcům

\begitems \style N
Zrušení výsledků v soutěži, v níž došlo k porušení antidopingových pravidel
* Porušení pravidel o dopingu v průběhu soutěže znamená zrušení všech výsledků dosažených atletem během příslušných soutěží se všemi následnými důsledky pro tohoto atleta, včetně odebrání všech titulů, odměn, medailí, bodů a cen a startovného, vyjma níže uvedeného.

Pokud atlet prokáže, že se při svém provinění nedopustil chyby nebo zanedbání povinností, výsledky tohoto atleta v ostatních disciplínách, kde nedošlo k porušení pravidel o dopingu, nebudou zrušeny, pokud pravděpodobně nebyly ovlivněny tím, že atlet pravidla o dopingu porušil.

Zákaz činnosti pro přítomnost, užití nebo pokus o užití nebo vlastnictví zakázané látky nebo zakázaného způsobu
* Doba zákazu činnosti (nezpůsobilosti) pro porušení ustanovení P 32.2.a) (přítomnost zakázané látky nebo jejího metabolitu či příznaku), P 32.2.b) (užití nebo pokus o užití zakázané látky nebo zakázaného postupu) nebo P 32.2.f) (držení zakázaní látky nebo zakázaného postupu), pokud nebudou splněny podmínky pro zrušení nebo snížení této doby ve smyslu P 40.4 a P 40.5 nebo podmínky pro prodloužení této doby ve smyslu P 40.6, je
  Při prvním porušení : Dva roky zákazu činnosti

Zákaz činnosti pro porušení jiných pravidel o dopingu
* Doba zákazu činnosti (nezpůsobilosti) pro porušení jiných ustanovení než uvedených v P 40.2 jsou následující :
  \begitems \style a
  * Při porušení P 32.2.c) (odmítnutí nebo neposkytnutí odebraného vzorku) nebo P 32.2.h) (poddání nebo pokus o podání zakázané látky nebo zakázaného postupu) musí být nařízen zákaz činnosti od čtyř let až na doživotí, pokud nebudou splněny podmínky uvedené v P 40.5. Porušení pravidel o dopingu týkající se nedospělých osob, musí být považováno za mimořádně vážné porušení pravidel, a pokud bylo způsobeno osobou z podpůrného týmu atleta v případě, který se netýká zvláštní látky definované v P 34.5, musí následovat uvalení doživotního zákazu činnosti na tuto osobu. Navíc, výrazná porušení P 32.2.g) a P 32.2.h) která mohou být současně porušením mimosportovních zákonů a pravidel, musí být ohlášena příslušným správním, profesním a soudním orgánům.
  * Při porušení P 32.2.g) (nelegální šíření nebi pokus o nelegální šíření), nebo porušení P 32.2.b) (podávání nebi pokus o podání zakázaní látky nebo zakázaného postupu) je doba uděleného zákazu činnosti minimálně čtyři (4) roky až doživotní zákaz činnosti, pokud nedojde k naplnění ustanovení P 40.5. Porušení pravidel o dopingu v případě nezletilých osob musí být považováno za zvláště vážný přestupek. Pokud se porušení pravidel dopustí člen podpůrného týmy atleta v případě, který se netýká specifických lýtek uvedených v P 34.5, je tento člen postižen doživotním zákazem činnosti. Navíc, při vážném porušení P 32.2.g) nebo P 32.2.h), kdy mohou být porušeny rovněž mimosportovní zákony a předpisy, bude případ oznámen též kompetentním správním, profesním a právním orgánům.
  * Při porušení P 32.2.d) (zanedbání zprávy o pobytu nebo zmeškání testu) je doba zákazu činnosti minimálně jeden (1) rok a nejvýše dva (2) roky podle stupně zavinění ze strany atleta.
  \enditems

Zrušení nebo snížení doby zákazu činnosti pro zvláštní látky nebo s ohledem na zvláštní okolnost
* Pokud atlet, nebo jiná osoba prokáže jak se specifická látka dostala do jeho těla nebo jak se dostala do jeho vlastnictví a přitom tato látka nebyla určena pro zvýšení jeho sportovní výkonnosti, nebo k zamaskování užití látky zvyšující výkonnost, bude doba zastavení činnosti uvedená v P 40.2 nahrazena následovně:

První provinění: minimálně důtka bez zákazu činnosti v dalších soutěžích, nejvýše dva roky zákazu činnosti.

Zrušení nebo snížení sankcí musí být založeno na tom, že atlet nebo jiná osoba musí kromě svého prohlášení předložit komisi, která řídí slyšení, přesvědčující důkazy o tom, že nešlo o úmysl zvýšit sportovní výkonnost nebo zamaskovat použití látky, která výkonnost zvyšuje. Kritériem pro jakékoliv snížení doby zákazu činnosti musí být stupeň zavinění zjištěné skutečnosti atletem nebo jinou osobou.

Tento článek se týká pouze těch okolností, kdy je komise řídící slyšení objektivními okolnostmi plně přesvědčena o tom, že užitím zakázané látky atlet neměl v úmyslu zvýšit svoji sportovní výkonnost.

Zrušení nebo snížení doby zákazu činnosti na základě výjimečných okolností

* \begitems \style a
  * Žádná chyba nebo zanedbání: Pokud atlet nebo jiná osoba v daném případě prokáže, že nenese žádnou vinu nebo zanedbání, doba zákazu činnosti, která by jej jinak postihla se ruší. Aby doba zákazu činnosti byla zrušena v případě zjištění zakázané látky nebo jejich metabolitů  nebo příznaků, což je porušením P 32.2.a), musí atlet prokázat, jak se zakázaná látka dostala do jeho systému.

  V případě uplatnění tohoto pravidla a příslušná doba zastavení činnosti je zrušena, porušení pravidel o dopingu nebude vzato v úvahu při stanovení doby zastavení činnosti v případě vícenásobného porušení pravidel podle P 40.7.

  * Žádná závažná chyba nebo zanedbání: Pokud osoba v daném případě  prokáže, že nenese žádnou významnou vinu nebo zanedbání, doba zákazu činnosti, která by jej jinak postihla, může být snížena, nejvýše na polovinu jinak možné doby zákazu činnosti. Pokud touto možnou dobou zákazu činnosti je doživotní zákaz, pak doba zákazu činnosti je nejméně 8 let. Aby doba zákazu činnosti byla snížena v případě zjištění zakázané látky nebo jejich metabolitů  nebo příznaků, což je porušením P 32.2.a), musí atlet prokázat, jak se zakázaná látka dostala do jeho systému.
  * Podstatná spolupráce při odhalení nebo zjištění porušení pravidel o dopingu. Příslušný tribunál člena může před vynesením konečného rozhodnutí, proti němuž lze podat odvolání podle P 42, nebo před uplynutím lhůty pro odvolání (podle toho, zda v případě atleta mezinárodní úrovně byl podle P 38.16 případ předán Dopingové revizní komisi k rozhodnutí), na čas zrušit část doby zákazu činnosti stanovené v daném případě pokud atlet nebo jiná osoba poskytli podstatnou spolupráci IAAF, národní federaci, antidopingové organizaci, orgánům činným v trestním řízení nebo profesnímu disciplinárnímu orgánu, na jejímž základě IAAF, národní federace nebo antidopingová organizace odhalila nebo zjistila porušení pravidel o dopingu nebo orgán činný v trestním řízení nebo profesní disciplinární orgán odhalil nebo zjistil trestný čin nebo porušení profesních pravidel jinou osobou. Po vynesení konečného rozhodnutí, nebo po uplynutí doby pro podání odvolání podle P 42, může být doba zákazu činnosti stanovená atletu nebo jiné osobě pozastavená členskou federací pouze tehdy, jestliže tak rozhodne Dopingová revizní komise a WADA udělí souhlas. Pokud Dopingová revizní komise rozhodně, že nedošlo k podstatné spolupráci, je toto rozhodnutí pro členskou federaci závazné a doba zákazu činnosti nemůže být pozastavená. Pokud Dopingová revizní komise rozhodně, že došlo k podstatné spolupráci, členská federace rozhodne o době, po níž bude zákaz činnosti pozastaven. Rozsah, v jakém může být příslušná doba zákazu činnosti pozastavena, závisí na vážnosti porušení pravidel o dopingu atletem nebo jinou osobou a významem podstatné spolupráce vykázané atletem nebo jinou osobou při úsilí o odstranění dopingu v atletice. Z příslušné doby zákazu činnosti nelze pozastavit víc, než tři čtvrtiny. Je-li touto dobou doživotní zákaz činnosti, nesmí být zbývající doba zákazu činnosti kratší než osm let.  Pokud členská federace pozastaví jakoukoliv část doby zákazu činnosti ve smyslu tohoto pravidla, musí odůvodnění svého rozhodnutí písemně sdělit IAAF a kterékoliv straně s právem podat odvolání. Pokud členská federace následně obnoví zákaz činnosti po libovolnou část pozastavené doby, neboť atlet nebo jiná osoba neposkytli tak podstatnou spolupráci, jak bylo očekáváno, může tento atlet nebo jiná osoba podat proti tomu odvolání.
  * Přiznání se k porušení pravidel o dopingu při nedostatku důkazů: Pokud se atlet nebo jiná osoba dobrovolně přizná k porušení pravidel o dopingu dříve, než se má podrobit k odběru vzorku, který by mohl vést k odhalení porušení pravidel o dopingu (nebo v případě jiného porušení pravidel o dopingu než jak je uvedeno v P 32.2.a) před obdržením první zprávy o porušení pravidel podle P 37) a toto přiznání je jediným spolehlivým důkazem porušení pravidel v době přiznání, může být doba zastavení činnosti snížena, ale nikoliv více než o polovinu jinak příslušné sankční doby.
  * Atlet nebo jiná osoba prokáže nárok na snížení sankce, na základě více než jednoho ustanovení těchto pravidel. Před využitím možnosti snížení nebo pozastavení sankcí podle P 40.5.b), c) nebo d),  musí být příslušná doba zastavení činnosti stanovena v souladu s P 40.2, P 40.3, P 40.4 a P 40.6. Pokud atlet nebo jiná osoba prokáže nárok na snížení sankce, na základě dvou nebo více ustanovení P 40.5.b), c), nebo d), pak doba zastavení činnosti musí být snížena, nikoliv však pod jednu čtvrtinu jinak příslušné doby sankce.
  \enditems

Přitěžující okolnosti pro zvýšení doby zastavení činnosti
* Je-li zjištěno, že v individuálním případě, kdy se jedná jiné porušení pravidel o dopingu, než jak je uvedeno v P 32.2.g) (nelegální šíření nebo pokus o nelegální šíření) a P 32.2.h) (podávání nebo pokus o podávání), existují přitěžující okolnosti, které ospravedlňují uložení zákazu činnosti po dobu delší, než je standardní sankce, bude jinak příslušná sankční doba prodloužena až na čtyřnásobek, pokud atlet nebo jiná osoba dostatečně před komisí při slyšení neprokáže, že pravidla o dopingu porušil bez vlastního vědomí.
  \begitems \style a
  * Příklady přitěžujících okolností, které mohou ospravedlnit uložení zákazu činnosti na dobu delší než je standardní sankce jsou :
    \begitems \style i
    * atlet nebo jiná osoba porušili pravidla o dopingu v rámci plánovaného dopingu, buď jako jednotlivci nebo v rámci dohody nebo společné iniciativy,
    * atlet nebo jiná osoba opakovaně použili nebo vlastnili několik druhů zakázaných látek nebo zakázaných postupů,
    * běžná osoba by mohla využít výhod zvýšené výkonnosti porušením pravidel o dopingu aniž by ji hrozilo příslušné zastavení činnosti;
    * atlet nebo jiná osoba jednala, klamala nebo se jinak snažila zabránit zjištění nebo rozhodnutí o porušení pravidel o dopingu.
    \enditems

  Kromě výše uvedených příkladů může být rozhodnutí o prodloužení doby zákazu činnosti založeno i na jiných přitěžujících okolnostech.

  * Atlet nebo jiná osoba se mohou vyhnout uplatnění tohoto pravidla přiznáním k porušení pravidel o dopingu z nějž je obviněn okamžitě, jakmile je seznámen se svým proviněním, což znamená nejpozději do konce lhůty pro poskytnutí písemného vyjádření ve smyslu P 37.4.c) a v každém případě dříve, než se zúčastní další soutěže.
  \enditems

Vícenásobná provinění
* \begitems \style a
  * Druhé porušení pravidel o dopingu: Doba zákazu činnosti atletu nebo jiné osobě za první porušení pravidel o dopingu je uvedená v P 40.2 a P 40.3, pokud se jedná o zrušení, snížení nebo dočasné zastavení činnosti P 40.4 nebo P 40.5, pokud se jedná o zvýšení sankcí P 406. Za druhé porušení pravidel o dopingu musí být uložena doba zákazu činnost v rozsahu, jak je uvedeno v následující tabulce :

\table{|6{l|}}{\crl
2. porušení & RS        & FFMT       & NSF        & St        & AS \crlp{1}
1. porušení &           &            &            &           & \crl
RS          & 1--4      & 2--4       & 2--4       & 4--6      & 8--10 \crl
FFMT        & 1--4      & 4--8       & 4--8       & 6--8      & 10--doživ. \crl
NSF         & 1--4      & 4--8       & 4--8       & 6--8      & 10--doživ. \crl
St          & 2--4      & 6--8       & 6--8       & 8--doživ. & doživ. \crl
AS          & 4--5      & 10--doživ. & 10--doživ. & doživ.    & doživ. \crl
TRA         & 8--doživ. & doživ.     & doživ.     & doživ.    & doživ. \crl
}

\noindent RS -- snížená sankce za specifikované látky podle P 40.4

\noindent FFMT -- zanedbané oznámení a/nebo zmeškaný test podle P 40.3.c)

\noindent NSF -- snížená sankce pro nezávažnou chybu nebo zanedbání podle P 40.5.b)

\noindent St -- Standardní sankce podle P 40.2 nebo P 40.3.a)

\noindent AS -- přitěžující okolnosti podle P 40.6

\noindent TRA -- rozšiřování a podávání dopingu podle P 40.6

  * Uplatnění P 40.5.c) a P 40.5.d) při druhém porušení pravidel o dopingu : Pokud atlet nebo jiná osoba, kteří poruší pravidla o dopingu podruhé, prokáží oprávnění pro odložení nebo snížení části doby zákazu činnosti podle P 40.5.c) nebo P 40.5.d), komise řídící slyšení musí nejprve stanovit jinak příslušnou dobu zákazu v rozmezí daném tabulkou v P 40.7.a) a poté uplatnit příslušné odložení nebo snížení doby zákazu. Po této úpravě musí zbývající doba zákazu činnosti musí činit alespoň jednu čtvrtinu jinak příslušné doby zákazu činnosti.
  * Třetí porušení pravidel o dopingu :  Po třetím porušení pravidel o dopingu musí vždy následovat doživotní zákaz činnosti, vyjma případu,  kdy jsou při tomto třetím porušení pravidla o dopingu splněny podmínky pro zrušení nebo snížení doby zákazu činnosti podle P 40.4 nebo se jedná o porušení ustanovení P 32.2.d) (opomenutí oznámení a/nebo zmeškaný test). V takových případech musí být uložen zákaz činnosti na dobu v rozmezí od 8 let až na doživotí.
  * Zvláštní pravidla týkající se opakovaného porušení pravidel:
    \begitems \style i
    * Pro účely ukládání sankcí podle P 40.7 o druhé prošení pravidla o dopingu se jedná pouze v případě, kdy lze prokázat, že atlet nebo jiná osoba již v souladu s P 37 (Správa výsledků) obdrželi sdělení o prvním porušení pravidel o dopingu nebo k tomu již bylo vynaloženo přiměřené úsilí. Pokud se to nelze prokázat, budou obě porušení pravidel o dopingu považována jako jediné, první porušení těchto pravidel a následné sankce budou vycházet ze závažnosti dané situace. Skutečnost, že došlo k opakovanému porušené pravidel, může být považováno za přitěžující okolnost ve smyslu P 40.6.
    * Pokud poté, co bylo rozhodnuto, že atlet nebo jiná osoba porušili pravidla o dopingu poprvé, bude zjištěno, že tento atlet nebo jiná osoba porušili pravidla již před oznámením o prvním porušení pravidel o dopingu, budou postihnutí sankcemi, které by je postihly při zjištění současného dvojnásobného porušení pravidel o dopingu. Výsledky dosažené ve všech soutěžích až po dříve zjištěné porušení pravidel budou zrušeny v souladu s P 40.8. Aby se vyhnuli možnosti, jim na základě dřívějšího, ale později zjištěného provinění budou přiznány přitěžující okolnosti (P 40.6), musí atlet nebo jiná osoba dobrovolně přiznat dřívější provinění ještě před tím, než obviněním z prvního porušení (což znamená nejpozději k datu, kdy musí podat písemné vyjádření podle P 37.4.c) a v každém případě před další účastí v soutěži. Stejné pravidlo platí, když je zjištěno další dřívější provinění po rozhodnutí o druhém porušení pravidel o dopingu.
    \enditems
  * Opakované porušení pravidel o dopingu během osmileté doby : Pro účely ustanovení P 40.7 musí ke každému porušení pravidel dojít během osmi po sobě jdoucích let, aby je bylo možno považovat za vícenásobná (opakovaná) porušení pravidel o dopingu.
  \enditems

Zrušení výsledků dosažených v soutěži po následném odebrání vzorků nebo rozhodnutí o porušení pravidel o dopingu
* Kromě automatického zrušení výsledků dosažených v soutěži, při níž byly odebrány pozitivní vzorky podle P 39 a P 40 budou zrušeny rovněž všechny výsledky dosažené v soutěžích po datu odebrání pozitivního vzorku (ať během soutěže nebo mimo soutěží) nebo datu, kdy došlo k jinému porušení pravidel o dopingu, až po počátek předběžného zákazu startů nebo rozhodnutí o zákazu startů. K tomu přistupují všechny důsledky,  zahrnující odebrání všech titulů, odměn, medailí bodů a cen i startovného.
* Odebrání peněžních cen v souladu s P 40.8
  \begitems \style a
  * Přidělení odebraných peněžních cen : pokud peněžní ceny nebyly atletu, který byl potrestán zákazem činnosti, dosud vyplaceny, budou rozděleny mezi atlety, kteří se v příslušné disciplíně nebo soutěži umístili za potrestaným atletem. Pokud byly peníze již vyplaceny, budou rozděleny mezi atlety, kteří se v příslušné disciplíně nebo soutěži umístili za potrestaným atletem pouze tehdy když a až budou vráceny potrestaným atletem příslušné osobě nebo společnosti a
  * podmínkou pro opětovné získání oprávnění ke startu poté, co bylo zjištěno, že atlet porušil pravidla o dopingu, musí provinivší se atlet všechny peněžní odměny, které mu byly odebrány podle výše uvedeného P 40.8 (viz P 40.12.a)
  \enditems

Počátek doby ztráty oprávnění k účasti na soutěžích (zákazu soutěžní činnosti)
* Vyjma dále uvedených ustanovení, doba zákazu soutěžní činnosti (ztráty způsobilosti) sankční doba začíná běžet dnem slyšení, kdy bylo o dané sankci rozhodnuto, nebo pokud atlet práva na slyšení nevyužil, dnem kdy byla sankce přijata nebo jinak uložena. Pokud před uložením sankce ztráty oprávnění k účasti v soutěžích (ať přijaté dobrovolně nebo nikoliv) byla atletu dočasně zastavena činnost, započítává se doba tohoto předchozího trestu do doby udělené sankce.
  \begitems \style a
  * Včasné doznání : Pokud atlet okamžitě poté, co byl obviněn z porušení pravidel o dopingu, se písemně přizná (což znamená nejpozději ke konci lhůty pro podání vysvětlení ve smyslu P 37.4.c), P 37.10 nebo sekce 6.16 Předpisů o dopingu, ale v každém případě před svým dalším startem) může doba zastavení činnosti začít běžet dnem odebrání vzorku nebo dnem, kdy došlo k poslednímu dalšímu porušení pravidel o dopingu. V každém případě však, při uplatnění tohoto pravidla, musí být atlet nebo jiná osoba postiženi zákazem činnosti v délce alespoň jedné poloviny sankce, která poběží od doby kdy tento atlet nebo jiná osoba uloženou sankci přijali nebo od data konání slyšení, na němž bylo o sankci rozhodnuto nebo od jiného data, kdy jim byla sankce uložena.
  * Při uložení sankce prozatímního zákazu činnosti, kterou atlet přijal, započítává se tato doba prozatímního zákazu činnosti do doby jakéhokoliv zákazu činnosti, kterým tento atlet je v konečné fázi postižen.
  * Pokud atlet písemným prohlášením dobrovolně přijme prozatímní zákaz činnosti (ve smyslu P 38.2) a následně se vzdá soutěžní činnosti, započítává se tato doba prozatímního zákazu činnosti do doby jakéhokoliv zákazu činnosti, kterým tento atlet je v konečné fázi postižen. V souladu s P 38.3 nabývá dobrovolné pozastavení činnosti účinnosti dnem, kdy je IAAF obdrží.
  * Do doby zákazu činnosti se nezapočítává jakákoliv doba, která uplyne před dočasným nebo dobrovolným dočasným zastavením činnosti, bez ohledu na to, zda se atlet rozhodnul nesoutěžit nebo nebyl na soutěže vyslán.
  \enditems

Postavení atleta během doby ztráty způsobilosti k účasti v soutěžích
* \begitems \style a
  * Zákaz účasti během doby zákazu činnosti: Žádný atlet nebo jiná osoba, které byla zastavena soutěžní činnost, se nesmí během sankční doby zúčastnit jakékoliv soutěže nebo činnosti (vyjma účasti na schválených programech antidopingové výchovy nebo rehabilitace) schválené nebo pořádané IAAF nebo kteroukoliv oblastní asociací, či členskou federací (nebo klubem nebo jinou organizací členské federace), nebo signatářem (nebo jeho členem, či klubem nebo jinou členskou organizací), nebo soutěže řízené či organizované jakoukoliv profesionální ligou nebo jakoukoliv mezinárodní či národní organizací Pojem „činnost“ pro účely tohoto pravidla zahrnuje účast v jakékoliv funkci, ať jako atlet trenér nebo člen podpůrného týmu atleta při tréninkových soustředěních, exhibicích, nebo ukázkách, organizovaných členskou federací (nebo klubem či jinou organizací členské federace) nebo signatářem (např. v národním tréninkovém středisku) stejně jako účast v administrativní funkci, jako činovník, vedoucí, zaměstnanec nebo dobrovolník jakékoliv organizace zménšné v těchto pravidlech. Atlet, kterému byl vysloven zákaz činnosti, může i být nadále testován. Atlet nebo jiná osoba, která byla potrestána zákazem činnosti delším než čtyři (4) roky, se může po uplynutí této doby zúčastnit místních soutěží v jiném sportovním odvětví než atletice, ale pouze potud, pokud se nejedná o úroveň, jíž by se tento atlet nebo jiná osoba přímo nebo kvalifikovali (byť jen pomocí nasbíraných bodů) na národní mistrovství nebo mezinárodní soutěž.
  * Porušení zákazu činnosti během doby zákazu činnosti, kdy atlet nebo jiná osoba nedbají zákazu a vyvíjejí činnost v rozporu s P 40.11, znamená diskvalifikaci v soutěžích jichž se zúčastnil, a původně uložená doba zákazu činnosti začíná běžet znovu, počínaje dnem porušení zákazu. Nová doba zákazu činnosti může být podle P 40.5.b) zkrácena, pokud atlet nebo jiná osoba prokáží, že k tomto porušení zákazu činnosti došlo bez jejich závažné chyby nebo zanedbání. Rozhodnutí, zda se atlet nebo jiná osoba provinili porušením zákazu činnosti a zda jim má být přiznáno snížení nového běhu sankce podle P 40.5.b), musí učinit stejný orgán, který je potrestal původním zákazem činnosti.
  * Zastavení finanční podpory během doby zákazu činnosti: Při jakémkoliv porušení pravidel o dopingu, kdy není trest snížen vzhledem k užití specifikované látky, jak je popsáno v P 40.4, musí být potrestané osobě zastaveny některé nebo všechny finanční podpory a výhody vyplývající z její sportovní činnosti
  \enditems

Návrat k soutěžím po ztrátě způsobilosti k účasti v soutěžích
* Podmínkou pro opětovné získání způsobilosti k činnosti po uplynutí doby zákazu, musí atlet nebo jiná osoba splnit následující podmínky
  \begitems \style a
  * Vrácení peněžních odměn: Atlet musí vrátit veškeré peněžní odměny, které obdržel v souvislosti se svými sportovními výkony od data soutěže, kde mu byl odebrán vzorek s nepříznivým analytickým nálezem nebo jinak porušil pravidla o dopingu nebo od data, kdy se dopustil jakéhokoliv jiného porušení pravidel o dopingu; a
  * Vrácení medailí: Atlet musí vrátit všechny medaile (z individuálních i týmových soutěží), které obdržel za výkony v soutěžích po datu odebrání vzorku, který byl pozitivní, nebo jiném porušení pravidel o dopingu, nebo od data přiznaného jiného porušení pravidel o dopingu dále; a
  * Obnovení kontrol: Během dočasného nebo trvalého zastavené činnosti musí být atlet k dispozici pro mimosoutěžní kontroly prováděné IAAF, jeho národní federací nebo jakoukoliv jinou organizací oprávněnou provádět testování podle těchto pravidel o dopingu, a pokud to bude po něm požadováno, musí podávat přesné informace o svém současném pobytu. Pokud byla činnost zastavena atletu mezinárodní úrovně na jeden (1) rok nebo více, musí se podrobit alespoň čtyřem (4) ověřovacím kontrolám, třem (3) mimosoutěžním kontrolám a jednomu (1) testu na úplný rozsah zakázaných látek a postupů a to těsně před koncem doby zákazu. Tyto ověřovací kontroly musí být provedeny na náklady atleta a musí být provedeny v odstupu alespoň tří měsíců mezi jednotlivými testy. Za provedení ověřovacích testů odpovídá IAAF v souladu s pravidly a předpisy o dopingu, ale IAAF se může pro daný účel spokojit i s ověřovacími testy provedenými jiným orgánem, pokud budou vzorky analyzovány laboratoří akreditovanou u WADA. V případě, kdy se proti pravidlům o dopingu provinil atlet, který startuje v běžeckých, chodeckých nebo vícebojařských soutěžích, musí být vzorky odebrané alespoň při jeho posledních dvou ověřovacích kontrolách analyzovány na látky stimulující erytropoetin a jejich uvolňující faktory. Výsledky všech testů spolu s kopiemi příslušných formulářů dopingových kontrol musí nýt předloženy IAAF ještě před návratem atleta k soutěžní činnosti. Pokud výsledkem některého z ověřovacích testů, provedených ve smyslu tohoto pravidla, bude nepříznivý analytický nález, nebo jiné porušení pravidel o dopingu, znamená to samostatné porušení pravidel o dopingu a atlet bude podroben disciplinárnímu řízení a příslušným dalším sankcím.
  * Jakmile uplyne doba zákazu činnosti a atlet splní požadavky uvedené v P 40.12, stává se automaticky znovu způsobilým k účasti v soutěžích a není třeba, aby atlet nebo jeho národní federace žádali IAAF o souhlas.
  \enditems
\enditems

\secc Sankce proti družstvům

\begitems \style N
* Pokud atlet, který porušil pravidla o dopingu, soutěžil jako člen družstva v běhu rozestavném, jeho družstvo bude automaticky v dotčeném závodě diskvalifikováno se všemi příslušnými důsledky, jako je odebrání všech titulů, odměn, medailí, bodů a peněžních cen.  Pokud atlet, který porušil pravidla o dopingu, soutěžil jako člen družstva v běhu rozestavném v následující disciplíně nebo soutěži, bude jeho družstvo diskvalifikováno v této následující soutěži se všemi uvedenými důsledky, jako je odebrání všech titulů, odměn, medailí, bodů a peněžních cen, pokud tento atlet neprokáže, že nenese žádnou vinu nebo nic nezanedbal a jeho účast v běhu rozestavném nebyla pravděpodobně ovlivněna skutečností, že byla porušena pravidla o dopingu.
* Pokud atlet, který porušil pravidla o dopingu, soutěžil jako člen jiného družstva než v běhu rozestavném, kdy pořadí družstev je založeno na součtu jednotlivých výsledků, jeho družstvo nebude diskvalifikováno, ale výsledek atleta, který se provinil proti pravidlům o dopingu, bude družstvu odečten a nahrazen výsledkem dalšího člena družstva. Pokud po odečtení výsledku provinivšího se atleta poklesne počet členů družstva pod požadovaný počet, bude toto družstvo diskvalifikováno. Stejný princip bude uplatněn v případě, kdy za družstvo startuje atlet, který se před danou soutěží provinil proti pravidlům o dopingu, pokud tento atlet neprokáže, že nenese žádnou vinu nebo nic nezanedbal a jeho účast v družstvu nebyla pravděpodobně ovlivněna skutečností, že byla porušena pravidla o dopingu.
* Kromě zrušení výsledků podle P 40.8
  \begitems \style a
  * výsledky kteréhokoliv družstva v běhu rozestavném, v němž startoval atlet v údobí od odběru pozitivního vzorku nebo jiného porušení pravidla o dopingu až do doby prozatímního zastavení činnosti nebo do zákazu činnosti, musí být zrušeny se všemi příslušnými důsledky, jako je odebrání všech titulů, odměn, medailí, bodů a peněžních cen; a
  * výsledky jakéhokoliv jiného družstva než v běhu rozestavném, za něž startoval atlet v údobí od odběru pozitivního vzorku nebo jiného porušení pravidla o dopingu až do doby prozatímního zastavení činnosti nebo do zákazu činnosti nebudou automaticky zrušeny, ale výsledek atleta, který se provinil proti pravidlům o dopingu, bude družstvu odečten a nahrazen výsledkem dalšího člena družstva. Pokud po odečtení výsledku provinivšího se atleta poklesne počet členů družstva pod požadovaný počet, bude toto družstvo diskvalifikováno.
  \enditems
\enditems

\secc Odvolání

\begitems \style N
Rozhodnutí, proti nimž lze podat odvolání
* Pokud není uvedeno jinak, proti všem rozhodnutím učiněným podle těchto pravidel o dopingu, lze, v souladu s dále uvedenými ustanoveními, podat odvolání. Všechna taková rozhodnutí zůstávají po dobu odvolacího řízení v platnosti, pokud odvolací orgán nerozhodne jinak nebo není jinak určeno těmito pravidly (viz P 42.15).  Před podáním odvolání musí být vyčerpány všechny možnosti, které po vynesení rozhodnutí tato pravidla o dopingu umožňují, vyjma případů, kde právo podat odvolání má WADA a žádná jiná strana se proti konečnému rozhodnutí neodvolala. V takovém případě může WADA podat odvolání přímo k CAS, aniž by vyčerpala jiné existující možnosti nápravy.

Odvolání proti rozhodnutí týkající se porušení pravidel o dopingu nebo následků.
* Následný přehled je nevyčerpávajícím seznamem rozhodnutím v případech porušení pravidel o dopingu, proti nimž je možné podle těchto pravidel podat odvolání;
  \begitems \style -
  * rozhodnutí, že došlo k porušení pravidel o dopingu;
  * rozhodnutí, jímž jsou uloženy sankce za porušení pravidel o dopingu;
  * rozhodnutí, že nedošlo k porušení pravidel o dopingu;
  * rozhodnutí, jímž při porušení pravidel o dopingu nebyly uloženy sankce podle těchto pravidel;
  * rozhodnutí Dopingové revizní komise podle P 38.21, že v případě atleta mezinárodní úrovně neexistují výjimečné/zvláštní okolnosti ospravedlňující vyloučení, nebo snížení sankcí;
  * rozhodnutí člena potvrzující, že atlet nebo jiná osoba přijali sankce za porušení pravidel o dopingu;
  * rozhodnutí, jednání o porušení pravidel o dopingu nemůže pokračovat z procedurálních důvodů (vč. např. předpisů);
  * rozhodnutí podle P 40.11 zda atlet porušil, či neporušil zákaz činnosti;
  * rozhodnutí, že člen není oprávněn rozhodnout o údajném porušení pravidel o dopingu nebo o příslušných sankcích;
  * rozhodnutí nepovažovat nepříznivý analytický nález za porušení pravidel o dopingu po provedeném vyšetřování podle P 37.10;
  * rozhodnutí samosoudce CAS v případě, který byl ve smyslu P 38.9 postoupen CAS;
  * jakékoliv rozhodnutí týkající se porušení pravidel o dopingu nebo sankcí, které IAAF považuje za chybné nebo procedurálně nesprávné.
  \enditems
* Odvolání týkající se atletů mezinárodní úrovně : V  případech, které se týkají atleta mezinárodní úrovně nebo člena jeho doprovodu, lze se proti odvolat pouze k CAS v souladu s dále uvedenými ustanoveními.
* Odvolání netýkající se atletů mezinárodní úrovně : V případech, které se netýkají se atleta mezinárodní úrovně nebo člena jeho doprovodu, lze (pokud neplatí P 42.8) proti rozhodnutí příslušného orgánu člena podat odvolání k nezávislému a nestrannému orgánu v souladu s pravidly stanovenými členem

Pravidla pro takové odvolání musí respektovat následující zásady :
  \begitems \style -
  * včasné slyšení;
  * spravedlivá, nestranná a nezávislá komise provádějící slyšení;
  * právo být zastoupen právníkem na vlastní náklady obviněného;
  * právo na tlumočníka na vlastní náklady obviněného;
  * včasné, písemné a odůvodněné rozhodnutí.
  \enditems

Proti rozhodnutí národního odvolacího orgánu lze podat odvolání podle P 42.7

* Strany oprávněné podat odvolání : Ve všech případech, které se týkají atleta mezinárodní úrovně nebo člena jeho doprovodu, mohou se k CAS odvolat tyto osoby :
  \begitems \style a
  * Atlet nebo jiná osoba, jíž se týká napadnutelné rozhodnutí
  * druhá strana v případu, v němž bylo vyneseno rozhodnutí;
  * IAAF;
  * národní Antidopingová organizace země, v níž atlet nebo jiná osoba sídlí, nebo kde mají občanství nebo jejíž licence jsou držiteli;
  * MOV v případech, kdy rozhodnutí se týká Olympijských her, vč. oprávnění startovat na OH); a WADA.
  \enditems
* V jakémkoliv případě, netýkajícím se atleta mezinárodní úrovně nebo člena jeho doprovodu, mohou podat odvolání k národnímu odvolacímu orgánu následující osoby :
  \begitems \style a
  * Atlet nebo jiná osoba, jíž se týká napadnutelné rozhodnutí
  * druhá strana v případu, v němž bylo vyneseno rozhodnutí;
  * členská federace;
  * národní Antidopingová organizace země, v níž atlet nebo jiná osoba sídlí, nebo kde mají občanství nebo jejíž licence jsou držiteli; a
  * WADA.
  \enditems

IAAF nemá právo odvolat se proti rozhodnutí k národní Antidopingová organizaci, ale je oprávněná zúčastnit se jako pozorovatel slyšení u národního odvolacího orgánu. Účast na takovém slyšení v uvedeném postavení nezbavuje IAAF práva odvolat se k CAS proti rozhodnutí národního odvolacího orgánu podle P 42.7.

* V jakémkoliv případě, netýkajícím se atleta mezinárodní úrovně nebo člena jeho doprovodu, mohou podat odvolání proti rozhodnutí národního odvolacího orgánu následující osoby :
  \begitems \style a
  * AAF;
  * MOV v případech, kdy rozhodnutí se týká Olympijských her, vč. oprávnění startovat na OH; a
  * WADA.
  \enditems
* V jakémkoliv případě, netýkajícím se atleta mezinárodní úrovně nebo člena jeho doprovodu, může MOV (v případech, kdy rozhodnutí může ovlivnit oprávnění startovat na OH) a WADA mají právo odvolat se proti rozhodnutí příslušného orgánu člena přímo k CAS v následujících případech :
* Kterákoliv strana, která podá odvolání podle těchto pravidel o dopingu má právo na pomoc  ze strany CAS k získání požadovaných informací od orgánu, proti jehož rozhodnutí je odvolání zaměřeno a taková informace musí být z nařízení CAS poskytnuta
  \begitems \style a
  * člen nemá na národní úrovni zavedený odvolací proces;
  * žádná ze stran uvedených v P 42.6 se neodvolala k národnímu odvolacímu orgánu člena;
  * je to postup podle pravidel člena.
  \enditems

Odvolání podaná WADA v případě, kdy rozhodnutí není vydáno včas
* Pokud, v určitém případě podle těchto pravidel o dopingu, IAAF nebo člen nerozhodne o tom, zda došlo k porušení pravidel o dopingu v časové lhůtě stanovené WADA, může se WADA podat odvolání přímo k CAS, jako by IAAF nebo člen rozhodli, že k porušení pravidel o dopingu  nedošlo.  Pokud komise CAS rozhodne, že k porušení pravidel o dopingu došlo a WADA jednala správně, že se obrátila přímo na CAS, pak náklady WADA a právní poplatky za podání odvolání musí uhradit orgán (IAAF nebo člen), který své rozhodnutí neučinil.

Odvolání proti rozhodnutí, které přiznává nebo odmítá výjimku pro terapeutické užití
* Proti rozhodnutí WADA, které mění rozhodnutí o přiznání nebo odmítnutí TUE je možné se odvolat pouze k CAS. Odvolání může buď atlet nebo IAAF nebo člen (nebo jím určený orgán podle P 34.9), jehož rozhodnutí bylo změněno. Proti jinému rozhodnutí WADA, než odmítnutí TUE, které WADA neobrátila, mohou atleti mezinárodní úrovně podat odvolání pouze k CAS a ostatní atleti svému národnímu odvolacímu orgánu uvedenému v P 42.4. Pokud národní odvolací orgán změní rozhodnutí, které TUE nepřiznává, může se proti tomuto řešení může WADA podat odvolání k CAS. Pokud IAAF nebo člen (buď sám nebo jím určený orgán podle P 34.9) neprojedná řádně podanou žádost o TUE v přiměřené lhůtě, může být chybějící rozhodnutí o této žádosti považováno za odmítnutí s ohledem na práva podat odvolání ve smyslu tohoto pravidla.

Odvolání proti sankci uvalené na člena za nedodržování závazků vyplývajících z pravidel o dopingu
* Proti rozhodnutí Rady v souladu s P 44 o sankci uvalené na člena za nedodržování závazků vyplývajících z pravidel o dopingu se člen může odvolat výhradně k CAS.

Lhůty pro podání odvolání k CAS
* Pokud není stanoveno jinak v těchto pravidlech (nebo dopingová revizní komise nerozhodne jinak v případech, kdy IAAF je případným odvolatelem), má odvolatel čtyřicet pět (45) dní na podání návrh odvolání k CAS, počínaje dnem kdy došlo písemné zdůvodnění rozhodnutí (v angličtině nebo francouzštině, je-li  případným odvolatelem IAAF), proti němuž se odvolání podává (v angličtině nebo francouzštině, je-li  případným odvolatelem nebo  posledním dny, kdy bylo možno podat odvolání na národní úrovni v souladu s P 42.8.b). Do 15 dní po konečné lhůtě k podání návrhu odvolání, musí odvolatel podat CAS zdůvodnění svého návrhu a do 30 dní od obdržení zdůvodnění musí odpůrce podat CAS své vyjádření.
* Konečná lhůta, kterou má WADA pro podání odvolání j CAS je buď 21 dní po posledním dni, kdy by se mohla odvolat některá z oprávněných stran nebo 21 dní poté co WADA obdržela úplný spis týkající se daného rozhodnutí, podle toho, která lhůta uplyne později.

Odvolání IAAF k CAS
* O tom, zda se má IAAF odvolat k CAS nebo zda má IAAF být účastníkem odvolání v němž není původní stranou (viz P 42.19),, rozhoduje Dopingová revizní komise.  V příslušných případech musí Dopingová revizní komise současně rozhodnout, zda bude atletu do rozhodnutí CAS dočasně zastavena činnost.

Protistrana při odvolání k CAS
* Obecně platí, že protistranou odvolání podaného k CAS je strana, proti jejímuž rozhodnutí směřuje dané odvolání. Pokud člen ve smyslu P 38.11 pověřil vedením slyšení podle těchto pravidel jiný orgán, komisi nebo tribunál, je protistranou odvolání podaného k CAS, proti takovému rozhodnutí, člen.
* Pokud odvolání k CAS podá IAAF, je IAAF oprávněna připojit se dle svého uvážení odvolání jako další protistrana k některé protistraně, vč. atleta, člena podpůrného týmu atleta nebo jiné osoby či organizace, která může být rozhodnutím dotčena.
* V případech, kdy je IAAF jedním ze dvou nebo více protistran při odvolání podaném k CAS, může se dohodnout se s ostatními protistranami na arbitrovi. Pokud nedojde k dohodě o tom, kdo má být tímto arbitrem, rozhoduje volba IAAF.
* V kterémkoliv případě, kdy IAAF není jednou ze stran odvolání podanému k CAS, může se IAAF přesto rozhodnout, že se odvolání zúčastní jako strana a v takovém případě má plná práva účastníka podle pravidel CAS.

Odvolání k CAS
* Všechna odvolání řešená CAS (mimo ustanovení P 42.21)  se konají formou nového slyšení s uvedením všech skutečností týkajících se daného případu a komise CAS svým rozhodnutím nahradí rozhodnutí příslušného tribunálu člena nebo IAAF, které považuje za chybné nebo nedostatečně podložené. Komice CAS může v každém případě rozšířit nebo zvýšit sankce, které byly uložené v napadeném rozhodnutí.
* V případech, kdy se odvolání k CAS týká rozhodnutí Dopingové revizní komise o přiznání výjimečných nebo zvláštních okolností, je slyšení před komisí CAS je omezeno na revizi materiálů předložených Dopingové revizní komise a její rozhodnutí. Komise CAS se bude zabývat stanoviskem Dopingové revizní komise pouze v případě kdy :
  \begitems \style a
  * stanovisko Dopingové revizní komise v této otázce není podloženo fakty;
  * stanovisko Dopingové revizní komise neodpovídá jejím rozhodnutím v předchozích případech a tato odlišnost není podložená fakty týkajícími se daného případu; nebo
  * stanovisko Dopingové revizní komise nemůže odůvodnit žádný odvolací orgán.
  \enditems
* Ve všech případech odvolání k CAS, které se týkají IAAF, jsou CAS a její komise vázány stanovami, pravidly a předpisy IAAF (vč. pravidel o dopingu). V případě rozporů mezi platnými pravidly CAS a stanovami, pravidly a předpisy IAAF, mají přednost ustanovení IAAF.
* Ve všech případech odvolání k CAS, které se týkají IAAF, jsou řídícím zákonodárstvím zákony Monackého knížectví a arbitrážní jednání musí probíhat v angličtině, pokud se strany nedohodnou jinak.
* Komise CAS může v příslušných případech přiznat straně náhradu všech nebo části  výdajů spojených s odvoláním k CAS.
* Rozhodnutí CAS je konečné a závazné pro všechny strany a pro všechny členské federace a proti rozhodnutí CAS není odvolání možné. Rozhodnutí CAS vstupuje v platnost okamžitě a všechny členské federace musí k tomu podniknout všechny nezbytné kroky.
\enditems

\secc Povinná hlášení členské federace

\begitems \style N
* Každá členská federace musí ihned nahlásit IAAF jména atletů, kteří podepsali písemné přijetí a uznání těchto Antidopingových pravidel a procedurálních směrnic a získali tak způsobilost k účasti v mezinárodních soutěžích (viz P 30.3). Kopie podepsané listiny musí členská federace zaslat kanceláři IAAF.
* Každá členská federace musí ihned nahlásit IAAF a WADA každé TUE, které bylo uděleno v souladu s P 34.9.b).
* Každá členská federace musí ihned a za všech okolností, nejpozději do 14 dní, nahlásit IAAF každý nepříznivý analytický nález při dopingových kontrolách prováděných touto federací nebo v zemi či na území této federace spolu se jménem atleta, kterého se nález týká.
* Každá členská federace musí nahlásit antidopingovému administrátorovi IAAF výsledky všech šetření, která provedla podle těchto pravidel o dopingu (viz P 37.2).
* Každá členská federace musí nahlásit, jako součást výroční zprávy, která se předkládá IAAF během prvních tří měsíců každého roku (viz článek 4.9 Statutu IAAF) všechny dopingové kontroly, které v předcházejícím roce provedla sama nebo byly v její zemi či na jejím území provedeny, s výjimkou kontrol provedených IAAF. Jednotlivé záznamy musí být rozděleny podle atletů, s uvedením, kdy byl kontrolován, kdo kontrolu provedl a zda se jednalo o kontrolu během nebo mimo soutěž. IAAF může rozhodnout o pravidelném zveřejňování údajů, které podle tohoto ustanovení od členských federací obdrží.
* IAAF musí každý druhý rok nahlásit WADA, že jak ona, tak členské federace vyhovují Kodexu.
\enditems

\secc Sankce proti členské federaci

\begitems \style N
* Rada je oprávněna, v souladu s článkem 14.7 statutu IAAF, uložit sankce každé členské federaci, která nedodrží své závazky podle těchto Antidopingových pravidel.
* Příkladem nedodržení závazků členské federace vyplývajících z Antidopingových pravidel jsou :
  \begitems \style a
  * opomenutí zahrnout tato Pravidla a předpisy o dopingu do svých pravidel nebo předpisů v souladu s P 30.2
  * opomenutí zajistit způsobilost atleta k účasti na soutěžích jeho písemným souhlasem s pravidly a předpisy o dopingu a zasláním kopie tohoto souhlasu kanceláři IAAF v souladu s P 30.3;
  * nerespektování rozhodnutí Rady v souladu s P 30.6;
  * neuskutečnění slyšení atleta do tří měsíců poté, co je o to požádána (viz P 38.9);
  * nevyvinutí dostatečného úsilí pro získání údajů o pobytu atletů na žádost IAAF (viz P 35.17) a/nebo zaslání neověřené či neúplné informace o pobytu atleta (viz P 38,19);
  * bránění, překážení nebo manipulování s mimosoutěžními kontrolami prováděnými IAAF, jiným členem WADA nebo jiným oprávněným orgánem (viz P 35.13).
  * nezaslání hlášení IAAF a WADA o udělených TUE podle P 34.5.b) (viz P 43.2);
  * opomenutí nejpozději do 14 dní nahlásit IAAF nepříznivý analytický nález při dopingových kontrolách prováděných touto federací nebo v zemi či na území této federace spolu se jménem atleta, kterého se nález týká (viz P 43.3);
  * opomenutí vést řádné disciplinární jednání v souladu s těmito Pravidly o dopingu, vč. opomenutí ohlásit Dopingové revizní komisi případy, týkající se atletů mezinárodní úrovně v záležitostech spojených výjimečnými / zvláštními okolnostmi (viz P 38.19);
  * opomenutí nahlásit antidopingovému administrátorovi IAAF výsledky všech šetření, která provedla podle těchto pravidel o dopingu (viz P 37.2).
  * opomenutí potrestat atleta za provinění proti pravidlům o dopingu v souladu s příslušnými pravidly;
  * odmítnutí nebo opomenutí vést na žádost IAAF šetření o možném porušení pravidel o dopingu anebo podat o takovém šetření zprávu ve lhůtě stanovené IAAF (viz P 37.12);
  * opomenutí uvést ve výroční zprávě, předkládané IAAF během prvních 3 měsíců každého roku, všechny dopingové kontroly, které v předcházejícím roce provedla sama nebo byly v její zemi či na jejím území provedeny (viz P 43.5).
  \enditems
* Pokud vše nasvědčuje, že členská federace nedodržela své závazky dle těchto Antidopingových pravidel, je Rada oprávněna učinit jeden nebo více následujících kroků
  \begitems \style a
  * zastavit členské federaci činnost do následujícího Kongresu nebo na jakoukoliv kratší dobu;
  * napomenout nebo odsoudit tuto členskou federaci;
  * udělit pokutu;
  * odebrat členské federaci grant nebo podporu;
  * vyloučit atlety této členské federace z jedné nebo více mezinárodních soutěží;
  * zrušit či odmítnout akreditaci činovníků i jiných představitelů této členské federace;
  * zavést jakékoliv jiné sankce, které považuje za vhodné.
  \enditems

 Rada může čas od času vydat přehled sankcí, jimiž mohou být členské federace postiženy za porušení svých závazků podle P 44.2. Každý takový seznam a jeho změny musí být sděleny všem členům a zveřejněny na internetových stránkách IAAF.

* V kterémkoliv případu zavedení sankcí proti členské federaci za porušení závazků, plynoucích z Antidopingových pravidel, musí Rada o svém rozhodnutí podat zprávu následujícímu Kongresu.
\enditems

\secc Uznání výsledků kontrol

\begitems \style N
* Každé konečné rozhodnutí učiněné v souladu s těmito antidopingovými pravidly musí být uznáno IAAF a jejími členskými organizacemi, které musí učinit veškerá opatření pro jeho účinnost
* S výhradou práva na odvolání podle P 42, kontroly a TUE provedené v atletice kterýmkoliv signatářem, které v souladu s antidopingovými pravidly a předpisy musí být uznány IAAF a jejími členskými organizacemi
* Rada IAAF může jménem všech členských federací uznat výsledky dopingových testů provedených v atletice jiným orgánem než signatářem podle pravidel a postupů lišících se od těchto antidopingových pravidel a předpisů, je-li přesvědčena, že testy byly řádně provedeny a pravidla organizace, která je prováděla, jinak odpovídají těmto antidopingovým pravidlům a předpisům.
* Rada může svoji odpovědnost za uznání výsledků dopingových kontrol podle P 45.3 delegovat na Dopingovou revizní komisi nebo na jinou osobu nebo orgán, který bude považovat za vhodný.
* Jestliže Rada (nebo osoba Radou určená podle P 45.4) rozhodne, že uznává výsledek dopingové zkoušky provedené jiným orgánem než signatářem, pak je dotyčný atlet považován za sportovce, který porušil odpovídající pravidla IAAF a vztahují se na něj stejná disciplinární řízení, jako při porušení obdobných ustanovení těchto Antidopingových pravidel. Všechny členské federace musí podniknout všechna nezbytná opatření, aby se jakékoliv rozhodnutí, které se týká porušení pravidel o dopingu, stalo účinné.
* Výsledky kontrol, rozhodnutí o TUE a závěry slyšení a ostatní konečná rozhodnutí kteréhokoliv signatáře učiněná v jiném sportu než atletice, která odpovídají těmto antidopingovým pravidlům a jsou učiněna v rámci oprávnění signatáře, musí být uznána a respektována IAAF a jejími členskými federacemi.
* IAAF a jejím členské federace musí uznat opatření uvedená v P 45.6 učiněná orgány, které nepřijaly kodex v jiném sportu než atletice, pokud pravidla těchto orgánů jinak odpovídají antidopingovým pravidlům a předpisům.
\enditems

\secc Časové omezení

\begitems \style N
* Podle těchto Antidopingových pravidel nesmí být zahájeno žádné disciplinární řízení za porušení některého z ustanovení těchto pravidel atletem nebo jinou osobou, pokud se tak nestane do osmi let od data, kdy k takovému porušení došlo.
\enditems

\secc Interpretace ustanovení těchto pravidel

\begitems \style N
* Antidopingová pravidla jsou svojí povahou soutěžními pravidly, stanovujícími podmínky, za nichž se má atletika konat. Nejsou určena k tomu, aby podléhala nebo byla omezována požadavky a právními normami platícími pro trestní řízení nebo zaměstnanecké záležitosti. Postoje a normy vyjádřené v Kodexu jako základ pro boj s dopingem ve sportu reprezentují široký konsenzus těch, kteří mají zájem o čestný sport a měly by být respektovány všemi soudy a rozhodčími orgány.
* Tato antidopingová pravidla musí být vykládána jako nezávislý a autonomní text a nikoliv pomocí odkazů na platné zákony a nařízení signatářů a vlád.
* Různé titulky a podtitulky použité v těchto Antidopingových pravidlech jsou použity pouze pro snazší orientaci v textu a netvoří podstatu těchto pravidel a nemají vliv na význam ustanovení, ke kterým se vztahují.
* Definice pojmů uvedených v Oddíle 3 tvoří nedílnou součást těchto Antidopingových pravidel..
* V případě konfliktu mezi těmito Antidopingovými pravidly a Kodexem, rozhoduje znění těchto antidopingových pravidel.
\enditems

\endinput