\sec SOUTĚŽE V POLI

\secc Všeobecná ustanovení -- soutěže v poli

\begitems \style N
Zkušební pokusy v soutěžním sektoru

* Před zahájením soutěže může mít každý atlet v soutěžním sektoru zkušební pokusy. V případě soutěží ve vrhu a hodech budou zkušební pokusy probíhat podle startovního pořadí a vždy za dozoru rozhodčího.
* Po zahájení soutěže není atletům dovoleno používat pro cvičné účely
  \begitems \style a
  * rozběhovou dráhu nebo odrazovou plochu;
  * atletické tyče;
  * náčiní;
  * kruhy nebo plochu celého sektoru jak s náčiním, tak i bez něho.
  \enditems

Pozn.: Porušení tohoto ustanovení je důvodem k okamžitému vyloučení atleta z další účasti v disciplíně bez předchozího upozornění nebo varování žlutou kartou.

Značky

* \begitems \style a
  * Ve všech soutěžích v poli, kde se využívá rozběžiště, mohou být podél něho umístěny značky kromě skoku do výšky, kde mohou být značky umístěny na rozběžišti. Atlet může použít jednu nebo dvě kontrolní značky (dodané nebo schválené pořadatelem) pro usnadnění rozběhu a odrazu. Pokud výše uvedené značky nejsou k dispozici, může atlet použít přilnavou pásku, ale nikoliv křídu nebo podobnou substanci a ani nic jiného co  zanechává nesmazatelnou stopu. Značky musí být umístěné podél rozběhové dráhy, vyjma skoku do výšky, kde mohou být značky umístěny na rozběžišti.
  * Při hodech z kruhu smí atlet použít pouze jednu značku. Tato značka může být umístěná pouze na zemi těsně za nebo přilehlá ke kruhu. Značka smí být na svém místě pouze po dobu trvání vlastního pokusu atleta a nesmí bránit rozhodčím ve výhledu. Žádné osobní značky  nesmějí být umístěny v nebo vedle sektoru pro dopad náčiní.

  POZN.: Každá značka musí být pouze jednodílná.

  * Pro skok o tyči by měl pořadatel umístit vedle rozběhové dráhy vhodné a bezpečné značky a to každých 0,5 m mezi vzdálenostmi  2,5 m až 5,0 m od „nulové“ čáry a dále každý 1,0 m od 5 m až do 18 m.
  * Rozhodčí musí příslušného atleta upozornit, aby upravil nebo odstranil jakoukoliv značku, která neodpovídá tomuto pravidlu. Pokud tak neučiní, musí je odstranit rozhodčí.

  POZN.: Vážnější případy mohou být dále projednány podle P125.5 a P145.2.
  \enditems

Vyznačení výkonů a větrná stélka

* \begitems \style a
  * Výraznou vlajkou nebo tabulí může být vyznačen platný světový rekord a, pokud je to vhodné, platný oblastní, národní rekord nebo rekord mítinku.
  * Jedna nebo více větrných stélek má být umístěno ve vhodné poloze u všech atletických disciplín a v blízkosti soutěžního sektoru pro hodu diskem a hodu oštěpem pro informování atletů o přibližném směru a síle větru.
  \enditems

Pořadí atletů a pokusy

* Vyjma situace, kdy je použito P180.6, atlet nastupují k soutěži v pořadí stanoveném losem

Pokud se kterýkoliv atlet samovolně rozhodne absolvovat svůj pokus v jiném, než stanoveném pořadí, bude uplatněno ustanovení P125.5 a 145.2. V případě napomenutí zůstává výsledek pokusu (zdařený či nezdařený) v platnosti. Pokud se nejprve koná kvalifikace, losuje se pořadí finálové soutěže samostatně.

Pozn.: V soutěžích oddílových družstev může tato pořadí stanovit příslušný řídící pracovník soutěže.

* Vyjma soutěží ve skoku do výšky a skoku o tyči, atlet může mít v každém kole pokusů zaznamenán pouze jediný pokus.

Ve všech disciplínách v poli, kromě soutěží ve skoku do výšky a skoku o tyči, kde soutěží více než osm účastníků, musí být každému z nich povoleny tři pokusy a osmi atletům s nejlepšími zdařenými pokusy musí být povoleny další tři pokusy, pokud není v příslušných propozicích soutěže určeno jinak.

Rovnost výkonů dvou nebo více atletů na posledním postupovém místě musí být řešena podle ustanovení P180.22. Zůstává-li nadále rovnost umístění, všem atletům s rovným umístěním musí být povoleny další pokusy, jejichž počet je dán příslušnými propozicemi soutěže.

Startuje-li osm nebo méně atletů, musí být všem povoleno šest pokusů, pokud není v příslušných propozicích soutěže určeno jinak.  Pokud více než jeden atlet nemá po prvních třech kolech zdařený pokus, nastupují tito atlet k dalším pokusům před atlety se zdařenými pokusy a to ve stejném vzájemném pořadí jaké měli podle vylosování.

V obou případech:
  \begitems \style a
  * atlet nastupují ke svým následujícím dvěma pokusům v obráceném sledu, než je jejich pořadí podle výkonů po prvních třech pokusech, pokud není v příslušných propozicích soutěže určeno jinak.
  * atlet nastupují k poslednímu pokusu v obráceném sledu, než je jejich pořadí podle výkonů po pěti pokusech, pokud není v příslušných propozicích soutěže určeno jinak.
  * Pokud v případě změny pořadí je na některém místě shoda umístění, nastupují tito atlet k dalším pokusům ve stejném vzájemném pořadí, jaké měli podle vylosování.
  \enditems

POZN. 1 : Ustanovení pro vertikální skoky viz P181.2.

POZN. 2:  Pokud se závěrečných tří pokusů účastní atlet, jemuž vrchní rozhodčí podle ustanovení P146.5 povolil dále soutěžit pod protestem, nastupuje k pokusům před ostatními atlety. Pokud dále pod protestem pokračuje více atletů, nastupují tito atleti k pokusům v  původním vzájemném pořadí.

POZN. 3 : řídící orgán soutěže může v propozicích dané soutěže určit počet pokusů (nejvýše šest) a počet startujících, kteří mohou postoupit do dalšího kola po prvních třech pokusech.

Záznam pokusů

* Vyjma skoku vysokého a skoku o tyči musí být zaznamenán naměřený výkon.

Standardně používané zkratky a symboly jsou uvedené v P132.4.

Ukončení pokusu

* Rozhodčí nesmí zvednout bílý praporek na znamení zdařeného pokusu, dokud není pokus ukončen. Rozhodčí může přehodnotit své rozhodnutí, pokud věří, že zvedl nesprávný praporek.

Pokus je ukončen jako zdařený:
  \begitems \style a
  * v případě vertikálního skoku, jakmile rozhodčí určí, že nedošlo k porušení pravidel ve smyslu ustanovení P 182.2, P183.2 nebo P183.4;
  * v případě horizontálního skoku, jakmile atlet opustí doskočiště v souladu s P185.2;
  * v případě vrhu nebo hodů, jakmile atlet opustí kruh či rozběhovou dráhu v souladu s P187.17.
  \enditems

Kvalifikační kola

* Je-li při soutěžích v poli tak velký počet atletů, že nedovoluje řádné uskutečnění soutěže v jediném kole, musí se konat kvalifikační kolo. Pokud se kvalifikační kolo koná, musí jím projít všichni startující a do dalšího kola se kvalifikovat, vyjma případů, kdy řídící orgán dané soutěže rozhodne, že v jedné nebo více jednotlivých soutěží se uskuteční předběžné kvalifikační kolo (nebo kola) a to buď během dané soutěže, nebo při jedné nebo více předcházejících soutěží. Toto předběžné kvalifikační kolo rozhodne o některých nebo všech atletech, kteří budou oprávnění soutěžit v některém z dalších kol soutěže. Takový postup a další podmínky (jako je dosažení účastnického limitu během stanovené doby, dosažení stanoveného umístění v předem určených soutěžích nebo umístění na žebříčku), za nichž je atlet oprávněn k účasti v určitém kole soutěže, musí být stanoveno v propozicích daných soutěží.

Výkony dosažené v kvalifikační soutěži nebo předběžném kvalifikačním kole se do finálové soutěže nezapočítávají.

* Kvalifikační soutěž musí být rozdělena alespoň do dvou skupin nahodile sestavených, ale kdykoliv je to možné, je třeba účastníky jedné členské země rozdělit do různých skupin. Pokud nejsou k dispozici zařízení, aby tyto skupiny soutěžily ve stejnou dobu a za stejných podmínek, začíná každá skupina své zkušební pokusy v soutěžním sektoru ihned po ukončení soutěže předcházející skupiny.

Pozn.: Pořadí, v němž jednotlivé skupiny soutěží, může být určeno losem.

* Doporučuje se, aby při soutěžích, trvajících déle než tři dny, byl u soutěží ve vertikálních skocích mezi kvalifikačním kolem a finálovou soutěží jeden den odpočinku.
* Podmínky kvalifikačního kola, kvalifikační limit a počet finalistů musí stanovit technický delegát(i). Pokud nebyl jmenován, určí tyto podmínky pořadatel. Při soutěžích uvedených v P1.1.a) ,b), c) má být ve finále alespoň 12 atletů, pokud v rozpisu soutěží není určeno jinak.
* V kvalifikační soutěži, vyjma skoku do výšky a skoku o tyči, má atlet povoleny až tři pokusy. Jakmile atlet splnil kvalifikační limit, nesmí v kvalifikační soutěži dále pokračovat.
* Při kvalifikační soutěži ve skoku do výšky a skoku o tyči musí atlet, který není vyřazen po třech po sobě následujících nezdařených pokusech, pokračovat v soutěži dle P181.2 (vč. vynechávání pokusů) až do posledního pokusu na výšce určené jako kvalifikační limit, pokud již předtím nebylo dosaženo stanoveného počtu atletů pro finále dle P180.12. Jakmile je rozhodnuto, že atlet postupuje do finále, nesmí v kvalifikační soutěži pokračovat.
* Pokud stanoveného kvalifikačního limitu nedosáhne žádný atlet nebo jej splní méně atletů než je požadovaný počet, musí být počet finalistů rozšířen na tento počet o další atlety podle výkonů dosažených v kvalifikační soutěži. Rovnost výkonů dvou nebo více atletů na posledním postupovém místě musí být řešena podle ustanovení P180.22 nebo P181.8. Zůstává-li nadále rovnost umístění, všichni atlet s rovným umístěním postupují do finále.
* Koná-li se kvalifikační soutěž ve skoku do výšky nebo skoku o tyči ve dvou skupinách souběžně, doporučuje se zvyšovat laťku na každou výšku současně v obou skupinách. Rovněž se doporučuje, aby obě skupiny byly přibližně stejně silné.

Doba na provedení pokusu

* Odpovědný rozhodčí dá atletovi najevo, že je vše připraveno k zahájení pokusu a tímto okamžikem běží doba, kterou má atlet pro pokus vymezen.

Ve skoku o tyči začíná čas běžet od okamžiku, kdy jsou stojany nastavené podle předchozího přání atleta. Další doba pro změnu nastavení není možná.

Pokud vymezený čas uplynul až poté, co atlet svůj pokus zahájil, nesmí být proto jeho pokus prohlášen za nezdařený. Takový pokus je platným pokusem.

Pokud se atlet rozhodne pokus neabsolvovat poté, co začal běžet čas pro tento pokus vymezený, bude tento pokus považován za nezdařený, jakmile uplyne doba pro něj vyměřená.

Pro provedení pokusu musí být dodrženy následující časy. Pokud uvedený čas není dodržený a není uplatněno P180.18, bude pokus zaznamenán jako nezdařený.

Individuální soutěže

počet atletů v soutěži	výška 	tyč	ostatní
více než 3
(nebo před zcela prvními
pokusy každého atleta)	1 min	1 min	1 min
2 nebo 3			1,5 mi.	2 min	1 min
1			3 min	5 min	     -
po sobě následující pokusy 	2 min	3 min	2 min

Víceboje

počet atletů v soutěži	výška 	tyč	ostatní
více než 3
(nebo před zcela prvními
pokusy každého atleta)	1 min.	1 min.	1 min.
2 nebo 3			1,5 min.	2 min    	1 min.
1 	                  		2 min.	3 min.	   -
po sobě následující pokusy 	2 min.	3 min.	2 min

POZN. 1: Hodiny ukazující zbývající čas mají být pro atleta viditelné. Navíc musí rozhodčí držet zvednutý žlutý praporek nebo jinak signalizovat, že zbývá posledních 15 sekund povoleného časového limitu.

POZN. 2 : Ve skoku vysokém a skoku o tyči se jakákoliv změna doby vymezené pro provedení pokusu (vyjma doby stanovené pro po sobě následující pokusy jednoho atleta) musí provést až po změně výšky laťky. V ostatních soutěžích v poli (vyjma doby stanovené pro po sobě následující pokusy jednoho atleta), se povolená doba nemění.

POZN. 3: Při stanovení počtu atletů zbývajících v soutěži je třeba vždy vzít v úvahu i ty, kteří by se mohli zúčastnit rozeskakování o první místo.

POZN. 4: Pouze pokud v soutěži ve skoku vysokém nebo skoku o tyči zůstal jediný atlet (a soutěž vyhrál) a pokouší se o světový rekord nebo jiný rekord vzhledem k dané soutěži, zvyšuje se výše uvedená doba o jednu minutu.

Náhradní pokusy

* Je-li atlet z nějakého, na něm nezávislého, důvodu v provádění pokusu omezován a pokus nemůže absolvovat nebo pokus nemůže být správně zaznamenán, příslušný vrchní rozhodčí má právo povolit mu náhradní pokus nebo vrátit zcela či částečně čas vymezený pro pokus. Náhradní pokus není možné provést mimo stanovené pořadí atletů. Pro provedení náhradního pokusu bude podle okolností povolena přiměřená doba. Pokud soutěž mezitím pokročila, musí být náhradní pokus proveden dříve než jakékoliv další pokusy.

Opuštění soutěžního sektoru

* Atlet může opustit soutěžní prostor během konání soutěže jen se souhlasem a v doprovodu rozhodčího. V opačném případě musí být nejdříve varován, ale při opakování nebo ve vážném případě musí být atlet diskvalifikován.

Změna místa nebo doby konání soutěže

* Technický delegát nebo příslušný vrchní rozhodčí má právo změnit místo nebo dobu konání soutěže, pokud to podle jeho mínění okolnosti vyžadují. Tato změna však může nastat pouze po ukončení probíhajícího soutěžního kola.

POZN.: Ani síla větru, ani změna jeho směru nejsou dostatečným důvodem ani pro změnu místa ani doby konání soutěže.

Konečné výsledky

* Každému atletu musí být do výsledků soutěže zapsán nejlepší výkon ze všech absolvovaných pokusů. V případě skoku vysokého a skoku o tyči vč. výkonu dosaženého při řešení shodného umístění na prvním místě.

Rovnost výkonů

* V soutěži v poli, vyjma skoku do výšky a skoku o tyči, rozhoduje o pořadí při rovnosti výkonů druhý nejlepší výkon atletů s rovností nejlepšího výkonu. Je-li i ten stejný, rozhoduje třetí nejlepší výkon, atd. Pokud rovnost umístění trvá i po uplatnění tohoto pravidla, je tato rovnost konečná, a to i v případě prvního místa.

Vyjma vertikálních skoků, v případě rovnosti umístění na kterémkoliv místě, včetně prvního, rovnost zůstává.

POZN.: Ustanovení pro vertikální skoky viz P181.8 a P181.9.
\enditems


A. Vertikální skoky

\secc Všeobecná ustanovení - vertikální skoky

\begitems \style N
* Před zahájením soutěže musí vrchní rozhodčí nebo vrchník oznámit atletům základní výšku a následné výšky, na něž bude laťka zvyšována po ukončení každého kola, dokud v soutěži nezůstane jediný atlet, který již soutěž vyhrál nebo nedojde k rovnosti výkonů na prvním místě. (Ustanovení pro víceboje viz P200.8.d)

Pokusy

* Atlet může začít skákat na kterékoliv výšce předtím oznámené vrchním rozhodčím nebo vrchníkem a pokračovat dle vlastního uvážení na kterékoliv následující výšce. Tři za sebou následující nezdařené pokusy, bez ohledu na výšku, na které k nezdařeným pokusům došlo, znamenají vyřazení atleta z další soutěže, s výjimkou případu rovnosti výkonů na prvním místě.

Důsledkem tohoto pravidla je, že atlet může vynechat druhý, případně třetí pokus na kterékoliv výšce po nezdařeném prvním, resp. druhém pokusu na této výšce, a přesto pokračovat v soutěži na následující výšce.

Jestliže atlet na určité výšce pokus vynechá, nemůže již na této výšce vykonat žádný další pokus, vyjma případu rovnosti výkonů na prvním místě.

Pokud při soutěži ve skoku vysokém nebo skoku o tyči není atlet přítomný, když všichni ostatní přítomní atlet soutěž ukončili, musí to být vrchním rozhodčím, poté, co uplynula příslušná doba pro provedení pokusu, považováno za ukončení soutěže tímto atletem

* Atlet má právo skákat dále, i když ostatní již byli ze soutěže vyřazeni, dokud sám neztratí právo v soutěži pokračovat.
* Dokud v soutěži nezbývá pouze jediný atlet, který již soutěž vyhrál, platí:
  \begitems \style a
  * po každém kole se laťka nesmí zvýšit nikdy méně než o 2 cm při skoku vysokém a o 5 cm při skoku o tyči a
  * hodnota, o níž se laťka zvyšuje, se v průběhu zvyšování nesmí nikdy zvýšit.
  \enditems

Ustanovení P181.4.a), b) neplatí, rozhodne-li se atlet pro zvýšení na úroveň nového světového rekordu (nebo jiného rekordu odpovídajícího soutěži).

Pozn.: Při soutěžích ČAS platí obdobné ustanovení pro pokus o nový ČR rekord.

Poté, co atlet soutěž vyhrál, rozhoduje o dalším zvyšování laťky tento atlet sám, po poradě s příslušným rozhodčím či vrchním rozhodčím.

POZN: Toto ustanovení neplatí pro soutěž v rámci víceboje.

Pozn.: Ustanovení POZN. platí i pro soutěže konané v ČR.

Měření výšky

* Při všech vertikálních skocích se výšky musí měřit v celých centimetrech, kolmo od země k nejnižšímu místu horního okraje laťky.
* Každé měření nové výšky musí být provedeno předtím, než se závodníci pokusí tuto výšku překonat. Při pokusech o rekord musí rozhodčí po umístění laťky na rekordní výšku nastavenou míru překontrolovat, a došlo-li k dotyku laťky, musí před následujícím pokusem nastavenou výšku přeměřit.

Laťka

* Laťka musí být zhotovena ze sklolaminátu či jiného vhodného materiálu, ale nikoliv z kovu. Musí mít kruhový průřez, vyjma dvou koncových dílů pro uložení na podpěrách. Musí mít takovou barvu, aby byla viditelné pro všechny vidoucí atlet. Celková délka laťky musí být 4,00 m ($\pm$0,02 m) pro skok vysoký a 4,50 m ($\pm$0,02 m) pro skok o tyči. Maximální hmotnost musí být 2,0 kg pro skok vysoký a 2,25 kg pro skok o tyči. Průměr dílu o kruhovém průřezu musí být 30 mm ($\pm$1 mm).

Každý z koncových dílů musí být široký 30 až 35 mm a dlouhý 0,150 m až 0,20 m.

Koncové díly musí mít kruhovitý nebo půlkruhovitý průřez s jednou jednoznačně definovanou rovnou plochou pro uložení laťky na podpěry. Rovná plocha nesmí být výše, než je střed svislého řezu laťkou. Koncové díly musí být tvrdé a hladké a nesmí být z gumy nebo pokryty gumou nebo jiným materiálem, zvyšujícím tření mezi nimi a podpěrami.

Po uložení na podpěry se laťka pro skok vysoký smí prohnout nejvýše o 20 mm a laťka pro skok o tyči nejvýše o 30 mm.

Kontrola tuhosti : Při zavěšení závaží o hmotnosti 3 kg uprostřed laťky se laťka pro skok vysoký smí prohnout nejvýše o 70 mm a laťka pro  skok o tyči nejvýše o 110 mm.

Obr. 181 - Koncové díly laťky - průřez

Rovnost výkonů

* Pokud dva nebo více atletů překoná stejnou konečnou výšku, bude o jejich pořadí rozhodnuto takto:
  \begitems \style a
  * Lepší umístění se přizná atletovi, který na nejvýše zdolané výšce, na níž k rovnosti výkonů došlo, měl nejméně pokusů.
  * Pokud při uplatnění ustanovení odstavce 8.a) rovnost umístění stále trvá, přizná se lepší umístění atletovi, který měl v soutěži nejmenší počet nezdařených pokusů až po nejvyšší zdolanou výšku včetně.
  * Trvá-li rovnost výkonů i po uplatnění ustanovení odstavce 8.b), bude atletům přiznáno stejné umístění v soutěži, pokud se nejedná o první místo.
  * Pokud se jedná o první místo, bude provedeno rozeskakování podle ustanovení P181.9, pokud tak bylo předem určeno soutěžním řádem nebo propozicemi dané soutěže nebo během soutěží, avšak před začátkem dané disciplíny, Technickým delegátem (nebo vrchním rozhodčím, pokud Technický delegát nebyl jmenován). Pokud se rozeskakování nekoná nebo se atlet, kteří by se měli rozeskakovat, v kterékoliv fázi soutěže rozhodnou dále neskákat, rovnost umístění na prvním místě zůstává.
  \enditems

POZN: Pravidlo 181.8.d) neplatí pro víceboje.

Rozeskakování

* \begitems \style a
  * Zúčastnění atlet musí skákat na každé výšce až do rozhodnutí, nebo dokud se všichni nerozhodnou dále neskákat.
  * Každý atlet má na každé výšce pouze jeden pokus.
  * Rozeskakování začíná další na výšce, určené podle ustanovení P181.1, která následuje po poslední výšce zdolané atlety.
  * Pokud nedojde k rozhodnutí, laťka se zvyšuje, pokud závodníci byli úspěšní, nebo snižuje, pokud nebyli, vždy o 2 cm při skoku do výšky a o 5 cm při skoku o tyči.
  * Pokud atlet na některé výšce svůj pokus neprovede, automaticky ztrácí právo na lepší umístění. Zůstane-li pak v soutěži již jen jeden další atlet, bude prohlášen vítězem bez ohledu na to, zda se pokusil danou výšku překonat.
  \enditems

POZN.: Ustanovení P181.9) neplatí pro soutěže družstev řízené ČAS.

Skok do výšky - PŘÍKLAD

Před  zahájením  soutěže  oznámil  vrchník  tyto postupné výšky: 180, 184, 188, 191, 194, 197, 199

Všech pět atletů zdolalo výšku 188, všichni neuspěli na dalších postupných výškách. Uplatňují se pravidla o rovnosti výkonů, P181.8 a P181.9.

Atlet A, B a C zdolali 188 na druhý pokus a všichni mají po dvou nezdařených pokusech. Proto se o první místo rozeskakovali a to na 191, následující (v zápise) po poslední výšce (188) zdolané atlety. Závodníci D a E překonali výšku 188 rovněž na druhý pokus, ale mají po třech nezdařených pokusech. Jelikož se nejedná o pořadí na prvním místě, je oběma přiznáno umístění na čtvrtém místě. Všichni tři atlet na výšce 191 neuspěli, proto se laťka snižuje na 189. Atlet A a B tuto výšku zdolali a proto pokračují na 191. Atlet C výšku 189 nyní nezdolal, a proto mu patří třetí místo.  Na výšce 191 tentokrát uspěl jen atlet B a stává se vítězem.

Mimořádné okolnosti

* Pokud laťka opustí stojany a přitom je evidentní, že to bylo způsobeno jinou silou, než v důsledku činnosti atleta při pokusu o překonání nastavené výšky (např. vlivem závanu větru), pak, pokud
  \begitems \style a
  * k pádu laťky došlo poté, kdy atlet výšku zdolal, aniž se laťky dotkl, musí být tento pokus považován za zdařený, nebo
  * k pádu laťky došlo za jakýchkoliv jiných okolností, jedná se o neplatný pokus a atletovi bude přiznán nový pokus.
  \enditems
\enditems

\secc Skok do výšky

\begitems \style N
* Atlet se v každém případě musí odrazit jednou nohou.
* Za nezdařený pokus se považuje, jestliže:
  \begitems \style a
  * po skoku laťka nezůstane na stojanech (podpěrách) v důsledku činnosti atleta při pokusu o její překonání nebo
  * atlet se kteroukoliv částí těla dotkne země, včetně doskočiště, za svislou rovinou proloženou předním okrajem laťky, ať již mezi stojany nebo mimo ně, aniž by napřed překonal laťku. Pokud se atlet při skoku dotkne nohou doskočiště, a podle názoru rozhodčího tím nezískal žádnou výhodu, takový skok nesmí být považován za nezdařený.

  Pozn.: Pro usnadnění aplikace tohoto ustanovení musí být na zemi mezi stojany vyznačena (přilnavou páskou nebo podobným materiálem) čára široká 50 mm, jejíž přední hrana leží ve svislé rovině procházející předním okrajem laťky, a je prodloužená do vzdálenosti 3 m za každý z obou stojanů.

  * Atlet se při rozběhu dotkne laťky nebo svislé části stojanů, aniž by skočil.
  \enditems

Rozběhová plocha a odraziště

* Minimální šířka rozběhové plochy musí být 16 m a minimální délka rozběhu musí být 15 m. Při soutěžích uvedených v P1.a),b),c), musí být minimální délka rozběhu 25 m.
* Nejvyšší sklon posledních 15 m rozběhové plochy nesmí překročit hodnotu 1:250, měřeno podél kterékoliv radiály polokruhové plochy se středem uprostřed stojanů a mající minimální poloměr v souladu s P182.3. Doskočiště by mělo být umístěno tak, že atlet se rozbíhá „směrem vzhůru“.
* Odraziště musí mít rovnou plochu a jakýkoliv sklon musí odpovídat ustanovení P182.4 a Manuálu IAAF pro Atletická zařízení a nářadí.("IAAF Track and Field Facilities Manual").

POZN.: Při použití přenosných pásů se všechna ustanovení týkající se plochy odraziště vztahují na plochu horního povrchu takových pásů.

Nářadí

* Lze použít jakýchkoliv stojanů tuhé konstrukce. Stojany musí mít pevně uchycené podpěry pro laťku. Stojany musí být tak vysoké, že příslušnou výšku, na níž je laťka zvednuta, přesahují vždy alespoň o 0,10 m.

Vzdálenost mezi stojany musí být v rozmezí 4,00 m až 4,04 m.

* V průběhu soutěže se stojany nesmějí přesunovat, pokud vrchní rozhodčí neuzná odraziště nebo doskočiště za nezpůsobilé.

V takovém případě může být změna provedena až po dokončení právě probíhajícího kola.

* Podpěry pro laťku musí být ploché, široké 40 mm, dlouhé 60 mm a musí se stojany svírat pravý úhel a musí vždy směřovat k protilehlému stojanu. Podpěry musí být ke stojanům pevně uchyceny a musí být během skoku nehybné. Laťka musí svými konci na podpěrách spočívat tak, aby dotykem atleta lehce spadla na zem, ať již dopředu nebo dozadu. Podpěry musí mít hladký povrch.

Podpěry nesmějí být zhotoveny z gumy či pokryty gumou nebo jiným materiálem, který by zvyšoval tření mezi nimi a povrchem laťky. Nesmějí mít žádné pružiny.

Podpěry musí mít těsně pod oběma konci laťky stejnou výšku nad odrazištěm.

Obr. 182 - Stojany a laťka pro skok do výšky

* Mezi konci laťky a stojany musí být mezera min. 10 mm.

Doskočiště

* Pro všechny soutěže uvedené v P1.1.a), b), c), e), f) rozměry doskočiště nesmí být menší než 6,0 m x 4,0 m x 0,7 m (tj. délka x šířka x výška), měřeno za svislou rovinou proloženou laťkou.

POZN.: Umístění stojanů a doskočiště je třeba řešit tak, aby mezi nimi byla mezera nejméně 0,10 m a laťka nemohla být shozena pohybem doskočiště a jeho následným dotykem se stojany.

Pozn. 1: Doskočiště musí být umístěno tak, aby žádná jeho část neprotínala svislou rovinu proloženou oběma stojany.

Pozn. 2: Pro ostatní soutěže doskočiště musí mít rozměry alespoň 5,0 m x 3,0 m x 0,7 m.
\enditems

\secc Skok o tyči

\begitems \style N
Závod

* Atlet si může nechat laťku posunout pouze směrem k doskočišti tak, že přední okraj laťky (směřující k rozběhové dráze) se dostane do kteréhokoliv místa od svislé roviny proložené vnitřní hranou horního okraje zarážecí desky skřínky do vzdálenosti 80 cm směrem do doskočiště.

Atlet musí před zahájením soutěže oznámit odpovědnému rozhodčímu, jakou polohu stojanů nebo podpěr si přeje nastavit pro své pokusy a tato hodnota musí být zaznamenána.

Pokud chce atlet provést následně nějaké změny, musí to oznámit odpovědnému rozhodčímu před nastavením stojanů dle jeho původního požadavku. Opomenutí tohoto nařízení znamená, že atletovi začne běžet časový limit na provedení pokusu.

POZN.: Na úrovni vnitřní hrany horního okraje zarážecí stěny skříňky musí být kolmo na osu rozběhové dráhy nakreslena 10 mm široká čára (tzv. nulová čára) výrazné barvy. Obdobná čára, široká až 50 mm, jejíž přední hrana navazuje na přední hranu nulové čáry, musí být nakreslena po povrchu doskočiště až po vnější hranu stojanů.

* Za nezdařený pokus se považuje, jestliže
  \begitems \style a
  * po skoku laťka nezůstane na obou kolících v důsledku činnosti atleta při pokusu o její překonání nebo
  * atlet se kteroukoliv částí těla nebo tyče dotkne země či doskočiště za svislou rovinou proloženou vnitřní hranou horního okraje zarážecí desky skřínky, aniž by předtím překonal laťku, nebo
  * po odrazu od země přesune na tyči spodní ruku nad vrchní nebo posune vrchní ruku výše po tyči nebo
  * během skoku atlet rukou (rukama) ustálí nebo vrátí laťku.
  \enditems

POZN. 1: Za nezdařený pokus se nepovažuje, pokud atlet při rozběhu překročí v kterémkoliv místě bílou čáru, která vymezuje rozběhovou dráhu.

POZN. 2: Dotkne-li se tyč během pokusu doskočiště, nelze to považovat za nezdařený pokus, pokud tato tyč byla předtím řádně zasunuta do skříňky.

* Atletům je během soutěže dovoleno nanášet na ruce nebo na tyč hmotu usnadňující držení tyče. Použití rukavic je dovoleno.
* Nikdo se nesmí dotknout tyče, dokud nepadá směrem od laťky či stojanů. Pokud se tak stane a vrchník usoudí, že bez tohoto zásahu by laťka byla shozena, musí být tento skok zaznamenán jako nezdařený.
* Jestliže se atletovi při pokusu zlomí tyč, musí to být považováno za neplatný pokus a atlet má právo na nový pokus.

Rozběhová dráha

* Délka rozběhové dráhy, měřeno od \uv{nulové} čáry, musí být alespoň 40 m a tam, kde je to možné, 45 m. Její šířka musí být 1,22 m $\pm$0,01 m. Musí být vyznačena bílými čarami o šířce 50 mm.

POZN.: Pro všechna rozběžiště zřízená před 1. 1. 2004 je maximální šířka 1,25 m. Při obnově povrchu musí šířka drah odpovídat tomuto pravidlu.

* Příčný sklon rozběhové dráhy nemá překročit hodnotu 1:100 (1 \%), pokud zvláštní okolnosti neospravedlňují výjimku udělenou IAAF a celkový sklon ve směru rozběhu na posledních 40 m nesmí překročit hodnotu 1:1000 (0,1 %).

Nářadí

* Při odrazu musí být tyč zasunuta do skříňky se zarážecí deskou. Skříňka musí být zhotovena z vhodného materiálu, zapuštěna do úrovně rozběžiště, se zakulacenými nebo měkkými horními hranami. Musí být dlouhá 1 m, měřeno podél vnitřního povrchu dna skříňky, které je na předním okraji široké 0,60 m a postupně se zužuje na 0,15 m na spodním okraji zarážecí desky. Délka skřínky v úrovni země a hloubka zarážecí desky jsou určeny úhlem $105^\circ$ mezi zarážecí deskou a dnem skříňky. (Tolerance rozměrů je $\pm$0,01 m, tolerance úhlů je $+1^\circ$).

Obr. 183a -- Skřínka pro skok o tyči

Dno skříňky se musí svažovat od přední hrany na úrovni rozběžiště do hloubky 0,20 m pod úrovní rozběžiště, v místě, kde se stýká se zarážecí deskou. Boční stěny skříňky se postupně rozevírají tak, že v místě styku se zarážecí deskou svírají se dnem skříňky úhel asi $120^\circ$.

POZN.: Atlet může skříňku před kterýmkoliv svým pokusem vyložit obložením pro dodatečnou ochranu. Musí tak učinit v době vyhrazené pro pokus a ihned po provedení svého pokusu musí obložení odstranit. Při soutěžích uvedených v P1.1.a), b), c), e), f) musí obložení poskytnout pořadatel.

* Lze použít jakýchkoliv stojanů tuhé konstrukce. Kovová konstrukce podstavce stojanů a spodní část stojanů nad úrovní doskočiště musí být bezpodmínečně pokryty polštářováním z vhodného materiálu, poskytujícího ochranu atletům a tyčím.
* Laťka musí být pokládána na vodorovně umístěné podpěry ve tvaru kolíků, upevněných na ramenech, vertikálně posuvných po stojanech. Laťka musí na kolících spočívat tak, aby lehce spadla na zem směrem do doskočiště, jakmile se jí atlet nebo jeho tyč dotkne. Kolíky nesmějí mít jakékoliv zářezy nebo vrypy a musejí mít po celé délce jednotný průměr nepřevyšující 13 mm.

Kolíky nesmějí vyčnívat více než 55 mm z ramen, která musí být hladká. Svislá opěra kolíků musí být rovněž hladká a upravená tak, aby laťka nemohla zůstat ležet na jejím vrchu, který musí ležet 35 až 40 mm nad kolíky.

Obr. 183b - podpěry pro laťku

Vzdálenost mezi kolíky musí být 4,28 až 4,37 m. Kolíky nesmějí být zhotoveny z gumy nebo pokryty gumou či jiným materiálem, který zvyšuje tření mezi nimi a laťkou a nesmějí mít žádná pera. Laťka se má o kolíky opírat středy svých koncových dílů.

POZN.: Pro snížení nebezpečí zranění atletů pádem na podstavce stojanů mohou být kolíky podpírající laťku umístěny na prodlužovací ramena natrvalo připevněná ke stojanům. To umožňuje umístit stojany dále od sebe bez prodloužení laťky (viz obr.183b).

Pozn.: Jsou-li stojany nebo prodlužovací ramena opatřena nástavci se soustavou kolíků (pro snadnější nastavování výšky laťky během tréninku), musí být při závodech laťka položena na nejvyšším kolíku takového zařízení.

Atletické tyče

* Závodníci mohou používat vlastní tyče. Žádný atlet nesmí použít tyč jiného atleta bez souhlasu vlastníka.

Tyč může být z libovolného materiálu nebo z kombinace materiálů, libovolné délky i průřezu, ale její základní povrch musí být hladký.

Tyč může být v místě úchopu (pro ochranu rukou) ovinuta vrstvami přilnavé pásky a v místě spodního konce (pro ochranu tyče) páskou nebo jiným vhodným materiálem. Páska v místě úchopu musí vytvořit plynulou vrstvu, vyjma místa vzájemného překrytí a nesmí vyvolat náhlou změnu průměru, např. ve tvaru prstence.

Doskočiště

* Pro všechny soutěže uvedené v P 1.1.a), b), c), e), f), doskočiště musí mít tyto minimální rozměry délka (mimo čelní díly) 6,0 m, šířka 6,0 m a výška 0,8 m. Čelní díly musí být dlouhé alespoň 2 m.

Postranní části doskočiště kolem skřínky musí být od ní vzdáleny 0,10 m až 0,15 m, se sklonem směrem od skřínky o úhlu asi $45^\circ$ (viz obr. 183c)

Obr. 183c - doskočiště pro skok o tyči
\enditems


B. HORIZONTÁLNÍ SKOKY

\secc Všeobecná ustanovení - horizontální skoky

\begitems \style N
Rozběhová dráha

* Délka rozběhové dráhy, měřeno od jejího konce po odrazové břevno, musí být 40 m a, pokud to podmínky dovolují, 45 m. Její šířka musí být 1,22 m $\pm$0,01 m. Musí být vyznačena bílými čarami o šířce 50 mm.

POZN.: Pro všechna rozběžiště zřízená před 1. 1. 2004 je maximální šířka 1,25 m. Při obnově povrchu musí šířka drah odpovídat tomuto pravidlu.

* Příčný sklon rozběhové dráhy nemá překročit hodnotu 1:100 (1 \%), pokud zvláštní okolnosti neospravedlňují výjimku udělenou IAAF a celkový sklon ve směru rozběhu na posledních 40 m nesmí překročit hodnotu 1:1000 (0,1 \%).

Odrazové břevno

* Místo odrazu musí být vyznačeno břevnem zapuštěným do úrovně rozběhové dráhy. Hrana břevna blíže doskočišti se nazývá odrazovou čarou. Bezprostředně za odrazovou čarou musí být umístěna deska s plastelínou pro usnadnění práce rozhodčích.
* Odrazové břevno musí být pravoúhlé, zhotovené ze dřeva nebo jiného vhodného tuhého materiálu, na kterém hřeby obuvi atleta drží a nesklouzne. Musí mít délku 1,22 m ($\pm$0,01 m), šířku 0,20 m ($\pm$0,002 m) a hloubku nejvýše 0,10 m. Musí mít bílou barvu.
* Deska s plastelínou musí být tuhá, zhotovená ze dřeva či jiného vhodného materiálu, po němž hřeby atletické obuvi nesklouznou. Deska musí být široká 0,10 m ($\pm$0,002 m), dlouhá 1,22 m ($\pm$0,01 m) a natřena kontrastní barvou vůči odrazovému břevnu. Pokud možno, plastelína musí mít kontrastní barvu vůči ostatní desce i břevnu. Deska musí být uložena ve výřezu nebo prohlubni rozběhové dráhy na straně odrazového břevna přivrácené k doskočišti a ve výřezu usazena s dostatečnou tuhostí, aby vydržela sílu dopadu nohy atleta. Povrch desky se z úrovně odrazového břevna ve směru rozběhu musí zvedat do výšky 7 mm $\pm$1 mm. Hrany musí být buď zkoseny pod úhlem $45^\circ$ a hrana desky přivrácená k rozběžišti pokrytá vrstvou plastelíny tlustou 1 mm, nebo opatřeny výřezem, který lze vyplnit vrstvou plastelíny se sklonem $45^\circ$ (viz obr. 184a).

Obr. 184a - odrazové břevno a deska s plastelínou

Horní strana desky musí být po celé délce své přední (náběžné) hrany pokryta vrstvou plastelíny o šířce asi 10 mm.

Deska musí být dostatečně tuhá, aby unesla plnou sílu nohy sportovce při odrazu.

Deska pod plastelínou musí být z vhodného materiálu, aby umožnila bezpečný odraz atleta.

Při odstraňování stop zanechaných tretrami atletů mohou být vrstvy plastelíny uhlazovány válečkem nebo vhodně tvarovanou škrabkou.

Pozn.: Je vhodné připravit náhradní desky s plastelínou, aby závod nebyl zdržován odstraňováním stop.

Doskočiště

* Doskočiště musí mít šířku nejméně 2,75 m a nejvýše 3 m. Musí být, je-li to možné, umístěno tak, že jeho osa je totožná s prodlouženou osou rozběhové dráhy.

POZN.: Není-li osa doskočiště totožná s osou rozběhové dráhy, povolená šířka doskočiště musí být vymezena páskou podél jedné, případně obou stran doskočiště (viz obr. 184b).

* Doskočiště musí být naplněno zkypřeným vlhkým pískem a jeho povrch musí být zarovnán do úrovně odrazového břevna.

Obr. 184b - Doskočiště pro skok do dálky a trojskok

Měření výkonů

* Při všech horizontálních skocích se výkony měří na celé centimetry. Pokud naměřená hodnota není v celých centimetrech, dosažený výkon se zaokrouhlí na nejblíže nižší celou hodnotu v centimetrech.
* Délka každého skoku musí být změřena ihned po každém zdařeném pokusu (nebo okamžitě po ústním protestu podle P 146.5), od nejbližší stopy v doskočišti způsobené kteroukoliv částí těla nebo čímkoliv, co bylo připojeno k tělu v okamžiku, kdy zanechalo stopu, kolmo na odrazovou čáru nebo její prodloužení .

Měření rychlosti větru

* Použije se stejný větroměr, jak je popsáno v P163.8 a 163.9. S větroměrem se pracuje v souladu s P163.11 a P184.12 a jeho údaje odečítají v souladu s P163.13.
* Příslušný vrchní rozhodčí zajistí, že větroměr je umístěn ve vzdálenosti 20 m od odrazového břevna. Měření musí být prováděno v úrovni 1,22 m od země a ve vzdálenosti do 2 m od rozběžiště.
* Při soutěžích musí být rychlost větru měřena po dobu 5 sekund od okamžiku, kdy atlet minul značku umístěnou u rozběhové dráhy ve vzdálenosti 40 m od odrazového břevna při skoku do dálky a 35 m při trojskoku. Pokud se atlet rozbíhá ze vzdálenosti kratší než 40 m, resp. 35 m, musí se rychlost větru měřit od okamžiku, kdy se atlet rozběhne.
\enditems

\secc Skok do dálky

\begitems \style N
Soutěž

* Za nezdařený pokus se považuje, jestliže:
  \begitems \style a
  * při vlastním skoku nebo při běhu, aniž skočí, se atlet kteroukoliv částí těla nebo končetin dotkne půdy za odrazovou čarou (vč. kterékoliv části desky s plastelínou),
  * se atlet odrazí na kterékoliv straně vedle odrazového prkna, a to ať před jeho prodloužením nebo za ním,
  * atlet použije při rozběhu či skoku přemetu či salta v jakékoliv podobě,
  * po odrazu, ale před prvním dotykem s doskočištěm, se atlet dotkne rozběhové dráhy nebo země kolem ní nebo země mimo doskočiště, nebo
  * při doskoku (vč. jakékoliv ztráty rovnováhy) dotkne ohraničení doskočiště nebo půdy mimo doskočiště v místě, které je blíže odrazové čáře, než je nejbližší stopa v doskočišti způsobená při tomto skoku, nebo
  * opustí doskočiště jiným způsobem, než jak je popsáno v P 185.2.
  \enditems
* Při opouštění doskočiště se atlet musí chodidlem dotknout jeho okraje nebo půdy vně doskočiště v místě, které je dále od odrazové čáry, než nejbližší stopa v doskočišti, kterou může být jakákoliv stopa uvnitř doskočiště způsobená ztrátou rovnováhy nebo chůzi zpět v místě, které je blíže odrazovému břevnu než počáteční stopa při doskoku.

POZN.: Tento první kontakt je považován za ukončení pokusu.

* Za nezdařený pokus se nepovažuje, jestliže atlet:
  \begitems \style a
  * při rozběhu v kterémkoliv místě překročí bílou čáru vymezující rozběhovou dráhu,
  * vyjma ustanovení P185.1.b)  se odrazí před odrazovým břevnem,
  * vyjma ustanovení P185.1.b)  se při odrazu částí své boty nebo chodidla dotkne půdy vedle odrazového prkna,
  * se při dopadu kteroukoliv částí svého těla nebo čímkoliv, co je v tomto okamžiku spojené s tělem, dotkne okraje doskočiště nebo země mimo doskočiště, nesmí to však být první dotyk nebo v rozporu s ustanovením P185,1.d) či e) výše,
  * se vrací doskočištěm, pokud jej předtím po skoku opustil v souladu s pravidlem P185.2
  \enditems

Odrazová čára

* Vzdálenost mezi odrazovou čárou a vzdálenějším okrajem doskočiště musí být alespoň 10 metrů.
* Přední hrana odrazového břevna musí být umístěna ve vzdálenosti 1 až 3 metry od bližšího okraje doskočiště.
\enditems

\secc Trojskok

\begitems \style N
Pro trojskok platí ustanovení P 184 a P 185 s následujícími odchylkami :

Soutěžní ustanovení

* Trojskok se skládá ze tří skoků (viz odst. 2).
* Skoky musí být provedeny tak, že atlet při prvním skoku doskočí na stejnou nohu, kterou se k prvnímu skoku odrazil, při druhém skoku dopadá na opačnou nohu a z ní se odráží k poslednímu skoku. Nepovažuje se za chybu, dotkne-li se atlet při skoku švihovou nohou země.

POZN.: Pro doskok při prvních dvou skocích neplatí ustanovení P 185.1.d).

Odrazová čára

* Mezi čárou odrazu pro soutěže mužů a vzdáleným okrajem doskočiště musí být vzdálenost alespoň 21 metrů.
* Pro mezinárodní závody musí být samostatné odrazové břevno pro soutěž mužů a pro soutěž žen.  Odrazová čára pro soutěž mužů musí být alespoň 13 m a pro soutěž žen alespoň 11 m od bližšího okraje doskočiště. Pro ostatní soutěže tato vzdálenost musí být úměrná úrovni soutěže.
* Mezi odrazovým břevnem a doskočištěm musí být pro odraz ke druhému a třetímu skoku k dispozici plocha široká 1,22 m $\pm$0,01 m poskytující pevnou a jednotnou plochu pro nášlap.

POZN.: Pro všechna zařízení zřízená před 1.01.2004 je maximální šířka odrazové plochy 1,25 m. Při obnově povrchu musí šířka drah odpovídat tomuto pravidlu.

\itemnum=30
* Pro soutěž dorostu se doporučuje stejná vzdálenost od odrazového břevna k bližšímu okraji doskočiště jako pro soutěž žen.
* V jedné a téže soutěži musí všichni závodníci použít odrazového prkna v jednotné vzdálenosti od doskočiště, pokud vrchní rozhodčí nerozhodne jinak.
\enditems


C. VRH a HODY

\secc Všeobecná ustanovení - vrhačské soutěže

\begitems \style N
Úřední náčiní

* Nářadí a náčiní používané při všech mezinárodních soutěžích musí odpovídat aktuálně platným předpisům IAAF. Může být použito pouze náčiní s certifikátem IAAF. V tabulce jsou váhy náčiní pro jednotlivé kategorie:

Náčiní	dorostenky	ženy	dorostenci	junioři	dorostenci
	U18	U20/ženy	U18	U20	U18
koule	3,000 kg	4,000 kg	5,000 kg	6,000 kg	7,260 kg
Disk	1,000 kg	1,000 kg	1,500 kg	1,750 kg	2,000 kg
kladivo	3,000 kg	4,000 kg	5,000 kg	6,000 kg	7,260 kg
oštěp	   500 g	   600 g	   700 g	   800 g	   800 g

POZN.: Standardní  formuláře žádosti o certifikaci nebo obnovu certifikace a informace o procesu certifikaci  je možno získat přímo z kanceláře  IAAF nebo na web-stránkách IAAF.

* Vyjma dále uvedených ustanovení, veškeré náčiní musí poskytnout pořadatel. Technický delegát může, na základě příslušných řádů, dovolit atletům použít vlastní nebo dodavatelem dodaná náčiní, pokud náčiní má certifikát IAAF, bylo zkontrolováno a označeno jako schválené pořadatelem a bylo dáno k dispozici všem atletům. Nebude ale přijato náčiní, které je již uvedené na seznamu náčiní poskytnutého pořadatelem.

POZN.: Pro účely ustanovení odst. 2 se pojem „náčiní s certifikátem IAAF“ vztahuje i na starší typy, které dříve dostaly certifikát, ale již se nevyrábějí.

* V průběhu soutěží nelze náčiní jakkoliv upravovat.

Pozn.:  Obdobné ustanovení jako jsou odstavce 2 a 3 výše, platí pro soutěže ČAS. Závodníci smějí používat vlastní náčiní ve vrhačských disciplínách (vrhu koulí, hodu diskem, kladivem a oštěpem), před soutěží však musí své náčiní předložit určenému rozhodčímu ke schválení. Rozhodčí označí náčiní značkou, prokazující jeho regulérnost.

Dopomoc

* Za nedovolenou dopomoc se považuje:
  \begitems \style a
  * Stažení dvou nebo více prstů dohromady. Je-li na ruce a prstech použita náplast, může tvořit bandáž, pokud se každý prst může pohybovat samostatně. Bandážování musí být před započetím soutěže ukázáno vrchníkovi.
  * Použití pomůcek jakéhokoliv typu, vč. závaží upevněného na těle během pokusů, které jakkoli může asistovat během pokusu.
  * Použití rukavic, vyjma hodu kladivem. V tomto případě rukavice musí mít na obou stranách hladký povrch a špičky prstů rukavice, kromě palců, musí být odstřižené.
  * Nastříkání nebo nanesení jakékoliv hmoty na povrch kruhu či na boty nebo zdrsnění povrchu kruhu, provedená atletem.
  \enditems

POZN.: Rozhodčí musí podle svého uvážení upozornit atlet, aby se upravili tak, že budou vyhovovat tomuto pravidlu. Pokud tak některý atlet neučiní, jeho pokusy budou prohlášeny za nezdařené. Pokud atlet provedl svůj pokus dříve, než rozhodčí postřehl takovou nesrovnalost, musí o řešení dané situace rozhodnout vrchní rozhodčí. V případech, kdy je přestupek považovaný za vážný, může být též uplatněno P125.5 a P145.2.

Pozn.: Rukou se rozumí úchopová část končetiny, prsty a dlaň, nikoliv zápěstí a předloktí či paže.

* Za nedovolenou dopomoc se nepovažuje:
  \begitems \style a
  * když pro snazší držení náčiní atlet si nanesou na ruce vhodnou hmotu, kladiváři mohou použít takovou hmotu na svých rukavicích a koulaři ji smějí nanést i na krk;
  * když atlet ve vrhu koulí a hodu diskem nanesou na svá náčiní křídu nebo podobnou hmotu.

  Všechny hmoty použité na rukou, rukavicích nebo na náčiní musí být z náčiní snadno odstranitelné použitím vlhké látky a nesmí zanechávat žádné pozůstatky. Pokud tomu tak nebude, musí být uplatněn postup podle Poznámky v P187.4.

  * Použití náplastí na ruce a prstech, které neporušuje ustanovení P187.4.a).
  \enditems

Kruh pro vrh koulí, hod diskem a hod kladivem

* Kruh musí být vymezen obručí zhotovené z páskové oceli, železa nebo jiného vhodného materiálu, a její okraj musí být v úrovni okolního terénu. Obruč musí mít tloušťku alespoň 6 mm a musí být bílá. Okolo kruhu může být beton, plast, asfalt, dřevo nebo jiný vhodný materiál.

Vnitřní plocha kruhu musí být z betonu, asfaltu či jiného pevného, nikoliv kluzkého materiálu. Povrch musí být vodorovný a o 20 mm ($\pm$6 mm) pod horním okrajem obruče kruhu.

Kruh pro vrh koulí může být přenosný, pokud vyhovuje zde uvedeným podmínkám.

* Vnitřní průměr obruče kruhu pro vrh koulí a hod kladivem musí být 2,135 m $\pm$0,005 m a pro hod diskem 2,50 m $\pm$0,005 m.

Pro hod kladivem lze použít kruhu pro hod diskem, pokud je do něj vložena obruč o vnitřním průměru 2,135 m $\pm$0,005 m.

POZN.: Vložená obruč má mít přednostně jinou barvu, než bílou, aby byly jasně zřetelné čáry podle P 187.8.

* Vně kruhu, v délce alespoň 0,75 m na obě strany od vnitřní hrany obruče, musí být vyznačeny bílé čáry široké 50 mm. Tyto čáry mohou být nakresleny nebo vytvořeny ze dřeva či jiného vhodného materiálu. Zadní hrana těchto čar musí tvořit prodloužení teoretické přímky vedené středem kruhu kolmo na osu výseče.

Obr. 187a - kruh pro vrh koulí

Obr. 187b - kruh pro hod diskem

Obr. 187c - kruh pro hod kladivem

Obr. 187d - uspořádání soustředných kruhů pro hod diskem a kladivem

Rozběhová dráha pro hod oštěpem

* Rozběhová dráha musí být dlouhá nejméně 30 m, vyjma soutěží uvedených v P 1.1.a),b),c), f), kde minimální délka musí být 33,5 m. Dovoluji-li to podmínky, má být minimální délka 36,5 m. Musí být podélně vyznačena dvěma rovnoběžnými čarami širokými 50 mm, vzdálenými od sebe 4,0 m. Hod musí být proveden před kruhovým obloukem o poloměru 8,0 m. Oblouk musí být vyznačen bílou čarou širokou 70 mm nebo stejně širokým, bíle natřeným břevnem ze dřeva, zapuštěným do úrovně okolní půdy. Na obou koncích oblouku musí být vyznačeny bílé čáry široké alespoň 70 mm a dlouhé alespoň 0,75 m (měřeno od vnitřní hrany postranních čar rozběhové dráhy) vedené kolmo na podélné čáry rozběhové dráhy. P

Příčný sklon rozběhové dráhy nemá překročit hodnotu 1:100 (1 \%), pokud zvláštní okolnosti neospravedlňují výjimku udělenou IAAF a celkový sklon ve směru rozběhu na posledních 20 m nesmí překročit hodnotu 1:1000 (0,1 \%).

Obr. 187e - rozběhová dráha pro hod oštěpem a výseč pro dopad náčiní (není kresleno v měřítku)

Pozn.: Rozběhová dráha pro hod oštěpem slouží i pro hod míčkem.

Výseč pro dopad náčiní

* Výseč musí být tvořena plochou pokrytou škvárou, trávou nebo jiným vhodným materiálem, na němž náčiní zanechá stopu.
* Nejvyšší sklon plochy výseče pro dopad náčiní ve směru vrhu nesmí překročit hodnotu 1:1000 (0,1 %).
* \begitems \style a
  * Výseč pro dopad náčiní při vrhu koulí a hodech diskem nebo kladivem musí být vyznačena bílými čarami širokými 50 mm, jejichž prodloužené vnitřní hrany se protínají ve středu kruhu a svírají spolu úhel $34,92^\circ$.

  POZN.: Výseč lze přesně vytyčit vyznačením dvou bodů na oblouku o poloměru 20 m, soustředném s kruhem, jejichž vzájemná vzdálenost je 12 m $\pm$0,05 m. Tato vzdálenost se zvětšuje o 0,6 m na každý další 1 m vzdálenosti od středu kruhu.

  * Výseč pro dopad náčiní při hodu oštěpem musí být vyznačena bílými čarami, jejichž vnitřní hrany musí procházet průsečíky odhodového oblouku s rovnoběžnými postranními čarami, jež vymezují rozběhovou dráhu a v prodloužení se musí protínat ve středu křivosti odhodového oblouku (viz náčrtek). Čáry, jež vyznačují výseč, tak svírají úhel $28,96^\circ$.
  \enditems

Pokusy

* Při vrhu koulí, hodu diskem a hodu kladivem se náčiní vypouští z kruhu, při hodu oštěpem z rozběhové dráhy. Pokusy prováděné z kruhu musí být zahájeny z klidového postavení uvnitř kruhu. Atlet se smí dotknout vnitřní stěny obruče, pří vrhu koulí i vnitřní plochy zarážecího břevna (viz P 188.2).
* Pokus je nezdařený, jestliže atlet během pokusu:
  \begitems \style a
  * vypustí oštěp nebo kouli způsobem v rozporu s pravidly, P 188.1 a 193.1;
  * poté, co vstoupil do kruhu a zahájil pokus, se kteroukoliv částí těla dotkne horní plochy obruče (nebo vnitřní hrany horní plochy obruče) nebo země vně kruhu;

  POZN.: Za nezdařený pokus se nepovažuje, pokud tento dotyk nezrychlí jeho pohyb a dojde k němu během první otáčky a je zcela před bílými čarami vyznačenými vně kruhu podle P 187.9.

  * při vrhu koulí se dotkne vnitřního horního okraje zarážecího břevna nebo jeho horní plochy;
  * při hodu oštěpem se kteroukoliv částí těla dotkne čar, které vymezují rozběhovou dráhu nebo terénu mimo ni.
  \enditems

Pozn.: Pokud se po vypuštění disk nebo kterákoliv část kladiva dotkne klece, nepovažuje se to za nezdařený pokus.

* Pokud v průběhu soutěže nebyla výše uvedená pravidla porušena, atlet může již započatý pokus přerušit, položit náčiní do kruhu či rozběhové dráhy nebo mimo a kruh či rozběhovou dráhu opustit.

POZN.: Čas spotřebovaný na přerušený pokus a do zahájení opakovaného pokusu se musí počítat do časového limitu pro přípravu k provedení pokusu podle ustanovení P 180.17.

* Za nezdařený musí být považován pokus, kdy koule, disk, kladivo nebo hlavice oštěpu se při prvním dopadu na zem dotkne čáry vymezující sektor pro dopad náčiní nebo země či jakéhokoliv předmětu (vyjma klece podle P187.14) vně jeho vnitřní hrany.
* Za nezdařený musí být považován pokus, kdy atlet opustí kruh nebo rozběhovou dráhu dříve, než se náčiní dotkne země, nebo:
  \begitems \style a
  * při pokusech prováděných z kruhu, se atlet při opouštění kruhu poprvé dotkne horního okraje obruče nebo terénu vně kruhu  ne zcela za bílou čarou vyznačenou po obou stranách kruhu a pomyslně probíhající jeho středem.

  POZN.: Za opuštění kruhu se považuje již první kontakt s horním okrajem obruče kruhu nebo země mimo kruh.

  * při hodu oštěpem atlet opustí rozběhovou dráhu tak, že jeho první kontakt s postranními čarami této dráhy nebo se zemí mimo rozběžiště není zcela za bílou čárou odhodového oblouku nebo čarami vedenými kolmo na postranní čáry rozběhové dráhy. Za správně opuštěné rozběžiště se považuje i, jestliže po dopadu náčiní udělal atlet první kontakt s nebo vně čáry(nakreslené, teoretické, pomyslné nebo vyznačené značkou vedle rozběžiště) vzdálené čtyři metry od odhodového oblouku. Pokud je atlet za touto čtyř metrovou pomyslnou čárou uvnitř rozběžiště ve chvíli dopadu náčiní je to taktéž považováno za správné opuštění rozběžiště. Jakmile se po dopadu náčiní atlet nachází ve vzdálenosti nejméně 4 m před koncovými čarami odhodového oblouku (pojmem před čarou se rozumí pohled ve směru hodu), je to rovněž považované za řádné opuštění rozběhové dráhy.
  \enditems

* Po provedeném pokusu v soutěži ve vrhu koulí nebo hodu diskem, kladivem či oštěpem, se náčiní při vracení zpět nikdy  nesmí házet.

Měření

* U všech hodů a vrhu, musí být vzdálenost zaznamenána s přesností 0.01 m na nejbližší nižší hodnotu jestliže naměřená hodnota není celý centimetr.
* Délka každého pokusu se musí měřit ihned po každém zdařeném pokusu (nebo po okamžitém ústním protestu podaném podle P 146.5)
  \begitems \style a
  * od nejbližšího místa dopadu koule, disku nebo hlavice kladiva k vnitřní hraně obruče kruhu, po přímce, probíhající od místa dopadu náčiní ke středu kruhu.
  * od místa, kde se hlavice oštěpu poprvé dotkla země, k vnitřní hraně odhodového oblouku po přímce probíhající od místa dopadu náčiní do středu křivosti odhodového oblouku
  \enditems

Pozn.: Místo dopadu náčiní určuje rozhodčí.

\itemnum=30
* Není-li možno měřit samostatně každý výkon všech atletů v hodu diskem, kladivem a oštěpem, použije se pro vyznačení průběžně nejlepšího výkonu každého atleta číselných značek, umístěných v místě dopadu náčiní v souladu s ustanovením odstavce 20. Čísla z dané číselné řady se atletům přidělují ve stejném pořadí, v jakém nastupují ke svým pokusům. V případě pochyb o tom, který výkon atleta je v daném okamžiku nejlepší, umístí rozhodčí obdobným způsobem k nové stopě dopadu kontrolní značku. Po ukončení soutěže se změří všechny vyznačené výkony.

Pozn.: Všechny číselné značky, vč. kontrolních, přidělené jednotlivým atletům, musí být uvedeny v zápise o průběhu soutěže.

* Při vrhu koulí se musí měřit každý pokus.
\enditems

\endinput