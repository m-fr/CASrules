\sec CHODECKÉ SOUTĚŽE


\secc Závodní chůze

\begitems \style N
Vzdálenosti
* Standardními vzdálenostmi jsou v hale 3000 m, 5000 m; venku na dráze 5000 m, 10000 m, 20000 m, 50000 m, na silnici 10 km, 20 km, 50 km.

Definice závodní chůze
* Závodní chůze je takový pohyb kroky, při němž nedojde k viditelné (lidským okem postřehnutelné) ztrátě dotyku chodce se zemí. Oporová noha musí bezpodmínečně být napnutá (tj. nepokrčená v koleně) od okamžiku prvního kontaktu se zemí až do okamžiku, kdy je ve svislé poloze.

Rozhodování
* \begitems \style a
  * Rozhodčí určení pro závod si zvolí vrchníka, pokud tento nebyl určen dříve.
  * Všichni rozhodčí musí jednat nezávisle jeden na druhém a jejich rozhodnutí musí být založena na pouhém pozorování lidským okem.
  * Při soutěžích uvedených v P 1.a) musí být všichni rozhodčí chůze uvedeni na IAAF Panelu mezinárodních rozhodčích chůze. Při soutěžích uvedených v P 1.b) a c) musí být všichni rozhodčí chůze uvedeni na IAAF Panelu mezinárodních rozhodčích chůze nebo na oblastním Panelu oblastních rozhodčích chůze.
  * Při soutěžích na silnici má normálně být nejméně šest a nejvýše devět rozhodčích, včetně vrchníka.
  * Při soutěžích na dráze má normálně být šest rozhodčích, včetně vrchníka.
  * Při soutěžích uvedených v P 1.a) může působit pouze jeden rozhodčí (nepočítaje vrchníka) příslušný jedné členské federaci.
  \enditems

  POZN.: Příslušností každého rozhodčího k určité členské federaci je ta, která je uvedena na seznamu oblastních nebo mezinárodních rozhodčí chůze.

Vrchník chůze
* \begitems \style a
  * Při soutěžích uvedených v P 1.a), b), c), d), f) má vrchník chůze pravomoc diskvalifikovat chodce na úseku posledních 100 m, pokud jeho/její způsob pohybu nevyhovuje ustanovení P 230.2., bez ohledu na počet červených karet, které vrchník chůze na atleta/y předtím obdržel. Chodec, který je vrchníkem chůze takto diskvalifikován, smí závod dokončit. O své diskvalifikaci musí být uvědomen vrchníkem chůze nebo jeho asistentem ukázáním červeného terče při nejbližší příležitosti poté, co prošel cílem.
  * Vrchník musí mít dozor nad celou soutěží a působí jako rozhodčí chůze pouze ve zvláštním případě uvedeném výše v odst. 4.a). Při soutěžích uvedených v P 1.1.a), b), c), f) musí být určeni dva nebo více asistentů vrchníka chůze, kteří pomáhají s uvědoměním atletů o diskvalifikacích. Tito asistenti nesmí působit jako rozhodčí chůze.
  * Při všech soutěžích uvedených v P 1.a), b), c), f), a pokud je to možné v dalších soutěžích, musí být určen zapisovatel vrchníka chůze a činovník, který bude mít na starosti návěstní tabuli pro vyznačení počtu udělených červených karet.
  \enditems

Žlutý terč
* Pokud rozhodčí není zcela spokojený s tím, jak dodržuje ustanovení P 230.2), musí atletovi, pokud je to možné, ukázat žlutý terč opatřený na obou stranách symbolem provinění proti pravidlům.

Atletovi nesmí být dvakrát ukázán žlutý terč za stejný přestupek stejným rozhodčím. O ukázaných žlutých terčích musí rozhodčí po závodě uvědomit vrchníka chůze.

Červené karty
* Když rozhodčí chůze zpozoruje, že chodec během kterékoliv části závodu nedodržuje ustanovení odstavce 1 tohoto pravidla tím, že viditelné ztrácí kontakt se zemí nebo pokrčuje koleno, musí na atleta zaslat vrchníkovi chůze červenou kartu.

Diskvalifikace
* \begitems \style a
  * Pokud vrchník chůze obdrží na jednoho atleta tři červené karty od tří různých rozhodčích, musí být takový atlet ze závodu vyloučen. Vrchník chůze nebo jeho asistent musí atleta o vyloučení uvědomit ukázáním červeného terče. I když se tak nestane, vyloučení zůstává v platnosti.
  * Při soutěžích uvedených v P 1.1.a), b), c), nebo e) nemůže být chodec za žádných okolností vyloučen na základě červených karet od dvou rozhodčích příslušných téže členské federaci.

  POZN.: Příslušností každého rozhodčího k určité členské federaci je ta, která je uvedena na seznamu oblastních nebo mezinárodních rozhodčí chůze.

  * Tam, kde to nařizuje řád dané soutěže nebo určí pořadatel, se pro závod zřídí trestné území (Penalty zone). Atlet, který obdrží tři červené karty, se na příkaz vrchníka chůze, nebo osoby vrchníkem určené, musí do tohoto území dostavit a setrvat v něm po určitou dobu.

  Pro jednotlivé závody se musí použít následující doby (trestné minuty)

\table{|l|r|}{\crl
Závod na, vč. & Doba \crl
5000 m/5 km   & 0,5 min \crl
10000 m/10 km & 1 min \crl
20000 m/20 km & 2 min \crl
30000 m/30 km & 3 min \crl
40000 m/40 km & 4 min \crl
50000 m/50 km & 5 min \crl
}

  Pokud tento atlet kdykoliv po návratu z trestu zpět do závodu obdrží další červenou kartu od jiného rozhodčího, než jednoho ze tří, kteří mu udělili předchozí tři červené karty, bude diskvalifikován. Atlet, který se do trestného území nedostaví, když k tomu bude vyzván nebo v trestném území nesetrvá po stanovenou dobu, bude vrchníkem chůze diskvalifikován.
  * Při soutěži na dráze musí vyloučený chodec okamžitě dráhu opustit a při soutěži mimo dráhu musí vyloučený chodec okamžitě sejmout startovní číslo a opustit trať. Kterýkoliv vyloučený atlet, který po své diskvalifikaci odmítne opustit dráhu nebo trať, nebo nedodrží ustanovení P230.7.c) o vstupu do trestného boxu a setrvání v něm po požadovanou dobu, může být vystaven dalším disciplinárním postihům ve smyslu P145.2.
  * Alespoň jedna návěstní tabule informující atlety o počtu červených karet, která byly podány na každého atleta, musí být umístěna na trati v blízkosti cíle. Na tabuli má být rovněž znázorněn přestupek, za nějž byla červená karta udělena.
  * Při všech soutěžích uvedených v P1.1.a) musí rozhodčí chůze pro předání všech červených karet zapisovateli vrchníka a na návěstní tabuli používat ruční zařízení výpočetní techniky s dálkovým přenosem dat.

  Při všech ostatních soutěžích, kde tento systém není použit, musí vrchník okamžitě po skončení soutěže nahlásit vrchnímu rozhodčímu všechny atlet, kteří byli vyloučeni podle znění P 230.4.a), 230.7.a) nebo 230.7c) s udáním startovního označení, času oznámení a druhu přestupku. Totéž musí učinit u všech atletů, kteří dostali červenou kartu.
  \enditems

Start
* Závod musí být odstartován výstřelem z pistole. Použijí se povely pro běhy delší než 400 m. (viz P 162.2.b). V závodech, kterých se účastí velký počet atletů, musí být dána výstražná znamení 5 minut, 3 minuty a 1 minutu před startem.  Po povelu "Připravte se!" se závodníci shromáždí na startu způsobem určeným organizátorem. Startér se přesvědčí, že žádný atlet se nohou (nebo kteroukoliv částí těla) nedotýká startovní čáry nebo země za ní (míněno ve směru chůze) a pak závod odstartuje.

Pozn.: V soutěži družstev je startovní prostor rozdělen do sektorů. Členové družstev se v sektoru řadí do zástupu.

Bezpečnost závodu
* Pořadatelé závodu v chůzi na silnici musí bezpodmínečně zajistit bezpečnost atletů. Při soutěžích uvedených v P 1.1.a), b), c) pořadatel musí zajistit, aby silnice, po níž závod probíhá, byla v obou směrech uzavřena pro motorizovanou dopravu.

Osvěžovaní a občerstvovací stanice při silničních soutěžích
* \begitems \style a
  * Na startu a v cíli všech závodů musí mít závodníci k dispozici vodu a další vhodné občerstvení.
  * Při všech závodech od 5 km vč. až do 10 km vč. musí být ve vhodných intervalech zřízeny osvěžovací stanice, kde se podává pouze voda, pokud k takovým opatřením povětrnostní podmínky opravňují.

  POZN.: Pokud to bude považováno za vhodné s ohledem na určité klimatické podmínky, mohou být rovněž zřízeny stanice, kde bude vytvářena vodní mlha na trati.

  * Při všech soutěžích na tratích delších než 10 km musí být v každém kole k dispozici občerstvovací stanice (kromě vody i jídlo). Dále musí být zřízeny osvěžovací stanice, kde se podává pouze voda, asi uprostřed mezi občerstvovacími stanicemi, případně častěji, pokud k takovým opatřením povětrnostní podmínky opravňují.
  * Občerstvení, které může být poskytnuto pořadatelem nebo je připraveno samotnými atlety, musí být uloženo na stanicích tak, aby bylo atletům snadno přístupné nebo jim může být podáváno přímo do rukou pověřenými osobami.
  * Pověřené osoby nesmí vstoupit na trať nebo překážet některému atletovi. Občerstvení mohou podat atletovi buď zpoza stolu nebo nejvýše 1 m z boku, ale nikoliv z postavení před stolem.
  * Při soutěžích uvedených v P1.1.a), b), c), f) mohou být za stolem současně pouze dva činovníci z každé členské federace Žádný činovník nebo pověřená osoba se za žádných okolností nesmí pohybovat vedle atleta, když přebírá občerstvení nebo vodu.

  POZN.: Při soutěžích, kde je členská federace reprezentována více než třemi atlety, mohou technické předpisy dovolit účast dalších činovníků u stolů občerstvovací stanice.

  * Atlet může kdykoliv nést vodu nebo občerstvení v rukou nebo je mít uchycené na těle, pokud je nese od startu nebo je převzal či obdržel na oficiální stanici.
  * Atlet, který přijme nebo obdrží občerstvení nebo vodu mimo oficiální občerstvovací stanici, pokud se nejedná o lékařské důvody a není přijato od nebo pod dozorem rozhodčích závodu, nebo přijme občerstvení od jiného atleta, bude při prvním provinění varován vrchním rozhodčím ukázáním žluté karty. Při druhém takovém provinění vrchní rozhodčí atleta diskvalifikuje ukázáním červené karty. Atlet pak musí trať ihned opustit.

  POZN.: Atlet může občerstvení obdržet od jiného atleta nebo mu předat občerstvení, vodu nebo houby, pokud je nese od startu nebo je vzal či obdržel na oficiální stanici. Nicméně trvalá pomoc mezi dvěma nebo více atlety takovým způsobem může být považovaná za nedovolenou dopomoc a mohou být proto uplatněna varování nebo diskvalifikace, jak je uvedeno výše.
  \enditems

Soutěže na silnici
* \begitems \style a
  * Okruh musí být dlouhý nejméně 1,0 km a nejvýše 2,0 km. Pro soutěže, které začínají a končí na stadionu, má být okruh umístěn co nejblíže stadionu.
  * Délka tratí silničních závodů musí být měřena v souladu s ustanoveními P 240.3.

  Pozn.: Toto ustanovení se doporučuje též pro soutěže konané v ČR a organizované ČAS.
  \enditems

Průběh soutěží
* Atlet smí opustit vyznačenou trať se svolením rozhodčího a za jeho dozoru, pokud si odchodem z trati nezkrátí předepsanou vzdálenost.
* Pokud se příslušný vrchní rozhodčí dozví od rozhodčího, úsekového rozhodčího nebo jinak, že chodec opustil vyznačenou trať, a zkrátil si tak předepsanou vzdálenost, musí takového chodce diskvalifikovat.
\itemnum=30
* Pro měření časů platí ustanovení P 165.
\enditems

\endinput