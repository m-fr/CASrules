\sec SOUTĚŽE NA DRÁZE

Ustanovení P 163.2, P 163.6 (s výjimkou P230.12 a P240.9), P163.14, P164.2, P165 a P167.1 platí též pro soutěže uvedené v kapitolách VII, VIII a IX.

\secc Atletký ovál a dráhy

\begitems \style N
* Standardní délka atletického oválu musí být 400 m. Musí mít dvě rovnoběžné rovinky a dvě zatáčky o stejném poloměru. Vnitřní okraj atletického oválu musí být ohraničen obrubníkem bílé barvy z vhodného materiálu, vysokým 50 až 65 mm a širokým 50 až 250 mm. Na obou rovinkách může být obrubník vynechán a nahrazen bílou čárou 50 mm širokou.

Pokud je třeba pro soutěže v poli dočasně odstranit část obrubníku v zatáčkách, musí být plochy pod ním označeny bílou čarou širokou 50 mm a kužely či praporky, vysokými min. 0,15 m, umístěnými tak, že okraj základny kužele nebo držáku praporku se dotýká okraje bílé čáry vymezující dráhu a jsou rozmístěny ve vzdálenosti 4 m. (Praporky musí být na čáře umístěny pod úhlem 60o směrem dovnitř atletického oválu.) Totéž (vč. dočasně umístěných obrubníků) platí pro zakřivenou část steeplechase tratě, kdy běžci opouštějí hlavní dráhu a překonávají přitom vodní příkop a pro vnější dráhy atletického oválu v případě startů podle P163.5.b) a případně i na příslušných rovných úsecích v intervalu do 10 m.

POZN.: Všechny body, kde zakřivené úseky oválu přecházejí do rovných nebo rovné do zakřivených, musí být vyznačené výrazně barevnou značkou 50x50 mm na bílé čáře a v těchto místech musí být během závodu umístěny kužely.

* Délka atletického oválu musí být měřena po čáře vzdálené 0,30 m od vnitřní hrany obrubníku směrem do dráhy. Není-li atletký ovál v zatáčce vymezen obrubníkem (a v úsecích steeplechase tratě, kdy běžci opouštějí hlavní dráhu a překonávají přitom vodní příkop), musí se měřit ve vzdálenosti 0,20 m od čáry označující vnitřní okraj oválu.

\picw=7cm \inspic img/mereni_drahy.jpg
\caption/f Měření délky dráhy.

* Délka dráhy musí být měřena od hrany startovní čáry vzdálenější od cíle po hranu cílové čáry bližší startu.
* Při všech bězích do 400 m vč., každý atlet musí běžet v samostatné dráze široké 1,22 m ($\pm$0,01 m včetně čáry na pravé straně dráhy, ohraničené čarami širokými 50 mm. Všechny dráhy musí být stejně široké. Vnitřní (tj. první) dráha musí být změřena tak, jak je uvedeno v bodě P160.2, ale ostatní dráhy musí být měřeny po čáře vedené 0,20 m od vnějšího okraje dráhy.

POZN.: U všech atletických oválů postavených před 1. lednem 2004 smí být maximální šířka jedné dráhy 1,25 m. Nicméně, při úplné obnově povrchu musí šířka čar splnit toto pravidlo.

* Atletký ovál pro mezinárodní soutěže uvedené v P 1.1.a) ,b), c) má mít alespoň 8 drah.
* Příčný sklon oválu směrem k vnitřnímu okraji nemá překročit hodnotu 1:100 (1\%), pokud není za zvláštních okolností udělena výjimka IAAF, sklon ve směru běhu nemá překročit hodnotu 1:1000  (0,1\%).
* Úplné technické informace o konstrukci atletických drah, jejich uspořádání a značení jsou uvedeny v manuálu IAAF "Atletická zařízení" (Track and Field Facilities Manual). Toto pravidlo uvádí jen základní zásady, které musí být dodrženy.
\enditems

\secc Startovní bloky

\begitems \style N
* Startovní bloky musí být bezpodmínečně používány při všech závodech do 400 m vč. (včetně prvního úseku běhů na 4x200 m, 1-2-3-400 m a 4x400 m) a nesmí být používány při jiných závodech. Při umístění na dráze nesmí jakákoliv jejich část přesahovat startovní čáru nebo zasahovat do jiné dráhy, vyjma situace, kdy nepřekážejí žádnému jinému atletovi, může zadní část rámu přesahovat přes vnější čáru dráhy.
* Startovní  bloky musí odpovídat těmto všeobecným požadavkům:
  \begitems \style a
  * Musí sestávat ze dvou opěr pro chodidla, o něž se opírají chodidla atleta ve startovní pozici a které jsou uchycené na tuhém rámu. Musí být zcela tuhé konstrukce a nesmějí atletu poskytovat žádnou výhodu. Rám nesmí jakkoliv omezovat pohyb chodidel atleta při výběhu z bloků.
  * Opěry musí být skloněné pro usnadnění startovní polohy atleta a mohou být ploché nebo lehce vyduté. Povrch opěr musí umožnit zasunutí hřebů atletické obuvi, buď pomocí drážek nebo výřezů v povrchu opěr nebo pokrytím vhodným materiálem umožňujícím použití obuvi s hřeby.
  * Opěry na rámu mohou být nastavitelné, ale nesmí umožnit jakýkoliv pohyb během vlastního startu. Ve všech případech musí být možné nastavit vzájemnou polohu obou opěr posunem vpřed či vzad. Nastavenou polohu musí být možné zajistit západkami nebo aretačním mechanismem, který může atlet snadno a rychle ovládat.
  * Musí být usazeny na dráze pomocí několika hřebů nebo bodců uspořádaných tak, aby co nejméně poškodily dráhu. Upevnění bloků musí dovolovat jejich rychlé a snadné odstranění. Počet, tloušťka a délka hřebů či bodců závisí na konstrukci dráhy. Ukotvení bloků nesmí během vlastního startu dovolovat žádný posun.
  * Pokud používá atlet vlastní startovní bloky tyto musí vyhovovat těmto pravidlům, ale mohou mít jinou konstrukci, pokud nepřekážejí ostatním atletům.
  \enditems
* Při soutěžích uvedených v P 1.1.a), b), c) a f) a pro výkon, který má být schválený jako světový rekord dle pravidla P261 nebo P262, musí být startovní bloky připojeny na Startovní informační systém, certifikovaný IAAF. Tento systém se důrazně doporučuje pro všechny ostatní soutěže.

POZN.: Navíc může být použit automatický systém pro vracení nezdařených startů, pracující v souladu s pravidly.

* Při soutěžích uvedených v P1.1.a) až f) závodníci musí používat startovní bloky připravené pořadateli závodů. Při ostatních soutěžích na drahách s umělým povrchem může pořadatel trvat na použití startovních bloků jím připravených.
\enditems

\secc Start

\begitems \style N
* Start závodu musí být vyznačen bílou čarou širokou 50 mm. Pro všechny závody, které se neběží v drahách, musí být startovní čára zakřivena tak, aby všichni atleti odstartovali ve stejné vzdálenosti od cíle. Startovní pozice pro všechny běhy jsou číslovány z leva doprava při pohledu ve směru běhu.

POZN .1: Při závodech konaných mimo stadion může být startovní čára široká až 0,30 m a může být jakékoliv barvy výrazně kontrastní vůči podkladu.

POZN. 2: Startovní čára běhu na 1500 m nebo kterákoliv zakřivená startovní čára může být prodloužena až za ohraničení vnější krajní dráhy, pokud to umožňuje stejný povrch, jaký je na atletickém oválu.

* Při všech mezinárodních závodech, vyjma jak uvedeno níže, musí startér dávat povely ve své vlastní řeči, v angličtině či ve francouzštině.
  \begitems \style a
  * V bězích na 400 m a kratších, (včetně 4x200 m, 4x400 m a 1-2-3-400 m podle P170.1) jsou povely "Připravte se !",  "Pozor !".
  * Při závodech delších než 400 m (vyjma  4x200 m, 4x400 m a 1-2-3-400 m) je povelem pouze "Připravte se !".
  * V kterémkoliv závodě, kdy jsou atleti v konečné startovní pozici v souladu s P162.5 a startér nepovažuje další pokračování startovního procesu za vhodné, nebo jinak start přeruší, vydá povel „Vstaňte !“.

  Všechny běhy musí normálně být odstartovány výstřelem startérovy pistole směrem vzhůru

  Pozn. : Při soutěžích uvedených v P 1.1.a), b), c), e) a i) dává startér povely pouze v angličtině.
  \enditems
* Při závodech do 400 m včetně (a na prvních úsecích štafet na 4x200 m,  4x400 m a 1-2-3-400 m (podle P170.1) je předepsán nízký startu a použití startovních bloků. Po povelu "Připravte se !" atlet musí zaujmout polohu, kdy je v jemu přidělené dráze a zcela za startovní čárou. Atlet se při startu v žádném případě nesmí nijak dotýkat ani startovní čáry, ani dráhy před ní. Obě jeho ruce a nejméně jedno koleno musí být v dotyku se zemí a obě chodidla v dotyku s opěrami nohou startovních bloků. Na povel "Pozor !" atlet musí ihned zaujmout konečnou startovní polohu při zachování dotyku rukou se zemí a chodidel se startovními bloky. Jakmile se startér ujistí, že všichni závodníci jsou v pozici „Pozor“ v klidu, vystřelí.
* Při závodech delších než 400 m (vyjma  4x200 m a 4x400 m a 1-2-3-400 m) všechny starty se provádějí ze vzpřímené polohy. Po povelu  "Připravte se !" atlet musí přistoupit ke startovní čáře a zaujmout startovní polohu za startovní čárou, při startu v drahách musí být zcela uvnitř jemu přidělené dráhy. Atlet se v této startovní poloze ani jednou rukou nesmí dotýkat země a ani jednou nohou se nesmí dotýkat ani startovní čáry, ani země před ní. Jakmile se startér ujistí, že všichni závodníci zaujali správnou startovní pozici a jsou v klidu, vystřelí.
* Na povel "Připravte se !“, nebo "Pozor !", podle toho o jaký závod se jedná, musí všichni atleti okamžitě a bez otálení zaujmout konečné startovní postavení. Pokud startér, z jakéhokoli důvodu, nepovažuje další pokračování startovního procesu za vhodné poté co závodníci zaujali konečnou startovní pozici na čáře, musí startovní přípravu přerušit a přikázat atletům, aby opustili svá místa.  Asistenti startéra/ musí atlety opět seřadit (viz též P130).

Pokud atlet podle názoru startéra,
  \begitems \style a
  * po povelu „Připravte se !“, nebo „Pozor !“ a před výstřelem pistole, způsobí přerušení startu, např. zvednutím ruky nebo povstáním či zpětným usednutím v případě nízkého startu, bez vážného důvodu (vážnost důvodu posoudí příslušný vrchní rozhodčí), nebo
  * neuposlechne povelu „Připravte se !“, či „Pozor !“, nebo nezaujme konečnou startovní pozici okamžitě a bez otálení, nebo
  * po povelu „Připravte se !“ či "Pozor !" ruší ostatní atlety běhu zvuky, pohybem či jinak,
  \enditems
startér přeruší startovní přípravu.

Vrchní rozhodčí může atleta napomenout (diskvalifikovat v případě druhého porušení pravidel během stejné soutěže) pro nesprávné chování podle ustanovení P125.5 a 145.2. V tomto případě zelenou kartu neukazuje. Nicméně pokud byl start přerušen z vnějších důvodů nebo vrchní rozhodčí nesouhlasí s rozhodnutím startéra, musí být všem atletům ukázána zelená karta potvrzující, že nezdařený start nebyl způsoben žádným z atletů na startu.

Chybný start

* Pokud je použit Startovní informační systém certifikovaný IAAF, startér nebo pověřený zástupce startéra musí mít nasazená sluchátka, aby jasně slyšel akustický signál indikující možný chybný start (tj. startovní reakce byla kratší než 0,100 s). Jakmile startér nebo zástupce startéra uslyší akustický signál a bylo vystřeleno z pistole, následuje vrácení startu a startér musí okamžitě prozkoumat reakční časy a ostatní informace poskytnuté systémem, aby si potvrdil, zda, a který atlet (-ci) je (jsou) odpovědný (-i) za vrácení startu.

POZN.: Při použití Startovního informačního systému certifikovaného IAAF musí být jeho záznamy příslušnými činovníky použity jako podklad pro správné rozhodnutí.

* Poté, co atlet zaujme konečnou startovní polohu, nesmí zahájit další startovní pohyb dříve, než zazní výstřel. Pokud podle rozhodnutí startéra (vč. ve smyslu P129.6) tak atlet učiní dříve, dopustil se tím chybného startu.

POZN.1: Jakýkoliv pohyb atleta, aniž noha/nohy atleta ztratí kontakt s nožní podpěrou startovního bloku, nebo ruka či obě ruce ztratí kontakt se zemí, nesmí být považován za zahájení startovního pohybu. Tyto okolnosti mohou být, pokud tak rozhodne rozhodčí, předmětem disciplinárního varování nebo diskvalifikace.

Nicméně pokud startér rozhodne, že atlet ještě před výstřelem zahájil pohyb, který nebyl zastaven a pokračoval i po výstřelu, jedná se chybný start.

POZN.2 : Pokud se více atletů, nacházející se ve startovní poloze, má sklon ke ztrátě rovnováhy a je-li tento pohyb považován za neúmyslný, je třeba start považovat za „neklidný“. Pokud je atlet vytlačen nebo vystrčen přes startovní čáru před startem, nebude potrestán. Kterýkoliv atlet, který tuto kolizi vyvolal, může být potrestán disciplinárním varováním nebo diskvalifikací.

* Vyjma soutěží ve víceboji každý atlet, který způsobí chybný start, musí být startérem ze závodu vyloučen.

Pro víceboje viz P200.8.c).

POZN.: Jestliže jeden nebo více atletů způsobí chybný start, jsou ostatní zpravidla strženi k následování. Přísně vzato, kterýkoliv atlet, který tak učinil, se dopustil chybného startu.  Startér však napomene nebo vyloučí ze závodu pouze toho atleta nebo ty atlety, kteří chybný start zavinili. To může znamenat, že může být napomenuto nebo vyloučeno i více atletů než jeden. Jestliže chybný start nezavinil žádný z atletů, nesmí být nikdo napomenut. Pokud nebyl chybný start způsoben některým ze atletů, nebude nikdo vyloučen (napomenut) a všem atletům bude ukázána zelená karta.

* V případě chybného startu postupuje asistent startéra následujícím způsobem:

Vyjma soutěže ve vícebojích, atlet(i), který (kteří) způsobil(i) chybný start je (jsou) diskvalifikován(i) a  musí před ním (nimi) být zvednuta červenočerná karta (úhlopříčně dělená).

V soutěžích ve vícebojích, v případě prvního chybného startu je atlet, který se chybného startu dopustil napomenut zvednutím žlutočerné karty (úhlopříčně dělené) před ním. Současně je umístěna rovněž na označení jeho dráhy (pokud existuje). Současně jsou žlutočernou kartou varováni i ostatní účastníci daného běhu na znamení, že každý viník dalšího chybného startu bude diskvalifikován. Pokud následně kdokoliv v daném běhu způsobí chybný start, bude diskvalifikován zvednutím červenočerné karty.

Pokud je použito označení drah, které to umožňuje, a atletovi, který je zodpovědný za chybný start, je ukázána karta, je na tomto označení vyvěšen odpovídající ukazatel.

Pozn.: Pro soutěže žactva v rámci ČR platí: V každém běhu je možný pouze jediný chybný start bez diskvalifikace atleta (atletů), který (kteří) jej způsobil(i). Kterýkoliv atlet, který způsobí další chybný start v témže běhu, musí být ze závodu vyloučen.

* Jestliže se startér nebo kterýkoliv z jeho zástupců domnívá, že start neproběhl podle pravidel, vrátí běžce dalším výstřelem ze startovací pistole.
\enditems

\secc Závod

\begitems \style N
* Při závodech s alespoň jednou zatáčkou je směr běhu a chůze levotočivý, tj. atletova levá paže je blíže vnitřnímu okraji drah. Dráhy musí být číslovány tak, že levá vnitřní má číslo 1.

POZN.: Pokud to okolnosti a stav dráhy dovolují, mohou být běhy na rovince uskutečněné v opačném směru.

Překážení

* Pokud je atlet strkán nebo blokován v průběhu závodu tak, že je mu bráněno v posunu vpřed, pak:
  \begitems \style a
  * toto strkání nebo bránění je považováno za neúmyslné, nebo není zaviněno atletem blízkým (v kontaktu), vrchní rozhodčí, pokud je přesvědčen, že atlet (nebo jeho družstvo) byl vážně poškozen, může v souladu s P 125.7 nebo P146.4 nařídit opakování závodu (jedním, několika, nebo všemi atlety) nebo atletovi (družstvu) dovolit start v následujícím kole soutěže.
  * a vrchní rozhodčí zjistí, že strkání nebo bránění způsobil jiný atlet, může takového atleta (nebo družstvo) z běhu vyloučit. Vrchní rozhodčí, pokud je přesvědčen, že atlet (či jeho družstvo) byl vážně poškozen, může v souladu s P 125.7 nebo P146.4 nařídit opakování závodu (jedním, několika nebo všemi atlety) bez účasti diskvalifikovaného atleta (nebo družstva) nebo dovolit kterémukoliv poškozenému atletovi (nebo družstvu) start v následujícím kole soutěže.

  POZN.: V případech považovaných za dostatečně vážné lze uplatnit rovněž P125.5 a P145.2.
  \enditems

V obou případech P163.2.a) a b) takový atlet (nebo družstvo) musí dokončit soutěž s „bona fide“ úsilím.

Opuštění (vyšlápnutí z) dráhy

* \begitems \style a
  * Při všech bězích v oddělených drahách se každý atlet musí pohybovat v jemu přidělené dráze od startu až do cíle. Totéž platí pro kteroukoliv část závodu, která se musí běžet v drahách.
  * Ve všech závodech (nebo kterékoliv části závodu) které se neběží v drahách, atlet, běžící v zatáčce, ve vnější části atletického oválu dle 163.5.b), či v zakřiveném úseku dráhy před a za vodním příkopem při steeplechase, nesmí vykročit nebo běžet na nebo pod obrubníkem nebo čárou vyznačující příslušnou hranici zakřiveného úseku dráhy.
  \enditems

Pokud však, vyjma ustanovení P 163.4, je vrchní rozhodčí, na základě zprávy rozhodčího či úsekového rozhodčího nebo jinak, přesvědčen, že atlet toto pravidlo porušil, musí jej diskvalifikovat.

* Atlet nebude diskvalifikován pokud
  \begitems \style a
  * byl jiným vytlačen nebo přinucen vykročit ze své dráhy nebo na či přes obrubník, nebo
  * vykročí mimo svou dráhu na rovince, mimo rovné úseky dráhy před a za vodním příkopem při steeplechase či do vnější dráhy (po své pravé ruce) v zatáčce nezískal tím žádnou materiální výhodu, a přitom nevrazil nebo nepřekážel jinému běžci. Pokud materiální výhodu získal, atlet musí být diskvalifikován.

  POZN.: Materiální výhoda zahrnuje zlepšení vlastní pozice, vč. uvolnění se z „uzavření“ v klubku běžců tím, že vykročí nebo běží vlevo od vnitřní hraniční čáry atletického oválu.
  \enditems

* Při soutěžích uvedených v P1.1. a kde to je vhodné i v ostatních soutěžích:
  \begitems \style a
  * běh na 800 m se musí běžet v drahách až po přední hranu obloukové čáry, kde závodníci mohou opustit své dráhy. Čára opuštění drah (seběhnutí k obrubníku) je oblouková čára vyznačená na výběhu z první zatáčky, vyznačená v šíři 50 mm napříč drahami. Pro usnadnění identifikace čáry dovolující běžcům seběhnout do první dráhy, musí být na čarách, bezprostředně před průsečíky obloukové čáry s čarami vymezujícími jednotlivé dráhy, umístěny malé kužely nebo hranoly nebo jiné vhodné značky, 50 x 50 mm, jejichž výška nepřesáhne 0,15 m, a jejichž barva je odlišná od barvy obloukové čáry.

  POZN.:  V soutěžích uvedených v P 1.d) a h) se účastníci mohou dohodnout, že se nepoběží v drahách.

  Pozn.: Ustanovení POZN.  platí i pro soutěže ČAS.

  * Při účasti více než 12 běžců v bězích na 1000 m, 2000 m, 3000 m, 5000 m nebo 10000 m je možno startující rozdělit do dvou skupin, z nichž první skupina, čítající přibližně dvě třetiny startovního pole, bude startovat z řádné zakřivené startovní čáry a druhá skupina ze samostatné vnější startovní čáry, vyznačené napříč vnější polovinou atletických drah. Druhá skupina poběží ve své polovině drah až na konec první zatáčky, která bude vyznačena kužely nebo praporky, jak je ustanoveno v P160.1.

  Samostatná vnější zakřivená startovní čára musí být vyznačena tak, aby všichni závodníci běželi stejnou vzdálenost.

  Oblouková čára pro běh na 800 m označuje místo, kde se závodníci vnější skupiny běhu na 2000 m a 10000 m mohou připojit k běžcům, kteří vyběhli ze základní startovní čáry.

  Na začátku cílové rovinky musí být obdobně vyznačeno, kde se běžci vnější skupiny při bězích na 1000 m, 3000 m a 5000 m mohou připojit k atletům, kteří vyběhli ze základní startovní čáry. Toto vyznačení je možné provést značkou 50 mm x 50 mm na čáře mezi čtvrtou a pátou dráhou (resp. třetí a čtvrtou při šesti drahách). Bezprostředně před touto značkou bude až do spojení obou skupin umístěn kužel nebo praporek.

  * Pokud atlet nedodrží toto pravidlo, bude diskvalifikován. V případě běhu rozestaveného bude diskvalifikováno jeho družstvo.
  \enditems

Opuštění atletického oválu

* Atlet, který dobrovolně opustí atletký ovál, nesmí v závodě pokračovat a v zápisu bude uvedeno, že běh nedokončil. Pokud se atlet pokusí do závodu vrátit, bude hlavním rozhodčím diskvalifikován.

Kontrolní značky

* S výjimkou ustanovení P170.4, na úsecích rozestavených běhů, které se běží v drahách, si závodníci nesmějí dělat na atletickém oválu kontrolní značky nebo na něm či podél něho umisťovat předměty, sloužící jako pomocné body. Rozhodčí musí příslušného atleta upozornit, aby značku nebo předmět upravil nebo odstranil značku nebo předmět, který neodpovídá tomuto pravidlu. Pokud tak neučiní, musí je odstranit rozhodčí.

POZN.: Vážnější případy mohou být dále projednány podle P125.5 a P145.2.

Měření rychlosti větru

* Všechny větroměry musí být vyrobené a kalibrované podle mezinárodních norem. Přesnost měřícího zařízení použitého při soutěži musí být ověřena příslušnou organizací akreditovanou u národního metrologického ústavu.
* Větroměr jiného než mechanického principu musí být použit při všech mezinárodních soutěžích uvedených v P1.a) až h) a u všech výkonů, které budou předloženy k ratifikaci jako světové rekordy.

Pokud je použit mechanický větroměr, má být opatřen vhodnou ochranou proti vlivu jakéhokoliv bočního větru. Při použití trubky má být její délka na obě strany od měřícího zařízení alespoň dvojnásobkem průměru trubky.

* Vrchní rozhodčí běhů musí zajistit, aby při atletických soutěžích větroměr byl umístěn vedle první dráhy, ve vzdálenosti 50 m od cílové čáry. Měření musí být prováděno v úrovni 1,22 m od země a ve vzdálenosti do 2 m od okraje atletického oválu.
* Větroměr může být spouštěn a zastavován automaticky, nebo na dálku a jeho údaj přenášen přímo do počítače soutěže.
* Rychlost větru musí být měřena od záblesku/kouře startérovy pistole nebo jiného schváleného startovacího zařízení po dále uvedenou dobu:

\table{ll}{
100m           & 10 sekund\cr
100 m překážek & 13 sekund\cr
110 m překážek & 13 sekund\cr
}

V běhu na 200 m musí být rychlost větru měřena po dobu deseti sekund od okamžiku, kdy vedoucí atlet vběhne do cílové rovinky.

* Rychlost větru musí být odečtena v metrech za sekundu, údaje zaokrouhlené na nejblíže vyšší desetinu (m/s) v kladném smyslu. (tzn., že údaj +2,01 m/s bude zaznamenán jako +2,1 m/s a údaj  2,01 m/s jako -2,0 m/s.). Měřiče s digitálním ukazatelem se čtením v desetinách m/s musí odpovídat tomuto pravidlu.

Hlášení mezičasů

* Průběžné mezičasy a předběžné časy vítězů mohou být úředně oznamovány určenou osobou nebo uvedeny na výsledkové tabuli. Jinak takové údaje nesmějí osoby v soutěžním sektoru hlásit atletům bez předběžného souhlasu příslušného vrchního rozhodčího. Takový souhlas je možno dát jen tehdy, pokud závodníci nemají k dispozici žádný viditelný ukazatel času v určitém místě a pokud jsou tyto časy sděleny i ostatním atletům.

Závodníci, kterým byly mezičasy sděleny jinak, než dovoluje toto pravidlo, jsou považováni za atlet, kteří přijali nepovolenou pomoc a vztahuje se na ně ustanovení P144.2.

POZN.: Soutěžním sektorem, který obvykle má i fyzické ohraničení, se pro tento účel rozumí oblast, kde se soutěž koná a kam je přístup povolen pouze sportovcům a osobám oprávněným v souladu s příslušnými pravidly a předpisy.

Osvěžení

* \begitems \style a
  * Při soutěžích na dráze na 5 km a delších může pořadatel poskytnout atletům osvěžení, pokud k tomu povětrnostní podmínky opravňují.
  * V bězích na dráze delších než 10000 m musí být poskytnuto osvěžení ve formě pitné vody a vody na opláchnutí. Osvěžení může být poskytnuto buď pořadatelem soutěže nebo atlety a musí být snadno dosažitelné samotnými atlety nebo předáno přímo do rukou atleta prostřednictvím oprávněných osob. Osvěžení dodané atlety musí být pod dohledem pořadatelem určených činovníků od okamžiku, kdy toto osvěžení atlet nebo jeho zástupce předloží. Tito činovníci musí zajistit, že osvěžení nebylo vyměněno nebo jakkoliv upraveno.
  * Atlet, který obdrží nebo přijme občerstvení nebo osvěžení mimo určenou stanici pro občerstvení, pokud se nejedná o lékařské důvody a není přijato od nebo pod dozorem rozhodčích závodu, nebo přijme občerstvení od jiného atleta, bude při prvním provinění varován vrchním rozhodčím ukázáním žluté karty. Při druhém takovém provinění vrchní rozhodčí atleta diskvalifikuje ukázáním červené karty. Atlet pak musí atletký ovál ihned opustit.
  \enditems

POZN.: Atlet může občerstvení obdržet od jiného atleta nebo mu předat občerstvení, vodu nebo houby, pokud je nese od startu nebo je vzal či obdržel na oficiální stanici. Nicméně trvalá pomoc mezi dvěma nebo více atlety takovým způsobem může být považovaná za nedovolenou dopomoc a mohou být proto uplatněna varování nebo diskvalifikace, jak je uvedeno výše.

\itemnum=30
* Rychlost větru musí být rovněž měřena

\table{ll}{
při běhu na 50 a 60 m     & po dobu 5 sekund\cr
při běhu na 80 m překážek & po dobu 10 sekund\cr
při běhu na 150 m         & stejné jako pro 200 m\cr
}
\enditems

\secc Cíl

\begitems \style N
* Cíl závodu musí být vyznačen bílou čarou širokou 50 mm.

POZN.: Při bězích konaných mimo stadion může být cílová čára široká až 0,30 m a může být jakékoliv barvy výrazně kontrastní vůči podkladu.

* Umístění atletů v cíli musí být stanoveno podle pořadí, v němž kterákoliv část jejich těla (tj. trupu, nikoliv hlavy, krku, paží, nohou, rukou či chodidel) dosáhne svislé roviny proložené bližším okrajem cílové čáry jak definováno výše.

Pozn.: Doporučuje se snímat dobíhající atlety kamerou, která je nasměrovaná proti směru běhu.

* Při závodech, kde rozhoduje vzdálenost dosažená za stanovenou dobu, musí startér vystřelit přesně jednu minutu před koncem závodu, aby upozornil atlety a rozhodčí, že se blíží konec závodu. Startér se musí řídit pokyny vedoucího časoměřiče. Přesně v okamžiku, kdy uplyne stanovená doba, oznámí startér ukončení závodu dalším výstřelem. V tomto okamžiku musí určení rozhodčí vyznačit místo, kde se každý atlet naposledy dotkl dráhy těsně před výstřelem nebo současně s ním.

Dosažená vzdálenost musí být změřena k nejbližšímu celému metru před touto značkou ve směru běhu. Před startem závodu musí být každému atletovi přidělen alespoň jeden rozhodčí k vyznačení dosažené vzdálenosti.

POZN.: IAAF Manuál pro pořádání závodu v běhu na jednu hodinu je možno stáhnout z webových stránek IAAF.
\enditems

\secc Měření časů, cílová kamera

\begitems \style N
* Oficiální časy musí být měřeny jedním ze tří způsobů:
  \begitems \style a
  * ručním měřením,
  * plně automaticky pomocí cílové kamery,
  * čipovým časoměrným systémem, tj. systémem pracujícím s prvky (čipy/transpondéry) pro automatický záznam průchodu atleta metou, avšak pouze při soutěžích podle P230, P240 a P250.
  \enditems
* V souladu s P165.1.a) a b) časy musí být měřeny do okamžiku, kdy kterákoliv část těla atleta (tj. trup, nikoliv hlava, krk, paže, noha, ruka, chodidlo) dosáhne svislé roviny procházející okrajem cílové čáry bližším startu.
* Musí být zaznamenány časy všech atletů v cíli. Je-li možné, navíc též mezičasy na každé kolo při bězích na 800 m a delších a mezičasy na každý kilometr při bězích na 3000 m a delších.

Ruční měření časů

* Časoměřiči musí být v rovině cílové čáry a, pokud je to možné, vně atletického oválu. Pokud je možné, jsou vzdáleni alespoň 5 metrů od vnější dráhy. Pro dobrý výhled všech časoměřičů na cílovou čáru se užije vyvýšeného stanoviště.
* Časoměřiči musí užívat ručně ovládaná elektronická zařízení pro měření času s digitálním čtením. Tato zařízení se v pravidlech IAAF nazývají stopkami.
* Mezičasy dle P165.3 musí být měřeny určenými časoměřiči pomocí stopek, jimiž lze měřit více časů, nebo dalšími časoměřiči nebo čipy.
* Časy musí být měřeny od záblesku či kouře pistole nebo schváleného startovacího zařízení.
* Tři určení časoměřiči, z nichž jedním musí být vedoucí časoměřič, a dále jeden nebo dva náhradní časoměřiči, musí měřit čas vítěze každého běhu a jakékoliv výkony pro rekordní účely (u vícebojů viz P200.8.b). K časům zaznamenaným náhradními časoměřiči se přihlíží jen tehdy, pokud stopky některého úředního časoměřiče nezaznamenají správný čas. V takovém případě se berou v úvahu časy náhradních časoměřičů, a to v předem stanoveném pořadí, takže úřední čas vítěze je vždy zaznamenán trojími stopkami.
* Každý časoměřič musí pracovat nezávisle na ostatních, nesmí nikomu ukazovat své stopky nebo se o čase s někým domlouval. Svůj čas zapíše na úřední formulář, který po podepsání předá vedoucímu časoměřiči, který si může na stopkách každého časoměřiče ověřit oznámený čas.
* Všechny ručně měřené časy musí být:
  \begitems \style a
  * při bězích na dráze odečítány s přesností desetinu sekundy, naměřené časy, které na druhém desetinném místě nekončí nulou, musí být zaokrouhleny na nejblíže vyšší celou desetinu sekundy, např. změřený čas 10,11 musí být zaznamenán jako 10,2.
  * Při závodech konaných částečně nebo zcela mimo stadion odečítány na nejblíže vyšší celou sekundu, např. naměřený čas 2:09:44,3 musí být zaznamenán jako 2:09:45.
  \enditems
* Pokud po úpravě, jak je uvedeno výše, souhlasí pouze údaj dvou ze tří stopek, pak je úředním časem údaj naměřený oběma souhlasícími stopkami. Pokud se rozcházejí údaje všech tří stopek, je úředním časem střední údaj. Jsou-li k dispozici pouze dva údaje, které se navzájem liší, musí být úředním časem horší údaj.
* Vedoucí časoměřič stanoví úřední čas každého atleta, přičemž musí uplatňovat příslušná ustanovení tohoto pravidla a připraví výsledky pro zveřejnění.

Plně automatické časoměrné zařízení

* Plně automatické časoměrné zařízení (dále cílová kamera), odpovídající pravidlům IAAF, má být používána při všech soutěžích.

Systém

* Systém musí být testován a mít certifikát přesnosti vydaný během 4 let před soutěží a dále:
  \begitems \style a
  * systém musí zaznamenávat cíl kamerou umístěnou v prodloužení cíle a poskytovat kompozitní obrázky.
    \begitems \style i
    * Pro soutěže uvedené v P1.1 musí být kompozitní obraz složený z alespoň 1000 záběrů za sekundu.
    * Pro ostatní soutěže musí být kompozitní obraz složený z alespoň 100 záběrů za sekundu.
    \enditems

  V obou případech musí být obraz synchronizovaný s časovou stupnicí se stejnoměrným dělením po 0,01 s.
  * systém musí být spouštěn automaticky signálem startéra tak, že celkové zpoždění mezi signálem z hlavně nebo jeho ekvivalentní vizuální indikací a spuštěním časoměrného systému je neměnné a rovné či menší než 0,001 sekunda.
  \enditems
* Pro ověření přesného postavení cílové kamery a usnadnění čtení jejího záznamu musí být průsečíky čar vymezujících jednotlivé dráhy s cílovou čarou vhodně vyznačeny černou barvou. Toto označení nesmí zasahovat před náběžnou hranu cílové čáry a nesmí zasahovat dále než 20 mm za tuto hranu. Podobné černé značky mohou být provedeny na každé straně průsečíků čar oddělujících jednotlivé dráhy s cílovou čárou pro usnadnění odečtů.
* Umístění jsou ze záznamu odečítány pomocí kursoru, který zaručuje kolmé postavení odečítací rysky vůči časové stupnici.
* Systém musí automaticky zaznamenávat časy atletů v cíli a musí být schopný poskytnout tištěný obraz, který ukazuje čas kteréhokoliv atleta. Navíc systém musí poskytnout tabulkový přehled ukazující časy nebo jiné výsledky každého atleta. Následné změny automatických stanovených hodnot a ručně vložené údaje (doba startu, ukončení závodu) musí být systémem zaznamenány automaticky na časové stupnici a tabulkovém přehledu.
* Časoměrné zařízení, které pracuje automaticky buď pouze při startu, nebo pouze v cíli, ale nikoliv v obou případech, nelze považovat ani za ruční, ani za automatickou časomíru, a nesmí být proto použito ke stanovení úředních časů. V takovém případě nemohou být časy odečtené na časové stupnici nikdy považovány za úřední, ale záznam lze použit jako průkazný materiál pro stanovení umístění atletů v cíli a časových intervalů mezi nimi.

POZN.: Není-li časoměrné zařízení spouštěno výstřelem startérovy pistole nebo schváleného startovacího zařízení, musí být tato skutečnost automaticky patrná ze záznamu.

Práce s cílovou kamerou

* Vedoucí rozhodčí cílové kamery je zodpovědný za funkci systému. Před začátkem soutěží sejde s technickou obsluhou a seznámí se s činností systému a zkontroluje jednotlivá nastavení.

Před zahájením každého bloku soutěží vedoucí rozhodčí cílové kamery ve spolupráci vrchním rozhodčím pro běhy na dráze a startérem připraví kontrolu nulového časového údaje pro ověření, že zařízení je spouštěno automaticky signálem startéra nebo schváleného startovacího zařízení v souladu s P165.14.b) (tj. s přesností alespoň na 0,001 s).

Dohlíží na kontrolu funkce zařízení, a zda je (jsou) kamera (y) správně v zákrytu s cílovou čárou.

* Pokud je to možné, použijí se alespoň dvě cílové kamery, každá z jedné strany drah. Oba systémy mají být technicky zcela odděleny, tj. napájeny z různých zdrojů a mají zaznamenávat a přenášet signál startéra nebo schváleného startovacího zařízení samostatnými zařízeními nebo kabely.

POZN. : Při použití více cílových kamer jedna z nich musí být technickým delegátem (nebo mezinárodním rozhodčím cílové kamery, je-li delegován) ještě před zahájením soutěží označena jako úřední. Časy a umístění stanovené ostatními kamerami se vezmou v úvahu pouze v případě pochybností o přesnosti měření úřední kamery nebo je-li potřeba dalšího snímku při řešení nejasností v pořadí v cíli (např. při překrytí atletů na úředním záběru).

* Vedoucí rozhodčí cílové kamery spolu se svými asistenty musí stanovit umístění atletů a jimi dosažené časy. Umístění běžců a úřední časy zaznamená do úředního zápisu a po podpisu jej předá sekretáři soutěží.
* Plně automatické elektrické měření musí být považováno za úřední, dokud příslušný rozhodčí nerozhodne, že provedené měření je z nějakého důvodu nepřesné. V takovém případě musí být jako úřední časy vzata měření záložních časoměřičů, pokud možno upravená na základě časových intervalů získaných z údajů cílové kamery.
* Časy se musí z cílového záznamu odečítat takto:
  \begitems \style a
  * Při všech závodech do 10 000 m vč., pokud čas není zaznamenán přesně na setinu sekundy, musí být odečtený čas zaokrouhlen a zaznamenán na nejbližší vyšší hodnotu v setinách sekundy, např. 26:17,533 musí být uveden jako 26:17,54.
  * Při všech delších závodech na dráze musí být všechny časy, které nekončí dvěma nulami, zaokrouhleny na nejblíže vyšší 0,1 sekundy a takto zaznamenány, např. čas 59:26,322 musí být zaznamenán jako 59:26,4.
  * Při všech závodech, které se částečně či zcela konají mimo stadion všechny časy nekončící třemi nulami musí být zaokrouhleny na nejblíže vyšší celou sekundu, např. dosažený čas 2:09:44,322 je zaznamenán jako 2:09:45.
  \enditems

Měření pomocí čipů (transpondérů)

* Čipová časomíra, tj. časoměrný systém pracující s čipy (transpondéry, tj. prvky pro automatický záznam průchodu atleta startovní a cílovou čarou), odpovídající pravidlům IAAF, je dovolen pro soutěže podle P230 (pro závody, které neprobíhají pouze na stadionu), P240, P250, P251 a P252, pokud:
  \begitems \style a
  * Systém nevyžaduje žádnou činnost atleta během soutěže, na startovní čáře či jakékoliv cílové čáře, a nezpůsobí žádné zpoždění ve zpracování výsledků.
  * Váha čipu a jeho krytu, které má atlet na sobě nebo na svém oblečení či obuvi, je nepodstatná.
  * Systém je spuštěn výstřelem startérovy pistole nebo je synchronizován se startovním signálem.
  * Zařízení použité na startu, podél tratě nebo na cílové čáře nepředstavuje pro atlety během soutěže žádnou podstatnou překážku nebo zábranu.
  * Při všech závodech musí být všechny časy, které nekončí nulou, zaokrouhleny a zaznamenány na nejbližší vyšší celou hodnotu, např. 2:09:44,3 musí být zaznamenán jako 2:09:45.

  Pozn.: Pro atletické a chodecké závody na silnici je oficiálním časem doba, která uplyne mezi výstřelem startéra (nebo synchronizovaného startovního signálu) a okamžikem, kdy atlet překročí cílovou čáru. Atletovi, který překročí startovní čáru po výstřelu startéra, je možno sdělit čas, který uplynul od překročení startovní čáry do překročení cílové čáry, ale tento čas nebude považován za oficiální. Pořadí, ve kterém závodníci překročili cílovou čáru, bude považováno za oficiální pořadí v cíli.

  * Zatímco zaznamenané pořadí v cíli a časy mohou být považovány za oficiální, pokud je vyžadováno, musí být uplatněna  P164.2 a P165.2.

  POZN.: Doporučuje se, aby stanovení správného pořadí v cíli a identifikaci atletů zajišťovali též rozhodčí nebo byl použit videozáznam.
  \enditems
* Vedoucí rozhodčí čipové časomíry je zodpovědný za správnou funkci systému. Před zahájením soutěže se setká s technickou obsluhou, seznámí se s činností systému a zkontroluje jednotlivá nastavení.  Dohlíží na zkoušku systému a ujistí se, že čip při průchodu přes cílovou čáru zaznamená čas atleta. Spolu s vrchním rozhodčím se ujistí, že je  zajištěno, pokud je to nezbytné, uplatnění P165.24.f).
\enditems

\secc Nasazování a kvalifikace v soutěžích na dráze

\begitems \style N
Kvalifikační kola

* Kvalifikační kola (rozběhy, meziběhy, semifinále) v atletických soutěžích se konají, pokud počet startujících nedovoluje, aby soutěž mohla řádně proběhnout v jediném kole (finále). Konají-li se kvalifikační kola, musí jimi projít všichni startující a do dalšího kola se kvalifikovat, vyjma případů, kdy vyjma případů, kdy řídící orgán dané soutěže rozhodne, že v jedné nebo více jednotlivých soutěží se uskuteční předběžné kvalifikační kolo (nebo kola) a to buď během dané soutěže, nebo při jedné nebo více předcházejících soutěží. Toto předběžné kvalifikační kolo rozhodne o některých nebo o všech atletech, kteří budou oprávnění soutěžit v některém z dalších kol soutěže. Takový postup a další podmínky (jako je dosažení stanoveného výkonu během stanovené doby, dosažení stanoveného umístění v předem určených soutěžích nebo umístění na žebříčku), za nichž je atlet oprávněn k účasti v určitém kole soutěže, musí být stanoveno v rozpisu daných soutěží.

POZN.: viz též P146.4.c).

* Kvalifikační kola v atletických disciplínách musí sestavovat jmenovaní techničtí delegáti jak je dále uvedeno. Pokud nebyl jmenován žádný technický delegát, nasazení provede pořadatel.
  \begitems \style a
  * Předpisy (soutěžní řád) každé soutěže mají zahrnovat tabulky, podle nichž, pokud nenastanou mimořádné okolnosti, bude určen počet kol, počet (roz)běhů v jednotlivých kolech a kvalifikační postup, tj. kdo postoupí podle umístění a kdo podle dosaženého času. Taková informace bude stanovena rovněž pro předběžnou kvalifikaci.

  Tabulky, které lze použít, pokud neexistují jakákoliv ustanovení v příslušných předpisech nebo ustanoveních pořadatele, budou uveřejněny na webových stránkách IAAF.
  * Kdykoliv je to možné, mají být reprezentanti jednoho státu či družstva rozděleni do různých běhů ve všech kvalifikačních kolech dané soutěže. Při aplikaci tohoto ustanovení po prvním kole lze požadované vzájemné přesuny mezi jednotlivými běhy provést atlety nasazenými do stejné „skupiny drah“ podle P166.4.b).
  * Při sestavování rozběhů se doporučuje vzít v úvahu co nejvíce informací o výkonnosti atletů a jednotlivé běhy obsadit tak, aby se za normálních okolností do finále probojovali nejlepší závodníci.

  Pozn.: Tohoto ustanovení se obdobným způsobem použije i při soutěžích klubových družstev, a to jak při soutěžích národních, tak i mezinárodních.
  \enditems

Seřazení běžců a rozdělení do jednotlivých běhů

* Atlet jsou do jednotlivých běhů rozděleni následujícím způsobem:
  \begitems \style a
  * Pro první kolo budou závodníci rozdělení střídavým (cik-cak) způsobem na základě příslušného seznamu platných výkonů dosažených v předem daném údobí.
  * Po prvním kole musí být závodníci nasazování do běhů v následujícím kole tímto postupem:
    \begitems \style i
    * Pro běhy od 100 m do 400 m včetně a běhy štafetové do 4x400 m včetně podle umístění a časů z každého přecházejícího kola. Pro tento účel budou závodníci seřazeni následovně:
      \begitems \style n
      * nejrychlejší vítěz rozběhů,
      * druhý nejrychlejší vítěz rozběhů,
      * třetí nejrychlejší vítěz rozběhů, atd.,
      * nejrychlejší atlet na druhém místě v rozbězích,
      * druhý nejrychlejší atlet na druhém místě,
      * třetí nejrychlejší atlet na druhém místě, atd.,

      načež následuje:

      * nejrychlejší atlet postupující podle dosaženého času,
      * druhý nejrychlejší atlet postupující podle času,
      * třetí nejrychlejší atlet postupující podle času, atd.
      \enditems
    * Pro ostatní atletické disciplíny zůstává původní seznam startujících dle výkonnosti pro nasazování v dalších kolech v platnosti a mění se jen na základě zlepšení dosažených v předcházejících kolech.
    \enditems
  * Závodníci jsou poté do jednotlivých běhů nasazeni střídavým systémem dle následujícího příkladu pro tři běhy A, B, a C

\table{l|llllllll}{
A & 1 & 6 & 7 & 12 & 13 & 18 & 19 & 24\cr
B & 2 & 5 & 8 & 11 & 14 & 17 & 20 & 23\cr
C & 3 & 4 & 9 & 10 & 15 & 16 & 21 & 22\cr
}

  * V každém případě pořadí, v němž budou běhy odstartovány, musí být určeno losem poté, co bylo rozhodnuto o jejich obsazení.
  \enditems

Rozdělení do drah

* V bězích od 100 m do 800 m vč. a v rozestavných bězích do 4x400 m vč., kde se konají kvalifikační kola, se postupuje takto:
  \begitems \style a
  * V prvém kole a kterémkoliv předběžném kvalifikačním kole podle P166.1 musí být obsazení drah určeno losem.
  * Pro další kola jsou závodníci seřazení na základě výsledků předcházejícího kola podle postupu uvedeného v P166.3.b.i), pro běh na 800 m podle P166.3.b.ii).

  Dráhy jsou losovány ve třech skupinách:
    \begitems \style i
    * V první skupině losují závodníci nebo týmy na prvních čtyřech místech o dráhy 3,4,5 a 6.
    * Ve druhé skupině losují závodníci či týmy na pátých a šestých místech o postavení v drahách 7 a 8.
    * Ve třetí skupině losují závodníci či týmy s nejnižším umístěním o postavení v drahách 1 a 2.
    \enditems
  \enditems

POZN. 1: Je-li méně nebo více  než osm drah, postupuje se obdobným způsobem s nutnou úpravou.

POZN. 2: V soutěžích uvedených v P1.1.d) až j) mohou v běhu na 800 m závodníci startovat po jednom nebo dvou v každé dráze nebo z hromadného startu z obloukové čáry. Při soutěžích uvedených v P1.1.a),b),c) to lze běžně uplatnit pouze v prvním kole, kromě stavu, kdy vlivem rovnosti umístění nebo rozhodnutím vrchního rozhodčího či jury postoupilo do následujícího kola více atletů, než se předpokládalo.

POZN. 3: V kterémkoliv běhu na 800, vč. finále, kde z jakéhokoliv důvodu má startovat více atletů než je počet drah, technický delegát(i) rozhodne, do které dráhy bude nalosován více než jeden atlet.

POZN. 4: Pokud startuje méně atletů než je počet drah, vnitřní (první) dráha má být ponechána volná.

* Při soutěžích uvedených v P1.1.a), b), c), u běhů delších než 800 m, rozestavných běhů delších než 4x400 m a všech běhů, které probíhají přímo jako finálové, se dráhy, resp. pořadí na startu, stanoví losem.
* Pokud se rozhodne, že místo kvalifikačních kol a finále proběhne soutěž jednorázově, v několika bězích, musí být v rozpisu soutěže uveden způsob nasazování atletů do jednotlivých běhů, losování drah a způsob stanovení konečného pořadí.

Pozn.: V soutěžích ČAS je rozhodující znění Soutěžního řádu pro daný rok.

* Atletovi nesmí být dovoleno startovat v jiném běhu, než do kterého byl zařazen, vyjma okolností, které podle názoru technického delegáta nebo vrchního rozhodčího změnu ospravedlňují.

Pozn.: V oddílových soutěžích ČAS má stejné právo změny řídící pracovník soutěže.

Postupy z kvalifikačních kol

* Z každého kvalifikačního běhu musí do dalšího kola postoupit alespoň první dva závodníci, je-li to možné, doporučuje se postup alespoň prvních tří běžců.

Vyjma situací, kdy platí P167, se další postupující mohou kvalifikovat do následujícího kola podle umístění či dosaženého času v souladu s P166.2, na základě příslušných ustanovení soutěžního řádu nebo z rozhodnutí Technického delegáta. V případě postupu dle dosažených časů je nezbytné použít pouze jediného časoměrného systému.

Jednodenní soutěže (mítinky)

* Při soutěžích uvedených v P1.1.e), i) a j) mohou být běžci nasazeni, seřazeni nebo rozmístěni do drah podle příslušného soutěžního řádu nebo jakékoliv jiné metody stanovené pořadatelem, s níž jsou atlet nebo jejich zástupci seznámeni, nejlépe před vlastní soutěží.

Minimální doba mezi soutěžními koly

* Tam, kde je to uskutečnitelné, je třeba dodržet následující minimální dobu mezi posledním během jednoho kvalifikačního kola a prvním během následujícího kola nebo finálovou soutěží:

\table{ll}{
do 200 m vč.            & 45 minut\cr
nad 200 m do 1000 m vč. & 90 minut\cr
nad 1000 m              & alespoň do následujícího dne\cr
}

\itemnum=30
* Při soutěžích ČAS se nasazování a postup do dalších kol řídí obdobnými pravidly. Zodpovědným za konečné rozhodnutí je řídící orgán soutěže.

Pozn.: V soutěžích oddílových družstev může takováto pořadí stanovit příslušný řídící pracovník soutěže.
\enditems

\secc Rovnost výkonů

\begitems \style N
* Pokud rozhodčí v cíli nebo rozhodčí cílové kamery nemohou stanovit pořadí atletů ve smyslu ustanovení P164.2, P165.18, P165.21 nebo P165.24 (podle okolností), musí být rozhodnuto, že došlo ke shodě umístění a tato shoda zůstává v platnosti.

Rovnost při řazení dle dosažených výkonů

* Pokud dojde k rovnosti při seřazování atletů podle výkonů podle P166.3.b), vedoucí rozhodčí cílové kamery musí vzít v úvahu skutečně zaznamenané časy na tisíciny sekundy. Pokud jsou i tak časy stejné, rozhoduje o lepším postavení v žebříčku los.

Pozn.: Pokud cílová kamera dovoluje odečtení času na desetitisíciny sekundy, je třeba vzít v úvahu takový čas.

Rovnost na posledním místě pro postup podle umístění

* Pokud i po uplatnění P167.1 trvá shoda na postupovém místě podle umístění, a je k dispozici místo nebo dráha (vč. sdílení drah v běhu na 800 m), postupují do dalšího kola oba běžci ve shodě. Pokud je to neproveditelné, rozhodne o postupu do dalšího kola los.
* Při rozhodování o postupu z kteréhokoliv kvalifikačního kola do dalšího kola soutěže na základě umístění a času (např. první tři z každého ze dvou běhů plus další dva nejrychlejší) a na posledním postupovém místě je shoda umístění na postupovém místě podle pořadí v cíli, je odpovídajícím počtem snížen počet atletů postupujících časem.

Rovnost na posledním místě pro postup podle časů

* Pokud dojde ke shodě na posledním místě při postupu podle dosažených časů, vedoucí rozhodčí cílové kamery musí vzít v úvahu skutečně zaznamenané časy na tisíciny sekundy. Pokud jsou i tak časy stejné a je k dispozici místo nebo dráha (vč. sdílení drah v běhu na 800 m), postupují do dalšího kola oba běžci ve shodě. Není-li to neproveditelné, rozhodne o postupu do dalšího kola los.
\enditems

\secc Běhy překážkové

\begitems \style N
* Standardními délkami překážkových závodů jsou:

MUŽI, muži U20 (junioři) a U18 (dorostenci): 110~m, 400~m

ŽENY, ženy U20 (juniorky) a U18 (dorostenky): 100~m, 400~m

V každé dráze musí být sada 10 překážek, rozestavených dle tabulky:

\table{|5{c|}}{\crl
           &           & \mspan3[c|]{vzdálenosti [m]}\crlp{3-5}
           & délka     & náběh na    & mezi       & doběh\cr
           & trati [m] & 1. překážku & překážkami & do cíle\crl
muži       & 110       & 13,72       & 9,14       & 14,02\cr
jun. + dci & 400       & 45,00       & 35,00      & 40,00\crl
ženy       & 100       & 13,00       & 8,50       & 10,50\cr
jun. + dky & 400       & 45,00       & 35,00      & 40,00\crl
}

Každá překážka musí být postavena tak, že základna překážky směřuje na stranu náběhu na překážku, přičemž hrana horní příčky, na straně náběhu na překážku, leží ve svislé rovině, procházející hranou příslušného značení dráhy, na straně náběhu.

* Překážky musí být zhotoveny z kovu nebo jiného nekovového vhodného materiálu s horní příčkou ze dřeva nebo obdobného vhodného materiálu. Skládají se ze dvou základen a dvou stojanů nesoucích obdélníkový rám zesílený alespoň jednou příčkou. Stojany musí být upevněny na konci základen. Překážka musí být zhotovena tak, že k jejímu poražení je třeba síly alespoň 3,6 kg působící vodorovně na střed horní hrany horní příčky. Výška překážky může být nastavitelná pro příslušný závod. Překážka musí být opatřena vyvažovacími závažími umístěnými v základnách, které musí být nastavitelné do takové polohy, aby při každé výšce překážky byla síla potřebná k poražení překážky rovna alespoň 3,6 kg, avšak nejvýše 4,0 kg.

Maximální vodorovné vyhnutí horní příčky (vč. výchylky stojanů) při působení hmotnosti 10 kg ve vodorovném směru na střed příčky nesmí překročit 35 mm.

\picw=5cm \inspic img/provedeni_prekazky.jpg
\caption/f Příklad provedení překážky.

* Rozměry. Standardní výšky (m) překážek jsou:

\table{c|5{c}}{
Trať      & muži  & junioři & dorostenci & ženy/juniorky & dorostenky\crl
100/110 m & 1,067	& 0,991   & 0,914      & 0,838         & 0,762\cr
400 m     &	0,914	& 0,914   & 0,838      & 0,762         & 0,762\cr
}

POZN.: Pro soutěže juniorů na 110 m je přípustná výška překážek až 1,00 m.

Pozn.: Pro závody ČAS platí výšky překážek podle soutěžního řádu.

Není-li uvedeno jinak, je pro předepsanou výšku překážek povolena tolerance 3 mm.Kromě zde uvedených výšek musí mít překážky šířku v rozmezí od 1,18 do 1,20 m, délku základny nejvýše 0,70 m a celková hmotnost překážky nesmí být menší než 10 kg.

* Horní příčka musí být 70 mm ($\pm$5 mm) vysoká, o tloušťce 10 až 25 mm a její horní hrany musí být zaobleny. Příčka musí být na obou koncích dobře upevněna.
* Horní příčka je pruhovaná černobíle nebo jinými výrazně kontrastními barvami (kontrastními i vůči okolí), a to tak, že světlejší pruhy jsou na okrajích příčky a jsou nejméně 225 mm široké. Musí mát takovou barvu, aby byly viditelné pro všechny vidoucí běžce.
* Všechny běhy musí být běženy v drahách. Každý atlet musí přejít přes překážky ve své dráze a musí po celý závod setrvat ve své dráze, vyjma případů, kdy platí P163.4. Vyjma situace, kdy to neovlivní běh nebo nebude bránit jinému atletu či atletům v běhu, bude atlet rovněž diskvalifikován, pokud přímo nebo nepřímo srazí nebo výrazně posune překážku v jiné dráze.
* Atlet musí překonat všechny překážky, jinak bude diskvalifikován.

Atlet bude diskvalifikován i v případě, že
  \begitems \style a
  * při přechodu přes překážku vede chodidlo nebo nohu pod vodorovnou úrovní horní hrany (na jedné nebo druhé straně) kterékoliv překážky
  * podle názoru vrchního rozhodčího či vrchníka úmyslně porazí některou překážku.
  \enditems

POZN.: Pokud je toto pravidlo jinak dodrženo a nedojde k posunu překážky nebo jejího snížení jakýmkoliv způsobem, vč. naklonění v kterémkoliv směru, atlet může překážku překonat jakýmkoliv způsobem.

* Vyjma případu uvedeného v P168.6 a  P168.7) tohoto pravidla, sražení překážky nesmí být důvodem k diskvalifikaci ani na závadu vytvoření rekordu.
\itemnum=30
* Kromě tratí uvedených v odst. 1 se pro jednotlivé kategorie pořádají závody na dále uvedených tratích.

Pro konstrukci překážek pro soutěže mužských i ženských složek platí ustanovení odstavců 2 až 5.

  \begitems \style a
  * Tratě, překážky a jejich rozestavení pro mužské složky:

\table{|7{c|}}{\crl
         & \mspan6[c|]{uvedeno v metrech}\crlp{2-7}
         & délka & výška    & náběh na & mezi  & doběh   & počet\cr
         & trati & překážky & 1. přek. & přek. & do cíle & přek.\crl
muži     & 200   & 0,762    & 18,29    & 18,29 & 17,10   & 10\crlp{2-7}
junioři  & 300   & 0,914    & 50,00    & 35,00 & 40,00   & 7\crl
dorci    & 300   & 0,838    & 50,00    & 35,00 & 40,00   & 7\crl
žáci     & 100   & 0,838    & 13,00    &  8,50 & 10,50   & 10\crlp{2-7}
starší   & 200   & 0,762    & 18,29    & 18,29 & 17,10   & 10\crl
žáci ml. & 60    & 0,762    & 11,70    &  7,70 &  9,80   & 6\crl
}

  * Tratě, překážky a jejich rozestavení pro ženské složky:

\table{|7{c|}}{\crl
           & \mspan6[c|]{uvedeno v metrech}\crlp{2-7}
           & délka & výška    & náběh na & mezi  & doběh   & počet\cr
           & trati & překážky & 1. přek. & přek. & do cíle & přek.\crl
ženy       & 200   & 0,762    & 18,29    & 18,29 & 17,10   & 10\crlp{2-7}
juniorky   & 300   & 0,762    & 50,00    & 35,00 & 40,00   & 7\crl
dorky      & 200   & 0,762    & 18,29    & 18,29 & 17,10   & 10\crlp{2-7}
           & 300   & 0,762    & 50,00    & 35,00 & 40,00   & 7\crl
žákyně     & 100   & 0,762    & 13,00    &  8,20 & 13,20   & 10\crlp{2-7}
starší     & 200   & 0,762    & 18,29    & 18,29 & 17,10   & 10\crl
žákyně ml. & 60    & 0,762    & 11,70    &  7,70 &  9,80   & 6\crl
}
  \enditems
\enditems

\secc Steeplechase

\begitems \style N
* Standardními vzdálenostmi jsou 2000 m a 3000 m.
* V závodě na 3000 m musí atlet překonat 28 pevných překážek a 7 překážek s vodním příkopem, v závodě na 2000 m 18 pevných překážek a 5 překážek s vodním příkopem.
* Při steeplechase závodě musí být v každém úplném kole překonáno 5 překážek, z nichž vodní příkop je čtvrtou v pořadí. Překážky mají být po trati rovnoměrně rozmístěny tak, že vzdálenost mezi nimi je přibližně jedna pětina jmenovité délky jednoho kola.

POZN. 1: Mezery mezi překážkami lze upravit tak, aby bezpečná vzdálenost od překážky či startovní čáry k následující překážce byla dodržena před i za cílovou čárou, jak je uvedeno v manuálu IAAF Atletická zařízení (IAAF Track and Field Facilities Manual).

POZN. 2: Při závodě na 2000 m překážek, pokud je vodní příkop uvnitř atletického oválu, je třeba překonat cílovou čáru dvakrát, než závodníci vběhnou do prvního kola s pěti překážkami.

* V závodě na 3000 m se úsek od startu k počátku prvního úplného kola musí běžet bez překážek, které musí být odstraněny, dokud všichni závodníci tímto úsekem neproběhnou. V závodě na 2000 m je první překážkou třetí překážka úplného kola, první dvě překážky musí být během tohoto kola odstraněny. Vodní příkop v prvním kole je druhou, a v dalších kolech čtvrtou překážkou.
* Překážky pro soutěž mužů musí být 0,914 m vysoké, ($\pm$3 mm) a musí být min. 3,94 m široké. Pro soutěže žen musí být 0,762 m ($\pm$3mm) vysoké a min. 3,94 m široké. Průřez horního břevna každé překážky, včetně překážky u vodního příkopu, musí být 0,127 m2.

\picw=9cm \inspic img/provedeni_prekazky_steeple.jpg
\caption/f Příklad provedení překážky.

Hmotnost každé překážky musí být v rozmezí 80 až 100 kg a každá překážka musí mít na každé straně základnu širokou 1,20 až 1,40 m (viz obr. 169a).

Překážka na vodním příkopu musí být 3,66 m ($\pm$0,02 m) široká a musí být pevně ukotvená v betonové stěně vodního příkopu tak, že je možný jen její minimální horizontální výkyv.

Horní břevno každé překážky musí být opatřeno pruhy v bílé a černé barvě nebo v jiných výrazně odlišných barvách (kontrastními i vůči okolí), a to tak, že světlejší pruhy, široké alespoň 225 mm, jsou na okrajích. Musí mát takovou barvu, aby byly viditelné pro všechny vidoucí běžce.

Překážka musí být na drahách umístěna tak, že alespoň 0,30 m délky horního břevna, měřeno od vnitřního okraje, zasahuje do vnitřního prostoru pole.

POZN.:  Doporučuje se, aby první překážka v závodě byla široká alespoň 5 m.

* Vodní příkop, včetně překážky, musí být 3,66 m ($\pm$2 cm) dlouhý a 4,00 m ($\pm$2 cm) široký.

Dno vodního příkopu musí být pokryto umělým povrchem nebo rohoží dostatečné tloušťky, zaručující bezpečný dopad běžců a poskytující dostatečnou oporu atletické obuvi. Na začátku závodu musí být hladina vody v úrovni povrchu dráhy s odchylkou 2 cm. Hloubka vody na straně překážky musí být 0,70 m do vzdálenosti 0,30 m od překážky. Od tohoto místa musí dno plynule stoupat až na úroveň dráhy na opačném konci příkopu.

POZN.: Hloubka vody na straně překážky může být po délce asi 1,2 m snížena ze 70 cm na 50 cm. Sklon příkopu ($12,4^\circ$ $\pm1^\circ$) musí zůstat tak, jak ukazuje obr. 169b. Doporučuje se, aby nové vodní příkopy měly nižší hloubku.

* Každý atlet musí překonat (přeskočit nebo přebrodit) vodní příkop a překonat každou překážku. Pokud tak neučiní, bude diskvalifikován.

Atlet bude rovněž diskvalifikován, pokud
  \begitems \style a
  * došlápne stranou od vodního příkopu,
  * v okamžiku překonávání překážky je jeho chodidlo nebo noha vedle překážky (na jedné či druhé straně) pod vodorovnou úrovní horní plochy překážky.
  \enditems

Je-li toto pravidlo dodrženo, může atlet překonat překážku jakýmkoliv způsobem.

\picw=8cm \inspic img/vodni_prikop.jpg
\caption/f Vodní příkop.

\itemnum=30
* V soutěžích řízených ČAS se rovněž vypisuje trať 1500 m, na níž je třeba překonat celkem 12 pevných překážek a 3 překážky s vodním příkopem. Výška překážek 0,762 m.
* V závodě na 1500 m se úsek od startu k počátku prvního úplného kola musí běžet bez překážek, které musí být odstraněny, dokud všichni závodníci nevběhnou do prvního kola.
\enditems

\secc Štafetové běhy

\begitems \style N
* Standardními soutěžemi ve štafetových bězích jsou 4x100 m, 4x200 m, 4x400 m, 4x800 m, 100-200-300-400 m (sprinterská úseková štafeta), 1200-400-800-1600 m (distanční úseková štafeta), 4x1500 m

POZN.: Štafetový běh 100-200-300-400 m lze běžet i v obráceném pořadí úseků. V takovém případě je třeba použít ustanovení P.170.14, P170.18, P170.19 a P170.20 v příslušně pozměněné podobě.

* Úsekové čáry vymezující jednotlivé úseky štafetových běhů a hraniční čáry vymezující jednotlivá předávací území musí být  50 mm široké a vedené napříč příslušnými drahami.
* Při štafetách na 4x100 m a 4x200 m a na prvním a druhém úseku štafety na 100-200-300-400 m je každé předávací území dlouhé 30 m, přičemž počáteční "čára se zaškrtnutím" (scratch line) je 20 m před čárou příslušného úseku a koncová "čára se zaškrtnutím" (scratch line) je 10 m za čárou daného úseku. Pro třetí předávku v běhu 100-200-300-400 m a všechny předávky běhů na 4x400m a delších jsou předávací území dlouhá 20 m, přičemž jejich středem je úseková čára příslušného úseku. Předávací území začíná a končí na hraně hraniční čáry bližší startu ve směru běhu. Pokud předávky probíhají v drahách, pověřený rozhodčí dbá na to, aby atleti byli rozmístění v jejich předávacím území. Pověřený rozhodčí současně dbá na dodržení pravidla P170.4.
* V úsecích, které se běží v drahách, si atlet může ve své vlastní dráze udělat jednu kontrolní značku pomocí přilnavé pásky o rozměru max. 0,05 x 0,40 m, výrazné barvy, kterou nelze zaměnit se stálým značením drah. Jiné kontrolní značky nemohou být v žádném případě použity. Rozhodčí musí příslušného atleta upozornit, aby upravil nebo odstranil jakoukoliv značku, která neodpovídá tomuto pravidlu. Pokud tak neučiní, musí je odstranit rozhodčí.

POZN.: Vážnější případy mohou být dále projednány podle P125.5 a P145.2.

Pozn.: Na škvárové nebo travnaté dráze si běžci mohou udělat kontrolní značku ve vlastní dráze vrypem do povrchu dráhy.

* Štafetový kolík musí být duté těleso kruhového průřezu, hladkého povrchu, zhotovené ze dřeva, kovu nebo jiného tuhého materiálu z jednoho kusu, jehož délka musí 280 být až 300 mm. Kolík musí mít vnější průměr 40 mm $\pm$2 mm a nesmí mít hmotnost menší než 50 gramů. Kolík může být barevný, aby byl jasně viditelný během závodu.
* \begitems \style a
  * Ve všech štafetových bězích konaných na stadionu musí být použit kolík, který musí být nesen v ruce po celý závod. Alespoň v soutěžích uvedených v P1.1.a), b), c) a f) musí být každý kolík očíslovaný a rozdílné barvy a může v něm být zabudovaný čip.

  POZN.: Pokud je to možné, barva kolíku přidělená pro každou z drah nebo pozici na startu má být uvedená ve startovní listině.

  * Závodníci nesmějí mít rukavice nebo nanášet na ruce materiál nebo látku (jinou než povolenou podle P144.4.c)), umožňující lepší držení kolíku.
  * Pokud kolík upadne na zem, musí jej zvednout atlet, který jej upustil, smí přitom opustit svou dráhu, nesmí si však takto zkrátit trať. Navíc, pokud kolík upadne tak, že se dostane mimo dráhu nebo dopředu (vč. překročení cílové čáry), atlet, který jej upustil a znovu uchopil, se musí vrátit do místa, kde kolík držel před upuštěním a teprve potom může pokračovat v závodě. Pokud je toto ustanovení dodrženo a není bráněno v závodě jinému atletovi, upuštění kolíku nesmí být důvodem k diskvalifikaci.
  \enditems

Pokud atlet toto pravidlo nedodrží, jeho družstvo bude diskvalifikováno.

* Při všech štafetových bězích musí být kolík předán uvnitř předávacího území. Předávka začíná v okamžiku, kdy se přebírající atlet poprvé kolíku dotkne a končí teprve v okamžiku, kdy je kolík pouze v ruce přebírajícího atleta. Rozhodující je poloha kolíku v předávacím území, nikoliv poloha těla nebo končetin atletů. Předání kolíku mimo předávací území znamená diskvalifikaci.
* Do okamžiku, kdy je kolík pouze v ruce přebírajícího běžce, platí P163.3, pouze pro předávajícího běžce. Následně pak platí pouze pro přebírajícího běžce.

Navíc před obdržením kolíku i po jeho předání musí závodníci zůstat ve svých drahách nebo dodržet polohu až do doby, kdy je volná, aby nedošlo ke vzájemné kolizi. Pro tyto atlety P163.3 a P163.4 neplatí. Pokud však kterýkoliv atlet brání členu jiného družstva, vč. tím, že běží mimo postavení nebo dráhu, bude uplatněno P163.2.

* Pokud během závodu atlet vezme nebo zvedne kolík jiného družstva, jeho družstvo bude diskvalifikováno. Pokud člen druhého družstva, které takto o svůj kolík přišlo, zvedne kolík prvního družstva a nezíská tím výhodu, druhé družstvo nebude postiženo penalizací.
* Každý atlet družstva může běžet pouze jeden úsek. Členy družstva pro běh štafetový mohou být jen atleti uvedení ve startovní listině soutěží, ať již pro tuto či jinou disciplínu. Nicméně v sestavě, v níž družstvo soutěž ve štafetovém běhu zahájilo, je možno v dalších kolech vyměnit pouze dva atlety. Atlet, který po startu v předcházejícím kole byl v dalším kole nahrazen jiným běžcem, se již do družstva nemůže vrátit. Pokud některé družstvo poruší toto ustanovení, bude diskvalifikováno.

POZN.: Příslušný řídící orgán (pokud není, pak pořadatel) může v rozpisu soutěží stanovit, že během soutěže v běhu štafetovém může být použito více náhradníků než dva). Platí od 20.06.2018.

* Složení družstva a pořadí jednotlivých běžců musí být úředně oznámeno nejpozději jednu hodinu před prvním příchodem do svolavatelny (tj. v čase, kdy závodníci již musí být ve svolavatelně) pro účastníky prvního běhu každého kola. Pozdější změny mohou být provedeny pouze ze zdravotních důvodů, na základě ověření lékařem, který byl určen pořadatelem, a to pouze do konce doby vymezené pro prezentaci příslušného běhu (tj. v čase, kdy závodníci mají opustit svolavatelnu), v němž družstvo startuje. Družstvo musí startovat v uvedeném složení a pořadí jednotlivých běžců. Pokud některé družstvo poruší toto ustanovení, bude diskvalifikováno.
* Štafetový běh 4x100 m se běží celý v drahách.
* Štafetový běh 4x200 m lze běžet jedním z následujících způsobů:
  \begitems \style a
  * pokud je to možné, zcela v drahách (čtyři zatáčky),
  * v drahách první dva úseky a část třetího úseku po nejbližší hranu čáry seběhu k mantinelu podle P163.5, kde běžci mohou své dráhy opustit (tj. celkem tři zatáčky v drahách),
  * první úsek v drahách po nejbližší hranu čáry seběhu k mantinelu podle P163.5, kde běžci mohou své dráhy opustit (tj. celkem jedna zatáčky v drahách).
  \enditems

POZN.: Pokud startují nejvýše čtyři týmy, a nelze použít variantu a), doporučuje se použít variantu c).

* V štafetovém běhu 100-200-300-400 m se mají běžet v drahách první dva úseky a část třetího úseku po nejbližší hranu čáry seběhu k mantinelu podle P163.5, kde běžci mohou své dráhy opustit (tj. celkem dvě zatáčky v drahách).
* Štafetový běh 4x400 m lze běžet jedním z následujících způsobů:
  \begitems \style a
  * v drahách první úsek a část druhého úseku po nejbližší hranu čáry seběhu k mantinelu podle P163.5, kde běžci mohou své dráhy opustit (tj. celkem tři zatáčky v drahách).
  * v drahách první úsek po nejbližší hranu čáry seběhu k mantinelu podle P163.5, kde běžci mohou své dráhy opustit (tj. jedna zatáčka v drahách).
  \enditems

Pozn.: Pokud startují nejvýše čtyři týmy, doporučuje se způsob ad b).

* Štafetový běh 4x800 m lze běžet jedním z následujících způsobů:
  \begitems \style a
  * v drahách první úsek po nejbližší hranu čáry seběhu k mantinelu podle P163.5, kde běžci mohou své dráhy opustit (tj. jedna zatáčka v drahách),
  * bez rozdělení do drah.
  \enditems
* Pokud atlet nedodrží ustanovení P170.13, P170.14, P170.15 nebo P170.16.a), bude jeho družstvo diskvalifikováno.
* Štafetové běhy 1200-400-800-1600 m a 4x1500 m se běží bez rozdělení do drah.
* Při všech předávkách se závodníci musí rozbíhat uvnitř předávacího území. Pokud atlet poruší toto pravidlo, jeho družstvo bude diskvalifikováno.
* Závodníci na třetím a čtvrtém úseku štafetového běhu na 4x400m a běžci na posledním úseku běhu na 1-2-3-400 m se za řízení určeným rozhodčím řadí do vyčkávací pozice na počátku předávacího území v pořadí (od vnitřní dráhy po vnější), v jakém přibíhající členové jednotlivých družstev doběhli do poslední zatáčky svých úseků. Toto své postavení na počátku předávacího území musí očekávající závodníci zachovat a nesmí je již měnit, i když se pořadí přibíhajících atletů mezitím změní. Pokud některý ze atletů toto ustanovení nedodrží, jeho družstvo bude diskvalifikováno.

POZN.: V běhu na 4x200 m, pokud není běžen celý v drahách, kde předcházející úsek se neběží v drahách, se závodníci řadí ve startovním pořadí (od vnitřní dráhy směrem ven).

* V štafetových bězích, na úsecích, které se neběží v drahách, vč. běhů na 4x200, 4x400 m a běhu na 1-2-3-400m, mohou očekávající závodníci zaujmout postavení u vnitřního okraje dráhy podle toho, jak členové družstev dobíhají, pokud nestrkají nebo nepřekážejí jiným atletům v běhu. V bězích na 4x200, 4x400 m a běhu na 1-2-3-400m však musí očekávající závodníci dodržet pořadí v souladu s P170.20. Pokud atlet poruší toto pravidlo, jeho družstvo bude diskvalifikováno.
\itemnum=30
* Pro závod na 4x60 m platí stejná ustanovení jako pro 4x100 m.
* Štafetový běh 4x300m se běží obdobně jako 4x400m, jedním z dále uvedených způsobů:
  \begitems \style a
  * Start s hendikepy 400m + 800m, běží se v drahách první úsek a část druhého úseku po nejbližší hranu čáry seběhu k mantinelu podle P163.5, kde běžci mohou své dráhy opustit (tj. celkem tři zatáčky v drahách). První předávka je na metě 100m, posunutá o hendikepy na 800m, druhá předávka na metě 200m, a třetí předávka na metě 300m. Všechna předávací území mají délku 20 m.
  * Start s hendikepy na 800 m, první úsek se běží v drahách po nejbližší hranu čáry seběhu k mantinelu podle P163.5, kde běžci mohou své dráhy opustit (tj. jedna zatáčka v drahách). Všechny předávky na metách 100 m, 200 m a 300 m mají předávací území o délce 20 m.
  \enditems

Pozn.: Pokud startují nejvýše čtyři týmy, doporučuje se způsob ad b).

* Štafetové běhy, kde první úsek je delší než 800 m, se startují z hromadného startu, z obloukovité startovní čáry.
* Štafetový běh na 100-200-300-400 m je možno běžet též v obráceném pořadí úseků, tj. 400-300-200-100 m, kdy se běží v drahách pouze první úsek a první zatáčka druhého úseku a závodníci pak sbíhají do první dráhy k vnitřnímu obrubníku. Čára pro seběhnutí do první dráhy musí být vyznačena v souladu s P 163.5. Startovní čáry na metě 200 m musí být dále posunuty o posuny startovních čar pro běh na 400 m (tj. celkem o odstupy na 200m + 400m).

Na všech předávkách se přebírající závodníci řadí od vnitřního obrubníku podle pořadí dobíhajících družstev, za dozoru rozhodčího.

* Běh na 400-300-200-100 m je možno alternativně běžet tak, že závodníci na prvním úseku běží v drahách pouze první zatáčku, pak sbíhají do první dráhy k vnitřnímu obrubníku. Startovní čáry musí být proto stupňovitě posunuty stejně, jako jsou posunuty startovní čáry běhu na 800 m. Čára označující metu seběhnutí do první dráhy musí být vyznačena podle  P163.5, resp. P 162.9.

Předávací území 1. předávky má úsekovou čáru na metě 200 m v první dráze.
Předávací území 2. předávky má úsekovou čáru na metě 300 m v první dráze.
Předávací území 3. předávky má úsekovou čáru na metě 100 m v první dráze.

Na všech předávkách se přebírající závodníci řadí od vnitřního obrubníku podle pořadí dobíhajících družstev, za dozoru rozhodčího.
\enditems

\endinput