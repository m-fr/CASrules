\input style

\tit PRAVIDLA ATLETIKY 2018

\vfill
\break

\nonum\notoc\chap Obsah

\maketoc

\vfill
\break

\pageno=1
\nonum\chap DEFINICE POJMŮ

\dt Atletika (Athletics)
\dd Soutěže na dráze a v poli, silniční běhy, chůze, přespolní běhy a běhy do vrchu.
\dend
\dt Atlet (Athlete)
\dd Sportovec, který se účastní mezinárodních soutěží (pro účely P9)
\dend
\dt Atlet mezinárodní úrovně (International-Level Atlete)
\dd Atlet, který je na soupisce testovaných sportovců (Registered Testing Pool) nebo atlet účastnící se mezinárodních soutěží podle definice v antidopingových pravidlech.
\dend
\dt Atletická disciplína (Event)
\dd jednotlivý závod nebo soutěž (např. běh na 100 m nebo hod oštěpem), včetně kvalifikačních kol.
\dend
\dt CAS (Court of Arbitration for Sports)
\dd Arbitrážní dvůr pro sport je nezávislý rozhodčí orgán se sídlem v Lausanne.
\dend
\dt ČAS (Český atletický svaz)
\dd Organizace sdružující atlety České republiky, člen IAAF
\dend
\dt Člen (Member)
\dd Národní organizace, řídící atletiku v příslušné zemi, přidružená k IAAF.
\dend
\dt Členství (Membership)
\dd Příslušnost k IAAF
\dend
\dt IAAF
\dd International Association of Athletics Federations (Mezinárodní asociace atletických federací).
\dend
\dt Klub (Club)
\dd Klub nebo společenství atletů, organizované v rámci členské federace v souladu s jejími pravidly.
\dend
\dt Komise (Commission)
\dd Komise ustanovená Radou v souladu se statutem IAAF.
\dend
\dt Mezinárodní soutěž (International Competition)
\dd Soutěže v světovém atletickém seriálu (jak je popsáno v pravidlech), atletický program olympijských her a další soutěže organizované IAAF nebo jménem IAAF jak je uvedeno v pravidlech a předpisech
\dend
\dt Mezinárodní zvací mítink (International Invitation Meeting)
\dd Atletická soutěž jíž se účastní jeden nebo dva členové na pozvání pořadatele mítinku.
\dend
\dt MOV (IOC)
\dd Mezinárodní olympijský výbor (International Olympic Committee).
\dend
\dt Národní federace (National Federation)
\dd Člen IAAF sdružující atlety, podpůrné týmy atletů a ostatní osoby, kterých se týkají tato pravidla.
\dend
\dt Neutrální atlet (Neutral Athlete)
\dd Atlet, kterému rada IAAF přiznala zvláštní způsobilost soutěžit v jedné nebo více mezinárodních soutěžích, jak uvádí pravidlo P 22.1a. (Pravidla způsobilosti). Pokud není uvedeno jinak pravidla pro neutrální sportovce se využijí i pro trenéry, manažery, zástupce atleta a další osoby spolupracující s neutrálním sportovcem.
\dend
\dt Občan
\dd Osoba, která má legální občanství v zemi nebo v případě teritoria legální příslušnost mateřské zemi v tomto teritoriu a odpovídající legální postavení v tomto teritoriu podle příslušných zákonů.
\dend
\dt Občanství
\dd Legální státní příslušnost k zemi nebo v případě teritoria, legální příslušnost k mateřské zemi teritoria a příslušný legální status v daném teritoriu podle příslušných zákonů.
\dend
\dt Oblast (Area)
\dd Geografická oblast zahrnující všechny země a teritoria sdružené v jednu ze šesti oblastí Asociace
\dend
\dt Oblastní asociace (Area Association)
\dd Oblastní asociace IAAF, která odpovídá za rozvoj atletiky v jedné ze šesti oblastí do nichž jsou podle Statutu rozděleni členské federace IAAF.
\dend
\dt Osoba (Person)
\dd Fyzická osoba, organizace nebo jiná společnost.
\dend
\dt Pravidla (Rules)
\dd Soubor pravidel pro soutěže v rámci IAAF, která jsou uvedená v této příručce.
\dend
\dt Pravidla technická (Technical Rules)
\dd Pravidla uvedená v Oddíle 5 příručky Soutěžní pravidla IAAF
\dend
\dt Předpisy (Regulations)
\dd Předpisy IAAF vydávané podle potřeby Radou IAAF.
\dend
\dt Předpisy IAAF o zastupování atletů ( Athlete's representative regulations)
\dd Předpis týkající se zastupování atletů, vydaný podle potřeby Radou IAAF.
\dend
\dt Rada (Council)
\dd Rada IAAF
\dend
\dt Sídlo (Residence)
\dd Místo nebo území, kde je atlet registrován u příslušné organizace a má svůj původní nebo trvalý pobyt.
\dend
\dt Soutěž (Competition)
\dd Jedna nebo více atletických disciplín konajících se během jednoho nebo více dní.
\dend
\dt Statut (Constitution)
\dd Základní dokument jímž se řídí veškerá činnost IAAF.
\dend
\dt Světový atletický sériál (World Athletics Series)
\dd Hlavní mezinárodní soutěže čtyřletého cyklu oficiálního soutěžního programu IAAF.
\dend
\dt Území (Territory)
\dd Geografické oblast nebo region, které není státem, ale má rysy samosprávy, alespoň co se týká autonomního postavení v otázkách řízení sportu a které je takto uznávané IAAF.
\dend
\dt Zabezpečovací tým atleta (Athletes Support Personnel)
\dd kterýkoliv kouč, trenér, manažér, oprávněný zástupce atleta, agent, člen týmu, činovník, lékař nebo zdravotník, vychovatel nebo kterákoliv jiná osoba pracující pro nebo spolupracující s atletem při jeho přípravě nebo účasti na atletické soutěži.
\dend
\dt Zástupce atleta (Athlete's representative)
\dd osoba, která je řádně oprávněna a registrována jako zástupce atleta ve smyslu Předpisů IAAF o zastupování atletů.
\dend
\dt Země (stát)
\dd Samosprávné geografické území světa, uznávané mezinárodním právem a mezinárodními řídícími organizacemi jako nezávislý státní útvar.
\dend

POZN. 1 : Výše uvedené definice platí pro všechna ustanovení pravidel, s výjimkou případů uvedených v Antidopingových pravidlech.

POZN. 2 : Všechny odkazy v pravidlech uvedené v mužském rodu platí i pro ženský rod a všechny odkazy uvedené v jednotlivém čísle platí i v množném čísle.

POZN. 3 : Zelené poznámky jsou výklad a uplatnění pravidel, které bylo dříve uvedeno v Publikace IAAF \uv{Referee} (Vrchní rozhodčí) - \uv{Le juge arbitre}.

\input sections/sec1

\input sections/sec5

\bye