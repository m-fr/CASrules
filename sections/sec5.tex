\chap TECHNICKÁ PRAVIDLA

\rule{99}
\secc Všeobecně

Všechny mezinárodní soutěže uvedené v P1.1. se musí konat podle pravidel IAAF.

Při všech soutěžích, vyjma Mistrovství světa a Olympijských her, mohou být jednotlivé disciplíny konány v jiném provedení, než je dáno technickými pravidly IAAF, ale nemohou být uplatněna pravidla, poskytující atletům širší práva, než by měli při uplatnění stávajících pravidel.
Provedení soutěží musí být určeno nebo schváleno orgánem, který příslušnou soutěž řídí.

V případě soutěží s masovou účastí konaných mimo stadion mají být tato pravidla uplatňována plně pouze vůči těm atletům, kteří startují v elitní nebo jiné, přesně stanovené skupině startujících, jako jsou věkové kategorie, kde je umístění ohodnoceno cenami nebo odměnami.
Organizátoři závodů mají v informacích pro ostatní účastníky uvést, která další pravidla platí pro jejich účast, zejména ustanovení týkající se jejich bezpečnosti.

POZN.: Doporučuje se, aby členské země přijaly Pravidla IAAF pro své vlastní soutěže.

Pozn.: Pravidla IAAF platí pro všechny soutěže konané na území České republiky (dále jen ČR) a řízené Českým atletickým svazem (dále jen ČAS).

\begitems \style N \itemnum=30
* V soutěžích všech kategorií na území ČR jsou přípustná ustanovení jiná, než jak je uvedeno v Pravidlech atletiky ČAS. Taková ustanovení musí být uvedena v propozicích (Soutěžním řádu) příslušných soutěží. V tomto případě je nutné v propozicích (Soutěžním řádu) soutěží uvést ustanovení „Soutěže se konají podle Pravidel atletiky s úpravami uvedenými v těchto propozicích (Soutěžním řádu).“
* V každém případě musí být dodržena ustanovení Pravidel atletiky ČAS týkající se bezpečnosti, zejména ustanovení pravidla .
\enditems

\input sections/sec5/chap1

\input sections/sec5/chap2

\input sections/sec5/chap3

\input sections/sec5/chap4

\endinput